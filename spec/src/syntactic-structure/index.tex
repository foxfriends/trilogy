\section{Syntax and Semantics}

As previously mentioned, grammars in this document are
\href{https://en.wikipedia.org/wiki/Parsing_expression_grammar}{PEGs},
albeit lacking in all syntax sugar.

In this section, terminals each refer to token types, though for tokens
with a single source text representation, their textual representation
is preferred for readability. The productions refer to nodes in the
abstract syntax tree, and are named as they are in the implementation
source code.

Between any two tokens may be any number of comment nodes. Additionally,
any whitespace (including line breaks) may occur between any two tokens
without issue. In certain marked cases, however, line breaks are required
as a statement terminator; in such cases, the line break is always interpreted
as a statement terminator, and not as meaningless whitespace.

The required end of line rule is formalized as follows:

\begin{bnf*}
    \bnfprod{EOL}{
        \bnfts{EndOfLine}\bnfor
        \bnfts{CommentLine}\bnfor
        \bnfts{CommentBlock}\bnfor
    } \\
    \bnfmore{
        \bnfts{DocInner}\bnfor
        \bnfts{DocOuter}
    }
\end{bnf*}

\subsection{Trilogy}

The top level of a \Trilogy{} file is the only part of the document that
is not written in one of the child languages. This section of the program
may only contain definitions.

\subsubsection{The Document}

A \Trilogy{} document \emph{should} end with a final line break,
but files without such a final line break may be accepted (with warning).
It is optional to implement rule~\ref{bnf:finaleol}.

As a notable exception, rule~\ref{bnf:empty} indicates that a truly empty
file \emph{is} accepted, regardless of the fact that it does not end with
a line break.

A byte order mark \emph{may} be accepted (with warning) at the start of
the file. It is optional to implement rule~\ref{bnf:bom}.

\begin{bnf}
    \bnfprod*{Document}{
        \bnfpn{Preamble}
        \bnfsp
        \bnfpn{Definitions}
        \bnfsp
        \bnfpn{Postamble}
        \bnfor
    } \\
    \bnfmore{\label{bnf:empty}\bnfpn{Preamble}\bnfsp\bnfts{EndOfFile}} \\
    \bnfprod*{Preamble}{\bnfts{StartOfFile}\bnfor} \\
    \bnfmore{
        \label{bnf:bom}
        \bnfts{StartOfFile}
        \bnfsp
        \bnfts{ByteOrderMark}
    } \\
    \bnfprod*{Postamble}{
        \bnfts{EOL}
        \bnfsp
        \bnfts{EndOfFile}
        \bnfor
    } \\
    \bnfmore{\label{bnf:finaleol}\bnfts{EndOfFile}} \\
    \bnfprod*{Definitions}{
        \bnfpn{Definition}\bnfsp\bnfpn{Definitions}\bnfor\bnfes
    } \\
    \bnfprod*{Definition}{
        \bnfpn{ModuleDefinition}\bnfor
        \bnfpn{ProcedureDefinition}\bnfor
    } \\
    \bnfmore*{
        \bnfpn{FunctionDefinition}\bnfor
        \bnfpn{RuleDefinition}\bnfor
    } \\
    \bnfmore*{\bnfpn{Import}\bnfor\bnfpn{Export}}
\end{bnf}

\subsubsection{Modules}

Modules act as containers for definitions, including procedures,
functions, rules, and further submodules. Every \Trilogy{} document is
implicitly a module, meaning a document may export any of its defined items,
to make them available to other modules that import that document. Any item
defined at the top level of a module may be exported: procedures, functions,
rules, modules, and imported items.

The \Trilogy{} module system is modelled after that of OCaml: modules may
be defined as functions which accept one or more other modules as parameters,
and generate a new module. As in OCaml, such modules functions are \emph{applicative}:
if a module function is applied to the same arguments multiple times, it will
generate the same resulting module. At this time, a document cannot specify
parameters, so such function modules may only be defined as a local submodule
within some existing module.

Modules defined as external modules via the \kw{module}-\kw{at} construct
\todo{Define and link module resolution algorithm.}
are resolved as documents located via the canonical absolute URL designated
by the (potentially relative) path \texttt{String} in rule~\ref{bnf:module-at}.
If two external module definitions paths resolve to the same canonical absolute
URL, those two modules will be considered identical, despite being ``defined''
in two places in the source code. The full details of the path resolution
algorithm are defined later.

\begin{bnf}
    \bnfprod{ModuleDefinition}{
        \label{bnf:module-at}
        \bnfts{\kw{module}}
        \bnfsp
        \bnfts{Identifier}
        \bnfsp
        \bnfts{\kw{at}}
        \bnfsp
        \bnfts{String}
        \bnfor
    } \\
    \bnfmore*{\bnfts{\kw{module}}\bnfsp\bnfts{\{}\bnfsp\bnfpn{Definitions}\bnfsp\bnfts{\}}\bnfor} \\
    \bnfmore*{\bnfts{\kw{module}}\bnfsp\bnfts{(}\bnfsp\bnfpn{ModuleParameters}\bnfsp\bnfts{)}\bnfsp\bnfts{\{}\bnfsp\bnfpn{Definitions}\bnfsp\bnfts{\}}} \\
    \bnfprod*{ModuleParameters}{\bnfts{Identifier}\bnfsp\bnfts{,}\bnfsp\bnfpn{ModuleParameters}\bnfor} \\
    \bnfmore*{\bnfts{Identifier}\bnfor\bnfes} \\
    \bnfprod*{Import}{
        \bnfts{\kw{import}}
        \bnfsp
        \bnfpn{ModuleReference}
        \bnfsp
        \bnfts{\kw{as}}
        \bnfsp
        \bnfts{Identifier}
        \bnfor
    } \\
    \bnfmore*{
        \bnfts{\kw{import}}
        \bnfsp
        \bnfpn{ImportList}
        \bnfsp
        \bnfts{\kw{from}}
        \bnfsp
        \bnfpn{ModuleReference}
    } \\
    \bnfprod*{ImportList}{
        \bnfts{Identifier}
        \bnfsp
        \bnfts{,}
        \bnfsp
        \bnfpn{ImportList}
        \bnfor
    } \\
    \bnfmore*{
        \bnfts{IdentifierBang}
        \bnfsp
        \bnfts{,}
        \bnfsp
        \bnfpn{ImportList}
        \bnfor
    } \\
    \bnfmore*{
        \bnfts{Identifier}
        \bnfor
        \bnfts{IdentifierBang}
        \bnfor
        \bnfes
    } \\
    \bnfprod*{ModuleReference}{
        \bnfts{Identifier}
        \bnfor
        \bnfts{Identifier}
        \bnfsp
        \bnfts{(}
        \bnfsp
        \bnfpn{ModuleArguments}
        \bnfsp
        \bnfts{)}
    } \\
    \bnfprod*{ModuleArguments}{
        \bnfts{ModuleReference}
        \bnfsp
        \bnfts{,}
        \bnfsp
        \bnfpn{ModuleArguments}
        \bnfor
    } \\
    \bnfmore*{
        \bnfts{ModuleReference}
        \bnfor
        \bnfes
    }
\end{bnf}

A \kw{module} definition puts into scope a namespace in which all
\kw{export}ed definitions contained within the module may be accessed.
An \kw{import} may be used to bring specific items from such a namespace
into scope, or to rename a module something more convenient.

When a module is defined locally (directly in the same file as its parent
module), the definitions within the module may access all definitions (including
private definitions) of the containing module. The reverse is not true; the
private members of a submodule remain private unless exported.

\begin{prooftree}
    \def\defaultHypSeparation{\hskip 0in}
    \AxiomC{$\Gamma,m_{1\hdots n},\Phi \vdash p : \tau$}
    \AxiomC{$\kw{module}\ m\ (m_{1\hdots n})\ \{\ K\ \}$}
    \AxiomC{$K \Rightarrow \kw{export}\ p$}
    \LeftLabel{Local Module}
    \TrinaryInfC{$\Gamma \vdash m\texttt{.}p : \tau$}
\end{prooftree}

\begin{prooftree}
    \AxiomC{$\Phi \vdash p : \tau$}
    \AxiomC{$\kw{module}\ m\ \kw{at}\ \sigma$}
    \AxiomC{$\text{read}(\sigma) \Rightarrow \kw{export}\ p$ }
    \LeftLabel{External Module}
    \TrinaryInfC{$\Gamma \vdash m\texttt{.}p : \tau$}
\end{prooftree}

\begin{prooftree}
    \AxiomC{$\Gamma \vdash m.p : \tau$}
    \AxiomC{$\kw{import}\ m\ \kw{as}\ n$}
    \LeftLabel{Import As}
    \BinaryInfC{$\Gamma \vdash n\texttt{.}p : \tau$}
\end{prooftree}

\begin{prooftree}
    \AxiomC{$\Gamma \vdash m.p : \tau$}
    \AxiomC{$\kw{import}\ p\ \kw{from}\ m$}
    \LeftLabel{Import From}
    \BinaryInfC{$\Gamma \vdash p : \tau$}
\end{prooftree}

\subsubsection{Definitions}

The other types of definitions each hand off to their respective child
languages.

\begin{bnf*}
    \bnfprod{ProcedureDefinition}{
        \bnfts{\kw{proc}}
        \bnfsp
        \bnfpn{ProcedureHead}
        \bnfsp
        \bnfts{\{}
        \bnfsp
        \bnfpn{ProcedureBody}
        \bnfsp
        \bnfts{\}}
    } \\
    \bnfprod{FunctionDefinition}{
        \bnfts{\kw{func}}
        \bnfsp
        \bnfpn{FunctionHead}
        \bnfsp
        \bnfts{=}
        \bnfsp
        \bnfpn{FunctionBody}
    } \\
    \bnfprod{RuleDefinition}{\bnfts{\kw{rule}}\bnfsp\bnfpn{RuleHead}\bnfor} \\
    \bnfmore{\bnfts{\kw{rule}}\bnfsp\bnfpn{RuleHead}\bnfsp{\op{:-}}\bnfsp\bnfpn{RuleBody}}
\end{bnf*}

\subsection{Prose}

When viewed on its own, \Prose{} has a fairly ``standard'' C-family syntax.
Arbitrary patterns are permitted in the procedure heads.
Trailing commas are permitted in all places where commas are found.

\begin{bnf*}
    \bnfprod{ProcedureHead}{
        \bnfts{IdentiferBang}
        \bnfsp
        \bnfts{(}
        \bnfsp
        \bnfpn{ParameterList}
        \bnfsp
        \bnfts{)}
    } \\
    \bnfprod{ParameterList}{
        \bnfpn{Pattern}
        \bnfsp
        \bnfts{,}
        \bnfsp
        \bnfpn{ParameterList}
        \bnfor
    } \\
    \bnfmore{
        \bnfpn{Pattern}
        \bnfor
        \bnfes
    }
\end{bnf*}

\begin{prooftree}
    \def\defaultHypSeparation{\hskip 0.05in}
    \AxiomC{$\kw{proc}\ p\texttt{!}(x_{1\hdots n})\ \{\ M\ \}$}
    \AxiomC{$\Gamma,x_{1\hdots n}:\tau_{1\hdots n} \vdash N : \tau$}
    \AxiomC{$M \Rightarrow \kw{return}\ N$}
    \LeftLabel{Procedure}
    \TrinaryInfC{$\Gamma \vdash p: \tau_{1\hdots n}\rightarrow\tau$}
\end{prooftree}

The body of a procedure is a sequence of statements. Statements in sequence
must be separated by a line break or an explicit separator (\op{,}).

\begin{bnf*}
    \bnfprod{ProcedureBody}{\bnfpn{Sequence}} \\
    \bnfprod{Sep}{\bnfpn{EOL}\bnfor\bnfts{,}} \\
    \bnfprod{Sequence}{\bnfpn{Statement}\bnfsp\bnfpn{Sep}\bnfsp\bnfpn{Sequence}\bnfor} \\
    \bnfmore{\bnfpn{Statement}\bnfor\bnfes}
\end{bnf*}

A sequence of statements evaluates to the result of the last statement
in the sequence. The values of preceding statements are lost.

\begin{prooftree}
    \AxiomC{$\Gamma \vdash M : \tau_0$}
    \AxiomC{$\Gamma \vdash N : \tau_1$}
    \LeftLabel{Sequencing}
    \BinaryInfC{$\Gamma \vdash M \op{,}\ N : \tau_1$}
\end{prooftree}

While most imperative languages bring some sort of distinction between
expressions and statements, \Prose{} does not. Statements in this
sense might have been more aptly referred to as expressions, but we
stick to the statement terminology to differentiate from expressions
in \Poetry{}.

Any statement or expression may occur within any other statement, and
all statements evaluate to some value, albeit sometimes \kw{unit}.
This can make for some pretty unexpected combinations
(e.g. \texttt{x == 5 or return 3}); whether it is a good idea to use
such combinations is left to the developer to decide.

While some statements are unique to \Prose{}, others look remarkably similar
to those found in the other languages, but while allowing other statements
to be nested within; for example, \texttt{Int readline!()} could almost be
a \Poetry{} application of \fn{Int} if not for the procedure call in argument
position. As procedure calls are not permitted in \Poetry{}, this ends up
producing very similar productions with very similar semantics. To
differentiate potentially ambiguous statement production names, they
are prefixed with \texttt{S}.

\begin{bnf*}
    \bnfprod{Statement}{
        \bnfpn{SLet}
        \bnfor
        \bnfpn{SAssignment}
        \bnfor
        \bnfpn{SIfElse}
        \bnfor
    } \\
    \bnfmore{
        \bnfpn{SMatch}
        \bnfor
        \bnfpn{SFor}
        \bnfor
        \bnfpn{SWhile}
        \bnfor
        \bnfpn{SLoop}
        \bnfor
    } \\
    \bnfmore{
        \bnfpn{SBreak}
        \bnfor
        \bnfpn{SContinue}
        \bnfor
        \bnfpn{SReturn}
        \bnfor
    } \\
    \bnfmore{
        \bnfpn{SYield}
        \bnfor
        \bnfpn{SWhen}
        \bnfor
        \bnfpn{SGiven}
        \bnfor
    } \\
    \bnfmore{
        \bnfpn{SCollection}
        \bnfor
        \bnfpn{SComprehension}
        \bnfor
    } \\
    \bnfmore{
        \bnfpn{SInfixOp}
        \bnfor
        \bnfpn{SApplication}
        \bnfor
    } \\
    \bnfmore{
        \bnfpn{ProcedureCall}
        \bnfor
        \bnfpn{DoBlock}
        \bnfor
        \bnfpn{Block}
        \bnfor
    } \\
    \bnfmore{
        \bnfpn{ValueAccess}
        \bnfor
        \bnfpn{ParenthesizedStatement}
    } \\
    \bnfprod{Substatement}{
        \bnfpn{SPrefixOp}\bnfor\bnfpn{Statement}
    } \\
    \bnfprod{ParenthesizedStatement}{
        \bnfts{(}
        \bnfsp
        \bnfpn{Substatement}
        \bnfsp
        \bnfts{)}
    } \\
    \bnfprod{Block}{
        \bnfts{\{}
        \bnfsp
        \bnfpn{Sequence}
        \bnfsp
        \bnfts{\}}
    }
\end{bnf*}

The one distinction is that a top-level statement may not begin with a
prefix operator. Given the lack of postfix operators in Trilogy, that
leaves infix operators being the only valid operators at the top-level
of a sequence.

Add the trivial evaluation semantics for parenthesized and block
expressions:

\begin{figure}[H]
    \centering
    \parbox[t]{0.35\linewidth}{
        \begin{prooftree}
            \AxiomC{$\Gamma \vdash M : \tau$}
            \LeftLabel{Unwrap}
            \UnaryInfC{$\Gamma \vdash \texttt{(}\ M\ \texttt{)} : \tau$}
        \end{prooftree}
    }
    \parbox[t]{0.35\linewidth}{
        \begin{prooftree}
            \AxiomC{$\Gamma \vdash M : \tau$}
            \LeftLabel{Unnest}
            \UnaryInfC{$\Gamma \vdash \texttt{\{}\ M\ \texttt{\}} : \tau$}
        \end{prooftree}
    }
\end{figure}

\subsubsection{Let}

\subsubsection{Assignment}

\subsubsection{Conditionals}

The conditional statement in \Trilogy{} comes in the form of \kw{if}-\kw{else}
chains. For \Prose{} they are defined as follows.

\begin{bnf*}
    \bnfprod{SIfElse}{
        \bnfts{\kw{if}}
        \bnfsp
        \bnfpn{Condition}
        \bnfsp
        \bnfpn{Block}
        \bnfsp
        \bnfpn{SElseChain}
        \bnfor
    } \\
    \bnfprod{SElseChain}{
        \bnfts{\kw{else}}
        \bnfsp
        \bnfts{\kw{if}}
        \bnfsp
        \bnfpn{Condition}
        \bnfsp
        \bnfpn{Block}
        \bnfsp
        \bnfpn{SElseChain}
        \bnfor
    } \\
    \bnfmore{
        \bnfts{\kw{else}}
        \bnfsp
        \bnfpn{Block}
        \bnfor
        \bnfes
    }
\end{bnf*}

In the common case, the condition is a statement that evaluates to a Boolean
result, \kw{true} or \kw{false}, and control flows as you might expect.
\todo{Define runtime failure}
If the result is not a Boolean, this triggers a runtime failure.

An \kw{if}-\kw{else} chain evaluates to the value of the block that is executed.
If no block is executed (there is no \kw{else} clause, and none of the
conditions matched) the resulting value is \kw{unit}.

\begin{prooftree}
    \AxiomC{$C : \kw{true}$}
    \AxiomC{$T : \tau_1$}
    \AxiomC{$F : \tau_2$}
    \LeftLabel{Branch Hit\textsuperscript{A}}
    \TrinaryInfC{$\kw{if}\ C\ T\ \kw{else}\ F : \tau_1$}
\end{prooftree}
\begin{prooftree}
    \AxiomC{$C : \kw{false}$}
    \AxiomC{$T : \tau_1$}
    \AxiomC{$F : \tau_2$}
    \LeftLabel{Branch Miss\textsuperscript{A}}
    \TrinaryInfC{$\kw{if}\ C\ T\ \kw{else}\ F : \tau_2$}
\end{prooftree}
\begin{prooftree}
    \AxiomC{$C : \kw{true}$}
    \AxiomC{$T : \tau$}
    \LeftLabel{Branch Hit\textsuperscript{B}}
    \BinaryInfC{$\kw{if}\ C\ T : \tau$}
\end{prooftree}
\begin{prooftree}
    \AxiomC{$C : \kw{false}$}
    \AxiomC{$T : \tau$}
    \LeftLabel{Branch Miss\textsuperscript{B}}
    \BinaryInfC{$\kw{if}\ C\ T : \kw{unit}$}
\end{prooftree}

In addition to Boolean evaluations, Conditions may be queries which bind
values to names available for the scope of the block of the first matching
condition. This is similar to the \texttt{if let} statement seen in other
languages, but extended to cover more than just pattern matching by taking
advantage of \Law{}. Conditions may not be empty, but trailing commas are
still supported, so we end up with this annoying extra bit of rule.

\begin{bnf*}
    \bnfprod{Condition}{
        \bnfpn{Statement}
        \bnfsp
        \bnfts{,}
        \bnfsp
        \bnfpn{Condition}
        \bnfor
        \bnfpn{Query}
        \bnfsp
        \bnfts{,}
        \bnfsp
        \bnfpn{Condition}
        \bnfor
    } \\
    \bnfmore{
        \bnfpn{Statement}
        \bnfsp
        \bnfts{,}
        \bnfor
        \bnfpn{Query}
        \bnfsp
        \bnfts{,}
        \bnfor
        \bnfpn{Statement}
        \bnfor
        \bnfpn{Query}
    } \\
    \bnfprod{Query}{
        \bnfpn{Statement}
        \bnfsp
        \bnfts{\kw{is}}
        \bnfsp
        \bnfpn{Pattern}
        \bnfor
        \bnfpn{RuleCheck}
    }
\end{bnf*}

A conditional is not an iterative context. If a pattern condition matches
to multiple results, the first result is chosen and the rest are skipped.
Meanwhile, if a pattern is a statement which evaluates to

