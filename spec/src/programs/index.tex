\section{Programs}

While the semantics of the language are fun and all, the practical matters of
executing a program must also be discussed. One of the goals of \Trilogy{} is
to be a practical language, which can be used to build actual programs after
all.

\subsection{Program Arguments}

As with any program run on the command line, \Trilogy{} programs may accept
command line arguments which alter the behaviour of the program.

When run through the interpreter command, the command line arguments passed
to the \Trilogy{} program are specified after the separator \texttt{--}. When
the path to the \Trilogy{} interpreter is specified via the shebang, arguments
may be passed directly.

From within a \Trilogy{} program, these command line arguments are received
as an array of strings which may be accessed via the built in procedure
\fn{args}.

\subsection{Errors}

Some errors are considered more non-recoverable than others, typically those
that are as a result of code that likely does not do what the author intended.
In fact, there are no built-in ``error handling'' mechanisms because indeed
a well formed \Trilogy{} program code contains no error states, only exceptional
circumstances. A true ``error'' is something that cannot be recovered from, and
immediately ends the \Trilogy{} program, which exits with exit code \texttt{255}
indicating an error.

The most common of these errors is the \emph{runtime type error}, where an
expression is expected by the runtime to evaluate to a value of a particular
type, but it does not. As this is most likely a developer error, the safest
result as directed by the fail-fast principle is to end the program immediately.

Another such error is an \emph{assertion error}, triggered by the \kw{assert} keyword's
condition evaluating to \kw{false}. This allows custom abstractions to enforce
their invariants and provide in a way that provides a clear indication that
something is being done incorrectly.

A fizzle is typically not treated as an error; fizzles are often expected
to cull unnecessary branches of execution. Fizzles are caused only by bindings
which do not have any solutions. The only time a fizzle is treated as an error
is when it is the last execution in the program, in which case it is an \emph{execution
fizzled error}.

Most things considered ``errors'' languages with typical exception
handling constructs (e.g. \texttt{try} and \texttt{catch}) are handled through
the effect system in \Trilogy{}; situations such as I/O failures and division by
zero are in this category. Though these situations are yielded via the effect
system, if you do not handle that effect, then it does become an
\emph{unhandled effect error}. This is true of any effect that is yielded in
a program, even effects of your own design; if they are not handled by any
surrounding handler, there can be no known course of action so the program
must end.

Finally, in the case of an unrecoverable error in the runtime implementation
itself, the program ends with an \emph{internal runtime error}. This situation
should be rare, maybe caused by the host process running out of memory, or
other unexpected hardware failure. Honestly, you should hope never to witness
this, as it means there is something seriously going wrong.

\subsection{Exit Codes}
\label{sec:exitcode}

Apart from runtime errors causing the program to exit with a code of \texttt{1},
the developer may choose their own exit codes in a few ways.

In the simplest case, a \Trilogy{} program's exit code is determined by the
return value of its \fn{main} procedure. The first execution to reach the end
of the \fn{main} procedure ends the program with the return value as the exit
code. If the return value of \fn{main} is not a valid exit code value, this
is a runtime type error and exit code is set to \texttt{255}.

Alternatively, the program may be ended at any time by using the \kw{end} keyword.
While the \kw{end} keyword with no argument ends the current execution (as a fizzle),
an \kw{end} with a value exit code value ends the program with that value. Again, if
the value is not a valid exit code value, it is a runtime type error.
