\section{Programs}

While the semantics of the language are nice, the practical matters of
executing a program must also be discussed.

\subsection{Exit Codes}
\label{sec:exitcode}

On all operating systems I have experienced, programs end with an exit
code indicating the status of the programs execution.

A \Trilogy{} program's exit code is determined by the return value (or
return values, in the case of parallel executions) of its \fn{main} procedure.

% TODO: this is just rambling, will have to make it more proper.

If a single evaluation makes it to the end of the program (i.e. runs off the end of the
\fn{main} procedure, rather than fizzling) the program is also considered a success (exit 0).
If not \emph{exactly} one (0 or more than 1) evaluation runs off the \fn{main} procedure,
the program is treated as unsuccessful due to indeterminite result, and exits 1.


If the program itself fizzles (i.e. the last remaining execution in
a program has fizzled), the program is considered to have ended in
failure, and will exit with an exit code of 1. This error status code
\emph{may} also be accompanied by a printed error message describing
to the user the circumstances of the program's failure, for debugging
purposes.

If the program terminates implicitly (i.e. not with the \kw{end} keyword)
through more than one execution it is again considered a failure, even if
all those executions were successful. A program running in the real world
is restricted to determinism and may only exit with a single value;
a program exiting by more than one path in this way is not something
supported by any operating system, so the only safe assumption is that
the programmer made an error.

Given that, the example program above will always end in failure, but
depending on the first character of the line read, may or may not
print multiple possible words first.

To avoid such confusing behaviour, it is generally recommended that
nondeterministic branching is not used directly at the top level without
some handling to combine the parallel executions or to recover from all
executions fizzling in a way that makes sense for the current program.
Alternatively, explicit use of the \kw{end} keyword to disambiguate the
exit codes may help, but is still a bit confusing.
