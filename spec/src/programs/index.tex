\section{Programs}

While the semantics of the language are fun and all, the practical matters of
executing a program must also be discussed. One of the goals of \Trilogy{} is
to be a practical language, which can be used to build actual programs after
all.

\subsection{Program Arguments}

As with any program run on the command line, \Trilogy{} programs may accept
command line arguments which alter the behaviour of the program.

When run through the interpreter command, the command line arguments passed
to the \Trilogy{} program are specified after the separator \texttt{--}. When
the path to the \Trilogy{} interpreter is specified via the shebang, arguments
may be passed directly.

From within a \Trilogy{} program, these command line arguments are received
as an array of strings which may be accessed via the built in procedure
\fn{args}.

\subsection{Errors}

Some errors are considered more non-recoverable than others, typically those
that are as a result of code that likely does not do what the author intended.
In fact, there are no built-in ``error handling'' mechanisms because indeed
a well formed \Trilogy{} program code contains no error states, only exceptional
circumstances. A true ``error'' is something that cannot be recovered from, and
immediately ends the \Trilogy{} program, which exits with a non-zero exit code
indicating an error.

The most common of these errors is the \emph{runtime type error}, where an
expression is expected by the runtime to evaluate to a value of a particular
type, but it does not. As this is most likely a developer error, the safest
result as directed by the fail-fast principle is to end the program immediately.

Another such error is an \emph{assertion error}, triggered by the \kw{assert} keyword's
condition evaluating to \kw{false}. This allows custom abstractions to enforce
their invariants and provide in a way that provides a clear indication that
something is being done incorrectly.

A fizzle is typically not treated as an error; fizzles are often expected
to cull unnecessary branches of execution. Fizzles are caused only by bindings
which do not have any solutions. The only time a fizzle is treated as an error
is when it is the last execution in the program, in which case it is an \emph{execution
fizzled error}.

Most things considered ``errors'' languages with typical exception
handling constructs (e.g. \texttt{try} and \texttt{catch}) are handled through
the effect system in \Trilogy{}; situations such as I/O failures and division by
zero are in this category. Though these situations are yielded via the effect
system, if you do not handle that effect, then it does become an
\emph{unhandled effect error}. This is true of any effect that is yielded in
a program, even effects of your own design; if they are not handled by any
surrounding handler, there can be no known course of action so the program
must end.

Finally, in the case of an unrecoverable error in the runtime implementation
itself, the program ends with an \emph{internal runtime error}. This situation
should be rare, maybe caused by the host process running out of memory, or
other unexpected hardware failure. Honestly, you should hope never to witness
this, as it means there is something seriously going wrong.

\subsection{Exit Codes}
\label{sec:exitcode}

Apart from runtime errors causing the program to exit with a code of \texttt{255},
the developer may choose their own exit codes in a few ways.

In the simplest case, a \Trilogy{} program's exit code is determined by the
return value of its \fn{main} procedure. The first execution to reach the end
of the \fn{main} procedure ends the program with the return value as the exit
code. If the return value of \fn{main} is not a valid exit code value, this
is a runtime type error and exit code is set to \texttt{255}.

Alternatively, the program may be ended at any time by using the \kw{exit} keyword.
The \kw{exit} keyword ends the program with the value it is provided. Again, if
the value is not a valid exit code value, it is a runtime type error and the exit code
is set to \texttt{255}.

\subsection{Module Resolution}
\label{sec:module-resolution}

The ability to define external modules is one of practical concern. Though it
would technically be possible to write your entire program in one file,
it simply wouldn't be a good time. External modules in \Trilogy{} are
referenced by strings, and the ``official'' method by which those strings
are resolved into the modules they are meant to represent is described here.
Alternative module resolution algorithms may be implemented as extensions by
user code in the host language.

Strings designating modules should come in one of a few forms: relative paths,
absolute paths, or absolute URLs. Given that, there are two phases to the module
resolution algorithm: first relative and absolute paths are resolved to absolute
URLs, then those absolute URLs are resolved to module.

An absolute path begins with a \texttt{/} character, or the sequence \texttt{file:///}
indicating an absolute URL with its scheme included, and is resolved to an absolute
URL as follows:

\begin{enumerate}
    \item The schema and authority of URL to the current \Trilogy{} file is prepended
    to the path (replacing any existing scheme if present).
    \begin{itemize}
        \item For files on the local system, this would be \texttt{file://}.
        \item For files located at a remote URL, these values are taken from that URL.
    \end{itemize}
    \item The path is \emph{normalized}.
\end{enumerate}

\noindent
A relative path begins with a \texttt{.} character, and is resolved to an absolute
URL as follows:

\begin{enumerate}
    \item The scheme, authority, and path up to and including its final \texttt{/} character
    are prepended to the relative path.
    \begin{itemize}
        \item For files on the local system, the scheme and authority are \texttt{file://}.
        \item For files located at a remote URL, the scheme, authority, and path taken
        from that URL.
    \end{itemize}
    \item The path is \emph{normalized}.
\end{enumerate}

\noindent
Normalization is performed as follows:

\begin{enumerate}
    \item Any empty path segments (\texttt{//}) are replaced with \texttt{/}
    \item Any useless path segments (\texttt{/./}) are replaced with \texttt{/}
    \item For any ``parent'' path segments following a named path segment
    (e.g. \texttt{/folder/../}), replace the two segments together with \texttt{/}
    \item If the final URL begins with \texttt{/../}, the path is marked invalid and
    the referenced module cannot be resolved.
\end{enumerate}

The resulting URL is now the absolute URL that is used to locate the module being
resolved. This is also the URL that is used to further resolve external modules
referenced by the resolved module. The file's contents are then retrieved depending
on its scheme. At this time, the following schemes are supported:

\begin{description}
    \item[\texttt{file}] the file is loaded from the current file system using
    the operating system's native means. Symbolic links should be resolved, but
    the module is still considered to be located at the original URL rather than
    the URL that the link pointed to.
    \item[\texttt{http} or \texttt{https}] the file is loaded by
    making an HTTP or HTTPS \texttt{GET} request (respectively) to the URL, as
    would be sent by following standard cURL command:

    \lstinputlisting[language=sh]{get-curl.sh}

    This request must respond with a response code of \texttt{200}, and the
    body of the response must contain the \Trilogy{} source code.

    It is the responsibility of the server hosting these paths that the requests
    are handled in an appropriate manner and that further external modules referenced
    in those files are resolvable.
\end{description}

In any case, the expected result is that the contents of the specified \Trilogy{}
file are retrieved as a UTF-8 encoded text file. If this is not the case for any
reason (invalid file contents, insufficient file system permissions, failed HTTP
request, etc.), then the module cannot not be resolved and the program is treated
as invalid.

Within a single invocation of the \Trilogy{} interpreter, any given module should
only be resolved once. If two paths in the program resolve to the same absolute
URL, those two modules are identical, as if aliases of each other. A reference to
such a module in the source code by any path is exactly equivalent.
