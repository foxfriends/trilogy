\subsection{Samples}

Of course we begin with Hello World and build up into all the crazy things this
language allows you to do.

Notice that procedures in \Trilogy{} are defined with \kw{proc} and end with an
\texttt{!}, and the entrypoint to the program is a procedure called \fn{main!}:

\inputcode{hello_world.tri}

Procedures are written in \Prose{}, which is an imperative language following Rust.
Since every program begins at the \fn{main!} procedure, every program naturally starts
in \Prose{}.

As an imperative language, \Prose{} features features familiar to other imperative
languages. Some of those features are seen here in a basic implementation of the
classic Guessing Game (typically my second program in any language, following a
successful Hello World).

\inputcode{imperative.tri}

A procedure is as it sounds: a series of steps to be taken in sequence in order
to produce effects or compute values. Such code is meant to be able to be read
in a straightforward manner. Control flow is described using explicit keywords.

This particular procedure should hopefully make sense to the average reader,
despite a few structures that may not be immediately familiar. Note that it is
assumed that the average reader of this document is a programmer.

A first look at \Poetry{} immediately dismisses some assumptions you may have
made previously about the direction of this language.

\inputcode{fibonacci.tri}

A function in \Trilogy{} is written in \Poetry{}, which is a functional
language that follows from Haskell. Beyond just being written using a more
traditionally functional syntax, functions written in \Poetry{} are \emph{pure}
(i.e.\ they cannot perform side effects). This is in contrast to \Prose{} which
performs side effects freely; IO and mutation were both seen in the previous
example.

Functions are defined with the \kw{func} keyword, to clearly differentiate them
from procedures. To further differentiate functions from procedures, they are called
without parentheses or commas (the function name and all following arguments are space
separated). They support partial application and currying as one might expect coming
from Haskell.

Functions may be called directly from \Prose{}, where these rules continue to apply.
This has the effect of bleeding functional programming into the imperative language.
Think of this like all the
\href{https://tolkiengateway.net/wiki/Poems_in_The_Lord_of_the_Rings}{poems in \emph{The Lord of the Rings}}.

\inputcode{functional.tri}

Here we have a more thoroughly functional example of a program that converts an
input number from decimal to hexadecimal. In particular, we make great use of
calling functions from \Prose{} with the help of the pipe operator (\op{|>}),
which simply calls the function on the right with the value on the left. Recall
that with functions being curried, an expression such as \texttt{map toHex} is a
function (given that \fn{map} is a function that takes a transformer function and
returns a function that applies that transformer over a list).

In addition to full-fledged functions, \Poetry{} supports lambdas (\kw{fn}) which
may be defined within functions and act as closures. Lambdas and regular functions
are both first class functions, and may be passed around as values in much the same
ways.

This is also our first taste of more complex data types in \Trilogy{}, but does
not provide a very clear look. Here's one that more obviously manipulates some
data by implementing a program like \href{https://ss64.com/bash/wc.html}{\texttt{wc}},
while manually implementing many functions that are actually part of the standard library.

\inputcode{wc.tri}

Notably, arrays are described within brackets (\op{[]}), where the colon
(\op{:}) separates the head of the array (a fixed number of elements) from the tai.
(an array onto which those fixed elements are attached, if any). This syntax is
reminiscent of the list syntax from other languages (think Prolog/Erlang), but
is implemented using actual arrays, and not linked lists.

Using \op{:} without the brackets, as seen in the definition of \fn{getWord},
creates a tuple. Those looking for linked lists might find them trivially implemented
using tuples.

Arrays being arrays makes them natural and efficient to manipulate in \Prose{}, as
seen in \fn{readlines!} where we extend the array by assigning to the index one
past the last existing index in the array, using a syntax more like that of most
languages which feature arrays. A notable difference is that the member access
``operator'' (\op{.}) is required before the index is supplied in brackets.

Arrays in \Trilogy{} are of dynamic length, they will grow and shrink as required
to contain all the elements assigned to them. That is, assigning to an index past
the end of the array will instead append a value to the array. Similarly, assigning
to an index before the beginning of the array will instead prepend that element, and
shift all existing elements to larger indices. Though assignment may occur to indices
well past the ends of the array, arrays always start at index 0 and may not contain
holes; immediately after assignment array elements will be shifted, maintaining order,
to satisfy those requirements.

In addition to arrays, and tuple-based linked lists, \Trilogy{} also natively
supports records (or dictionaries, maps, hashes, whatever you prefer to call them)
and sets. All four of these fundamental collections can be iterated as seen in the
following example.

\inputcode{meal.tri}

\Law{}, is a logic language, following Prolog, supporting the nondeterministic,
multiple-solution semantics that Prolog users might expect. A definition in \Law{}
is a \kw{rule} defining a pattern that must hold true, which may be implied by
other existing patterns.

Rules may be applied in most situations where patterns are being matched, making
for a very powerful pattern-matching experience. When used within an \emph{iterative
context}, if a pattern binds variables as a result of looking up a rule, those bindings
are iterated over successively.

In \Prose{}, the primary iterative context is the \kw{for} loop. Unlike other
\kw{for} loops you may have seen, the head of such a loop in \Prose{} is actually
a pattern matching, which is triggered multiple times by multi-solution patterns.
As an additional unusual feature, the \kw{for} loop may include an \kw{else} clause
which is called if the pattern in the head of the loop did not bind to anything.

In \Poetry{}, the primary iterative context is in comprehension. Comprehensions
collect each of their yielded items into a new iterator, or directly into a
collection. The four types of comprehensions (array \op{[]}, set \op{[||]},
record \op{\{||\}} and iterator \op{\$()}) are each seen in this example.

Most functions from the standard library that act on multiple values are built to
function over generic iterators, rather than arrays or lists or any other one
collection in particular, allowing us to pass the iterator directly to the
\fn{sum} function, which can then compute the sum without ever allocating storage
for any collection. That this is made possible by the fact that iterators
and comprehensions are lazy, which also implies that we may have infinite
iterators. Collections, however, remain finite, and the act of collecting
an iterator into a collection triggers the eager evaluation of that iterator.

\inputcode{foodchain.tri}

Looking more closely at \Law{}, notice that as patterns are reflexive, we are
able to check both directions along the food chain with just one definition of
\fn{chain}, while defining the food chain using the flat structure of \fn{eats}.

Now may also be a good time to bring up atoms, which take the form of identifiers
prefixed by an apostrophe (\texttt{'}). Atoms are a staple of logic languages
such as Prolog, which also make appearances in functional languages where
they are sometimes called symbols. They are values without inherent meaning,
but can be compared in constant time, and are great for matching in patterns.

Atoms have actually appeared in most of the previous examples, in the \kw{when}
clauses of various expressions.

\inputcode{effects.tri}

The \kw{when} clause is \Trilogy{}'s implementation effect handlers, where
effects are triggered by the \kw{yield} keyword.

Effects can be seen as the parent of generators, iterators, monads,
exceptions\dots many control flow or fancy patterns can be implemented using
effects and, in fact, many of the control flow and fancy patterns seen so far
are actually powered by this effect system deep down.

In this latest example, we have used effects to implement a sort of try-catch,
but with the bonus of being able to handle the exception and resume the program
from where it left off, as if the exception had never happened. Specifically,
\kw{when} the \fn{Int} function \kw{yield}s that \var{line} is \val{'NAN}
(``not a number''), \fn{readsum!} in turn \kw{yield}s the string in hopes that
some handler will be able to figure out what to do with it and pass back something
more useful. The \kw{when} clauses in the \fn{main!} procedure guard the call to
\fn{readsum!} and \kw{resume} with numbers parsed from the string, when possible.

As we have seen in most of the examples so far, \kw{when} clauses are also able
to \kw{cancel} in response to an effect, behaving more similarly to a traditional
\texttt{catch} handler.

\inputcode{functional_effects.tri}

Luckily for us, effects are functionally pure, so they can be both used and
handled from \Poetry{} as well, everything works pretty much the same. What's
interesting is that their handlers \emph{don't} need to be functionally pure.
By making use of effects, we can bring IO to \Poetry{}.

This latest example implements a guessing game equivalent to that seen earlier
written purely in \Prose{}, but now taking advantage of effects. Why you would
do this particular thing, I cannot say, but at least now you \emph{can}.

While I cannot explain this effect system any better than
\href{https://www.eff-lang.org/handlers-tutorial.pdf}{existing literature},
eventually maybe you can see how this can be used to implement more than just
a janky version of ``pure'' IO.

\inputcode{collect.tri}

In this example, we implement something like a generator function, and then
also a procedure to collect those results into a collection (as list comprehension
does for standard iterators), but all purely as user-written code. The
\fn{collect!} procedure accepts another procedure as an argument, which is
made easy using first-class procedures (\kw{do}).

More interesting than simply collecting is when we also apply a \fn{map!}
procedure to transform the previously \kw{yield}ed values using the provided
transformation function, much like \fn{map} when used on iterators.

Even more interesting is when we take control into our own hands. It
turns out, what comes after the \kw{when} may be one of three things. So
far we have seen \kw{cancel}, which aborts the entire watched expression
before running a provided expression in the surrounding context. We have
also seen \kw{resume} which evaluates the expression and then resumes
from the \kw{yield} where the effect originated with the value of that
expression.

The last option is to \kw{invert} the flow of the program, turning \kw{resume}
into a \emph{first class continuation}. The continuation can be thought of as
a function which accepts a single value, which is the value that the
\kw{yield} will resume with when the continuation is called.

\inputcode{continuations.tri}

What really makes continuations powerful is that they can be called
multiple times. In this highly contrived example, we use \op{==} to
check if a provided word equals the ``allowed'' word, which is retrieved
as an effect. What we didn't tell this function is that sometimes there
are multiple allowed words, but that's ok, it all works out anyway.

By resuming the continuation multiple times, once with each allowed
word, and then collecting the results of those multiple resumptions
later, what appears to be nondeterminism can be implemented very
easily, and without the function even needing to know that it is
maybe running nondeterministically.

While all that is fun and cool, this particular problem would be
more easily and obviously implemented using dynamic rules
and the natural nondeterminism introduced by \Law{}.

\inputcode{dynamic_rules.tri}

The \kw{given} keyword annotates an expression in which one or more
rules (\kw{rl}) are made available, similarly to how the \kw{when}
keyword annotates the expression with an effect handler.

Meanwhile, the \kw{is} keyword seen here introduces a Boolean context,
converting the result of calling a rule of \Law{} into a Boolean,
rather than letting it act as a branch in control flow as it would
have by default. An \kw{if} statement acts similarly, but with the
capability of conditionally introducing bindings for its branches,
similar to the \texttt{if let} statement of some other languages.

\inputcode{fizzle.tri}

Omitting the \kw{is} keyword, rules of \Law{} may still be called
when in a default binding context, such as that of \kw{let}. In this case,
nondeterminism in execution is exposed leading to some unusual but
interesting behaviour.

When a pattern in a binding context fails to match, we say the
execution \emph{fizzles}, ending it silently. Meanwhile, if a pattern in
the binding context binds more than once, the execution \emph{branches} into
multiple parallel executions, which run independently of each other
(in different execution contexts).
