\subsection{Additional Features}

In addition to the core programming language, \Trilogy{}'s ecosystem includes
facilities for code sharing, scripting, documentation, and testing, which
take advantage of specific features of the syntax.

\subsubsection{Code Sharing}

Code sharing in \Trilogy{} comes in the form of modules, which follow after
the module system of OCaml.

\inputcode{module.tri}

Beginning with the \kw{module} keyword, we can define a simple module. A
module may contain any number of declarations of any type (procedures,
rules, functions, and even nested modules). Names listed in an \kw{export}
statement are available to other modules that \kw{import} the module and
\kw{use} those declarations.

\inputcode{adt.tri}

Modules may alternatively (or additionally) be imported ``whole''. \Trilogy{}
modules are first class values, somewhat similar to read-only records, so they
can be passed to functions or procedures, allowing the contents to be used
generically. This technique can be used to build something similar to ADTs
and generic functions.

\inputcode{functor.tri}

Modules may also take parameters, granted that those parameters can be found
and applied statically. When doing so, \kw{from} must be used to denote that
module application and separate it from the final name.

An \kw{import} statement is just a scoping mechanism, so one may be written
within another declaration, even with parameters, but they will still be
resolved statically.

This module system is \emph{applicative}; that is, when applied to the same
(resolved) arguments, a module function will return the exact same module.
Given the lack of named and structured data types in \Trilogy{}, this is
naturally the only option - there is nothing that could be made generative.
Supposing syntactically named types were eventually added to \Trilogy{},
such applicativity will be preserved, following similar semantics to that of
OCaml.

For truly dynamic ``modules'', using records and first class procedures,
functions and rules may be a suitable alternative with similar syntax.

Of course, no code sharing solution is complete without allowing code
to be split into files and shared between projects. A \kw{module} declaration
in \Trilogy{} may be found in another file.

Much like with Deno's modules, external modules may either be found locally
given a relative file path, or via the Internet given an absolute URL. For
now, these are the only options, but further loading strategies may come
later.

\inputcode[title=\texttt{main.tri}]{external_main.tri}

In this one file we have defined three modules, each in different locations.
The first, \mod{HereModule} is as we have seen previously. The second,
\mod{LocalModule} is written the file \texttt{local\_module.tri} located
beside \texttt{main.tri}. Its contents are just the contents of the module
(no \kw{module} declaration is required).

\inputcode[title=\texttt{local\_module.tri}]{external_local.tri}

The third module was written externally and hosted on the Internet.
Its contents are similarly just the contents of the module, without
any declaration.

\inputcode[title=\texttt{https://cameldridge.com/module.tri}]{external_url.tri}

All three of these modules define and export a single procedure, which
are each made available to \fn{main!}.

\subsubsection{Documentation}

Just as important as writing code is being able to write high quality
documentation that other developers (including your own future self)
are able to find and read in a predictable location and format.

Comments in \Trilogy{} come in the typical two forms of line comments
(\texttt{\#}) and block comments (\texttt{\#-} until \texttt{-\#}).

In addition there are external doc comments (\texttt{\#\#}) and internal
doc comments (\texttt{\#!}), as found in Rust.

\inputcode{documentation.tri}

Doc comments will be transformed into external documentation by a built
in documentation command, similar to Rust's \texttt{cargo doc} command.
Such exported documentation can be made available by whatever means the
author desires, but more importantly, can exported locally alongside the
documentation of all other referenced code, so it can be found in a
predictable location alongside any project that uses it.

\subsubsection{Testing}

Tests are defined similarly to procedures, but using the keyword \kw{test}
instead of \kw{proc}. A test is run as if it was the entrypoint to the
application. If execution reaches the end successfully, the test is
considered a success; otherwise it is a failure.

\inputcode{tests.tri}

To make tests easier to write, the \kw{assert} keyword will abort the
current process when its parameter does not evaluate to \val{true}.
