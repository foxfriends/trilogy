\section{Introduction}

As previously alluded to, \Trilogy{} may be affectionally  described
as three languages in a trenchcoat. These languages, each very distinct in
their syntax and semantics, work together to make up \Trilogy{}:

\begin{description}[labelindent=1.5em]
    \item[Prose] a straight forward imperative language;
    \item[Poetry] a more elegant, purely functional language; and
    \item[Law] a fact-and-rule based logical language.
\end{description}

\noindent
Of course, seeing as this trilogy is yet to be written, those languages do not
exist at this time. The following chapters will explore these languages to
gradually build up a formal definition of each and how they weave together
into a cohesive whole.

\subsection{Motivation}

My motivation behind \Trilogy{} is threefold (obviously). The further I go
into this language, the more things I find that come in threes.

\subsubsection{Expressivity}

As someone who has tried many languages, for little reason other than
to try them out, I find that nearly every language has something that
makes it interesting and unique. Conversely, every language is missing
the interesting and unique feature of every other languages, and so
is disappointing in some way.

In an attempt to cover as many such features as possible, it's pretty common these
days to see languages incorporating features of multiple paradigms at once. I might even
go so far as to say it would be \emph{wrong} to commit a new language to
any single paradigm, as developers are so accustomed to having at least a few
options at their disposal to solve any particular problem.

Many inherently imperative languages, think Javascript, Python, or Rust,
incorporate ideas from functional programming, allowing for the concise and
composable code that functional languages are known for when manipulating simple
data structures, often lists or arrays. These languages typically don't embrace
all of the power of functional programming though, and instead choose to
maintain a consistent and familiar imperative ideal, giving up the full
conveniences afforded by features like currying and partial application.
Such languages also often include aspects of object-oriented programming
with classes (or similar) as a form of encapsulation and abstraction
without going so far as to make \emph{every single thing} a class (like Java).

Meanwhile there are functional languages, taking Scala as an extreme, which
incorporate typically imperative constructs in order to make traditionally
imperative operations, such as I/O, easier. In other functional languages, common
imperative tasks, like updating an element of a list (without mutation), or
reading all lines of input into an array (without loops), become somewhat
convoluted constructs of recursion for which it becomes hard to reason about
in what order side effects will be performed.

In other cases, a second paradigm is tacked on as an afterthought to catch up with
other more ``modern'' languages. \CC{} and Java are like this, with the addition of
lambdas in later versions of each. Sadly, in both of those cases, using lambdas
feels clumsy; they don't quite fit in with the rest of the language or
pre-existing standard library. The lambdas are forced to squeeze into the
language as a mostly runtime construct via strange classes with special cased
compiler level syntax sugar.

Similarly, Haskell, which requires everything to be purely functional, encapsulates
concepts taken for granted in imperative languages through monads, which (as much
as I enjoy the mental gymnastics they involve) I have only found to lead to
unnecessarily complicated abstractions and hard to read documentation written
by scientists who have forgone human communication in favour of mathematical supremacy.

What I have trouble finding are languages which fully embrace the strengths of
every paradigm involved. Neither can I find languages which incorporate logic programming
as one of its core ideas. \Trilogy{} aims to solve this by incorporating
three paradigms (imperative, functional, and logic programming) at once, while
maintaining the full expressivity of each.

With three distinct sub-languages (and no fear of adding further branches if
ever there comes such a reason to), \Trilogy{} is able to take on the latest
and greatest features of every paradigm it chooses to embrace. Each can play
to its strengths, while gracefully handing off to the other branches when
its usefulness wears off.

Though it may be a bit tricky to have to switch between three different
syntaxes, it should be remembered that notation is simply a tool by which
we are able to codify our thoughts, and what is really important is the
structure of those thoughts. With three notations at your disposal, it is
much easier to express those thoughts exactly as you think them, rather than
having to force them into the restrictions of just one.

\subsubsection{Experimentation}

By being a truly multi-paradigm language, there is no choice but to accept the
fact that something of this whole idea must be an experiment. \Trilogy{} is a
mixture of features not often all seen in a single language. Indeed it is all
the features I wish I had a language in which I could experiment with them,
but since they simply didn't exist, I could not.

From the research I've read, and the thought I've put into this language, what
I have found is that the differences between languages that already exist are
typically based on value, binding, and control flow. By providing different
combinations of each, we arrive at different paradigms: highly value based
languages with minimal control flow operators end up looking functional; highly
binding focused languages tend to turn out more logical but value manipulation
becomes hard; traditional imperative languages have all sorts of control flow
constructs, but are relatively elementary when it comes to manipulating values
and bindings.

\Trilogy{} takes those three pillars and raises the bar on each. To do so must
take advantage of the more advanced control flow options provided by continuations,
expressed directly to the language through the effect system. Both from an
implementation and language design standpoint, \Trilogy{} is an experiment.
Additionally, \Trilogy{} is an environment in which we can experiment with
all these properties.

\subsubsection{Practicality}

Despite supporting features and a design that are fundamentally experimental,
\Trilogy{} tries to remain practical at all times. Bringing along all of the
common tooling, features, and developer experience we have come to expect of
modern languages, \Trilogy{} should be easy and enjoyable to work with.

Additionally, performance must be passable; a language can't be considered
practical if it takes too long to run a program. A very conservative minimum
requirement is to be at least as fast as Python.

Though it's likely that not every feature of \Trilogy{} will be used to its
fullest, nor are some even a good idea to use to begin with, their existence
will not get in the way of the usefulness of the language. Rather the opposite,
there will be no situation where \Trilogy{} \emph{can't} do something that you
need it to do; what could be more practical than that.

\subsection{Prose}

\Prose{} is the imperative language of \Trilogy{}, named for the
straightforward nature of reading prose, which is the same way
well written imperative code should be read.

\Prose{} contains all of the standard fixings of an imperative language:
\begin{itemize}
    \item Sequenced operations
    \item Immutable constants (\kw{let}) and mutable variables (\kw{let mut})
    \item Branching in the form of \kw{if}, \kw{else if}, and \kw{else}
    \item The (iterative) \kw{for} loop, \kw{while} loop, and infinite \kw{loop}
    \item Callable \kw{proc}edures
    \item \kw{match} as a supercharged switch
    \item Exception handling (\texttt{try}/\texttt{catch}) via the effect system (\kw{yield}/\kw{when})
    \item Traditionally indexable arrays
\end{itemize}

\Prose{} is meant to handle the simple and mundane tasks of your
average program; it acts as the entrypoint of every program,
and is expected to handle all of the I/O operations and general
sequencing of the flow of the program.

It then frictionlessly gets out of the way by allowing the other languages
to interject when necessary. In a sense, everything possible in any of the
three languages can be spliced almost seamlessly into \Prose{}, as if it
were all the same thing.

\subsection{Poetry}

\Poetry{} is the functional language of \Trilogy{}, named for
the concise, cerebral nature of reading poetry, which I think
holds a literary parallel to the mathematical poetry that is
functional programming.

\Poetry{} of course contains the features one might expect from a purely
functional language:
\begin{itemize}
    \item Expression-first; everything has a value
    \item First class \kw{func}tions
    \item Lambda functions (\kw{fn}), closures
    \item Automatic currying and partial application
    \item Function composition (\op{>>}) and pipelining (\op{|>}) operators
    \item Pattern matching, even in function heads
    \item Monadic encapsulation of imperative concepts via the effect system
\end{itemize}

Once the flow of the program transitions to \Poetry{}, there is no going
back; \Poetry{} remains purely functional to the core. From \Poetry{}, it
is impossible to call code written in \Prose{}, only by \kw{yield}ing effects
is \Prose{} able to react to \Poetry{} and respond with values, maintaining
functional purity while allowing interaction with the outside world.

\subsection{Law}

\Law{} is the logical language of \Trilogy{}, named for the way that
the law defines rules, and the rule based nature in which logical
programs are constructed.

\Law{} contains those features typical of logic programming languages
(if ``typical'' can be used to describe so relatively few):
\begin{itemize}
    \item Facts and \kw{rule}s
    \item Non-deterministic branching behaviour
    \item Pattern matching to the extreme
    \item Bi-directional parameters
\end{itemize}

Unlike other logical languages in existence, \Law{} is able to remain more
tightly tied to the pure-logic nature, and leave the non-logical tasks of
computation and side effects to languages better suited to them.

Where \Law{} really shines is in bringing powerful pattern matching
abilities to the other languages, allowing you to implement expressions
which are capable of not just consuming iterators for loops and
comprehensions, but also generating them.

More experimentally, \Law{} is also able to take advantage of the
effect system to fall back on the other languages and access the
results of computation or side effects. While I suspect this does
break out of the purely logical nature of the language, maybe a
more cohesive model for effects will come up later on.
