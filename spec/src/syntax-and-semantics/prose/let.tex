\subsubsection{Let}

The \kw{let} statement in \Prose{} exposes a ``binding context'' in which
bindings may be introduced, similar to that of \Law{}. Depending on how many times
the binding pattern matches, this may introduce a branch or cause a fizzle.
Most often, however, such bindings are done via a direct single unification
which is determinisitic, and so control flow will progress intuitively.

\begin{bnf*}
    \bnfprod{SLet}{
        \bnfts{\kw{let}}
        \bnfsp
        \bnfpn{Query}
    } \\
\end{bnf*}

Semantically, a \kw{let} statement introduces an implicit scope from its location
in the source code until the end of the nearest explicit scope (e.g. the end of a block).
The bindings declared in this statement are available within that scope.

What is more unique to \Trilogy{} is the branching and fizzling capability of the
\kw{let} statement when provided a query via a more complex rule of \Law{}.
Depending on how many possible bindings there are for the provided rule, a \kw{let}
statement may lead to multiple executions of the program running in parallel, or none.

\begin{prooftree}
    \AxiomC{$\Gamma \vdash Q \Rightarrow \Phi$}
    \LeftLabel{Branch}
    \UnaryInfC{$\Gamma \vdash \kw{let}\ Q \Rightarrow \Phi$}
\end{prooftree}
