\subsubsection{Assert}

The last statement, of very low importance, is the \kw{assert} statement.
In a real program the \kw{assert} statement is not likely to be used, as
it is truly only a convenience when writing tests, further discussed in
\S\ref{sec:tests}. Technically you could use an \kw{assert} statement
in a real program, but its dramatic behaviour of triggering an assertion error
and immediately ending the program in failure makes that typically undesirable.

\begin{bnf*}
    \bnfprod{SAssert}{
        \bnfts{\kw{assert}}
        \bnfsp
        \bnfpn{Expression}
        \bnfor
    } \\
    \bnfmore{
        \bnfts{\kw{assert}}
        \bnfsp
        \bnfpn{Expression}
        \bnfsp
        \bnfts{\kw{as}}
        \bnfsp
        \bnfpn{Expression}
    }
\end{bnf*}

The \kw{assert} statement is optionally provided an expression (before \kw{as})
which is only evaluated in case of failure to be included in an error message.

The other (or only) expression is the condition which is used to determine
whether to end the program. If it evaluates to \kw{true}, the assertion was
a success and execution continues.

When \kw{false}, the message expression (if any) is evaluated. The message
expression may not call any continuations, and may not fail for any reason.
If either of these requirements is violated, or the evaluation of the
message expression takes longer than one second, it is abandoned. Any further
runtime errors encountered during this expression are ignored. The program
then ends with an assertion error.

\begin{prooftree}
    \AxiomC{$\Gamma\vdash C = \kw{true}$}
    \LeftLabel{Assertion}
    \UnaryInfC{$\Gamma\vdash \kw{assert}\ M\ \kw{as}\ C : \top$}
\end{prooftree}

\begin{prooftree}
    \AxiomC{$\Gamma\vdash C = \kw{false}$}
    \LeftLabel{Assertion}
    \UnaryInfC{$\Gamma\vdash \kw{assert}\ M\ \kw{as}\ C : \bot$}
\end{prooftree}
