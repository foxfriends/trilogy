\subsubsection{Sequences}

Sequences are statements which are to be executed in order, in typical
imperative fashion. Statements in sequence must be separated by a line break or
an explicit separator (\op{;}).

\begin{bnf*}
    \bnfprod{Sep}{\bnfpn{EOL}\bnfor\bnfts{;}} \\
    \bnfprod{Sequence}{\bnfpn{Statement}\bnfsp\bnfpn{Sep}\bnfsp\bnfpn{Sequence}\bnfor} \\
    \bnfmore{\bnfpn{Statement}\bnfor\bnfes}
\end{bnf*}

A sequence of statements executes each statement in order, performing some effect on the context in which
it is being run. The value of the sequence is the value of the final statement in that sequence. Note, however,
that only expression statements have values; if the final statement of a sequence is a non-expression statement,
then the value of the whole sequence is just \kw{unit}.

\begin{prooftree}
    \AxiomC{$\Gamma \vdash M \Rightarrow \Phi$}
    \AxiomC{$\Phi \vdash N: \tau \Rightarrow \Sigma$}
    \LeftLabel{Sequence}
    \BinaryInfC{$\Gamma \vdash M \op{;}\ N: \tau \Rightarrow \Sigma$}
\end{prooftree}

\begin{prooftree}
    \AxiomC{$\Gamma \vdash M \Rightarrow \bot$}
    \LeftLabel{Sequence\textsuperscript{$\bot$}}
    \UnaryInfC{$\Gamma \vdash M \op{;}\ N: \bot$}
\end{prooftree}

Other than directly as a procedure body, sequences can be placed in any block.
Blocks may be used in any situation expressions are allowed, and act as a
scope barrier. Bindings declared within a block go out of scope at the end
of the block.

\begin{bnf*}
    \bnfprod{Block}{
        \bnfts{\{}
        \bnfsp
        \bnfpn{Sequence}
        \bnfsp
        \bnfts{\}}
    }
\end{bnf*}

\begin{prooftree}
    \AxiomC{$\Gamma :> \Phi$}
    \AxiomC{$\Phi \vdash M : \tau$}
    \LeftLabel{Unnest}
    \BinaryInfC{$\Gamma \vdash \block{M} : \tau$}
\end{prooftree}
