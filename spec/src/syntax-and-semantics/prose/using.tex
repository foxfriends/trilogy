\subsubsection{Using}

The \kw{using} statement, though not adding anything you couldn't do already,
provides a convenient syntax for avoiding ``callback hell''. We steal this
basically straight out of Gleam.

\begin{bnf*}
    \bnfprod{Using}{
        \bnfts{\kw{do}}
        \bnfsp
        \bnfts{(}
        \bnfsp
        \bnfpn{PatternList}
        \bnfsp
        \bnfts{)}
        \bnfsp
        \bnfts{\kw{using}}
        \bnfsp
        \bnfpn{Expression}
        \bnfor
    } \\
    \bnfmore{
        \bnfts{\kw{do}}
        \bnfsp
        \bnfts{\kw{using}}
        \bnfsp
        \bnfpn{Expression}
        \bnfor
    } \\
    \bnfmore{
        \bnfts{\kw{using}}
        \bnfsp
        \bnfpn{Expression}
    } \\
\end{bnf*}

As this is purely syntax sugar, its behaviour can be defined as a syntax transformation,
including the remainder of the lexical scope. In particular, all statements in
the current scope following the \kw{using} statement are encapsulated in a callback,
and passed as the final parameter to the function expression on the right side of the
\kw{using} keyword.

\begin{align*}
    \texttt{\kw{using} \$e} &\Rightarrow \texttt{\kw{do}() \kw{using} \$e} \\ %
    \texttt{\kw{do} \kw{using} \$e} &\Rightarrow \texttt{\kw{do}() \kw{using} \$e} \\ %
    \texttt{\kw{do}(\$p) \kw{using} \$e; \$(\$t)*} &\Rightarrow \texttt{\$e (\kw{do}(\$p) \{ \$(\$t)* \})} \\ %
\end{align*}
