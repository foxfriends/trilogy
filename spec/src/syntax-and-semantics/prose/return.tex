\subsubsection{Return}

Similar to \kw{break} and \kw{continue}, the \kw{return} statement causes execution
to end and be picked up elsewhere. The \kw{return} statement may designate the value
that a procedure call evaluates to. This should be intuitive to any who have written
imperative code before.

A \kw{return} statement with no value is assumed to return \kw{unit}, the same as if
the \kw{return} statement was omitted entirely and execution ran off the end of the
procedure.

Also similar to \kw{break} and \kw{continue}, the \kw{return} keyword is a first-class
keyword, allowing it to be assigned to a variable within a procedure and later passed
around to other parts of the code.

\begin{bnf*}
    \bnfprod{SReturn}{
        \bnfts{\kw{return}}
        \bnfor
        \bnfts{\kw{return}}
        \bnfsp
        \bnfpn{Expression}
    }
\end{bnf*}

Semantically, the \kw{return} statement on its own is simple. Meanwhile, its meaning
to a procedure was already covered as part of the semantics of procedure definitions
(Proof~\ref{proof:return}).

\begin{prooftree}
    \AxiomC{$\Gamma\vdash N:\top$}
    \LeftLabel{Return}
    \UnaryInfC{$\Gamma\vdash\kw{return}\ N : \bot$}
\end{prooftree}
