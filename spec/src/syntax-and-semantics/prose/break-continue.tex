\subsubsection{Break and Continue}

The statements \kw{break} and \kw{continue} may be used in loops to control iteration.
The \kw{break} statement will exit the loop immediately, skipping all further iteration,
whereas the \kw{continue} statement will end the current iteration and move on to the
next, without executing any further code. This applies for both \kw{for} and \kw{while}
loops.

Either of these statements given a label will act on that labelled loop.
Without a label, the closest enclosing loop is assumed. If the label corresponds to a
statement that is not a loop (or block for a \kw{break}), the syntax is to be considered
invalid.

The \kw{break} statement may be used in labelled blocks as well, breaking out of
those blocks immediately. Unlabelled blocks cannot be used with the \kw{break}
statement in order to give priority to the loop usage, which is likely to be
more frequent. The \kw{continue} statement may not be used on regular blocks,
only on loops.

\begin{bnf*}
    \bnfprod{SBreak}{
        \bnfts{\kw{break}}
        \bnfor
        \bnfts{\kw{break}}
        \bnfsp
        \bnfts{Identifier}
    } \\
    \bnfprod{SContinue}{
        \bnfts{\kw{continue}}
        \bnfor
        \bnfts{\kw{continue}}
        \bnfsp
        \bnfts{Identifier}
    }
\end{bnf*}

Labels are a concept only at the syntax level, so are not considered bindings
that may be used at runtime, and may not be passed as parameters. When a \kw{break}
or \kw{continue} statement uses a label, they may only refer to labels on blocks
or loops syntactically surrounding the statement using the label. In the semantics
below, it is assumed that the block or loop that the \kw{break} or \kw{continue}
is meant to correspond to has already been correctly resolved, and so the labels are
omitted.

A \kw{break} or \kw{continue} statement on its own does not evaluate to anything.

\begin{figure}[H]
    \centering
    \parbox[t]{0.45\linewidth}{
        \begin{prooftree}
            \AxiomC{}
            \LeftLabel{Break}
            \UnaryInfC{$\Gamma \vdash \kw{break} : \bot$}
        \end{prooftree}
    }
    \parbox[t]{0.45\linewidth}{
        \begin{prooftree}
            \AxiomC{}
            \LeftLabel{Continue}
            \UnaryInfC{$\Gamma \vdash \kw{continue} : \bot$}
        \end{prooftree}
    }
\end{figure}

Though the statement ends the local execution, it changes the behaviour of the loop
or block it corresponds to, which picks up the execution before it fizzles completely.

\begin{prooftree}
    \AxiomC{$\Gamma\vdash \texttt{\{}\ M\ \texttt{\}} : \top$}
    \LeftLabel{Break\textsuperscript{Block}}
    \UnaryInfC{$\Gamma \vdash \texttt{\{}\ M\ \op{,}\ \kw{break}\ \texttt{\}} : \top$}
\end{prooftree}

\begin{prooftree}
    \AxiomC{$\Gamma \vdash \kw{for}\ Q\ \texttt{\{}\ M\ \texttt{\}} : \top$}
    \LeftLabel{Break\textsuperscript{for}}
    \UnaryInfC{$\Gamma \vdash \kw{for}\ Q\ \texttt{\{}\ M\ \op{,}\ \kw{break}\ \texttt{\}} : \top$}
\end{prooftree}

\begin{prooftree}
    \AxiomC{$\Gamma \vdash \kw{while}\ E\ \texttt{\{}\ M\ \texttt{\}} : \top$}
    \LeftLabel{Break\textsuperscript{while}}
    \UnaryInfC{$\Gamma \vdash \kw{while}\ E\ \texttt{\{}\ M\ \op{,}\ \kw{break}\ \texttt{\}} : \top$}
\end{prooftree}

\begin{prooftree}
    \AxiomC{$\Gamma \vdash \texttt{\{}\ M\ \texttt{\}} : \top$}
    \LeftLabel{Continue}
    \UnaryInfC{$\Gamma \vdash \texttt{\{}\ M\ \texttt{,}\ \kw{continue}\ \texttt{\}} : \top$}
\end{prooftree}
