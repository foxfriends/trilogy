\subsubsection{Break and Continue}

The statements \kw{break} and \kw{continue} may be used in loops to control iteration.
The \kw{break} statement will exit the loop immediately, skipping all further iteration,
whereas the \kw{continue} statement will end the current iteration and move on to the
next, without executing any further code. This applies for both \kw{for} and \kw{while}
loops.

Unlike in other languages, \Trilogy{} does not have labelled statements. Instead, to break
out of the outer loop from within a nested loop, \Trilogy{} supports \emph{first-class keywords}.
The \kw{break} or \kw{continue} keywords, when not used in statement form, are treated as
nullary procedures, meaning they can be assigned to variables and later called as any
other procedure. These keywords are tied syntactically to nearest enclosing loop, so when
invoked from an inner loop, the captured \kw{break} or \kw{continue}'s original loop is
the one that is affected.

Of course, grammatically that case is covered later, as part of the expression grammar.
Here are only listed the syntax of the \kw{break} and \kw{continue} statements, which are
trivial.

\begin{bnf*}
    \bnfprod{SBreak}{\bnfts{\kw{break}}} \\
    \bnfprod{SContinue}{\bnfts{\kw{continue}}}
\end{bnf*}

A \kw{break} or \kw{continue} statement on its own does not evaluate to anything, so
their semantics are uninteresting. When captured in a variable or closure, the semantics
are unchanged, only the syntax is adjusted to disambiguate.

\begin{figure}[H]
    \centering
    \parbox[t]{0.45\linewidth}{
        \begin{prooftree}
            \AxiomC{}
            \LeftLabel{Break}
            \UnaryInfC{$\Gamma \vdash \kw{break} : \bot$}
        \end{prooftree}
    }
    \parbox[t]{0.45\linewidth}{
        \begin{prooftree}
            \AxiomC{}
            \LeftLabel{Continue}
            \UnaryInfC{$\Gamma \vdash \kw{continue} : \bot$}
        \end{prooftree}
    }
\end{figure}

Though the statement ends the local execution, it changes the behaviour of the loop
or block it corresponds to, which picks up the execution before it fizzles completely.

The behaviour of binding the keyword to a loop is only of syntactic concern, so in the
semantics we assume that the break or continue in question has already been resolved to
the correct containing loop.

\begin{prooftree}
    \AxiomC{$\Gamma \vdash \kw{for}\ Q\ \texttt{\{}\ M\ \texttt{\}} : \top$}
    \LeftLabel{Break\textsuperscript{for}}
    \UnaryInfC{$\Gamma \vdash \kw{for}\ Q\ \texttt{\{}\ M\ \op{,}\ \kw{break}\ \texttt{\}} : \top$}
\end{prooftree}

\begin{prooftree}
    \AxiomC{$\Gamma \vdash \kw{while}\ E\ \texttt{\{}\ M\ \texttt{\}} : \top$}
    \LeftLabel{Break\textsuperscript{while}}
    \UnaryInfC{$\Gamma \vdash \kw{while}\ E\ \texttt{\{}\ M\ \op{,}\ \kw{break}\ \texttt{\}} : \top$}
\end{prooftree}

\begin{prooftree}
    \AxiomC{$\Gamma \vdash \texttt{\{}\ M\ \texttt{\}} : \top$}
    \LeftLabel{Continue}
    \UnaryInfC{$\Gamma \vdash \texttt{\{}\ M\ \texttt{,}\ \kw{continue}\ \texttt{\}} : \top$}
\end{prooftree}

It is worth noting, however, that if either the \kw{break} or \kw{continue} keyword escapes the
loop it was bound to and is invoked outside of the loop, the program is to be considered invalid.
Syntactically this may be hard to enforce, so instead it is enforced as a runtime invariant.

Rather than fizzling, the expected behaviour in this situation is the immediate end of the
program as an ``execution failure''. This fact is not included in any proof of semantics,
and is instead defined only as a safeguard against undefined behaviour until more nuanced
options may be explored at a later time.
