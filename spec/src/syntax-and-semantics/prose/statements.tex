\subsubsection{Statements}

A statement is a single ``step'' of a procedure in \Prose{}. Statements do not
evaluate to any value, but they perform effects on the state of the program and
its environment.

To differentiate with similar constructs in \Poetry{}, while also saving space,
the names of the statement productions are prefixed with S, short for statement.
Meanwhile, expressions are actually all \Poetry{}, so such specification is left
until later.

\begin{bnf*}
    \bnfprod{Statement}{
        \bnfpn{SLet}
        \bnfor
        \bnfpn{SAssignment}
        \bnfor
        \bnfpn{SIf}
        \bnfor
        \bnfpn{SMatch}
        \bnfor
    } \\
    \bnfmore{
        \bnfpn{SWhile}
        \bnfor
        \bnfpn{SFor}
        \bnfor
        \bnfpn{SBreak}
        \bnfor
        \bnfpn{SContinue}
        \bnfor
    } \\
    \bnfmore{
        \bnfpn{SReturn}
        \bnfor
        \bnfpn{SEnd}
        \bnfor
        \bnfpn{SExit}
        \bnfor
        \bnfpn{SYield}
        \bnfor
    } \\
    \bnfmore{
        \bnfpn{SResume}
        \bnfor
        \bnfpn{SCancel}
        \bnfor
        \bnfpn{SDefer}
        \bnfor
    } \\
    \bnfmore{
        \bnfpn{SAssert}
        \bnfor
        \bnfpn{Block}
        \bnfor
        \bnfpn{Expression}
    } \\
    \bnfprod{Block}{
        \bnfts{\{}
        \bnfsp
        \bnfpn{Sequence}
        \bnfsp
        \bnfts{\}}
    }
\end{bnf*}

While expressions may be written in statement position, do note that a statement
beginning with a keyword that may also start a statement is always parsed as a
statement. For example, a \kw{match} written in \Prose{} is always a \kw{match}
statement, and never a \kw{match} expression. The purpose of the expression
statement is to permit procedure calls and other such side-effecting expressions.
Used in this way, an expression's value is lost. Additionally, since an expression
cannot affect scope outside of itself, no changes to the evaluation context are
possible.

\begin{prooftree}
    \AxiomC{$\Gamma\vdash E : \tau$}
    \LeftLabel{Expression}
    \UnaryInfC{$\Gamma\vdash E : \top$}
\end{prooftree}

Blocks may be used in any situation statements are allowed, and act as
a scope barrier. Bindings declared within a block go out of scope at
the end of the block.

\begin{prooftree}
    \AxiomC{$\Gamma :> \Phi$}
    \AxiomC{$\Phi \vdash M : \top$}
    \LeftLabel{Unnest}
    \BinaryInfC{$\Gamma \vdash \block{M} : \top$}
\end{prooftree}
