\subsubsection{Statements}

A statement is a single ``step'' of a procedure in \Prose{}. Statements do not
evaluate to any value, but they perform effects on the state of the program and
its environment.

To differentiate with similar constructs in \Poetry{}, while also saving space,
the names of the statement productions are prefixed with S, short for statement.
Meanwhile, expressions are actually all \Poetry{}, so such specification is left
until later.

\begin{bnf*}
    \bnfprod{Statement}{
        \bnfpn{SLet}
        \bnfor
        \bnfpn{SAssignment}
        \bnfor
        \bnfpn{SIf}
        \bnfor
        \bnfpn{SMatch}
        \bnfor
    } \\
    \bnfmore{
        \bnfpn{SWhile}
        \bnfor
        \bnfpn{SFor}
        \bnfor
        \bnfpn{SBreak}
        \bnfor
        \bnfpn{SContinue}
        \bnfor
    } \\
    \bnfmore{
        \bnfpn{SReturn}
        \bnfor
        \bnfpn{SEnd}
        \bnfor
        \bnfpn{SYield}
        \bnfor
        \bnfpn{SResume}
        \bnfor
    } \\
    \bnfmore{
        \bnfpn{SCancel}
        \bnfor
        \bnfpn{SProcedureCall}
        \bnfor
        \bnfts{(}
        \bnfsp
        \bnfpn{Expression}
        \bnfsp
        \bnfts{)}
        \bnfor
    } \\
    \bnfmore{
        \bnfpn{Block}
    } \\
    \bnfprod{Block}{
        \bnfts{\{}
        \bnfsp
        \bnfpn{Sequence}
        \bnfsp
        \bnfts{\}}
        \bnfor
        \bnfts{\{}
        \bnfsp
        \bnfpn{Sequence}
        \bnfsp
        \bnfts{\}}
        \bnfsp
        \bnfpn{EffectHandlers}
    }
\end{bnf*}

Evaluating an expressions as a statement is allowed, but is typically
recommended against. In particular cases, however, it may be necessary,
so wrapping the expression in parentheses will allow this. Bare expressions
are not permitted, except for procedure calls which are allowed by special
case of the statement syntax.

\begin{prooftree}
    \AxiomC{$\Gamma \vdash M : \top$}
    \LeftLabel{Unwrap}
    \UnaryInfC{$\Gamma \vdash \texttt{(}\ M\ \texttt{)} : \top$}
\end{prooftree}

Blocks may be used in any situation statements are allowed, and act as
a scope barrier. Bindings declared within a block go out of scope at
the end of the block.

\begin{prooftree}
    \AxiomC{$\Gamma :> \Phi$}
    \AxiomC{$\Phi \vdash M : \top$}
    \LeftLabel{Unnest}
    \BinaryInfC{$\Gamma \vdash \block{M} : \top$}
\end{prooftree}

In addition to being a scope barrier, blocks serve as anchor points onto
which effect handlers may be attached. The semantics of effect handlers
will be addressed in \S\ref{sec:effects}.
