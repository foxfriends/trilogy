\subsubsection{Match}
\label{sec:prose-match}

The \kw{match} statement allows for pattern matching, allowing code to branch
on the structure of a value, rather than a Boolean evaluation of an expression.
As \Trilogy{} is a dynamically typed language, it is hard to ensure completeness
of a \kw{match} statement, so completeness is not enforced; a mismatch is skipped
in much the same way a \kw{false} evaluation of an \kw{if} statement is skipped.

The \kw{match} statement takes an expression and is followed by multiple \kw{case}
statements. Each \kw{case} statement has a pattern, a guard, or both, and then a block
of code to run if that case is selected. An \kw{else} clause may be included as the
last case, to be run in the situation that no \kw{case} was selected.

\begin{bnf*}
    \bnfprod{Match}{
        \bnfts{\kw{match}}
        \bnfsp
        \bnfpn{Expression}
        \bnfsp
        \bnfts{\{}
        \bnfsp
        \bnfpn{Cases}
        \bnfsp
        \bnfts{\}}
    } \\
    \bnfprod{Cases}{
        \bnfpn{Case}
        \bnfsp
        \bnfpn{Cases}
        \bnfor
        \bnfpn{Case}
        \bnfor
        \bnfpn{ElseCase}
    } \\
    \bnfprod{Case}{
        \bnfts{\kw{case}}
        \bnfsp
        \bnfpn{Pattern}
        \bnfsp
        \bnfpn{CaseGuard}
        \bnfsp
        \bnfpn{CaseBody}
    } \\
    \bnfprod{CaseGuard}{
        \bnfts{\kw{if}}
        \bnfsp
        \bnfpn{Expression}
        \bnfor
        \bnfes
    } \\
    \bnfprod{CaseBody}{
        \bnfts{\kw{then}}
        \bnfsp
        \bnfpn{Expression}
        \bnfor
        \bnfpn{Block}
    } \\
    \bnfprod{ElseCase}{
        \bnfts{\kw{else}}
        \bnfsp
        \bnfpn{Binding}
        \bnfsp
        \bnfpn{CaseBody}
        \bnfor
    } \\
    \bnfmore{
        \bnfts{\kw{else}}
        \bnfsp
        \bnfpn{CaseBody}
    }
\end{bnf*}

To evaluate a \kw{match} statement, the original expression is evaluated, and compared
against the pattern of each \kw{case}; an omitted pattern is interpreted like an underscore
(\texttt{\_}) pattern, matching with any value.

Before committing to a \kw{case} with a matching pattern the guard, if any, is evaluated.
Identifiers bound in the pattern are in scope for this evaluation. If the guard expression
evaluates to \kw{true} (or there are no guards), the case is selected. If the guard expression
evaluates to \kw{false}, it is skipped. If the guard expression evalutes to a non-Boolean result,
it is a runtime type error. An omitted guard is treated to be always \kw{true}.

The block following the first selected case is then run. If no cases are selected,
they are simply ignored and the program continues.

\begin{prooftree}
    \AxiomC{$\Gamma \vdash E$}
    \AxiomC{$P = E$}
    \AxiomC{$\Gamma,\setof{a\given a \defby P} \vdash C = \kw{true}$}
    \LeftLabel{Match}
    \TrinaryInfC{$\Gamma \vdash \kw{match}\ E\ \kw{case}\ P\ \kw{if}\ C\ B \Rightarrow \Phi$}
\end{prooftree}

\begin{prooftree}
    \def\extraVskip{3.5pt}
    \AxiomC{$\Gamma \vdash E$}
    \AxiomC{$P = E$}
    \AxiomC{$\Gamma,\setof{a\given a \defby P} \vdash C = \kw{false}$}
    \TrinaryInfC{$R_1 = \kw{case}\ P\ \kw{if}\ C\ B$}
    \AxiomC{$\Gamma \vdash \kw{match}\ E\ R_2 \cdots R_n \Rightarrow \Phi$}
    \LeftLabel{Match\textsuperscript{\kw{false}}}
    \insertBetweenHyps{\hskip -17pt}
    \BinaryInfC{$\Gamma \vdash \kw{match}\ E\ R_1 R_2 \cdots R_n \Rightarrow \Phi$}
\end{prooftree}

\begin{prooftree}
    \def\extraVskip{3.5pt}
    \AxiomC{$\Gamma \vdash E$}
    \AxiomC{$P \neq E$}
    \BinaryInfC{$R_1 = \kw{case}\ P\ \kw{if}\ C\ B$}
    \AxiomC{$\Gamma \vdash \kw{match}\ E\ R_2 \cdots R_n \Rightarrow \Phi$}
    \LeftLabel{Match\textsuperscript{$\bot$}}
    \BinaryInfC{$\Gamma \vdash \kw{match}\ E\ R_1 R_2 \cdots R_n \Rightarrow \Phi$}
\end{prooftree}

The omitted clauses of the \kw{case} statement, and the behaviour of the \kw{else} case,
are formalized by syntax transformation:

\begin{align*}
    \texttt{\kw{case} \$p:pat} &\Rightarrow \texttt{\kw{case} \$p \kw{if} \kw{true}} \\ %
    \texttt{\kw{case} \kw{if} \$c:expr} &\Rightarrow \texttt{\kw{case} \_ \kw{if} \$c} \\ %
    \texttt{\kw{else} \_ \kw{then}} &\Rightarrow \texttt{\kw{case} \_ \kw{then}} \\ %
    \texttt{\kw{else} \$i:id \kw{then}} &\Rightarrow \texttt{\kw{case} \$i \kw{then}} \\ %
\end{align*}
