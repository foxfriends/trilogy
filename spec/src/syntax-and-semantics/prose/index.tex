\subsection{Prose}

When viewed on its own, \Prose{} has a fairly ``standard'' C-family syntax.

Arbitrary patterns are permitted in the procedure heads, just as in \Law{}'s
rule definitions. This does imply that some procedure calls will fail silently
via fizzling as procedures may not be defined multiple times with different
patterns, in the way that rules or functions may be. It is recommended to only
use such arbitrary patterns in the most certain of cases, and otherwise use
single identifier patterns whenever possible and handle alternative inputs
accordingly.

The ability to define multiple procedures with the same name but different
patterns may be added in future.

\begin{bnf*}
    \bnfprod{ProcedureHead}{
        \bnfts{IdentiferBang}
        \bnfsp
        \bnfts{(}
        \bnfsp
        \bnfpn{PatternList}
        \bnfsp
        \bnfts{)}
    } \\
    \bnfprod{ProcedureBody}{\bnfpn{Sequence}} \\
\end{bnf*}

\noindent A procedure declaration binds the name in the scope of the current module.
As with all kinds of declarations in \Trilogy{}, procedure declarations may be
referenced out of order.

When called, the body of the procedure is run with the formal parameters bound to
the arguments with which the procedure was called. A procedure may use the \kw{return}
keyword to end its evaluation, with its call evaluating to the returned value.
If control runs off the end of the procedure's body, its call evaluates to \kw{unit}.
A procedure may cause a fizzle or branch, in which case all evaluations are propgated
to the call site, conceptually causing the call to fizzle or branch respectively.

\begin{prooftree}
    \AxiomC{$\kw{proc}\ p\texttt{!}(x_{1\hdots n})\ \{\ M,\ \kw{return}\ N\ \}$}
    \AxiomC{$\Gamma,x_{1\hdots n} \vdash M\op{,}\ N $}
    \LeftLabel{Procedure}
    \ProofLabel{proof:return}
    \BinaryInfC{$\Gamma \vdash p\texttt{!(}x_{1\hdots n}\texttt{)}=N$}
\end{prooftree}

\begin{prooftree}
    \AxiomC{$\kw{proc}\ p\texttt{!}(x_{1\hdots n})\ \{\ M\ \}$}
    \AxiomC{$\Gamma,x_{1\hdots n} \vdash M : \top$}
    \LeftLabel{Procedure\textsuperscript{$\top$}}
    \BinaryInfC{$\Gamma \vdash p\texttt{!(}x_{1\hdots n}\texttt{)}=\kw{unit}$}
\end{prooftree}

\begin{prooftree}
    \AxiomC{$\kw{proc}\ p\texttt{!}(x_{1\hdots n})\ \{\ M\ \}$}
    \AxiomC{$\Gamma,x_{1\hdots n} \vdash M : \bot$}
    \LeftLabel{Procedure\textsuperscript{$\bot$}}
    \BinaryInfC{$\Gamma \vdash p\texttt{!(}x_{1\hdots n}\texttt{)}:\bot$}
\end{prooftree}

\subsubsection{Sequences}

Sequences are statements which are to be executed in order, in typical
imperative fashion. Statements in sequence must be separated by a line break or
an explicit separator (\op{,}).

\begin{bnf*}
    \bnfprod{Sep}{\bnfpn{EOL}\bnfor\bnfts{,}} \\
    \bnfprod{Sequence}{\bnfpn{Statement}\bnfsp\bnfpn{Sep}\bnfsp\bnfpn{Sequence}\bnfor} \\
    \bnfmore{\bnfpn{Statement}\bnfor\bnfes}
\end{bnf*}

A sequence of statements has no value, but rather each step of the sequence
performs some effect on the context in which it is being run, eventually
producing the result of the program.

\begin{prooftree}
    \AxiomC{$\Gamma \vdash M \Rightarrow \Phi$}
    \AxiomC{$\Phi \vdash N \Rightarrow \Sigma$}
    \LeftLabel{Sequence}
    \BinaryInfC{$\Gamma \vdash M \op{,}\ N \Rightarrow \Sigma$}
\end{prooftree}

\begin{prooftree}
    \AxiomC{$\Gamma \vdash M \Rightarrow \bot$}
    \LeftLabel{Sequence\textsuperscript{$\bot$}}
    \UnaryInfC{$\Gamma \vdash M \op{,}\ N: \bot$}
\end{prooftree}

\subsubsection{Statements}

A statement is a single ``step'' of a procedure in \Prose{}. Statements do not
evaluate to any value, but they perform effects on the state of the program and
its environment.

To differentiate with similar constructs in \Poetry{}, while also saving space,
the names of the statement productions are prefixed with S, short for statement.
Meanwhile, expressions are actually all \Poetry{}, so such specification is left
until later.

\begin{bnf*}
    \bnfprod{Statement}{
        \bnfpn{SLet}
        \bnfor
        \bnfpn{SAssignment}
        \bnfor
        \bnfpn{SIf}
        \bnfor
        \bnfpn{SMatch}
        \bnfor
    } \\
    \bnfmore{
        \bnfpn{SWhile}
        \bnfor
        \bnfpn{SFor}
        \bnfor
        \bnfpn{SBreak}
        \bnfor
        \bnfpn{SContinue}
        \bnfor
    } \\
    \bnfmore{
        \bnfpn{SReturn}
        \bnfor
        \bnfpn{SEnd}
        \bnfor
        \bnfpn{SYield}
        \bnfor
        \bnfpn{SResume}
        \bnfor
    } \\
    \bnfmore{
        \bnfpn{ProcedureCall}
        \bnfor
        \bnfts{(}
        \bnfsp
        \bnfpn{Expression}
        \bnfsp
        \bnfts{)}
        \bnfor
        \bnfpn{Block}
        \bnfor
    } \\
    \bnfmore{
        \bnfts{\texttt{Identifier}}
        \bnfsp
        \bnfts{\texttt{:}}
        \bnfsp
        \bnfpn{Statement}
    } \\
    \bnfprod{Block}{
        \bnfts{\{}
        \bnfsp
        \bnfpn{Sequence}
        \bnfsp
        \bnfts{\}}
    }
\end{bnf*}

Evaluating an expressions as a statement is allowed, but is typically
recommended against. In particular cases, however, it may be necessary,
so wrapping the expression in parentheses will allow this. Bare expressions
are not permitted, except for procedure calls which are allowed by special
case of the statement syntax.

\begin{prooftree}
    \AxiomC{$\Gamma \vdash M : \tau$}
    \LeftLabel{Unwrap}
    \UnaryInfC{$\Gamma \vdash \texttt{(}\ M\ \texttt{)} : \tau$}
\end{prooftree}

Blocks may be used in any situation statements are allowed, and act as
a scope barrier. Bindings declared within a block go out of scope at
the end of the block.

\begin{prooftree}
    \AxiomC{$\Gamma :> \Phi$}
    \AxiomC{$\Phi \vdash M : \tau$}
    \LeftLabel{Unnest}
    \BinaryInfC{$\Gamma \vdash \texttt{\{}\ M\ \texttt{\}} : \tau$}
\end{prooftree}

Labelled statements evaluate as if they were unlabelled, but
may be referred to by their name in substatements. Though this uses the
same colon (\texttt{:}) as the cons operator, this does not produce any
tuple.

\subsubsection{Let}

The \kw{let} statement in \Prose{} exposes a ``binding context'' in which
bindings may be introduced, similar to that of \Law{}. Depending on how many times
the binding pattern matches, this may introduce a branch or cause a fizzle.
Most often, however, such bindings are done via a direct single unification
which is determinisitic, and so control flow will progress intuitively.

\begin{bnf*}
    \bnfprod{SLet}{
        \bnfts{\kw{let}}
        \bnfsp
        \bnfpn{Unification}
    } \\
\end{bnf*}

Semantically, a \kw{let} statement introduces an implicit scope from its location
in the source code until the end of the nearest explicit scope (e.g. the end of a block).
The bindings declared in this statement are available within that scope.

What is more unique to \Trilogy{} is the branching and fizzling capability of the
\kw{let} statement when provided a unification via a more complex rule of \Law{}.
Depending on how many possible bindings there are for the provided rule, a \kw{let}
statement may lead to multiple executions of the program running in parallel, or none.

\begin{prooftree}
    \AxiomC{$\Gamma \vdash U \Rightarrow \Phi$}
    \LeftLabel{Branch}
    \UnaryInfC{$\Gamma \vdash \kw{let}\ U \Rightarrow \Phi$}
\end{prooftree}

\subsubsection{Assignment}

Variables (specifically, mutable bindings) may be assigned to. Similar to
declaration via \kw{let}, the left-hand side of standard assignment may be
any pattern so long as it does not introduce new bindings. The pattern may only
use previously bound names. The assignment statement updates the values of the
bindings in the pattern.

Function assignment is a special syntax for applying a function to a value and
then simultaneously reassigning that value with the result of the function; a
generalization of operator assignment (e.g. \op{+=}) to any function. In this form
of assignment, the left-hand side of the assignment operator is a single identifier,
and the right side is the arguments to apply to the function being used as the
assignment operator. The left hand side is then applied as the last argument to
the function.

Regular operator assignment (\op{+=}, \op{-=}, etc.) is supported as well, in the
intuitive way: the left hand identifer is used as the left hand side of the operator.

\begin{bnf*}
    \bnfprod{SAssignment}{
        \bnfpn{PatternAssignment}
        \bnfor
        \bnfpn{FunctionAssignment}
        \bnfor
    } \\
    \bnfmore{
        \bnfpn{OperatorAssignment}
    } \\
    \bnfprod{PatternAssignment}{
        \bnfpn{Pattern}
        \bnfsp
        \bnfts{\op{=}}
        \bnfsp
        \bnfpn{Expression}
    } \\
    \bnfprod{FunctionAssignment}{
        \bnfpn{Identifier}
        \bnfsp
        \bnfts{IdentifierEq}
        \bnfsp
        \bnfpn{ApplicationList}
    } \\
    \bnfprod{OperatorAssignment}{
        \bnfpn{Identifier}
        \bnfsp
        \bnfpn{OperatorEq}
        \bnfsp
        \bnfpn{Expression}
    } \\
    \bnfprod{OperatorEq}{
        \bnfts{+=}\bnfor
        \bnfts{-=}\bnfor
        \bnfts{*=}\bnfor
        \bnfts{/=}\bnfor
        \bnfts{\%=}\bnfor
        \bnfts{**=}\bnfor
        \bnfts{//=}\bnfor
        \bnfts{<>=}\bnfor
    } \\
    \bnfmore{
        \bnfts{|=}\bnfor
        \bnfts{\&=}\bnfor
        \bnfts{\textasciicircum=}\bnfor
        \bnfts{\textasciitilde=}\bnfor
        \bnfts{<\textasciitilde=}\bnfor
        \bnfts{\textasciitilde>=}
    }
\end{bnf*}

\begin{prooftree}
    \AxiomC{$\Gamma \vdash y$}
    \AxiomC{$P = y$}
    \AxiomC{$\Gamma, \{a\ |\ a := P\} :> \Phi$}
    \LeftLabel{Assignment}
    \TrinaryInfC{$\Gamma\vdash P\ \op{=}\ y \Rightarrow \Phi$}
\end{prooftree}

Rather than defining the exact semantics of function or operator assignment,
instead understand those by syntax transformation, and use the same assignment
semantics as normal:

\begin{align*}
    \texttt{\$x:id \$fn:id= \$(\$a:expr)+} &\Rightarrow \texttt{\$x = \$f \$(\$a)+ \$x} \\ %
    \texttt{\$x:id \$op:op= \$a:expr} & \Rightarrow \texttt{\$x = \$x \$op \$a} \\ %
\end{align*}

\subsubsection{Conditionals}

The primary conditional statement in \Prose{} is the \kw{if} statement,
which appears as you might expect coming from nearly any other language.

\begin{bnf*}
    \bnfprod{SIf}{
        \bnfts{\kw{if}}
        \bnfsp
        \bnfpn{Expression}
        \bnfsp
        \bnfpn{Block}
        \bnfor
    } \\
    \bnfmore{
        \bnfts{\kw{if}}
        \bnfsp
        \bnfpn{Expression}
        \bnfsp
        \bnfpn{Block}
        \bnfsp
        \bnfts{\kw{else}}
        \bnfsp
        \bnfpn{SIf}
        \bnfor
    } \\
    \bnfmore{
        \bnfts{\kw{if}}
        \bnfsp
        \bnfpn{Expression}
        \bnfsp
        \bnfpn{Block}
        \bnfsp
        \bnfts{\kw{else}}
        \bnfsp
        \bnfpn{Block}
    }
\end{bnf*}

In the common case, the condition is a statement that evaluates to a Boolean
result, \kw{true} or \kw{false}, and control flows as you might expect.
If the result is not a Boolean, the execution fizzles.

\begin{prooftree}
    \AxiomC{$\Gamma \vdash C = \kw{true}$}
    \AxiomC{$\Gamma \vdash T \Rightarrow \Phi$}
    \LeftLabel{If Else\textsuperscript{\kw{true}}}
    \BinaryInfC{$\Gamma \vdash \kw{if}\ C\ T\ \kw{else}\ F \Rightarrow \Phi$}
\end{prooftree}
\begin{prooftree}
    \AxiomC{$\Gamma \vdash C = \kw{false}$}
    \AxiomC{$\Gamma \vdash F \Rightarrow \Phi$}
    \LeftLabel{If Else\textsuperscript{\kw{false}}}
    \BinaryInfC{$\Gamma \vdash \kw{if}\ C\ T\ \kw{else}\ F \Rightarrow \Phi$}
\end{prooftree}
\begin{prooftree}
    \AxiomC{$\Gamma \vdash C = \kw{true}$}
    \AxiomC{$\Gamma \vdash T \Rightarrow \Phi$}
    \LeftLabel{If\textsuperscript{\kw{true}}}
    \BinaryInfC{$\Gamma \vdash \kw{if}\ C\ T \Rightarrow \Phi$}
\end{prooftree}
\begin{prooftree}
    \AxiomC{$\Gamma \vdash C = \kw{false}$}
    \LeftLabel{If\textsuperscript{\kw{false}}}
    \UnaryInfC{$\Gamma \vdash \kw{if}\ C\ T : \top$}
\end{prooftree}

\subsubsection{Match}

The \kw{match} statement allows for pattern matching, allowing code to branch
on the structure of a value, rather than a Boolean evaluation of an expression.
As \Trilogy{} is a dynamically typed language, it is hard to ensure completeness
of a \kw{match} statement, so completeness is not enforced; a mismatch is skipped
in much the same way a \kw{false} evaluation of an \kw{if} statement is skipped.

The \kw{match} statement takes an expression and is followed by multiple \kw{case}
statements. Each \kw{case} statement has a pattern, a guard, or both, and then a block
of code to run if that case is selected.

\begin{bnf*}
    \bnfprod{SMatch}{
        \bnfts{\kw{match}}
        \bnfsp
        \bnfpn{Expression}
        \bnfsp
        \bnfpn{Cases}
    } \\
    \bnfprod{Cases}{
        \bnfpn{Case}\bnfsp\bnfpn{Cases}\bnfor\bnfpn{Case}
    } \\
    \bnfprod{Case}{
        \bnfts{\kw{case}}
        \bnfsp
        \bnfpn{Pattern}
        \bnfsp
        \bnfpn{Block}
        \bnfor
    } \\
    \bnfmore{
        \bnfts{\kw{case}}
        \bnfsp
        \bnfpn{Pattern}
        \bnfsp
        \bnfts{\kw{if}}
        \bnfsp
        \bnfpn{Expression}
        \bnfsp
        \bnfpn{Block}
        \bnfor
    } \\
    \bnfmore{
        \bnfts{\kw{case}}
        \bnfsp
        \bnfts{\kw{if}}
        \bnfsp
        \bnfpn{Expression}
        \bnfsp
        \bnfpn{Block}
        \bnfor
    }
\end{bnf*}

To evaluate a \kw{match} statement, the original expression is evaluated, and compared
against the pattern of each \kw{case}; an omitted pattern is interpreted like an underscore
(\texttt{\_}) pattern, matching with any value.

Before committing to a \kw{case} with a matching pattern, the guard is evaluated, if any.
If the guard expression evaluates to \kw{true} (or there are no guards), the case is selected.
Identifiers bound in the pattern are in scope for this evaluation. Any value that is not \kw{true},
including non-Boolean values, are treated as \kw{false}, and the case is not selected. An omitted
guard is treated to be always \kw{true}.

The block following the first selected case is then run. If no cases are selected,
they are simply ignored and the program continues.

\begin{prooftree}
    \AxiomC{$\Gamma \vdash E$}
    \AxiomC{$P = E$}
    \AxiomC{$\Gamma,\{a\ |\ a := P\} \vdash C = \kw{true}$}
    \LeftLabel{Match}
    \TrinaryInfC{$\Gamma \vdash \kw{match}\ E\ \kw{case}\ P\ \kw{if}\ C\ B \Rightarrow \Phi$}
\end{prooftree}

\begin{prooftree}
    \def\extraVskip{3.5pt}
    \AxiomC{$\Gamma \vdash E$}
    \AxiomC{$P = E$}
    \AxiomC{$\Gamma,\{a\ |\ a := P\} \vdash C \neq \kw{true}$}
    \TrinaryInfC{$R_1 = \kw{case}\ P\ \kw{if}\ C\ B$}
    \AxiomC{$\Gamma \vdash \kw{match}\ E\ R_2 \cdots R_n \Rightarrow \Phi$}
    \LeftLabel{Match\textsuperscript{\kw{false}}}
    \insertBetweenHyps{\hskip -12pt}
    \BinaryInfC{$\Gamma \vdash \kw{match}\ E\ R_1 R_2 \cdots R_n \Rightarrow \Phi$}
\end{prooftree}

\begin{prooftree}
    \def\extraVskip{3.5pt}
    \AxiomC{$\Gamma \vdash E$}
    \AxiomC{$P \neq E$}
    \BinaryInfC{$R_1 = \kw{case}\ P\ \kw{if}\ C\ B$}
    \AxiomC{$\Gamma \vdash \kw{match}\ E\ R_2 \cdots R_n \Rightarrow \Phi$}
    \LeftLabel{Match\textsuperscript{$\bot$}}
    \BinaryInfC{$\Gamma \vdash \kw{match}\ E\ R_1 R_2 \cdots R_n \Rightarrow \Phi$}
\end{prooftree}

The omitted clauses of the \kw{case} statement are formalized by syntax transformation:

\begin{align*}
    \texttt{\kw{case} \$p:pat} &\Rightarrow \texttt{\kw{case} \$p \kw{if} \kw{true}} \\ %
    \texttt{\kw{case} \kw{if} \$c:expr} &\Rightarrow \texttt{\kw{case} \_ \kw{if} \$c} \\ %
\end{align*}

\subsubsection{While}

The \kw{while} loop accepts a Boolean condition and repeats a block of code
until that condition evaluates to \kw{false}. \Trilogy{}'s \kw{while} loop is
very standard, as far as such loops go. There is no \texttt{do}\dots\texttt{while}
loop in \Trilogy{}.

\begin{bnf*}
    \bnfprod{SWhile}{
        \bnfts{\kw{while}}
        \bnfsp
        \bnfpn{Expression}
        \bnfsp
        \bnfpn{Block}
    }
\end{bnf*}

Branching may occur in the body of a \kw{while} loop, leading to more iterations
than expected if not careful. However, since the \kw{while} loop provides no collection
mechanism, once those branches break out of the loop they will continue to execute
beyond the end of the loop.

\begin{prooftree}
    \AxiomC{$\Gamma\vdash C = \kw{true}$}
    \AxiomC{$\Gamma\vdash B \Rightarrow \Phi$}
    \AxiomC{$\Phi\vdash \kw{while}\ C\ B \Rightarrow\Sigma$}
    \LeftLabel{Loop}
    \TrinaryInfC{$\Gamma\vdash \kw{while}\ C\ B \Rightarrow \Sigma$}
\end{prooftree}

\begin{prooftree}
    \AxiomC{$\Gamma\vdash C = \kw{false}$}
    \LeftLabel{End Loop}
    \UnaryInfC{$\Gamma\vdash \kw{while}\ C\ B:\top$}
\end{prooftree}

\subsubsection{For}

The \kw{for} loop repeats a block of code for all solutions to a provided query.
This is more similar to a for-each loop than the three part conditional loop seen in
some imperative languages. No such conditional \kw{for} loop exists in \Trilogy{},
but similar semantics can be constructed using \kw{while}.

\begin{bnf*}
    \bnfprod{SFor}{
        \bnfts{\kw{for}}
        \bnfsp
        \bnfpn{Query}
        \bnfsp
        \bnfpn{Block}
        \bnfor
    } \\
    \bnfmore{
        \bnfts{\kw{for}}
        \bnfsp
        \bnfpn{Query}
        \bnfsp
        \bnfpn{Block}
        \bnfsp
        \bnfts{\kw{else}}
        \bnfsp
        \bnfpn{SFor}
        \bnfor
    } \\
    \bnfmore{
        \bnfts{\kw{for}}
        \bnfsp
        \bnfpn{Query}
        \bnfsp
        \bnfpn{Block}
        \bnfsp
        \bnfts{\kw{else}}
        \bnfsp
        \bnfpn{Block}
    }
\end{bnf*}

Notably different from other languages is that a \kw{for} loop may have an \kw{else} clause,
similar to an \kw{if} statement. This \kw{else} clause is triggered in the case that the
unification fails.

The query of the \kw{for} loop is performed only once, at the beginning of the loop.
Any mutation to the values as a result of the loop's execution will not affect the remaining
iterations.

\begin{prooftree}
    \AxiomC{$\Gamma\vdash U_{1\hdots n}$}
    \AxiomC{$Q = U_i$}
    \AxiomC{$\Gamma,U_1\vdash B \Rightarrow\Phi_1$}
    \noLine
    \UnaryInfC{$\Phi_1,U_2\vdash B \Rightarrow\Phi_2$}
    \noLine
    \UnaryInfC{$\vdots$}
    \noLine
    \UnaryInfC{$\Phi_{n-1},U_n\vdash B \Rightarrow\Sigma$}
    \LeftLabel{For}
    \TrinaryInfC{$\Gamma\vdash \kw{for}\ Q\ B\ \kw{else}\ F\Rightarrow \Sigma$}
\end{prooftree}

\begin{prooftree}
    \AxiomC{$\Gamma\vdash Q:\bot$}
    \AxiomC{$\Gamma\vdash F\Rightarrow\Phi$}
    \LeftLabel{For\textsuperscript{$\bot$}}
    \BinaryInfC{$\Gamma\vdash \kw{for}\ Q\ B\ \kw{else}\ F\Rightarrow \Phi$}
\end{prooftree}

A \kw{for} loop is distinct from the branching \kw{let} statement in how execution
continues after each branch has been evaluated. Where a \kw{let} statement would
result in multiple parallel executions with separate execution contexts, the
\kw{for} loop executes the block for each binding of the query in sequence,
all within one continuous execution context. A failed \kw{let} binding will cause
the current execution to fizzle, while a \kw{for} loop query that does not
have a solution simply causes the loop to be skipped, but execution does not end.

\subsubsection{Break and Continue}

The statements \kw{break} and \kw{continue} may be used in loops to control iteration.
The \kw{break} statement will exit the loop immediately, skipping all further iteration,
whereas the \kw{continue} statement will end the current iteration and move on to the
next, without executing any further code. This applies for both \kw{for} and \kw{while}
loops.

Either of these statements given a label will act on that labelled loop.
Without a label, the closest enclosing loop is assumed. If the label corresponds to a
statement that is not a loop (or block for a \kw{break}), the syntax is to be considered
invalid.

The \kw{break} statement may be used in labelled blocks as well, breaking out of
those blocks immediately. Unlabelled blocks cannot be used with the \kw{break}
statement in order to give priority to the loop usage, which is likely to be
more frequent. The \kw{continue} statement may not be used on regular blocks,
only on loops.

\begin{bnf*}
    \bnfprod{SBreak}{
        \bnfts{\kw{break}}
        \bnfor
        \bnfts{\kw{break}}
        \bnfsp
        \bnfts{Identifier}
    } \\
    \bnfprod{SContinue}{
        \bnfts{\kw{continue}}
        \bnfor
        \bnfts{\kw{continue}}
        \bnfsp
        \bnfts{Identifier}
    }
\end{bnf*}

Labels are a concept only at the syntax level, so are not considered bindings
that may be used at runtime, and may not be passed as parameters. When a \kw{break}
or \kw{continue} statement uses a label, they may only refer to labels on blocks
or loops syntactically surrounding the statement using the label. In the semantics
below, it is assumed that the block or loop that the \kw{break} or \kw{continue}
is meant to correspond to has already been correctly resolved, and so the labels are
omitted.

A \kw{break} or \kw{continue} statement on its own does not evaluate to anything.

\begin{figure}[H]
    \centering
    \parbox[t]{0.45\linewidth}{
        \begin{prooftree}
            \AxiomC{}
            \LeftLabel{Break}
            \UnaryInfC{$\Gamma \vdash \kw{break} : \bot$}
        \end{prooftree}
    }
    \parbox[t]{0.45\linewidth}{
        \begin{prooftree}
            \AxiomC{}
            \LeftLabel{Continue}
            \UnaryInfC{$\Gamma \vdash \kw{continue} : \bot$}
        \end{prooftree}
    }
\end{figure}

Though the statement ends the local execution, it changes the behaviour of the loop
or block it corresponds to, which picks up the execution before it fizzles completely.

\begin{prooftree}
    \AxiomC{$\Gamma\vdash \texttt{\{}\ M\ \texttt{\}} : \top$}
    \LeftLabel{Break\textsuperscript{Block}}
    \UnaryInfC{$\Gamma \vdash \texttt{\{}\ M\ \op{,}\ \kw{break}\ \texttt{\}} : \top$}
\end{prooftree}

\begin{prooftree}
    \AxiomC{$\Gamma \vdash \kw{for}\ Q\ \texttt{\{}\ M\ \texttt{\}} : \top$}
    \LeftLabel{Break\textsuperscript{for}}
    \UnaryInfC{$\Gamma \vdash \kw{for}\ Q\ \texttt{\{}\ M\ \op{,}\ \kw{break}\ \texttt{\}} : \top$}
\end{prooftree}

\begin{prooftree}
    \AxiomC{$\Gamma \vdash \kw{while}\ E\ \texttt{\{}\ M\ \texttt{\}} : \top$}
    \LeftLabel{Break\textsuperscript{while}}
    \UnaryInfC{$\Gamma \vdash \kw{while}\ E\ \texttt{\{}\ M\ \op{,}\ \kw{break}\ \texttt{\}} : \top$}
\end{prooftree}

\begin{prooftree}
    \AxiomC{$\Gamma \vdash \texttt{\{}\ M\ \texttt{\}} : \top$}
    \LeftLabel{Continue}
    \UnaryInfC{$\Gamma \vdash \texttt{\{}\ M\ \texttt{,}\ \kw{continue}\ \texttt{\}} : \top$}
\end{prooftree}

\subsubsection{Return}

Similar to \kw{break} and \kw{continue}, the \kw{return} statement causes execution
to end and be picked up elsewhere. The \kw{return} statement may designate the value
that a procedure call evaluates to. This should be intuitive to any who have written
code before.

A \kw{return} statement with no value is assumed to return \kw{unit}, the same as if
the \kw{return} statement was omitted entirely and execution ran off the end of the
procedure.

\begin{bnf*}
    \bnfprod{SReturn}{
        \bnfts{\kw{return}}
        \bnfor
        \bnfts{\kw{return}}
        \bnfsp
        \bnfpn{Expression}
    }
\end{bnf*}

Semantically, the \kw{return} statement on its own is simple. Meanwhile, its meaning
to a procedure was already covered as part of the semantics of procedure definitions
(Proof~\ref{proof:return}).

\begin{prooftree}
    \AxiomC{$\Gamma\vdash N:\top$}
    \LeftLabel{Return}
    \UnaryInfC{$\Gamma\vdash\kw{return}\ N : \bot$}
\end{prooftree}

\subsubsection{End}

The last execution ending statement is the \kw{end} statement, of which the only
purpose is to explicitly end an execution.

The \kw{end} statement may be bare, or be provided a single value. In the case of
a bare \kw{end} statement, the execution is simply stopped, the same as any fizzle.

Meanwhile, provided a value, the execution is treated to have been completed and
the value is its final result. While this value is not currently accessible from
within the program, it is used to treat the current execution as if it had returned
that value from the \fn{main} procedure, which has some effect on the resulting
exit code of the program (\S\ref{sec:exitcode}).

\begin{bnf*}
    \bnfprod{SEnd}{
        \bnfts{\kw{end}}
        \bnfor
        \bnfts{\kw{end}}
        \bnfsp
        \bnfpn{Expression}
    }
\end{bnf*}

Semantically, the \kw{end} statement is very simple, as with the other execution
ending statements.

\begin{prooftree}
    \AxiomC{}
    \LeftLabel{End}
    \UnaryInfC{$\Gamma\vdash\kw{end}\ : \bot$}
\end{prooftree}

\begin{prooftree}
    \AxiomC{$\Gamma\vdash N:\top$}
    \LeftLabel{End}
    \UnaryInfC{$\Gamma\vdash\kw{end}\ N : \bot$}
\end{prooftree}

\subsubsection{Yield and Resume}

The \kw{yield} statement, however simple in appearance, is one of the most
semantically ambiguous statements in all of \Trilogy{}, as it provides direct
access to the effect system that underpins the entirety of \Trilogy{}'s control
flow structure.

The \kw{resume} statement is a bit of a special case in that it is
treated both as a keyword and a value. The \kw{resume} statement is available only
in an effect handler, and is syntactically invalid when not within an effect handler.
In the case of nested effect handlers, the \kw{resume} keyword refers to the innermost
handler's continuation.

% TODO: use first-class keyword for break and continue, for consistency. No labels!
% Would additionally make it possible to pass break and continue to functions, which
% would be an interesting control flow experiment consistent with how break and continue
% would work were they user-built as effects and handlers.

\begin{bnf*}
    \bnfprod{SYield}{
        \bnfts{\kw{yield}}
        \bnfsp
        \bnfpn{Expression}
    }
\end{bnf*}

On its own, the \kw{yield} keyword is used to perform an ``effect'', which is
handled by an upstream handler, designated by \kw{when} (which will be seen later).
The effect in this situation is simply any value which is unified with the patterns
in the handlers to find the first one which matches, running its body in order to
handle the effect and determine how execution should proceed.

Seen here, the \kw{yield} keyword is used in statement position, so its resulting
value, if any, is discarded (\kw{yield} is more often used in expression position).
However, due to an effect handler being possibly useful for side effects, \kw{yield}
may be used in statement position as well.

\begin{prooftree}
    \AxiomC{$\Gamma\vdash E$}
    \AxiomC{$\Gamma\vdash \text{when}\ P\ B$}
    \UnaryInfC{$\Gamma\vdash B$}
    \AxiomC{$P = E$}
    \TrinaryInfC{$\Gamma\vdash \kw{yield}\ E : \top$}
\end{prooftree}
