\subsection{Prose}

When viewed on its own, \Prose{} has a fairly ``standard'' C-family syntax.
Arbitrary patterns are permitted in the procedure heads.
Trailing commas are permitted in all places where commas are found.

\begin{bnf*}
    \bnfprod{ProcedureHead}{
        \bnfts{IdentiferBang}
        \bnfsp
        \bnfts{(}
        \bnfsp
        \bnfpn{ParameterList}
        \bnfsp
        \bnfts{)}
    } \\
    \bnfprod{ParameterList}{
        \bnfpn{Pattern}
        \bnfsp
        \bnfts{,}
        \bnfsp
        \bnfpn{ParameterList}
        \bnfor
    } \\
    \bnfmore{
        \bnfpn{Pattern}
        \bnfor
        \bnfes
    }
\end{bnf*}

\noindent A procedure declaration binds the name in the scope of the current module.
As with all kinds of declarations in \Trilogy{}, procedure declarations may be
referenced out of order.

When called, the body of the procedure is run with the formal parameters bound to
the arguments with which the procedure was called. A procedure may use the \kw{return}
keyword to end its evaluation, with its call evaluating to the returned value.
If control runs off the end of the procedure's body, its call evaluates to \kw{unit}.

It is also possible for a procedure to never return (e.g. built-in \fn{exit!}), in
which case, the return type is Never, and such a value may never be realized.

\begin{prooftree}
    \def\defaultHypSeparation{\hskip 0pt}
    \AxiomC{$\kw{proc}\ p\texttt{!}(x_{1\hdots n})\ \{\ M,\ \kw{return}\ N\ \}$}
    \AxiomC{$\Gamma,x_{1\hdots n}:\tau_{1\hdots n} \vdash M,\ N : \tau$}
    \LeftLabel{Procedure\textsuperscript{ret}}
    \BinaryInfC{$\Gamma \vdash p: \texttt{!(}\tau_{1\hdots n}\texttt{)}\rightarrow\tau$}
\end{prooftree}

\begin{prooftree}
    \AxiomC{$\kw{proc}\ p\texttt{!}(x_{1\hdots n})\ \{\ M\ \}$}
    \AxiomC{$\Gamma,x_{1\hdots n}:\tau_{1\hdots n} \vdash M : \top$}
    \LeftLabel{Procedure\textsuperscript{end}}
    \BinaryInfC{$\Gamma \vdash p: \texttt{!(}\tau_{1\hdots n}\texttt{)}\rightarrow\kw{unit}$}
\end{prooftree}

\begin{prooftree}
    \AxiomC{$\kw{proc}\ p\texttt{!}(x_{1\hdots n})\ \{\ M\ \}$}
    \AxiomC{$\Gamma,x_{1\hdots n}:\tau_{1\hdots n} \vdash M : \bot$}
    \LeftLabel{Procedure\textsuperscript{$\bot$}}
    \BinaryInfC{$\Gamma \vdash p:\ \texttt{!(}\tau_{1\hdots n}\texttt{)}\rightarrow\bot$}
\end{prooftree}

\subsubsection{Sequences}

The body of a procedure is a sequence of statements. Statements in sequence
must be separated by a line break or an explicit separator (\op{,}).

\begin{bnf*}
    \bnfprod{ProcedureBody}{\bnfpn{Sequence}} \\
    \bnfprod{Sep}{\bnfpn{EOL}\bnfor\bnfts{,}} \\
    \bnfprod{Sequence}{\bnfpn{Statement}\bnfsp\bnfpn{Sep}\bnfsp\bnfpn{Sequence}\bnfor} \\
    \bnfmore{\bnfpn{Statement}\bnfor\bnfes}
\end{bnf*}

A sequence of statements evaluates to the result of the last statement
in the sequence. The values of preceding statements are lost. This rule
may be overridden by the rules in the following sections depending on
the specific types of statements.

\begin{prooftree}
    \AxiomC{$\Gamma \vdash M : \tau_0$}
    \AxiomC{$\Gamma \vdash N : \tau_1$}
    \LeftLabel{Sequencing}
    \BinaryInfC{$\Gamma \vdash M \op{,}\ N : \tau_1$}
\end{prooftree}

\subsubsection{Statements}

While most imperative languages bring some sort of distinction between
expressions and statements, \Prose{} does not. Statements in this
sense might have been more aptly referred to as expressions, but we
stick to the statement terminology to differentiate from expressions
in \Poetry{}.

Any statement or expression may occur within any other statement, and
all statements evaluate to some value, albeit sometimes \kw{unit}.
This can make for some possibly unexpected but useful combinations
(e.g. \texttt{x == 5 or return 3}); whether it is a good idea to use
such combinations is left to the developer to decide.

While some statements are unique to \Prose{}, others look remarkably similar
to those found in the other languages, but while allowing other statements
to be nested within; for example, \texttt{Int readline!()} could almost be
a \Poetry{} application of \fn{Int} if not for the procedure call in argument
position. As procedure calls are not permitted in \Poetry{}, this ends up
producing very similar productions with very similar semantics. To
differentiate potentially ambiguous statement production names, they
are prefixed with \texttt{S}.

\begin{bnf*}
    \bnfprod{Statement}{
        \bnfpn{SLet}
        \bnfor
        \bnfpn{SAssignment}
        \bnfor
        \bnfpn{SIfElse}
        \bnfor
    } \\
    \bnfmore{
        \bnfpn{SMatch}
        \bnfor
        \bnfpn{SFor}
        \bnfor
        \bnfpn{SWhile}
        \bnfor
        \bnfpn{SLoop}
        \bnfor
    } \\
    \bnfmore{
        \bnfpn{SBreak}
        \bnfor
        \bnfpn{SContinue}
        \bnfor
        \bnfpn{SReturn}
        \bnfor
    } \\
    \bnfmore{
        \bnfpn{SYield}
        \bnfor
        \bnfpn{SWhen}
        \bnfor
        \bnfpn{SGiven}
        \bnfor
    } \\
    \bnfmore{
        \bnfpn{SCollection}
        \bnfor
        \bnfpn{SComprehension}
        \bnfor
    } \\
    \bnfmore{
        \bnfpn{SInfixOp}
        \bnfor
        \bnfpn{SApplication}
        \bnfor
    } \\
    \bnfmore{
        \bnfpn{ProcedureCall}
        \bnfor
        \bnfpn{DoBlock}
        \bnfor
        \bnfpn{Block}
        \bnfor
    } \\
    \bnfmore{
        \bnfpn{ValueAccess}
        \bnfor
        \bnfpn{ParenthesizedStatement}
    } \\
    \bnfprod{Substatement}{
        \bnfpn{SPrefixOp}\bnfor\bnfpn{Statement}
    } \\
    \bnfprod{ParenthesizedStatement}{
        \bnfts{(}
        \bnfsp
        \bnfpn{Substatement}
        \bnfsp
        \bnfts{)}
    } \\
    \bnfprod{Block}{
        \bnfts{\{}
        \bnfsp
        \bnfpn{Sequence}
        \bnfsp
        \bnfts{\}}
    }
\end{bnf*}

The one distinction is that a top-level statement may not begin with a
prefix operator. Given the lack of postfix operators in Trilogy, that
leaves infix operators being the only valid operators at the top-level
of a sequence.

Add the evaluation semantics for parenthesized and block statements.
Notably, a block introduces a new scopes in which names are bound,
while parentheses do not.

\begin{prooftree}
    \AxiomC{$\Gamma \vdash M : \tau$}
    \LeftLabel{Unwrap}
    \UnaryInfC{$\Gamma \vdash \texttt{(}\ M\ \texttt{)} : \tau$}
\end{prooftree}

\begin{prooftree}
    \AxiomC{$\Gamma :> \Phi$}
    \AxiomC{$\Phi \vdash M : \tau$}
    \LeftLabel{Unnest}
    \BinaryInfC{$\Gamma \vdash \texttt{\{}\ M\ \texttt{\}} : \tau$}
\end{prooftree}

\subsubsection{Let}

The \kw{let} statement in \Prose{} is a non-fallible pattern matching context.
If the value on the right side of the \op{=} does not match the pattern on the
left, the execution fizzles.

Only a single pattern may be bound in each let statement, though tuple patterns
may be used to simulate binding multiple values at once.

The \kw{let} introduces an implicit scope from its location in the source
code until the end of the nearest explicit scope (e.g. the end of a block).

\begin{bnf*}
    \bnfprod{SLet}{
        \bnfts{\kw{let}}
        \bnfsp
        \bnfpn{Pattern}
        \bnfsp
        \bnfts{\op{=}}
        \bnfsp
        \bnfpn{Substatement}
    }
\end{bnf*}

\begin{prooftree}
    \def\defaultHypSeparation{\hskip 0in}
    \AxiomC{$\Gamma \vdash y:\pi$}
    \AxiomC{$\Gamma \vdash y \mapsto x$}
    \AxiomC{$\Gamma \vdash \{ a_i : \pi_i\ |\ a_i := x \}$}
    \AxiomC{$\Gamma, a_i: \pi_i \vdash M : \tau$}
    \LeftLabel{\label{sem:declaration}Declaration}
    \QuaternaryInfC{$\Gamma \vdash \kw{let}\ x = y\op{,}\ M : \tau$}
\end{prooftree}

\subsubsection{Assignment}

Variables (specifically, mutable bindings) may be assigned to. Similar to
declaration via \kw{let}, the left-hand side of standard assignment may be
any pattern that does not introduce \emph{new} bindings. The pattern may only
use previously bound names.

Function assignment is a special syntax for applying a function to a value and
then simultaneously reassigning that value with the result of the function; a
generalization of operators like \op{+=} to any function. In this form of assignment,
the left-hand side of the assignment operator is a single identifier, and the right
side is the arguments to apply to the function being used as the assignment operator.
The left hand side is then applied as the last argument to the function.

Regular operator assignment (\op{+=}, \op{-=}, etc.) is supported as well, in the
intuitive way (the left hand identifer is used as the left hand side of the operator).

\begin{bnf*}
    \bnfprod{SAssignment}{
        \bnfpn{PatternAssignment}
        \bnfor
        \bnfpn{FunctionAssignment}
        \bnfor
    } \\
    \bnfmore{
        \bnfpn{OperatorAssignment}
    } \\
    \bnfprod{PatternAssignment}{
        \bnfpn{Pattern}
        \bnfsp
        \bnfts{\op{=}}
        \bnfsp
        \bnfpn{Substatement}
    } \\
    \bnfprod{FunctionAssignment}{
        \bnfpn{Identifier}
        \bnfsp
        \bnfts{IdentifierEq}
        \bnfsp
        \bnfpn{ApplicationList}
    } \\
    \bnfprod{OperatorAssignment}{
        \bnfpn{Identifier}
        \bnfsp
        \bnfpn{OperatorEq}
        \bnfsp
        \bnfpn{Substatement}
    } \\
    \bnfprod{OperatorEq}{
        \bnfts{+=}\bnfor
        \bnfts{-=}\bnfor
        \bnfts{*=}\bnfor
        \bnfts{/=}\bnfor
        \bnfts{\%=}\bnfor
        \bnfts{**=}\bnfor
        \bnfts{//=}\bnfor
        \bnfts{<>=}\bnfor
    } \\
    \bnfmore{
        \bnfts{|=}\bnfor
        \bnfts{\&=}\bnfor
        \bnfts{\textasciicircum=}\bnfor
        \bnfts{\textasciitilde=}\bnfor
        \bnfts{<\textasciitilde=}\bnfor
        \bnfts{\textasciitilde>=}
    }
\end{bnf*}

\begin{prooftree}
    \def\defaultHypSeparation{\hskip 0.05in}
    \AxiomC{$\Gamma \vdash y:\pi$}
    \AxiomC{$y \mapsto x$}
    \AxiomC{$\Gamma \vdash \{a_i : \pi_i\ |\ a_i := x\ \}$}
    \AxiomC{$\Gamma,a_i:\pi_i \vdash M : \tau$}
    \LeftLabel{Assignment}
    \QuaternaryInfC{$\Gamma\vdash x\ \op{=}\ y\op{,}\ M : \tau$}
\end{prooftree}

Rather than defining the exact semantics of function or operator assignment,
instead understand those by syntax transformation, and use the same assignment
semantics as normal:

\begin{align*}
    \texttt{\$x:id \$fn:id= \$(\$a:expr)+} &\Rightarrow \texttt{\$x = \$f \$(\$a)+ \$x} \\ %
    \texttt{\$x:id \$op:op= \$a:expr} & \Rightarrow \texttt{\$x = \$x \$op \$a} %
\end{align*}

\subsubsection{Conditionals}

The conditional statement in \Trilogy{} comes in the form of \kw{if}-\kw{else}
chains. For \Prose{} they are defined as follows.

\begin{bnf*}
    \bnfprod{SIfElse}{
        \bnfts{\kw{if}}
        \bnfsp
        \bnfpn{Condition}
        \bnfsp
        \bnfpn{Block}
        \bnfsp
        \bnfpn{SElseChain}
        \bnfor
    } \\
    \bnfprod{SElseChain}{
        \bnfts{\kw{else}}
        \bnfsp
        \bnfts{\kw{if}}
        \bnfsp
        \bnfpn{Condition}
        \bnfsp
        \bnfpn{Block}
        \bnfsp
        \bnfpn{SElseChain}
        \bnfor
    } \\
    \bnfmore{
        \bnfts{\kw{else}}
        \bnfsp
        \bnfpn{Block}
        \bnfor
        \bnfes
    }
\end{bnf*}

In the common case, the condition is a statement that evaluates to a Boolean
result, \kw{true} or \kw{false}, and control flows as you might expect.
If the result is not a Boolean, the execution fizzles.

An \kw{if}-\kw{else} chain evaluates to the value of the block that is executed.
If no block is executed (there is no \kw{else} clause, and none of the
conditions matched) the resulting value is \kw{unit}.

\begin{prooftree}
    \AxiomC{$C : \kw{true}$}
    \AxiomC{$T : \tau_1$}
    \AxiomC{$F : \tau_2$}
    \LeftLabel{Branch Hit\textsuperscript{A}}
    \TrinaryInfC{$\kw{if}\ C\ T\ \kw{else}\ F : \tau_1$}
\end{prooftree}
\begin{prooftree}
    \AxiomC{$C : \kw{false}$}
    \AxiomC{$T : \tau_1$}
    \AxiomC{$F : \tau_2$}
    \LeftLabel{Branch Miss\textsuperscript{A}}
    \TrinaryInfC{$\kw{if}\ C\ T\ \kw{else}\ F : \tau_2$}
\end{prooftree}
\begin{prooftree}
    \AxiomC{$C : \kw{true}$}
    \AxiomC{$T : \tau$}
    \LeftLabel{Branch Hit\textsuperscript{B}}
    \BinaryInfC{$\kw{if}\ C\ T : \tau$}
\end{prooftree}
\begin{prooftree}
    \AxiomC{$C : \kw{false}$}
    \AxiomC{$T : \tau$}
    \LeftLabel{Branch Miss\textsuperscript{B}}
    \BinaryInfC{$\kw{if}\ C\ T : \kw{unit}$}
\end{prooftree}

In addition to Boolean evaluations, Conditions may be queries which bind
values to names available for the scope of the block of the first matching
condition. This is similar to the \texttt{if let} statement seen in other
languages, but extended to cover more than just pattern matching by taking
advantage of \Law{}. Conditions may not be empty, but trailing commas are
still supported, so we end up with this annoying extra bit of rule.

\begin{bnf*}
    \bnfprod{Condition}{
        \bnfpn{Substatement}
        \bnfsp
        \bnfts{,}
        \bnfsp
        \bnfpn{Condition}
        \bnfor
        \bnfpn{Query}
        \bnfsp
        \bnfts{,}
        \bnfsp
        \bnfpn{Condition}
        \bnfor
    } \\
    \bnfmore{
        \bnfpn{Substatement}
        \bnfsp
        \bnfts{,}
        \bnfor
        \bnfpn{Query}
        \bnfsp
        \bnfts{,}
        \bnfor
        \bnfpn{Substatement}
        \bnfor
        \bnfpn{Query}
    } \\
    \bnfprod{Query}{
        \bnfpn{Pattern}
        \bnfsp
        \bnfts{=}
        \bnfsp
        \bnfpn{Substatement}
        \bnfor
        \bnfpn{RuleCheck}
    }
\end{bnf*}

A conditional is not an iterative context. If a pattern condition matches
to multiple results, the first result is chosen and the rest are skipped.
