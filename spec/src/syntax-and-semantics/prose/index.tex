\subsection{Prose}

When viewed on its own, \Prose{} has a fairly ``standard'' C-family syntax.

Arbitrary patterns are permitted in the procedure heads, just as in \Law{}'s
rule definitions. This does imply that some procedure calls will fail silently
via fizzling as procedures may not be defined multiple times with different
patterns, in the way that rules or functions may be. It is recommended to only
use such arbitrary patterns in the most certain of cases, and otherwise use
single identifier patterns whenever possible and handle alternative inputs
accordingly.

The ability to define multiple procedures with the same name but different
patterns may be added in future.

\begin{bnf*}
    \bnfprod{ProcedureHead}{
        \bnfts{Identifer}
        \bnfsp
        \bnfts{!(}
        \bnfsp
        \bnfpn{PatternList}
        \bnfsp
        \bnfts{)}
    } \\
    \bnfprod{ProcedureBody}{\bnfpn{Sequence}} \\
\end{bnf*}

\noindent A procedure declaration binds the name in the scope of the current module.
As with all kinds of declarations in \Trilogy{}, procedure declarations may be
referenced out of order.

When called, the body of the procedure is run with the formal parameters bound to
the arguments with which the procedure was called. A procedure may use the \kw{return}
keyword to end its evaluation, with its call evaluating to the returned value.
If control runs off the end of the procedure's body, its call evaluates to \kw{unit}.
A procedure may cause a fizzle or branch, in which case all evaluations are propgated
to the call site, conceptually causing the call to fizzle or branch respectively.

\begin{prooftree}
    \AxiomC{$\kw{proc}\ p\texttt{!}(P_{1\hdots n})\ \block{M,\ \kw{return}\ N}$}
    \AxiomC{$\Sigma\vdash x_i: \tau_i$}
    \AxiomC{$x_i = P_i$}
    \def\extraVskip{3.5pt}
    \BinaryInfC{$\Gamma,\setof{a_i\given a_i\defby P_i} \vdash M\Rightarrow\Phi$}
    \UnaryInfC{$\Phi\vdash N : \rho$}
    \LeftLabel{Procedure}
    \ProofLabel{proof:return}
    \def\extraVskip{2pt}
    \BinaryInfC{$\Gamma \vdash p : \texttt{!(}\tau_{1\hdots n}\texttt{)} \rightarrow \rho$}
\end{prooftree}

\begin{prooftree}
    \AxiomC{$\kw{proc}\ p\texttt{!}(P_{1\hdots n})\ \block{M}$}
    \AxiomC{$\Sigma\vdash x_i: \tau_i$}
    \AxiomC{$x_i = P_i$}
    \def\extraVskip{3.5pt}
    \BinaryInfC{$\Gamma,\setof{a_i\given a_i\defby P_i} \vdash M: \top$}
    \LeftLabel{Procedure\textsuperscript{$\top$}}
    \def\extraVskip{2pt}
    \BinaryInfC{$\Gamma \vdash p : \texttt{!(}\tau_{1\hdots n}\texttt{)} \rightarrow \kw{unit}$}
\end{prooftree}

\begin{prooftree}
    \AxiomC{$\kw{proc}\ p\texttt{!}(P_{1\hdots n})\ \block{M}$}
    \AxiomC{$\Sigma\vdash x_i: \tau_i$}
    \AxiomC{$x_i = P_i$}
    \def\extraVskip{3.5pt}
    \BinaryInfC{$\Gamma,\setof{a_i\given a_i\defby P_i} \vdash M: \top$}
    \LeftLabel{Procedure\textsuperscript{$\bot$}}
    \BinaryInfC{$\Gamma \vdash p : \texttt{!(}\tau_{1\hdots n}\texttt{)} \rightarrow \bot$}
\end{prooftree}

\subsubsection{Sequences}

Sequences are statements which are to be executed in order, in typical
imperative fashion. Statements in sequence must be separated by a line break or
an explicit separator (\op{;}).

\begin{bnf*}
    \bnfprod{Sep}{\bnfpn{EOL}\bnfor\bnfts{;}} \\
    \bnfprod{Sequence}{\bnfpn{Statement}\bnfsp\bnfpn{Sep}\bnfsp\bnfpn{Sequence}\bnfor} \\
    \bnfmore{\bnfpn{Statement}\bnfor\bnfes}
\end{bnf*}

A sequence of statements executes each statement in order, performing some effect on the context in which
it is being run. The value of the sequence is the value of the final statement in that sequence. Note, however,
that only expression statements have values; if the final statement of a sequence is a non-expression statement,
then the value of the whole sequence is just \kw{unit}.

\begin{prooftree}
    \AxiomC{$\Gamma \vdash M \Rightarrow \Phi$}
    \AxiomC{$\Phi \vdash N: \tau \Rightarrow \Sigma$}
    \LeftLabel{Sequence}
    \BinaryInfC{$\Gamma \vdash M \op{;}\ N: \tau \Rightarrow \Sigma$}
\end{prooftree}

\begin{prooftree}
    \AxiomC{$\Gamma \vdash M \Rightarrow \bot$}
    \LeftLabel{Sequence\textsuperscript{$\bot$}}
    \UnaryInfC{$\Gamma \vdash M \op{;}\ N: \bot$}
\end{prooftree}

Other than directly as a procedure body, sequences can be placed in any block.
Blocks may be used in any situation expressions are allowed, and act as a
scope barrier. Bindings declared within a block go out of scope at the end
of the block.

\begin{bnf*}
    \bnfprod{Block}{
        \bnfts{\{}
        \bnfsp
        \bnfpn{Sequence}
        \bnfsp
        \bnfts{\}}
    }
\end{bnf*}

\begin{prooftree}
    \AxiomC{$\Gamma :> \Phi$}
    \AxiomC{$\Phi \vdash M : \tau$}
    \LeftLabel{Unnest}
    \BinaryInfC{$\Gamma \vdash \block{M} : \tau$}
\end{prooftree}


\subsubsection{Statements}

A statement is a single ``step'' of a procedure in \Prose{}. Statements do not
evaluate to any value, but they perform effects on the state of the program and
its environment.

To differentiate with similar constructs in \Poetry{}, while also saving space,
the names of the statement productions are prefixed with S, short for statement.
Meanwhile, expressions are actually all \Poetry{}, so such specification is left
until later.

\begin{bnf*}
    \bnfprod{Statement}{
        \bnfpn{SLet}
        \bnfor
        \bnfpn{SAssignment}
        \bnfor
        \bnfpn{SIf}
        \bnfor
        \bnfpn{SMatch}
        \bnfor
    } \\
    \bnfmore{
        \bnfpn{SWhile}
        \bnfor
        \bnfpn{SFor}
        \bnfor
        \bnfpn{SBreak}
        \bnfor
        \bnfpn{SContinue}
        \bnfor
    } \\
    \bnfmore{
        \bnfpn{SReturn}
        \bnfor
        \bnfpn{SEnd}
        \bnfor
        \bnfpn{Exit}
        \bnfor
        \bnfpn{SYield}
        \bnfor
    } \\
    \bnfmore{
        \bnfpn{SResume}
        \bnfor
        \bnfpn{SCancel}
        \bnfor
        \bnfpn{SProcedureCall}
        \bnfor
    } \\
    \bnfmore{
        \bnfts{(}
        \bnfsp
        \bnfpn{Expression}
        \bnfsp
        \bnfts{)}
        \bnfor
        \bnfpn{Block}
        \bnfor
        \bnfpn{SAssert}
    } \\
    \bnfprod{Block}{
        \bnfts{\{}
        \bnfsp
        \bnfpn{Sequence}
        \bnfsp
        \bnfts{\}}
        \bnfor
        \bnfts{\{}
        \bnfsp
        \bnfpn{Sequence}
        \bnfsp
        \bnfts{\}}
        \bnfsp
        \bnfpn{EffectHandlers}
    }
\end{bnf*}

Evaluating an expressions as a statement is allowed, but is typically
recommended against. In particular cases, however, it may be necessary,
so wrapping the expression in parentheses will allow this. Bare expressions
are not permitted, except for procedure calls which are allowed by special
case of the statement syntax.

\begin{prooftree}
    \AxiomC{$\Gamma \vdash M : \top$}
    \LeftLabel{Unwrap}
    \UnaryInfC{$\Gamma \vdash \texttt{(}\ M\ \texttt{)} : \top$}
\end{prooftree}

Blocks may be used in any situation statements are allowed, and act as
a scope barrier. Bindings declared within a block go out of scope at
the end of the block.

\begin{prooftree}
    \AxiomC{$\Gamma :> \Phi$}
    \AxiomC{$\Phi \vdash M : \top$}
    \LeftLabel{Unnest}
    \BinaryInfC{$\Gamma \vdash \block{M} : \top$}
\end{prooftree}

In addition to being a scope barrier, blocks serve as anchor points onto
which effect handlers may be attached. The semantics of effect handlers
will be addressed in \S\ref{sec:effects}.


\subsubsection{Let}

The \kw{let} statement in \Prose{} exposes a ``binding context'' in which
bindings may be introduced, similar to that of \Law{}. Depending on how many times
the binding pattern matches, this may introduce a branch or cause a fizzle.
Most often, however, such bindings are done via a direct single unification
which is determinisitic, and so control flow will progress intuitively.

\begin{bnf*}
    \bnfprod{SLet}{
        \bnfts{\kw{let}}
        \bnfsp
        \bnfpn{Query}
    } \\
\end{bnf*}

Semantically, a \kw{let} statement introduces an implicit scope from its location
in the source code until the end of the nearest explicit scope (e.g.\ the end of a block).
The bindings declared in this statement are available within that scope.

What is more unique to \Trilogy{} is the branching and fizzling capability of the
\kw{let} statement when provided a query via a more complex rule of \Law{}.
Depending on how many possible bindings there are for the provided rule, a \kw{let}
statement may lead to multiple executions of the program running in parallel, or none.

\begin{prooftree}
    \AxiomC{$\Gamma \vdash Q \Rightarrow \Phi$}
    \LeftLabel{Branch}
    \UnaryInfC{$\Gamma \vdash \kw{let}\ Q \Rightarrow \Phi$}
\end{prooftree}


\subsubsection{Assignment}

Variables (specifically, mutable bindings) may be assigned to. Similar to
declaration via \kw{let}, the left-hand side of standard assignment may be
any pattern so long as it does not introduce new bindings. The pattern may only
use previously bound names. The assignment statement updates the values of the
bindings in the pattern.

The left-hand side may also be a member access expression (\S\ref{sec:member-access}),
in which case the member in the collection is updated in-place but otherwise the
assignment works as any other normal.

Function assignment is a special syntax for applying a function to a value and
then simultaneously reassigning that value with the result of the function; a
generalization of operator assignment (e.g. \op{+=}) to any function\footnote{A
generalization which I cannot take credit for as I was inspired to it by
\href{https://github.com/betaveros/noulith}{Noulith}, the same project
which inspired me to attempt \Trilogy{} at all.}. In this form
of assignment, the left-hand side of the assignment operator is a single identifier,
and the right side is the arguments to apply to the function being used as the
assignment operator. The left hand side is then applied as the last argument to
the function.

Regular operator assignment (\op{+=}, \op{-=}, etc.) is supported as well, in the
intuitive way: the left hand identifer is used as the left hand side of the operator.

\begin{bnf*}
    \bnfprod{SAssignment}{
        \bnfpn{Assignment}
        \bnfor
        \bnfpn{FunctionAssignment}
        \bnfor
    } \\
    \bnfmore{
        \bnfpn{OperatorAssignment}
    } \\
    \bnfprod{Assignment}{
        \bnfpn{LValue}
        \bnfsp
        \bnfts{\op{=}}
        \bnfsp
        \bnfpn{Expression}
    } \\
    \bnfprod{FunctionAssignment}{
        \bnfpn{LValue}
        \bnfsp
        \bnfts{IdentifierEq}
        \bnfsp
        \bnfpn{ApplicationList}
    } \\
    \bnfprod{OperatorAssignment}{
        \bnfpn{LValue}
        \bnfsp
        \bnfpn{OperatorEq}
        \bnfsp
        \bnfpn{Expression}
    } \\
    \bnfprod{LValue}{
        \bnfpn{Pattern}
        \bnfor
        \bnfpn{MemberAccess}
    } \\
    \bnfprod{OperatorEq}{
        \bnfts{+=}\bnfor
        \bnfts{-=}\bnfor
        \bnfts{*=}\bnfor
        \bnfts{/=}\bnfor
        \bnfts{\%=}\bnfor
        \bnfts{**=}\bnfor
        \bnfts{//=}\bnfor
        \bnfts{<>=}\bnfor
    } \\
    \bnfmore{
        \bnfts{|=}\bnfor
        \bnfts{\&=}\bnfor
        \bnfts{\textasciicircum=}\bnfor
        \bnfts{<\textasciitilde=}\bnfor
        \bnfts{\textasciitilde>=}\bnfor
        \bnfts{<<=}\bnfor
        \bnfts{>>=}
    }
\end{bnf*}

Semantically, the assignment statement evaluates its value and then updates
the binding in the current context. Assignment in this way is the only way
a binding in a scope is modified (other operations typically generate a new scope
that extends from the previous, introducing new bindings). In particular, the
update of the binding updates it in the scope that it was defined, not in the
latest extending scope, therefore making it possible to update the value of a
variable defined outside of a block such that the modification persists to be
seen outside of that block.

\begin{prooftree}
    \AxiomC{$\Gamma \vdash y$}
    \AxiomC{$P = y$}
    \AxiomC{$a \defby P$}
    \LeftLabel{Assignment}
    \TrinaryInfC{$\Gamma\vdash P\ \op{=}\ y \Rightarrow \Gamma[a\mapsto y]$}
\end{prooftree}

For a member access assignment, this update simply happens not just within
a scope but within the value itself as well.

\begin{prooftree}
    \AxiomC{$\Gamma\vdash v$}
    \AxiomC{$\Gamma\vdash E = x$}
    \AxiomC{$\Gamma\vdash y$}
    \LeftLabel{Assignment}
    \TrinaryInfC{$\Gamma\vdash v\ \texttt{.[}\ E\ \texttt{]}\ \op{=}\ y \Rightarrow \Gamma[v\mapsto v[x\mapsto y]]$}
\end{prooftree}

Rather than defining the exact semantics of function or operator assignment,
instead understand those by syntax transformation, and use the same assignment
semantics as normal:

\begin{align*}
    \texttt{\$x:id \$fn:id= \$(\$a:expr)+} &\Rightarrow \texttt{\$x = \$f \$(\$a)+ \$x} \\ %
    \texttt{\$x:id \$op:op= \$a:expr} & \Rightarrow \texttt{\$x = \$x \$op \$a} \\ %
\end{align*}


\subsubsection{Conditionals}

The primary conditional statement in \Prose{} is the \kw{if} statement,
which appears as you might expect coming from nearly any other language.

\begin{bnf*}
    \bnfprod{SIf}{
        \bnfts{\kw{if}}
        \bnfsp
        \bnfpn{Expression}
        \bnfsp
        \bnfpn{Block}
        \bnfor
    } \\
    \bnfmore{
        \bnfts{\kw{if}}
        \bnfsp
        \bnfpn{Expression}
        \bnfsp
        \bnfpn{Block}
        \bnfsp
        \bnfts{\kw{else}}
        \bnfsp
        \bnfpn{SIf}
        \bnfor
    } \\
    \bnfmore{
        \bnfts{\kw{if}}
        \bnfsp
        \bnfpn{Expression}
        \bnfsp
        \bnfpn{Block}
        \bnfsp
        \bnfts{\kw{else}}
        \bnfsp
        \bnfpn{Block}
    }
\end{bnf*}

In the common case, the condition is a statement that evaluates to a Boolean
result, \kw{true} or \kw{false}, and control flows as you might expect.
If the result is not a Boolean, this is a runtime type error.

\begin{prooftree}
    \AxiomC{$\Gamma \vdash C = \kw{true}$}
    \AxiomC{$\Gamma \vdash T \Rightarrow \Phi$}
    \LeftLabel{If Else\textsuperscript{\kw{true}}}
    \BinaryInfC{$\Gamma \vdash \kw{if}\ C\ T\ \kw{else}\ F \Rightarrow \Phi$}
\end{prooftree}
\begin{prooftree}
    \AxiomC{$\Gamma \vdash C = \kw{false}$}
    \AxiomC{$\Gamma \vdash F \Rightarrow \Phi$}
    \LeftLabel{If Else\textsuperscript{\kw{false}}}
    \BinaryInfC{$\Gamma \vdash \kw{if}\ C\ T\ \kw{else}\ F \Rightarrow \Phi$}
\end{prooftree}
\begin{prooftree}
    \AxiomC{$\Gamma \vdash C = \kw{true}$}
    \AxiomC{$\Gamma \vdash T \Rightarrow \Phi$}
    \LeftLabel{If\textsuperscript{\kw{true}}}
    \BinaryInfC{$\Gamma \vdash \kw{if}\ C\ T \Rightarrow \Phi$}
\end{prooftree}
\begin{prooftree}
    \AxiomC{$\Gamma \vdash C = \kw{false}$}
    \LeftLabel{If\textsuperscript{\kw{false}}}
    \UnaryInfC{$\Gamma \vdash \kw{if}\ C\ T : \top$}
\end{prooftree}


\subsubsection{Match-Else}

The \kw{match}-\kw{else} expression unifies a value with its cases, and evaluates the
branch associated with the first unification that succeeds. This is much
like the \kw{match} statement of \Prose{}, but as all expressions are required
to evaluate to some value, the \kw{match}-\kw{else} expression requires a specific
\kw{else} clause to be run in the situation that no \kw{case} was selected.

Repeated here are the same grammar for Cases as seen in \S\ref{sec:prose-match}.

\begin{bnf*}
    \bnfprod{MatchElse}{
        \bnfts{\kw{match}}
        \bnfsp
        \bnfpn{Expression}
        \bnfsp
        \bnfpn{Cases}
        \bnfsp
        \bnfts{\kw{else}}
        \bnfsp
        \bnfpn{ElseCase}
    } \\
    \bnfprod{Cases}{
        \bnfpn{Case}
        \bnfsp
        \bnfpn{Cases}
        \bnfor
        \bnfpn{Case}
    } \\
    \bnfprod{Case}{
        \bnfts{\kw{case}}
        \bnfsp
        \bnfpn{Pattern}
        \bnfsp
        \bnfpn{CaseGuard}
        \bnfsp
        \bnfpn{CaseBody}
    } \\
    \bnfprod{CaseGuard}{
        \bnfts{\kw{if}}
        \bnfsp
        \bnfpn{Expression}
        \bnfor
        \bnfes
    } \\
    \bnfprod{CaseBody}{
        \bnfts{\kw{then}}
        \bnfsp
        \bnfpn{Expression}
        \bnfor
        \bnfpn{Block}
    } \\
    \bnfprod{ElseCase}{
        \bnfts{\kw{else}}
        \bnfsp
        \bnfpn{Binding}
        \bnfsp
        \bnfpn{CaseBody}
        \bnfor
    } \\
    \bnfmore{
        \bnfts{\kw{else}}
        \bnfsp
        \bnfpn{Wildcard}
        \bnfsp
        \bnfpn{CaseBody}
    } \\
\end{bnf*}

The \kw{match}-\kw{else} expression is evaluated in much the same way as the \kw{match}
statement of \Prose{} (\S\ref{sec:prose-match}). The only addition is the inclusion of
the \kw{else} case, and the fact that the result of evaluating the matched case is used
as the resulting value of the whole \kw{match}-\kw{else} expression.

\begin{prooftree}
    \AxiomC{$\Gamma \vdash E$}
    \AxiomC{$P = E$}
    \AxiomC{$\Gamma,\setof{a\given a \defby P} \vdash C = \kw{true}, B : \tau$}
    \LeftLabel{Match}
    \TrinaryInfC{$\Gamma \vdash \kw{match}\ E\ \kw{case}\ P\ \kw{if}\ C\ \kw{then}\ B : \tau$}
\end{prooftree}

\begin{prooftree}
    \def\extraVskip{3.5pt}
    \AxiomC{$\Gamma \vdash E$}
    \AxiomC{$P = E$}
    \AxiomC{$\Gamma,\setof{a\given a \defby P} \vdash C = \kw{false}$}
    \TrinaryInfC{$R_1 = \kw{case}\ P\ \kw{if}\ C\ \kw{then}\ B$}
    \AxiomC{$\Gamma \vdash \kw{match}\ E\ R_2 \cdots R_n : \tau$}
    \LeftLabel{Match\textsuperscript{\kw{false}}}
    \insertBetweenHyps{\hskip -12pt}
    \BinaryInfC{$\Gamma \vdash \kw{match}\ E\ R_1 R_2 \cdots R_n : \tau$}
\end{prooftree}

\vskip 0.5em

\begin{prooftree}
    \def\extraVskip{3.5pt}
    \AxiomC{$\Gamma \vdash E$}
    \AxiomC{$P \neq E$}
    \BinaryInfC{$R_1 = \kw{case}\ P\ \kw{if}\ C\ \kw{then}\ B$}
    \AxiomC{$\Gamma \vdash \kw{match}\ E\ R_2 \cdots R_n : \tau$}
    \LeftLabel{Match\textsuperscript{$\bot$}}
    \BinaryInfC{$\Gamma \vdash \kw{match}\ E\ R_1 R_2 \cdots R_n : \tau$}
\end{prooftree}

The omitted clauses of the \kw{case} statement are formalized by the same syntax
transformation as before, as well as the additional \kw{else} case:

\begin{align*}
    \texttt{\kw{case} \$p:pat} &\Rightarrow \texttt{\kw{case} \$p \kw{if} \kw{true}} \\ %
    \texttt{\kw{case} \kw{if} \$c:expr} &\Rightarrow \texttt{\kw{case} \_ \kw{if} \$c} \\ %
    \texttt{\kw{else} \_ \kw{then}} &\Rightarrow \texttt{\kw{case} \_ \kw{then}} \\ %
    \texttt{\kw{else} \$i:id \kw{then}} &\Rightarrow \texttt{\kw{case} \$i \kw{then}} \\ %
\end{align*}


\subsubsection{While}

The \kw{while} loop accepts a Boolean condition and repeats a block of code
until that condition evaluates to \kw{false}. \Trilogy{}'s \kw{while} loop is
very standard, as far as such loops go. There is no \texttt{do}\dots\texttt{while}
loop in \Trilogy{}.

\begin{bnf*}
    \bnfprod{While}{
        \bnfts{\kw{while}}
        \bnfsp
        \bnfpn{Expression}
        \bnfsp
        \bnfpn{Block}
    }
\end{bnf*}

Branching may occur in the body of a \kw{while} loop, leading to more iterations
than expected if not careful. However, since the \kw{while} loop provides no collection
mechanism, once those branches break out of the loop they will continue to execute
beyond the end of the loop.

If the expression in the condition of the \kw{while} loop does not evaluate to a
Boolean, it is a runtime type error.

\begin{prooftree}
    \AxiomC{$\Gamma\vdash C = \kw{true}$}
    \AxiomC{$\Gamma\vdash B \Rightarrow \Phi$}
    \AxiomC{$\Phi\vdash \kw{while}\ C\ B \Rightarrow\Sigma$}
    \LeftLabel{Loop}
    \TrinaryInfC{$\Gamma\vdash \kw{while}\ C\ B \Rightarrow \Sigma$}
\end{prooftree}

\begin{prooftree}
    \AxiomC{$\Gamma\vdash C = \kw{false}$}
    \LeftLabel{End Loop}
    \UnaryInfC{$\Gamma\vdash \kw{while}\ C\ B:\top$}
\end{prooftree}


\subsubsection{For}

The \kw{for} loop repeats a block of code for all solutions to a provided query.
This is more similar to a for-each loop than the three part conditional loop seen in
some imperative languages. No such conditional \kw{for} loop exists in \Trilogy{},
but similar semantics can be constructed using \kw{while}.

\begin{bnf*}
    \bnfprod{SFor}{
        \bnfts{\kw{for}}
        \bnfsp
        \bnfpn{Query}
        \bnfsp
        \bnfpn{Block}
        \bnfor
    } \\
    \bnfmore{
        \bnfts{\kw{for}}
        \bnfsp
        \bnfpn{Query}
        \bnfsp
        \bnfpn{Block}
        \bnfsp
        \bnfts{\kw{else}}
        \bnfsp
        \bnfpn{SFor}
        \bnfor
    } \\
    \bnfmore{
        \bnfts{\kw{for}}
        \bnfsp
        \bnfpn{Query}
        \bnfsp
        \bnfpn{Block}
        \bnfsp
        \bnfts{\kw{else}}
        \bnfsp
        \bnfpn{Block}
    }
\end{bnf*}

Notably different from other languages is that a \kw{for} loop may have an \kw{else} clause,
similar to an \kw{if} statement. This \kw{else} clause is triggered in the case that the
unification fails.

The query of the \kw{for} loop is performed only once, at the beginning of the loop.
Any mutation to the values as a result of the loop's execution will not affect the remaining
iterations.

\begin{prooftree}
    \AxiomC{$\Gamma\vdash U_{1\hdots n}$}
    \AxiomC{$Q = U_i$}
    \AxiomC{$\Gamma,U_1\vdash B \Rightarrow\Phi_1$}
    \noLine
    \UnaryInfC{$\Phi_1,U_2\vdash B \Rightarrow\Phi_2$}
    \noLine
    \UnaryInfC{$\vdots$}
    \noLine
    \UnaryInfC{$\Phi_{n-1},U_n\vdash B \Rightarrow\Sigma$}
    \LeftLabel{For}
    \TrinaryInfC{$\Gamma\vdash \kw{for}\ Q\ B\ \kw{else}\ F\Rightarrow \Sigma$}
\end{prooftree}

\begin{prooftree}
    \AxiomC{$\Gamma\vdash Q:\bot$}
    \AxiomC{$\Gamma\vdash F\Rightarrow\Phi$}
    \LeftLabel{For\textsuperscript{$\bot$}}
    \BinaryInfC{$\Gamma\vdash \kw{for}\ Q\ B\ \kw{else}\ F\Rightarrow \Phi$}
\end{prooftree}

A \kw{for} loop is distinct from the branching \kw{let} statement in how execution
continues after each branch has been evaluated. Where a \kw{let} statement would
result in multiple parallel executions with separate execution contexts, the
\kw{for} loop executes the block for each binding of the query in sequence,
all within one continuous execution context. A failed \kw{let} binding will cause
the current execution to fizzle, while a \kw{for} loop query that does not
have a solution simply causes the loop to be skipped, but execution does not end.


\subsubsection{Break and Continue}
\label{sec:break-continue}

The statements \kw{break} and \kw{continue} may be used in loops to control iteration.
The \kw{break} statement will exit the loop immediately, skipping all further iteration,
whereas the \kw{continue} statement will end the current iteration and move on to the
next, without executing any further code. This applies for both \kw{for} and \kw{while}
loops.

You might note that these keywords do not accept any label, nor does \Trilogy{} even
have labelled statements. That does not prevent us from breaking out of a parent loop
from a child loop, instead \Trilogy{} supports \emph{first class keywords}
(\S\ref{sec:first-class-keyword}) for this purpose.

\begin{bnf*}
    \bnfprod{SBreak}{\bnfts{\kw{break}}} \\
    \bnfprod{SContinue}{\bnfts{\kw{continue}}}
\end{bnf*}

A \kw{break} or \kw{continue} statement on its own does not evaluate to anything, so
their semantics are uninteresting.

\begin{figure}[H]
    \centering
    \parbox[t]{0.45\linewidth}{
        \begin{prooftree}
            \AxiomC{}
            \LeftLabel{Break}
            \UnaryInfC{$\Gamma \vdash \kw{break} : \bot$}
        \end{prooftree}
    }
    \parbox[t]{0.45\linewidth}{
        \begin{prooftree}
            \AxiomC{}
            \LeftLabel{Continue}
            \UnaryInfC{$\Gamma \vdash \kw{continue} : \bot$}
        \end{prooftree}
    }
\end{figure}

Though the statement ends the local execution, it changes the behaviour of the loop
or block it corresponds to, which picks up the execution before it fizzles completely.

The behaviour of binding the keyword to a loop is only of syntactic concern, so in the
semantics we assume that the break or continue in question has already been resolved to
the correct containing loop.

\begin{prooftree}
    \AxiomC{$\Gamma \vdash \kw{for}\ Q\ \block{M} : \top$}
    \LeftLabel{Break\textsuperscript{for}}
    \UnaryInfC{$\Gamma \vdash \kw{for}\ Q\ \block{M\ \op{;}\ \kw{break}} : \top$}
\end{prooftree}

\begin{prooftree}
    \AxiomC{$\Gamma \vdash \kw{while}\ E\ \block{M} : \top$}
    \LeftLabel{Break\textsuperscript{while}}
    \UnaryInfC{$\Gamma \vdash \kw{while}\ E\ \block{M\ \op{;}\ \kw{break}} : \top$}
\end{prooftree}

\begin{prooftree}
    \AxiomC{$\Gamma \vdash \block{M} : \top$}
    \LeftLabel{Continue}
    \UnaryInfC{$\Gamma \vdash \block{M\ \op{;}\ \kw{continue}} : \top$}
\end{prooftree}


\subsubsection{Return}

Similar to \kw{break} and \kw{continue}, the \kw{return} statement causes execution
to end and be picked up elsewhere. The \kw{return} statement may designate the value
that a procedure call evaluates to. This should be intuitive to any who have written
code before.

A \kw{return} statement with no value is assumed to return \kw{unit}, the same as if
the \kw{return} statement was omitted entirely and execution ran off the end of the
procedure.

\begin{bnf*}
    \bnfprod{SReturn}{
        \bnfts{\kw{return}}
        \bnfor
        \bnfts{\kw{return}}
        \bnfsp
        \bnfpn{Expression}
    }
\end{bnf*}

Semantically, the \kw{return} statement on its own is simple. Meanwhile, its meaning
to a procedure was already covered as part of the semantics of procedure definitions
(Proof~\ref{proof:return}).

\begin{prooftree}
    \AxiomC{$\Gamma\vdash N:\top$}
    \LeftLabel{Return}
    \UnaryInfC{$\Gamma\vdash\kw{return}\ N : \bot$}
\end{prooftree}


\subsubsection{End}

The \kw{end} keyword can be used in \Poetry{}, with much the same
behaviour as in \Prose{}: called without a value it triggers a fizzle,
with a value it ends the program using that value as the exit code.
See \S\ref{sec:prose-end} for slightly more detail, but not really.

\begin{bnf*}
    \bnfprod{End}{
        \bnfts{\kw{end}}
        \bnfor
        \bnfts{\kw{end}}
        \bnfsp
        \bnfpn{Expr}
    }
\end{bnf*}

In fact, the syntax and semantics are exactly the same. There is truly no
reason why this needs to be distinct from the statement form, but that might
just be a coincidence so it remains a distinct piece of syntax for no reason
other than formality.

\begin{prooftree}
    \AxiomC{}
    \LeftLabel{End}
    \UnaryInfC{$\Gamma\vdash\kw{end}\ : \bot$}
\end{prooftree}

\begin{prooftree}
    \AxiomC{$\Gamma\vdash N:\top$}
    \LeftLabel{End}
    \UnaryInfC{$\Gamma\vdash\kw{end}\ N : \bot$}
\end{prooftree}


\subsubsection{Yield and Resume}

The \kw{yield} statement, however simple in appearance, is one of the most
semantically complex statements in all of \Trilogy{}, as it provides direct
access to the effect system that underpins the entirety of \Trilogy{}'s control
flow structure.

The \kw{resume} statement is the response to a \kw{yield}, providing the value
that the \kw{yield} keyword evaluates to (when used in an expression). The
\kw{resume} statement is available only in an effect handler, and is syntactically
invalid when not within an effect handler. In the case of nested effect handlers,
the \kw{resume} keyword refers to the innermost handler's continuation. This is
much like \kw{break} or \kw{continue} on loops.

The \kw{resume} keyword is a first-class keyword, and so follows the now familiar
rules of first-class keywords. When used in value position, the \kw{resume} keyword
becomes a unary function.

In either case, the \kw{resume} keyword can be considered to represent the delimited
continuation of the \kw{yield} statement that is performing the current effect, until the
expresson to which the handler for the current \kw{resume} is attached.

While the \kw{yield} statement requires a value (describing the intended effect), the
\kw{resume} statement may be used without a value, in which case the \kw{yield} that
is performing the effect evaluates to \kw{unit}.

\begin{bnf*}
    \bnfprod{SYield}{
        \bnfts{\kw{yield}}
        \bnfsp
        \bnfpn{Expression}
    } \\
    \bnfprod{SResume}{
        \bnfts{\kw{resume}}
        \bnfor
        \bnfts{\kw{resume}}
        \bnfsp
        \bnfpn{Expression}
    }
\end{bnf*}

While the syntax of yield and resume are easy to specify at this time, the semantics
will be left for later, when we look at the effect system as a whole in \S\ref{sec:effects}.


\subsubsection{Procedure Call}

Much like a procedure call in \Prose{}, procedures may be called in \Poetry{}.
Though this sort of breaks the ``pureness'' of \Poetry{}, I think it's also
sort of justifiable if you consider the procedure call to actually be syntax
sugar for a \kw{yield} that corresponds to a handler that calls that procedure
and resumes with the result. Assume that handler is applied automatically by
the procedure's definition.

There's no need to define the syntax of a procedure call in this situation,
as it is exactly the same as the syntax of a procedure call in \Prose{}
(\S\ref{sec:prose-procedure-call}). The semantics however are slightly
different when called in an expression context, adding the relevance of
the procedure call's returned value. Honestly, the definition earlier
could have just had this type to begin with, and ignored it.

\begin{prooftree}
    \AxiomC{$\Gamma\vdash p : \texttt{!}(x_{1\hdots n})\rightarrow\tau$}
    \AxiomC{$a_i = x_i$}
    \LeftLabel{Procedure Call}
    \BinaryInfC{$\Gamma\vdash p\texttt{!}(a_{1\hdots n}):\tau$}
\end{prooftree}


\subsubsection{Assert}

The last statement, of very low importance, is the \kw{assert} statement.
In a real program the \kw{assert} statement is not likely to be used, as
it is truly only a convenience when writing tests, further discussed in
\S\ref{sec:tests}. Technically you could use an \kw{assert} statement
in a real program, but its dramatic behaviour of triggering an assertion error
and immediately ending the program in failure makes that typically undesirable.

\begin{bnf*}
    \bnfprod{Assert}{
        \bnfts{\kw{assert}}
        \bnfsp
        \bnfpn{Expression}
        \bnfor
    } \\
    \bnfmore{
        \bnfts{\kw{assert}}
        \bnfsp
        \bnfpn{Expression}
        \bnfsp
        \bnfts{\kw{as}}
        \bnfsp
        \bnfpn{Expression}
    }
\end{bnf*}

The \kw{assert} statement is optionally provided an expression (before \kw{as})
which is only evaluated in case of failure to be included in an error message.

The other (or only) expression is the condition which is used to determine
whether to end the program. If it evaluates to \kw{true}, the assertion was
a success and execution continues.

When \kw{false}, the message expression (if any) is evaluated. The message
expression may not call any continuations, and may not fail for any reason.
If either of these requirements is violated, or the evaluation of the
message expression takes longer than one second, it is abandoned. Any further
runtime errors encountered during this expression are ignored. The program
then ends with an assertion error.

\begin{prooftree}
    \AxiomC{$\Gamma\vdash C = \kw{true}$}
    \LeftLabel{Assertion}
    \UnaryInfC{$\Gamma\vdash \kw{assert}\ M\ \kw{as}\ C : \top$}
\end{prooftree}

\begin{prooftree}
    \AxiomC{$\Gamma\vdash C = \kw{false}$}
    \LeftLabel{Assertion}
    \UnaryInfC{$\Gamma\vdash \kw{assert}\ M\ \kw{as}\ C : \bot$}
\end{prooftree}

