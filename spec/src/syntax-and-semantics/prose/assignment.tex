\subsubsection{Assignment}

Variables (specifically, mutable bindings) as well as members of containers
may be assigned to. Syntactically, the left hand side of the assignment
statement must either:

\begin{itemize}
    \item Be able to be parsed both as a pattern (\S\ref{sec:patterns}) and an expression (\S\ref{sec:expressions})
    \item End with a member access operation (\S\ref{sec:member-access})
\end{itemize}

As that is an inconvenience to write out grammatically, and also an inconvenience
to have to implement, note that the syntax for expressions is superset of that
of patterns, so the left hand side of an assignment is parsed as an expression, and
the validity of this expression as an assignment target is detected separately.

Function assignment is a special syntax for applying a function to a value and
then simultaneously reassigning that value with the result of the function; a
generalization of operator assignment (e.g. \op{+=}) to any function\footnote{A
generalization which I cannot take credit for as I was inspired to it by
\href{https://github.com/betaveros/noulith}{Noulith}, the same project
which inspired me to attempt \Trilogy{} at all.}. In this form of assignment,
the right side is a list of arguments to apply to the function being used as
the assignment operator. The left hand side is then applied as the last argument
to the function before being re-assigned.

In terms of precedence, these arguments are parsed as if in any other function
application, meaning that most complex expressions will require explicit
parenthesization. Additionally, as expressions do not typically span multiple lines,
these arguments must also fit on one line or do strange things to ensure a specific
parsing. It is generally not recommended to use this syntax for any heavy lifting, it
is merely intended as syntax sugar for quick one-liners.

Regular operator assignment (\op{+=}, \op{-=}, etc.) is supported as well, in the
intuitive way: the left hand identifer is used as the left hand side of the operator.

\begin{bnf*}
    \bnfprod{SAssignment}{
        \bnfpn{Assignment}
        \bnfor
        \bnfpn{FunctionAssignment}
        \bnfor
    } \\
    \bnfmore{
        \bnfpn{OperatorAssignment}
    } \\
    \bnfprod{Assignment}{
        \bnfpn{Expression}
        \bnfsp
        \bnfts{\op{=}}
        \bnfsp
        \bnfpn{Expression}
    } \\
    \bnfprod{FunctionAssignment}{
        \bnfpn{Expression}
        \bnfsp
        \bnfpn{IdentifierEq}
        \bnfsp
        \bnfpn{ApplicationList}
    } \\
    \bnfprod{OperatorAssignment}{
        \bnfpn{Expression}
        \bnfsp
        \bnfpn{OperatorEq}
        \bnfsp
        \bnfpn{Expression}
    } \\
    \bnfprod{OperatorEq}{
        \bnfts{+=}\bnfor
        \bnfts{-=}\bnfor
        \bnfts{*=}\bnfor
        \bnfts{/=}\bnfor
        \bnfts{\%=}\bnfor
        \bnfts{**=}\bnfor
        \bnfts{//=}\bnfor
    } \\
    \bnfmore{
        \bnfts{<>=}\bnfor
        \bnfts{:=}\bnfor
        \bnfts{|=}\bnfor
        \bnfts{\&=}\bnfor
        \bnfts{\textasciicircum=}\bnfor
        \bnfts{<\textasciitilde=}\bnfor
        \bnfts{\textasciitilde>=}\bnfor
    } \\
    \bnfmore{
        \bnfts{<<=}\bnfor
        \bnfts{>>=}\bnfor
        \bnfts{.=}
    }
\end{bnf*}

Semantically, the assignment statement evaluates its value and then updates
the binding in the current context. Assignment in this way is the only way
a binding in a scope is modified (other operations typically generate a new scope
that extends from the previous, introducing new bindings). In particular, the
update of the binding updates it in the scope that it was defined, not in the
latest extending scope, therefore making it possible to update the value of a
variable defined outside of a block such that the modification persists to be
seen outside of that block.

\begin{prooftree}
    \AxiomC{$\Gamma \vdash y$}
    \AxiomC{$P = y$}
    \AxiomC{$a \defby P$}
    \LeftLabel{Assignment}
    \TrinaryInfC{$\Gamma\vdash P\ \op{=}\ y \Rightarrow \Gamma[a\mapsto y]$}
\end{prooftree}

For a member access assignment, this update simply happens not just within
a scope but within the value itself as well.

\begin{prooftree}
    \AxiomC{$\Gamma\vdash v$}
    \AxiomC{$\Gamma\vdash E = x$}
    \AxiomC{$\Gamma\vdash y$}
    \LeftLabel{Assignment}
    \TrinaryInfC{$\Gamma\vdash v\ \texttt{.}\ E\ \op{=}\ y \Rightarrow \Gamma[v\mapsto v[x\mapsto y]]$}
\end{prooftree}

Rather than defining the exact semantics of function or operator assignment,
instead understand those by syntax transformation, and use the same assignment
semantics as normal:

\begin{align*}
    \texttt{\$x:expr \$fn:expr= \$(\$a:expr)+} &\Rightarrow \texttt{\$x = \$fn \$(\$a)+ \$x} \\ %
    \texttt{\$x:expr \$op:op= \$a:expr} & \Rightarrow \texttt{\$x = \$x \$op \$a} \\ %
\end{align*}
