\subsubsection{Yield and Resume}

The \kw{yield} statement, however simple in appearance, is one of the most
semantically ambiguous statements in all of \Trilogy{}, as it provides direct
access to the effect system that underpins the entirety of \Trilogy{}'s control
flow structure.

The \kw{resume} statement is the response to a \kw{yield}, providing the value
that the \kw{yield} keyword evaluates to (when used in an expression). The
\kw{resume} statement is available only in an effect handler, and is syntactically
invalid when not within an effect handler. In the case of nested effect handlers,
the \kw{resume} keyword refers to the innermost handler's continuation. This is
much like \kw{break} or \kw{continue} on loops.

The \kw{resume} keyword is a first-class keyword, and so follows the now familiar
rules of first-class keywords. When used in value position, the \kw{resume} keyword
becomes a unary function.

In either case, the \kw{resume} keyword can be considered to represent the delimited
continuation of the \kw{yield} statement that is performing the current effect, until the
expresson to which the handler for the current \kw{resume} is attached.

While the \kw{yield} statement requires a value (describing the intended effect), the
\kw{resume} statement may be used without a value, in which case the \kw{yield} that
is performing the effect evaluates to \kw{unit}.

\begin{bnf*}
    \bnfprod{SYield}{
        \bnfts{\kw{yield}}
        \bnfsp
        \bnfpn{Expression}
    } \\
    \bnfprod{SResume}{
        \bnfts{\kw{resume}}
        \bnfor
        \bnfts{\kw{resume}}
        \bnfsp
        \bnfpn{Expression}
    }
\end{bnf*}

On its own, the \kw{yield} keyword is used to perform an ``effect'', which is
handled by an upstream handler, designated by \kw{when} (which will be seen later).
The effect in this situation is simply any value which is unified with the patterns
in the handlers to find the first one which matches, running its body in order to
handle the effect and determine how execution should proceed.

Seen here, the \kw{yield} keyword is used in statement position, so its resulting
value, if any, is discarded (\kw{yield} is more often used in expression position).
However, due to an effect handler being possibly useful for side effects, \kw{yield}
may be used in statement position as well.

\begin{prooftree}
    \AxiomC{$\Gamma\vdash E$}
    \AxiomC{$\Gamma\vdash \text{when}\ P\ B$}
    \UnaryInfC{$\Gamma\vdash B$}
    \AxiomC{$P = E$}
    \TrinaryInfC{$\Gamma\vdash \kw{yield}\ E : \top$}
\end{prooftree}
