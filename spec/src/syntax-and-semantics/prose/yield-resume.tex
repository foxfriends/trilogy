\subsubsection{Yield and Resume}

The \kw{yield} statement, however simple in appearance, is one of the most
semantically ambiguous statements in all of \Trilogy{}, as it provides direct
access to the effect system that underpins the entirety of \Trilogy{}'s control
flow structure.

The \kw{resume} statement is a bit of a special case in that it is
treated both as a keyword and a value. The \kw{resume} statement is available only
in an effect handler, and is syntactically invalid when not within an effect handler.
In the case of nested effect handlers, the \kw{resume} keyword refers to the innermost
handler's continuation.

% TODO: use first-class keyword for break and continue, for consistency. No labels!
% Would additionally make it possible to pass break and continue to functions, which
% would be an interesting control flow experiment consistent with how break and continue
% would work were they user-built as effects and handlers.

\begin{bnf*}
    \bnfprod{SYield}{
        \bnfts{\kw{yield}}
        \bnfsp
        \bnfpn{Expression}
    }
\end{bnf*}

On its own, the \kw{yield} keyword is used to perform an ``effect'', which is
handled by an upstream handler, designated by \kw{when} (which will be seen later).
The effect in this situation is simply any value which is unified with the patterns
in the handlers to find the first one which matches, running its body in order to
handle the effect and determine how execution should proceed.

Seen here, the \kw{yield} keyword is used in statement position, so its resulting
value, if any, is discarded (\kw{yield} is more often used in expression position).
However, due to an effect handler being possibly useful for side effects, \kw{yield}
may be used in statement position as well.

\begin{prooftree}
    \AxiomC{$\Gamma\vdash E$}
    \AxiomC{$\Gamma\vdash \text{when}\ P\ B$}
    \UnaryInfC{$\Gamma\vdash B$}
    \AxiomC{$P = E$}
    \TrinaryInfC{$\Gamma\vdash \kw{yield}\ E : \top$}
\end{prooftree}
