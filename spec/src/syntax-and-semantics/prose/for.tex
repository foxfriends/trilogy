\subsubsection{For}

The \kw{for} loop repeats a block of code for all solutions to a provided query.
This is more similar to a for-each loop than the three part conditional loop seen in
some imperative languages. No such conditional \kw{for} loop exists in \Trilogy{},
but similar semantics can be constructed using \kw{while}.

\begin{bnf*}
    \bnfprod{SFor}{
        \bnfts{\kw{for}}
        \bnfsp
        \bnfpn{Query}
        \bnfsp
        \bnfpn{Block}
        \bnfor
    } \\
    \bnfmore{
        \bnfts{\kw{for}}
        \bnfsp
        \bnfpn{Query}
        \bnfsp
        \bnfpn{Block}
        \bnfsp
        \bnfts{\kw{else}}
        \bnfsp
        \bnfpn{SFor}
        \bnfor
    } \\
    \bnfmore{
        \bnfts{\kw{for}}
        \bnfsp
        \bnfpn{Query}
        \bnfsp
        \bnfpn{Block}
        \bnfsp
        \bnfts{\kw{else}}
        \bnfsp
        \bnfpn{Block}
    }
\end{bnf*}

Notably different from other languages is that a \kw{for} loop may have an \kw{else} clause,
similar to an \kw{if} statement. This \kw{else} clause is triggered in the case that the
unification fails.

The query of the \kw{for} loop is performed only once, at the beginning of the loop.
Any mutation to the values as a result of the loop's execution will not affect the remaining
iterations.

\begin{prooftree}
    \AxiomC{$\Gamma\vdash U$}
    \AxiomC{$\Gamma\vdash \forall\setof{U \given Q = U}.\ B\Rightarrow\Sigma$}
    \LeftLabel{For}
    \BinaryInfC{$\Gamma\vdash \kw{for}\ Q\ B\ \kw{else}\ F\Rightarrow \Sigma$}
\end{prooftree}

\begin{prooftree}
    \AxiomC{$\Gamma, U_1\vdash B \Rightarrow\Phi$}
    \AxiomC{$\Phi\vdash \forall\setof{U_2, \cdots, U_n}.\ B\Rightarrow\Sigma$}
    \LeftLabel{For}
    \BinaryInfC{$\Gamma\vdash \forall\setof{U_1, U_2, \cdots, U_n}.\ B\Rightarrow\Sigma$}
\end{prooftree}

\begin{prooftree}
    \AxiomC{$\Gamma\vdash Q:\bot$}
    \AxiomC{$\Gamma\vdash F\Rightarrow\Phi$}
    \LeftLabel{For\textsuperscript{$\bot$}}
    \BinaryInfC{$\Gamma\vdash \kw{for}\ Q\ B\ \kw{else}\ F\Rightarrow \Phi$}
\end{prooftree}

A \kw{for} loop is distinct from the branching \kw{let} statement in how execution
continues after each branch has been evaluated. Where a \kw{let} statement would
result in multiple parallel executions with separate execution contexts, the
\kw{for} loop executes the block for each binding of the query in sequence,
all within one continuous execution context. A failed \kw{let} binding will cause
the current execution to fizzle, while a \kw{for} loop query that does not
have a solution simply causes the loop to be skipped, but execution does not end.
