\subsection{Effect System}
\label{sec:effects}

By no means is \Trilogy{} the first language to implement an effect system;
much of what is presented here is inspired by, and (loosely) cross checked against,
existing research on effects\cite{eff}.

Sadly, mainstream languages featuring a full-fledged first class effect system
are\dots{} none? At least none I have ever come across. Most languages with an
effect system are experimental or research based, and appear far from practical
to me. While I can't really claim that \Trilogy{} is much better on being ``practical''
and not ``experimental'', it does aim to take a more user and productivity first
approach to implementing effects than other research languages.

\Trilogy{} includes effects as its core advanced control flow mechanism because
I believe it to be the most concretely conceptualized of the equivalent competing
models (namely monads and delimited control\cite{effmondel}).

Now, with a complete lack of true expertise, I attempt to specify the effect
system of \Trilogy{}, beginning by finally addressing the syntax of effect
handlers.

\begin{bnf*}
    \bnfprod{EffectHandlers}{
        \bnfpn{EffectHandler}
        \bnfsp
        \bnfpn{EffectHandlers}
        \bnfor
    } \\
    \bnfmore{
        \bnfpn{EffectHandler}
        \bnfor
        \bnfts{\kw{else}}
        \bnfsp
        \bnfpn{Handler}
    } \\
    \bnfprod{EffectHandler}{
        \bnfts{\kw{when}}
        \bnfsp
        \bnfpn{Pattern}
        \bnfsp
        \bnfpn{WhenGuard}
        \bnfsp
        \bnfpn{Handler}
        \bnfor
    } \\
    \bnfprod{WhenGuard}{
        \bnfts{\kw{if}}
        \bnfsp
        \bnfpn{Expression}
        \bnfor
        \bnfes
    } \\
    \bnfprod{Handler}{
        \bnfts{\kw{resume}}
        \bnfsp
        \bnfpn{Expression}
        \bnfor
        \bnfts{\kw{cancel}}
        \bnfsp
        \bnfpn{Expression}
        \bnfor
    } \\
    \bnfmore{
        \bnfts{\kw{then}}
        \bnfsp
        \bnfpn{Block}
        \bnfor
        \bnfts{\kw{yield}}
        \bnfsp
        \bnfpn{Expression}
        \bnfor
        \bnfts{\kw{yield}}
    }
\end{bnf*}

Effect handlers are defined with \kw{when} and are either written in functional
or imperative style. At the end of a chain of handlers, an \kw{else} clause may be
included to describe what to do in the case of no handler matching; if omitted,
the default is as if it were written \kw{else yield}.

Handlers install a contextual handler around a piece of code. When an effect is
yielded, the ``nearest'' handler with a matching pattern and guard will be used
to handle that effect. The nearest handler in a chain of them on one expression
is the first handler in the chain, meanwhile the nearest handler when nested is
the one least scopes away from the point of the \kw{yield}.

The three keywords \kw{resume}, \kw{cancel}, and \kw{then} describe how the
handler responds to the effect. The \kw{yield} keyword may also be used on a
handler to denote that no handling is done, and the effect is simply re-yielded
to a handler in the parent scope (a re-yielded effect never gets handled by another
handler on the same chain).

A \kw{cancel} handler will not respond to the yielder, instead completing the entire
expression or statement on which the handler was attached, as if the \kw{with} keyword
evaluated to the value that \kw{cancel} was supplied.

A \kw{resume} handler will respond with the evaluation of the expression following
the keyword. The \kw{yield} that performed the effect will evaluate to that value.
Once the resumed continuation complets, the \kw{resume} handler will evaluate to
that final value, as if it were written \kw{then cancel resume}.

The \kw{then} handler neither cancels nor resumes, but allows the programmer to fully
describe exactly the control flow that is desired. When evaluation reaches the end of
the block of the \kw{then} handler, execution simply fizzles, the \kw{cancel}
or \kw{become} must be written explicitly.

In each case, the first-class keywords \kw{resume}, \kw{cancel} and \kw{become} are
bound to the current effect handler, and may be used as part of the handler body.

The \kw{resume} in this situation is a bounded continuation that ranges from the
\kw{yield} expression that is performing the effect until the \kw{with} keyword
of the expression onto which the handler is attached. When applied, this continuation
runs until it reaches its bound and then the value that the last expression evaluates
to is used as the value of the \kw{resume} application.

Only when the effect handler's body is encounters the \kw{cancel} keyword does the original
expression finally evaluate, specifically to the value that was passed to \kw{cancel}.
If a handler runs off its scope without using \kw{cancel} (or \kw{become}), execution
fizzles instead. \kw{cancel} itself is also a first class keyword that can be captured
to get the unbounded continuation following the end of the original expression to be
passed along to other code, if so desired.

\kw{become} is just syntax sugar for a combined \kw{cancel resume}.

As you can see, the bare \kw{then} form is the most controlled, while the \kw{yield},
\kw{resume} and \kw{cancel} forms are just special cases, the following syntax
transformation formalizes them so that we only need to specify the semantics of
\kw{then}. Included as well is the transformation for the \kw{else} clause.

\begin{align*}
    & \texttt{\kw{when} \$p:pat \kw{resume} \$e:expr} \Rightarrow \texttt{\kw{when} \$p \kw{then} \{ \kw{cancel} \kw{resume} \$e \}} \\ %
    & \texttt{\kw{when} \$p:pat \kw{cancel} \$e:expr} \Rightarrow \texttt{\kw{when} \$p \kw{then} \{ \kw{cancel} \$e \}} \\ %
    & \texttt{\kw{when} \$p:pat \kw{yield}} \Rightarrow \texttt{\kw{when} \$p \kw{then} \{ \kw{cancel} \kw{resume} \kw{yield} \$p \}} \\ %
    & \texttt{\kw{else} \$n:ident \$h:handler} \Rightarrow \texttt{\kw{when} \$n \$h} \\ %
    & \texttt{\kw{else} \$h:handler} \Rightarrow \texttt{\kw{when} \_ \$h} \\ %
\end{align*}

Beginning with the \kw{when} itself, this keyword adds an effect handler to the scope
of the expression or statement it is attached to. That expression or statement now
evaluates to either its original result or the cancellation value of the handler that
matches a yielded effect.

\begin{prooftree}
    \AxiomC{$\Gamma,P \triangleleft B \vdash E : \tau$}
    \AxiomC{$\Gamma,\setof{a\given a \defby P} \vdash B \triangleright \tau$}
    \LeftLabel{When}
    \BinaryInfC{$\Gamma\vdash E\ \kw{when}\ P\ \kw{then}\ B : \tau$}
\end{prooftree}

The \kw{cancel} keyword provides that cancellation value.

\begin{prooftree}
    \AxiomC{$\Gamma\vdash E : \tau$}
    \LeftLabel{Cancel}
    \UnaryInfC{$\Gamma\vdash \kw{cancel}\ E\triangleright\tau$}
\end{prooftree}

Within the expression, the \kw{yield} keyword yields a value as an effect to be
performed by the first matching handler.

\begin{prooftree}
    \AxiomC{$P\triangleleft B \in \Gamma$}
    \AxiomC{$P = \eta$}
    \AxiomC{$\Gamma,\setof{a\given a \defby P}\vdash B\ \triangledown\ \tau$}
    \AxiomC{$\Gamma:>\Phi$}
    \LeftLabel{Yield}
    \QuaternaryInfC{$\Phi\vdash\vartriangle \eta\ : \tau$}
\end{prooftree}

The behaviour and final value of the $\vartriangle\eta$ form depends on the
behaviour of the handler that $\eta$ is handled by. When the handler uses
the keyword \kw{resume}, it evaluates to the value that was \kw{resume} was
applied to.

\begin{prooftree}
    \AxiomC{$\Gamma\vdash E : \tau$}
    \LeftLabel{Resume}
    \UnaryInfC{$\Gamma\vdash \kw{resume}\ E\ \triangledown\ \tau$}
\end{prooftree}

Though all of these keywords can be used in both expression and statement
position, the semantics are the same, so won't be repeated; the only difference
is that in statement position, all the values are lost and the effect system
acts as pure control flow.

Finally, \kw{become} is just the composition of \kw{cancel} and \kw{resume},
and is defined as:

\begin{align*}
    \texttt{\kw{become}} & \Rightarrow \texttt{(\kw{cancel} << \kw{resume})}
\end{align*}
