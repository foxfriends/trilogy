\subsection{Effect System}
\label{sec:effects}

By no means is \Trilogy{} the first language to implement an effect system;
much of what is presented here is inspired by, and cross checked against,
existing research on effects\cite{eff}.

Sadly, mainstream languages featuring a full-fledged first class effect system
are\dots{} none? At least none I have ever come across. Most languages with an
effect system are experimental or research based, and appear far from practical
to me. While I can't really claim that \Trilogy{} is much better on being ``practical''
and not ``experimental'', it does aim to take a more user and productivity first
approach to implementing effects than other research languages.

\Trilogy{} includes effects as its core advanced control flow mechanism because
I believe it to be the most concretely conceptualized of the equivalent competing
models: monads and delimited control\cite{effmondel}.

Now, with a complete lack of true expertise, I attempt to specify the effect
system of \Trilogy{}, beginning by finally addressing the syntax of effect
handlers.

\begin{bnf*}
    \bnfprod{EffectHandlers}{
        \bnfpn{EffectHandler}
        \bnfsp
        \bnfpn{EffectHandlers}
        \bnfor
        \bnfpn{EffectHandler}
    } \\
    \bnfprod{EffectHandler}{
        \bnfts{\kw{when}}
        \bnfsp
        \bnfpn{Pattern}
        \bnfsp
        \bnfts{\kw{resume}}
        \bnfsp
        \bnfpn{Expression}
        \bnfor
    } \\
    \bnfmore{
        \bnfts{\kw{when}}
        \bnfsp
        \bnfpn{Pattern}
        \bnfsp
        \bnfts{\kw{resume}}
        \bnfsp
        \bnfpn{Block}
        \bnfor
    } \\
    \bnfmore{
        \bnfts{\kw{when}}
        \bnfsp
        \bnfpn{Pattern}
        \bnfsp
        \bnfts{\kw{cancel}}
        \bnfsp
        \bnfpn{Expression}
        \bnfor
    } \\
    \bnfmore{
        \bnfts{\kw{when}}
        \bnfsp
        \bnfpn{Pattern}
        \bnfsp
        \bnfts{\kw{cancel}}
        \bnfsp
        \bnfpn{Block}
        \bnfor
    } \\
    \bnfmore{
        \bnfts{\kw{when}}
        \bnfsp
        \bnfpn{Pattern}
        \bnfsp
        \bnfts{\kw{invert}}
        \bnfsp
        \bnfpn{Expression}
        \bnfor
    } \\
    \bnfmore{
        \bnfts{\kw{when}}
        \bnfsp
        \bnfpn{Pattern}
        \bnfsp
        \bnfts{\kw{invert}}
        \bnfsp
        \bnfpn{Block}
        \bnfor
    } \\
    \bnfmore{
        \bnfts{\kw{given}}
        \bnfsp
        \bnfpn{RuleHead}
        \bnfor
    } \\
    \bnfmore{
        \bnfts{\kw{given}}
        \bnfsp
        \bnfpn{RuleHead}
        \bnfsp
        \bnfts{\op{<-}}
        \bnfsp
        \bnfpn{RuleBody}
    }
\end{bnf*}

Effect handlers come in two varieties: the \kw{when} form for functional
or imperative effect handlers; and the \kw{given} form for dynamic rules,
which take the place of both closures and predicate-level effects in \Law{}
(\Poetry{}'s \kw{yield} expression may be used in expressions in \Law{}
in the conventional effect system way).

Further broken down, the \kw{when} form allows the body to be either an expression
or a block. Though not encoded in the grammar at all, note that the block
option is only permitted in procedures. Encountered in a function, the block
will be reported as a syntax error. For specification and parsing purposes, however
it is easier to leave it like it is; the grammar will be unambiguous either way,
it's mostly just for logical consistency that this is forbidden.

It is worth noting that dynamic rules are not necessarily here to stay, as
they may be a hack given my relative lack of experience with both practical
logic programming and effect systems. In particular, there are ways to both
implement both higher-order predicates\cite{hologic} as well as effect
handlers in logic languages\cite{prologeffects}. Most likely they aren't really
meant to be part of the effect system at all and I have just smashed its syntax
into the effect handler syntax because it was convenient. This constrained
version of \texttt{assertz/1} is likely a good idea though, so it will make
it into the first version of \Trilogy{}.

In fact, to get it out of the way, the semantics of the \kw{given} clause
will be discussed first, as they are basically exactly like rule definitions,
only limited to the scope of one expression or block rather than the entire
program.

\begin{prooftree}
    \AxiomC{$\Gamma, R(P_{1\hdots n}) \Rightarrow \Phi\vdash E : \tau$}
    \LeftLabel{Given}
    \UnaryInfC{$E\ \kw{given}\ R\ \texttt{(}\ P_{1\hdots n}\ \texttt{)} : \tau$}
\end{prooftree}

\begin{prooftree}
    \def\extraVskip{3.5pt}
    \AxiomC{$A_i = P_i$}
    \AxiomC{$\Gamma,\setof{a \given a \defby P_i}\vdash B \Rightarrow \top$}
    \BinaryInfC{$\Gamma,R(P_{1\hdots n}) \Rightarrow \Phi \vdash E : \tau$}
    \LeftLabel{Given}
    \UnaryInfC{$E\ \kw{given}\ R\ \texttt{(}\ P_{1\hdots n}\ \texttt{)}\ \op{<-}\ B : \tau$}
\end{prooftree}

Effects are similar in that they install a contextual handler around a piece
of code. When an effect is yielded, the ``nearest'' handler with a matching pattern
will be used to handle that effect. The nearest handler in a chain of them on one
expression is the first handler in the chain, meanwhile the nearest handler when
nested is the one least scopes upwards from the point of the \kw{yield}.

The three keywords \kw{resume}, \kw{cancel}, and \kw{invert} describe how the
handler responds to the effect.

A \kw{cancel} handler will not respond, instead aborting the entire expression
or statement on which the handler was attached. The expression following the
\kw{cancel} is used as the evaluated value of the expression; in the case of a
block, the value is \kw{unit}.

A \kw{resume} handler will respond with the evaluation of the expression following
the keyword. The \kw{yield} that performed the effect will evaluate to that value.
If a block is used instead of an expression, the \kw{yield} evaluates to \kw{unit}.

The \kw{invert} handler is the most powerful, and is the one that binds the
first-class keyword \kw{resume} to the current effect handler. The block or expression
following the \kw{invert} keyword is run in response to the effect, and may use \kw{resume} to
respond to the \kw{yield}, possibly more than one time.

The \kw{resume} in this situation is a bounded continuation that ranges from the
\kw{yield} expression that is performing the effect until the expression onto
which the handler is attached. When applied, this continuation runs until it reaches
its bound and then the value that the last expression evaluates to is used as the
value of the \kw{resume} application.

Only when the effect handler's body is finished executing does the original
expression finally evaluate. In the case that the body is an expression, the
final value of that expression is used. In the case of a block, the \kw{cancel}
keyword is used to denote the value of that evaluation. If the block runs off
the end without \kw{cancel}, the value is \kw{unit}. Of course, \kw{cancel} itself
is also a first class keyword that can be captured to get the unbounded continuation
following the end of the original expression to be passed along to other code, if
so desired.

As you can see, the \kw{invert} form holds the most power, while the \kw{resume}
and \kw{cancel} forms are just special cases, the following syntax transformation
formalizes them so that we only need to specify the semantics of \kw{invert}.

\begin{align*}
    & \texttt{\kw{when} \$p:pat \kw{resume} \$e:expr} \Rightarrow \texttt{\kw{when} \$p \kw{invert} \{ \kw{cancel} \kw{resume} \$e \}} \\ %
    & \texttt{\kw{when} \$p:pat \kw{resume} \$block} \Rightarrow \texttt{\kw{when} \$p \kw{invert} \{ \$b; \kw{cancel} \kw{resume} \}} \\ %
    & \texttt{\kw{when} \$p:pat \kw{cancel} \$e:expr} \Rightarrow \texttt{\kw{when} \$p \kw{invert} \{ \kw{cancel} \$e \}} \\ %
    & \texttt{\kw{when} \$p:pat \kw{cancel} \$b:block} \Rightarrow \texttt{\kw{when} \$p \kw{invert} \{ \$b; \kw{cancel} \}} \\ %
\end{align*}

Beginning with the \kw{when} itself, this keyword adds an effect handler to the scope
of the expression or statement it is attached to. That expression or statement now
evaluates to either its original result or the cancellation value of the handler that
matches a yielded effect.

\begin{prooftree}
    \AxiomC{$\Gamma,P \triangleleft B \vdash E : \tau$}
    \AxiomC{$\Gamma,\setof{a\given a \defby P} \vdash B \triangleright \tau$}
    \LeftLabel{When}
    \BinaryInfC{$\Gamma\vdash E\ \kw{when}\ P\ \kw{invert}\ B : \tau$}
\end{prooftree}

The \kw{cancel} keyword provides that cancellation value.

\begin{prooftree}
    \AxiomC{$\Gamma\vdash E : \tau$}
    \LeftLabel{Cancel}
    \UnaryInfC{$\Gamma\vdash \kw{cancel}\ E\triangleright\tau$}
\end{prooftree}

Within the expression, the \kw{yield} keyword yields a value as an effect to be
performed by the first matching handler.

\begin{prooftree}
    \AxiomC{$P\triangleleft B \in \Gamma$}
    \AxiomC{$P = \eta$}
    \AxiomC{$\Gamma,\setof{a\given a \defby P}\vdash B\ \triangledown\ \tau$}
    \AxiomC{$\Gamma:>\Phi$}
    \LeftLabel{Yield}
    \QuaternaryInfC{$\Phi\vdash\vartriangle \eta\ : \tau$}
\end{prooftree}

The behaviour and final value of the $\vartriangle\eta$ form depends on the
behaviour of the handler that $\eta$ is handled by. When the handler uses
the keyword \kw{resume}, it evaluates to the value that was \kw{resume} was
applied to.

\begin{prooftree}
    \AxiomC{$\Gamma\vdash E : \tau$}
    \LeftLabel{Resume}
    \UnaryInfC{$\Gamma\vdash \kw{resume}\ E\ \triangledown\ \tau$}
\end{prooftree}

Though all of these keywords can be used in both expression and statement
position, the semantics are the same, so won't be repeated; the only difference
is that in statement position, all the values are lost and the effect system
acts as pure control flow.
