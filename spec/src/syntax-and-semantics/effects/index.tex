\subsection{Effect System}
\label{sec:effects}

By no means is \Trilogy{} the first language to implement an effect system;
much of what is presented here is inspired by, and cross checked against,
existing research on effects\cite{eff}.

Sadly, mainstream languages featuring a full-fledged first class effect system
are\dots{} none? At least none I have ever come across. Most languages with an
effect system are experimental or research based, and appear far from practical
to me. While I can't really claim that \Trilogy{} is much better on being ``practical''
and not ``experimental'', it does aim to take a more user and productivity first
approach to implementing effects than other research languages.

\Trilogy{} includes effects as its core advanced control flow mechanism because
I believe it to be the most concretely conceptualized of the equivalent competing
models: monads and delimited control\cite{effmondel}.

Now, with a complete lack of true expertise, I attempt to specify the effect
system of \Trilogy{}, beginning by finally addressing the syntax of effect
handlers.

\begin{bnf*}
    \bnfprod{EffectHandlers}{
        \bnfpn{EffectHandler}
        \bnfsp
        \bnfpn{EffectHandlers}
        \bnfor
        \bnfpn{EffectHandler}
    } \\
    \bnfprod{EffectHandler}{
        \bnfts{\kw{when}}
        \bnfsp
        \bnfpn{Pattern}
        \bnfsp
        \bnfts{\kw{resume}}
        \bnfsp
        \bnfpn{Expression}
        \bnfor
    } \\
    \bnfmore{
        \bnfts{\kw{when}}
        \bnfsp
        \bnfpn{Pattern}
        \bnfsp
        \bnfts{\kw{cancel}}
        \bnfsp
        \bnfpn{Expression}
        \bnfor
    } \\
    \bnfmore{
        \bnfts{\kw{when}}
        \bnfsp
        \bnfpn{Pattern}
        \bnfsp
        \bnfts{\kw{cancel}}
        \bnfsp
        \bnfpn{Block}
        \bnfor
    } \\
    \bnfmore{
        \bnfts{\kw{when}}
        \bnfsp
        \bnfpn{Pattern}
        \bnfsp
        \bnfts{\kw{invert}}
        \bnfsp
        \bnfpn{Block}
        \bnfor
    } \\
    \bnfmore{
        \bnfts{\kw{given}}
        \bnfsp
        \bnfpn{RuleHead}
        \bnfor
    } \\
    \bnfmore{
        \bnfts{\kw{given}}
        \bnfsp
        \bnfpn{RuleHead}
        \bnfsp
        \bnfts{\op{<-}}
        \bnfsp
        \bnfpn{RuleBody}
    }
\end{bnf*}

Effect handlers come in two varieties: the \kw{when} form for a functional
or imperative effect handler; and the \kw{given} form for a logical effect
handler, or dynamic rule, which takes the place of both closures and effects
in \Law{} due to its vastly different syntax and semantics that do not work
smoothly with traditional effects.

It is worth noting that dynamic rules are not necessarily here to stay, as
they may be a hack given my relative lack of experience with both practical
logic programming and effect systems. In particular, there are ways to both
implement both higher-order predicates\cite{hologic} as well as effect
handlers in logic languages\cite{prologeffects}.
