\subsubsection{Unary Operation}
\label{sec:unaryop}

There are a few unary operators in \Poetry{}, all of which are prefix operators.
There are no postfix operators in \Trilogy{}, as they end up being a parsing
nightmare for little value (the only commonly seen postfix operators are
ones like \texttt{++}, and we prefer \texttt{+= 1} in general these days).

\begin{bnf*}
    \bnfprod{UnaryOp}{
        \bnfpn{UnaryOperator}
        \bnfsp
        \bnfpn{Expression}
    } \\
    \bnfprod{UnaryOperator}{
        \bnfts{\op{-}}
        \bnfor
        \bnfpn{OnlyUnaryOperator}
    } \\
    \bnfprod{OnlyUnaryOperator}{
        \bnfts{\kw{not}}
        \bnfor
        \bnfts{\op{\textasciitilde}}
        \bnfor
        \bnfts{\kw{yield}}
    }
\end{bnf*}

It turns out, even prefix unary operators cause some parsing ambiguity, particularly
when it comes to application of a function to a negative number. To resolve this
ambiguity, we simply declare unary operators to be the last choice when parsing;
only if there is no other interpretation of the expression but the interpretation
where the operator is used as a unary operator will it be a interpreted unary operation.

To illustrate that point, the following table summarizes some syntax and a
less ambiguous ``formal'' representation. The formal representation is actually
valid \Trilogy{} code as well, showing that \Trilogy{} could be used like a Lisp
if you really wanted to.

\begin{table}[H]
    \centering
    \begin{tabular}{lll}
        \hline
        \textbf{Conventional} & \textbf{Formal} & \textbf{Explanation} \\
        \hline
        \tri{x - 5} & \tri{((-) x 5)} & Infix takes precedence \\
        \tri{x (- 5)} & \tri{(x ((-) 5))} & Explicitly disambiguated \\
        \tri{x (-) 5} &  \tri{((x (-)) 5))} & \tri{(-)} is a reference \\
        \tri{x not true} & \tri{(x ((not) true))} & \tri{not} cannot be infix \\
        \tri{x (not) true} & \tri{((x (not)) true))} & \tri{(not)} is a reference \\
        \tri{x + - 5} & \tri{((+) x ((-) 5))} & One of two operators is unary \\
        \hline
    \end{tabular}
\end{table}

The \kw{not} operator is used to perform Boolean negation. \texttt{\kw{not}~\kw{true}}
evaluates to \kw{false} and \texttt{\kw{not}~\kw{false}} evaluates to \kw{true}, as
you might have expected. If the expression is not a Boolean, it causes a runtime type
error. This is much like the operator \texttt{!} seen in other languages.

\begin{figure}[H]
    \centering
    \parbox[t]{0.4\linewidth}{
        \begin{prooftree}
            \AxiomC{$\Gamma\vdash E=\kw{true}$}
            \LeftLabel{Not}
            \UnaryInfC{$\Gamma\vdash \kw{not}\ E=\kw{false}$}
        \end{prooftree}
    }
    \parbox[t]{0.4\linewidth}{
        \begin{prooftree}
            \AxiomC{$\Gamma\vdash E=\kw{false}$}
            \LeftLabel{Not}
            \UnaryInfC{$\Gamma\vdash \kw{not}\ E=\kw{true}$}
        \end{prooftree}
    }
\end{figure}

The operator \op{-} is the numeric negation operator. In \Trilogy{} there are no
negative numeric literals, so all negative numbers must go through this operator.
Used on a non-Number value, it causes a runtime type error.

\begin{prooftree}
    \AxiomC{$\Gamma\vdash E = n : \ty{Number}$}
    \LeftLabel{Negation}
    \UnaryInfC{$\Gamma\vdash \op{-}\ E = -n$}
\end{prooftree}

The operator \op{\textasciitilde} is the bitwise negation operator. Used on a value
of Bits type, the state of each bit in the value is inverted.
Used on a non-Bits value, it causes a runtime type error.

\begin{prooftree}
    \AxiomC{$\Gamma\vdash E:\ty{Bits}$}
    \LeftLabel{Bitwise Negation}
    \UnaryInfC{$\Gamma\vdash \op{\textasciitilde}\ E:\ty{Bits}$}
\end{prooftree}

The \kw{yield} keyword used in \Poetry{} is actually also an operator, mostly for
convenience. This operator works similarly to how the \kw{yield} statement of \Prose{}
works, but the returned value is accessible. It may return more or less than once if
the effect handler resumes more or less that once. Though it would be nice to explain
it now, once again leave the detailed explanation to \S\ref{sec:effects}.
