\subsubsection{Literals}
\label{sec:literals}

Starting easy, the simplest expression is a primitive value.

\begin{bnf*}
    \bnfprod{Primitive}{
        \bnfts{\kw{unit}}
        \bnfor
        \bnfts{\kw{false}}
        \bnfor
        \bnfts{\kw{true}}
        \bnfor
        \bnfpn{Numeric}
        \bnfor
    } \\
    \bnfmore{
        \bnfpn{Atom}
        \bnfor
        \bnfpn{String}
        \bnfor
        \bnfpn{Character}
        \bnfor
        \bnfpn{Bits}
    }
\end{bnf*}

The evaluation of a literal primitive value is simply the value it represents.
For completeness, some trivial specifications:

\begin{figure}[H]
    \centering
    \parbox[t]{0.3\linewidth}{
        \begin{prooftree}
            \AxiomC{}
            \LeftLabel{Unit}
            \UnaryInfC{$\kw{unit} : \ty{Unit}$}
        \end{prooftree}
    }
    \parbox[t]{0.3\linewidth}{
        \begin{prooftree}
            \AxiomC{}
            \LeftLabel{Boolean}
            \UnaryInfC{$\kw{false} : \ty{Bool}$}
        \end{prooftree}
    }
    \parbox[t]{0.3\linewidth}{
        \begin{prooftree}
            \AxiomC{}
            \LeftLabel{Boolean}
            \UnaryInfC{$\kw{true} : \ty{Bool}$}
        \end{prooftree}
    }
    \parbox[t]{0.4\linewidth}{
        \begin{prooftree}
            \AxiomC{}
            \LeftLabel{Numeric}
            \UnaryInfC{$\langle\text{Numeric}\rangle : \ty{Number}$}
        \end{prooftree}
    }
    \parbox[t]{0.4\linewidth}{
        \begin{prooftree}
            \AxiomC{}
            \LeftLabel{String}
            \UnaryInfC{$\langle\text{String}\rangle : \ty{String}$}
        \end{prooftree}
    }
    \parbox[t]{0.4\linewidth}{
        \begin{prooftree}
            \AxiomC{}
            \LeftLabel{Atom}
            \UnaryInfC{$\langle\text{Atom}\rangle : \ty{Atom}$}
        \end{prooftree}
    }
    \parbox[t]{0.4\linewidth}{
        \begin{prooftree}
            \AxiomC{}
            \LeftLabel{Character}
            \UnaryInfC{$\langle\text{Character}\rangle : \ty{Char}$}
        \end{prooftree}
    }
    \parbox[t]{0.4\linewidth}{
        \begin{prooftree}
            \AxiomC{}
            \LeftLabel{Bits}
            \UnaryInfC{$\langle\text{Bits}\rangle : \ty{Bits}$}
        \end{prooftree}
    }
\end{figure}

Of similar literal-ness are the literals for compound types. Notably
tuples are missing from this section, as they are implemented in terms
of the cons operator, rather than a literal form.

\begin{bnf*}
    \bnfprod{Compound}{
        \bnfpn{Struct}
        \bnfor
        \bnfpn{Array}
        \bnfor
        \bnfpn{Set}
        \bnfor
        \bnfpn{Record}
    } \\
    \bnfprod{Struct}{
        \bnfpn{Atom}
        \bnfsp
        \bnfts{(}
        \bnfsp
        \bnfpn{Expression}
        \bnfsp
        \bnfts{)}
    } \\
    \bnfprod{Array}{
        \bnfts{[}
        \bnfsp
        \bnfpn{CompoundElements}
        \bnfsp
        \bnfts{]}
    } \\
    \bnfprod{Set}{
        \bnfts{[|}
        \bnfsp
        \bnfpn{CompoundElements}
        \bnfsp
        \bnfts{|]}
    } \\
    \bnfprod{CompoundElements}{
        \bnfpn{CompoundElement}
        \bnfsp
        \bnfts{,}
        \bnfsp
        \bnfpn{CompoundElements}
        \bnfor
    } \\
    \bnfmore{
        \bnfpn{CompoundElement}
        \bnfor
        \bnfes
    } \\
    \bnfprod{CompoundElement}{
        \bnfpn{Expression}
        \bnfor
        \bnfts{\op{..}}
        \bnfsp
        \bnfpn{Expression}
    } \\
    \bnfprod{Record}{
        \bnfts{\{|}
        \bnfsp
        \bnfpn{KeyValueElements}
        \bnfsp
        \bnfts{|\}}
    } \\
    \bnfprod{KeyValueElements}{
        \bnfpn{KeyValueElement}
        \bnfsp
        \bnfts{,}
        \bnfsp
        \bnfpn{KeyValueElements}
        \bnfor
    } \\
    \bnfmore{
        \bnfpn{KeyValueElement}
        \bnfor
        \bnfes
    } \\
    \bnfprod{KeyValueElement}{
        \bnfpn{Expression}
        \bnfsp
        \bnfts{\op{=>}}
        \bnfsp
        \bnfpn{Expression}
        \bnfor
        \bnfts{\op{..}}
        \bnfsp
        \bnfpn{Expression}
    } \\
\end{bnf*}

As you might expect, the evaluation of a compound literal is a value of the
compound type, where the items it contains are the values the evaluate to.
The only thing of particular interest is the use of spread (\op{..}) in
Array, Set, and Record literals, for which the expression must evaluate
to a value of the same compound type, and that value is then inserted
as if its elements were written in place of the spread.

\begin{prooftree}
    \AxiomC{$\Gamma\vdash E : \tau$}
    \LeftLabel{Struct}
    \UnaryInfC{$\Gamma\vdash\texttt{'}l\ \texttt{(}\ E\ \texttt{)} : \texttt{'} l(\tau)$}
\end{prooftree}

\begin{prooftree}
    \AxiomC{}
    \LeftLabel{Array\textsuperscript{$\epsilon$}}
    \UnaryInfC{$\texttt{[]} : \ty{List}\ \tau$}
\end{prooftree}

\begin{prooftree}
    \AxiomC{$\Gamma\vdash X_1 : \tau$}
    \AxiomC{$\Gamma\vdash \texttt{[}X_2\texttt{,}\ \cdots\texttt{,}\ X_n\texttt{]} : \ty{List}\ \tau$}
    \LeftLabel{Array}
    \BinaryInfC{$\Gamma\vdash\texttt{[}X_1\texttt{,}\ X_2\texttt{,}\ \cdots\texttt{,}\ X_n\texttt{]} : \ty{List}\ \tau$}
\end{prooftree}

\begin{prooftree}
    \AxiomC{$\Gamma\vdash X_1 : \ty{List}\ \tau$}
    \AxiomC{$\Gamma\vdash \texttt{[}X_2\texttt{,}\ \cdots\texttt{,}\ X_n\texttt{]} : \ty{List}\ \tau$}
    \LeftLabel{Array\textsuperscript{\op{..}}}
    \BinaryInfC{$\Gamma\vdash\texttt{[}\op{..}X_1\texttt{,}\ X_2\texttt{,}\ \cdots\texttt{,}\ X_n\texttt{]} : \ty{List}\ \tau$}
\end{prooftree}

\begin{prooftree}
    \AxiomC{}
    \LeftLabel{Set\textsuperscript{$\epsilon$}}
    \UnaryInfC{$\texttt{[||]} : \ty{Set}\ \tau$}
\end{prooftree}

\begin{prooftree}
    \AxiomC{$\Gamma\vdash X_1 : \tau$}
    \AxiomC{$\Gamma\vdash \texttt{[|}\ X_2\ \texttt{,}\cdots\texttt{,}\ X_n\ \texttt{|]} : \ty{Set}\ \tau$}
    \LeftLabel{Set}
    \BinaryInfC{$\Gamma\vdash\texttt{[|}\ X_1\ \texttt{,}\ X_2\ \texttt{,}\cdots\texttt{,}\ X_n\ \texttt{|]} : \ty{Set}\ \tau$}
\end{prooftree}

\begin{prooftree}
    \AxiomC{$\Gamma\vdash X_1 : \ty{Set}\ \tau$}
    \AxiomC{$\Gamma\vdash \texttt{[|}\ X_2\ \texttt{,}\cdots\texttt{,}\ X_n\ \texttt{|]} : \ty{Set}\ \tau$}
    \LeftLabel{Set\textsuperscript{\op{..}}}
    \BinaryInfC{$\Gamma\vdash\texttt{[|}\ \op{..}X_1\ \texttt{,}\ X_2\ \texttt{,}\cdots\texttt{,}\ X_n\ \texttt{|]} : \ty{Set}\ \tau$}
\end{prooftree}

\begin{prooftree}
    \AxiomC{}
    \LeftLabel{Record\textsuperscript{$\epsilon$}}
    \UnaryInfC{$\texttt{\{||\}} : \ty{Record}\ \kappa\ \nu$}
\end{prooftree}

\begin{prooftree}
    \def\defaultHypSeparation{\hskip 0in}
    \AxiomC{$\Gamma\vdash K_1 : \kappa$}
    \AxiomC{$\Gamma\vdash V_1 : \nu$}
    \AxiomC{$\Gamma\vdash \texttt{\{|}K_2:V_2\ \texttt{,}\cdots\texttt{,}\ K_n:V_n\texttt{|\}} : \ty{Record}\ \kappa\ \nu$}
    \LeftLabel{Record}
    \TrinaryInfC{$\Gamma\vdash\texttt{\{|}K_1\ \texttt{=>}\ V_1\ \texttt{,}\ K_2\ \texttt{=>}\ V_2\ \texttt{,}\cdots\texttt{,}\ K_n\ \texttt{=>}\ V_n\texttt{|\}} : \ty{Record}\ \kappa\ \nu$}
\end{prooftree}

\begin{prooftree}
    \def\defaultHypSeparation{\hskip 0in}
    \AxiomC{$\Gamma\vdash X : \ty{Record}\ \kappa\ \nu$}
    \AxiomC{$\Gamma\vdash \texttt{\{|}K_2\ \texttt{=>}\ V_2\ \texttt{,}\cdots\texttt{,}\ K_n\ \texttt{=>}\ V_n\texttt{|\}} : \ty{Record}\ \kappa\ \nu$}
    \LeftLabel{Record\textsuperscript{\op{..}}}
    \BinaryInfC{$\Gamma\vdash\texttt{\{|}\op{..}X\ \texttt{,}\ K_2\ \texttt{=>}\ V_2\ \texttt{,}\cdots\texttt{,}\ K_n\ \texttt{=>}\ V_n\texttt{|\}} : \ty{Record}\ \kappa\ \nu$}
\end{prooftree}
