\subsubsection{Reference}

Defined names may be referenced at any time, their naming evaluating to the value
they currently hold. This is a typical variable reference as in any language.
Notably this is \emph{only} for local references, looking up a name in another
scope is handled by the module access operator.

As a bit of a special case, the built-in operators may be referenced as if they
were functions by wrapping them in parentheses. This allows operators to be passed
as values and applied as if they were functions to their arguments, occasionally
useful in some functional compositions. The operators themselves are specified
in \S\ref{sec:binop} and \S\ref{sec:unaryop} (the member access operator is included
as well, but since it is described separately in \S\ref{sec:member-access}, it is
also listed separately here).

The one odd bit here is that the \op{-} symbol is always referenced as the
subtraction operator, and never the unary negation operator. Unary negation
can be constructed from subtraction by pre-supplying the zero as in \tri{((-) 0)}.

It is here also that first class keywords are converted into values.
First class keywords are described in \S\ref{sec:first-class-keyword}.

\begin{bnf*}
    \bnfprod{Reference}{
        \bnfpn{Identifier}
        \bnfor
        \bnfts{(}
        \bnfsp
        \bnfpn{OnlyUnaryOperator}
        \bnfsp
        \bnfts{)}
        \bnfor
        \bnfts{(}
        \bnfsp
        \bnfts{.}
        \bnfsp
        \bnfts{)}
        \bnfor
    } \\
    \bnfmore{
        \bnfts{(}
        \bnfsp
        \bnfpn{BinaryOperator}
        \bnfsp
        \bnfts{)}
        \bnfor
        \bnfts{(}
        \bnfsp
        \bnfpn{FirstClassKeyword}
        \bnfsp
        \bnfts{)}
    }
\end{bnf*}

Admittedly the following semantics look a little wonky due to an ambiguous choice
of notation. Rest assured, in the hypotheses the $x$ refers to the binding in the
context, while in the inference the $x$ refers to the symbol in source code, so
this is not an infinitely recursive deduction.

\begin{prooftree}
    \AxiomC{$\Gamma\vdash x : \tau$}
    \LeftLabel{Reference}
    \UnaryInfC{$\Gamma\vdash x : \tau$}
\end{prooftree}

For brevity, the semantics of the operator and keyword references are condensed
into this more symbolic form. Repeating the same tree for every operator would
be a real pain for everyone.

\begin{prooftree}
    \AxiomC{$\Gamma\vdash E : \tau$}
    \AxiomC{$\Gamma\vdash \langle \text{UnaryOperator} \rangle\ E : \rho$}
    \LeftLabel{Reference}
    \BinaryInfC{$\Gamma\vdash \texttt{(}\ \langle \text{UnaryOperator} \rangle\ \texttt{)}\ : \tau \rightarrow \rho$}
\end{prooftree}

\begin{prooftree}
    \AxiomC{$\Gamma\vdash L : \tau$}
    \AxiomC{$\Gamma\vdash R : \rho$}
    \AxiomC{$\Gamma\vdash L\ \langle \text{BinaryOperator} \rangle\ R : \phi$}
    \LeftLabel{Reference}
    \TrinaryInfC{$\Gamma\vdash \texttt{(}\ \langle \text{BinaryOperator} \rangle\ \texttt{)}\ : \tau \rightarrow \rho \rightarrow \phi$}
\end{prooftree}

\begin{prooftree}
    \AxiomC{$\Gamma\vdash E : \tau$}
    \AxiomC{$\Gamma\vdash \langle \text{FirstClassKeyword} \rangle\ E : \rho$}
    \LeftLabel{Reference}
    \BinaryInfC{$\Gamma\vdash \texttt{(}\ \langle \text{FirstClassKeyword} \rangle\ \texttt{)}\ : \tau \rightarrow \rho$}
\end{prooftree}
