\subsubsection{Conditionals}

\Poetry{}'s take on the conditional uses slightly modified syntax when
compared to that of \Prose{}. Rather than using blocks as the bodies of
the branches, the bodies are expressions. To separate the condition from
the body, the \kw{then} is used.

Also worth noting is that there must always be an \kw{else} clause, as
the expression must always evaluate to some value.

\begin{bnf*}
    \bnfprod{IfElse}{
        \bnfts{\kw{if}}
        \bnfsp
        \bnfpn{Expression}
        \bnfsp
        \bnfts{\kw{then}}
        \bnfsp
        \bnfpn{Expression}
        \bnfsp
        \bnfts{\kw{else}}
        \bnfsp
        \bnfpn{Expression}
    }
\end{bnf*}

This rigid form makes for a relatively simple semantics. Similar to in the
conditional statement, the condition is an expression that must evaluate
to a Boolean value. When \kw{true}, the \kw{then} block is evaluated.
Whe \kw{false}, the \kw{else} block is evaluated. When not a Boolean value,
then it is considered a runtime type error.

\begin{prooftree}
    \AxiomC{$\Gamma\vdash E = \kw{true}$}
    \AxiomC{$\Gamma\vdash T : \tau$}
    \LeftLabel{If}
    \BinaryInfC{$\Gamma\vdash \kw{if}\ E\ \kw{then}\ T\ \kw{else}\ F : \tau$}
\end{prooftree}

\begin{prooftree}
    \AxiomC{$\Gamma\vdash E = \kw{false}$}
    \AxiomC{$\Gamma\vdash F : \tau$}
    \LeftLabel{Else}
    \BinaryInfC{$\Gamma\vdash \kw{if}\ E\ \kw{then}\ T\ \kw{else}\ F : \tau$}
\end{prooftree}
