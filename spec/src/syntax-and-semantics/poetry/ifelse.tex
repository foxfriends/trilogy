\subsubsection{If-Else}
\label{sec:if-else}

The primary conditional statement in \Trilogy{} is the \kw{if}-\kw{else} expression,
which appears as you might expect coming from nearly any other language.
Different than the \kw{if} statement of \Prose{}, which may not have a bare
\kw{else} clause, the \kw{if}-\kw{else} expression must always have an \kw{else}
clause, as the expression must always evaluate to some value.

\begin{bnf*}
    \bnfprod{IfElse}{
        \bnfts{\kw{if}}
        \bnfsp
        \bnfpn{Expression}
        \bnfsp
        \bnfts{\kw{then}}
        \bnfsp
        \bnfpn{Expression}
        \bnfsp
        \bnfpn{Else}
        \bnfor
    } \\
    \bnfmore{
        \bnfts{\kw{if}}
        \bnfsp
        \bnfpn{Expression}
        \bnfsp
        \bnfpn{Block}
        \bnfsp
        \bnfpn{Else}
    } \\
    \bnfprod{Else}{
        \bnfts{\kw{else}}
        \bnfsp
        \bnfpn{Expression}
    }
\end{bnf*}

This rigid form makes for a relatively simple semantics. Similar to in the
conditional statement, the condition is an expression that must evaluate
to a Boolean value. When \kw{true}, the \kw{then} expression or body block
is evaluated. Whe \kw{false}, the \kw{else} body is evaluated. When not a
Boolean value, then it is considered a runtime type error.

\begin{prooftree}
    \AxiomC{$\Gamma\vdash E = \kw{true}$}
    \AxiomC{$\Gamma\vdash T : \tau$}
    \LeftLabel{If}
    \BinaryInfC{$\Gamma\vdash \kw{if}\ E\ \kw{then}\ T\ \kw{else}\ F : \tau$}
\end{prooftree}

\begin{prooftree}
    \AxiomC{$\Gamma\vdash E = \kw{false}$}
    \AxiomC{$\Gamma\vdash F : \tau$}
    \LeftLabel{Else}
    \BinaryInfC{$\Gamma\vdash \kw{if}\ E\ \kw{then}\ T\ \kw{else}\ F : \tau$}
\end{prooftree}

When the body of the \kw{if} clause is a block, the \kw{then} keyword may be
omitted. This is handled by the following syntax transformation:

\begin{align*}
    \texttt{\kw{if} \$b:block} & \Rightarrow \texttt{\kw{if} \kw{then} \$b}
\end{align*}
