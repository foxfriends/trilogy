\subsubsection{First Class Keywords}
\label{sec:first-class-keyword}

The four keywords \kw{resume}, \kw{return}, \kw{break}, and \kw{continue} are
considered ``first class keywords'', meaning they may be used in expressions,
stored in variables, and passed to functions to be called later.

These four keywords are bound syntactically to their respective structures:
\kw{resume} to the nearest enclosing effect handler, \kw{break} and \kw{continue}
to the nearest enclosing loop, and \kw{return} to the nearest enclosing procedure.
When used in an expression, no matter how deeply nested in other structures or even
closures, they continue to refer to these scopes (though do note that a new \kw{return}
is introduced for nested procedure closures).

The syntactic binding becomes more important when the keyword is assigned to variables:
even when passed as a value to another function or stored in a variable, the keyword
continues to refer to the structure that syntactically enclosed it.

\begin{bnf*}
    \bnfprod{FirstClassKeyword}{
        \bnfts{\kw{resume}}
        \bnfor
        \bnfts{\kw{return}}
        \bnfor
        \bnfts{\kw{break}}
        \bnfor
        \bnfts{\kw{continue}}
    } \\
    \bnfprod{Resume}{
        \bnfts{\kw{resume}}
        \bnfsp
        \bnfpn{Expression}
    } \\
    \bnfprod{Return}{
        \bnfts{\kw{return}}
        \bnfsp
        \bnfpn{Expression}
    } \\
    \bnfprod{Break}{
        \bnfts{\kw{break}}
        \bnfsp
        \bnfpn{Expression}
    } \\
    \bnfprod{Continue}{
        \bnfts{\kw{continue}}
        \bnfsp
        \bnfpn{Expression}
    } \\
\end{bnf*}

In each case, these keywords represent a \emph{continuation}---a point in the
construction of the program to jump to---and accept a value that the continuation is
resumed with. Note that for \kw{break} and \kw{return}, the value is simply lost as
there is no way to receive that value at this time. It is convention to use \kw{unit}
in this situation to acknowledge that the value will not be received.

The value supplied to the \kw{return} keyword is used as the value that the function
it is bound to returns when called, as it would be when used regularly as a statement.

Semantically, none of these are particularly interesting, though \kw{resume}, being
part of the effect system and notably harder to define semantics for, is once again
left for \S\ref{sec:effects}.

\begin{prooftree}
    \AxiomC{$\Gamma\vdash E:\top$}
    \LeftLabel{Break}
    \UnaryInfC{$\Gamma\vdash \kw{break}\ E : \bot$}
\end{prooftree}

\begin{prooftree}
    \AxiomC{$\Gamma\vdash E:\top$}
    \LeftLabel{Continue}
    \UnaryInfC{$\Gamma\vdash \kw{continue}\ E : \bot$}
\end{prooftree}

\begin{prooftree}
    \AxiomC{$\Gamma\vdash E:\top$}
    \LeftLabel{Return}
    \UnaryInfC{$\Gamma\vdash \kw{return}\ E : \bot$}
\end{prooftree}

More interesting than the formal semantics of these statements is a discussion on
the practical semantics, which I believe are quietly hidden in the formal semantics,
but also I am no expert and so maybe they are overlooked.

The \kw{break} keyword is the least interesting; it simply ends the loop that it
is bound to, and the program resumes execution from the point after the end of the
loop. However, if the \kw{break} keyword \emph{escapes} the loop, and is called
a second time, execution once again jumps back to the end of that loop. Honestly, not
the most useful behaviour, but it is consistent with the way continuations work, and
so it remains.

The \kw{continue} keyword is slightly more interesting: it's continuation resumes
at the beginning of the next iteration of the loop. While syntactically it is visibly
bound to a loop, it is also semantically bound to a particular \emph{iteration} of
that loop. If the \kw{continue} keyword escapes its iteration and is called,
the loop will start from an iteration that has already occurred. Combined with
\kw{break} it is also now possible to escape a loop, do some things, then go back
into the loop and continue. What this would be useful for is unknown (why didn't
you just put that code in the loop to begin with?) but again, it is consistent with
how continuations work, and so it is how it is.

Finally, the \kw{return} statement is also interesting in that it is, to some degree,
a realization of the operator \textbf{J}, described by Peter J. Landin\cite{j}.
Passing the return value to another function allows for a returned value to skip
multiple layers and return from the function the first class keyword was bound to.
If the \kw{return} escapes its function (e.g. by being itself returned), we now have
a way to pass a value ``back in time''. It was shown by John C. Reynolds that \textbf{J}
can be used to implement the call/cc function, much like the one seen in some Lisps\cite{defint}.
