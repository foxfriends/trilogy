\subsubsection{First Class Keywords}
\label{sec:first-class-keyword}

The seven keywords \kw{resume}, \kw{cancel}, \kw{become}, \kw{return}, \kw{break},
\kw{continue}, and \kw{next} are considered ``first class keywords'', meaning they may be
used in expressions, stored in variables, and passed to functions to be called later.

These four keywords are bound syntactically to their respective structures:
\kw{resume}, \kw{cancel}, and \kw{become} to the nearest enclosing effect handler,
\kw{break}, \kw{continue}, and \kw{next} to the nearest enclosing loop,
and \kw{return} to the nearest enclosing procedure or function.

The syntactic binding becomes more important when the keyword is assigned to variables:
even when passed as a value to another function or stored in a variable, the keyword
continues to refer to the structure that syntactically enclosed it.

\begin{bnf*}
    \bnfprod{FirstClassKeyword}{
        \bnfts{\kw{resume}}
        \bnfor
        \bnfts{\kw{cancel}}
        \bnfor
        \bnfts{\kw{become}}
        \bnfor
    } \\
    \bnfmore{
        \bnfts{\kw{return}}
        \bnfor
        \bnfts{\kw{break}}
        \bnfor
        \bnfts{\kw{continue}}
        \bnfor
        \bnfts{\kw{next}}
    } \\
    \bnfprod{Resume}{
        \bnfts{\kw{resume}}
        \bnfsp
        \bnfpn{Expression}
    } \\
    \bnfprod{Cancel}{
        \bnfts{\kw{cancel}}
        \bnfsp
        \bnfpn{Expression}
    } \\
    \bnfprod{Become}{
        \bnfts{\kw{become}}
        \bnfsp
        \bnfpn{Expression}
    } \\
    \bnfprod{Return}{
        \bnfts{\kw{return}}
        \bnfsp
        \bnfpn{Expression}
    } \\
    \bnfprod{Break}{
        \bnfts{\kw{break}}
        \bnfsp
        \bnfpn{Expression}
    } \\
    \bnfprod{Continue}{
        \bnfts{\kw{continue}}
        \bnfsp
        \bnfpn{Expression}
    } \\
    \bnfprod{Next}{
        \bnfts{\kw{next}}
        \bnfsp
        \bnfpn{Expression}
    } \\
\end{bnf*}

In each case, these keywords represent a \emph{continuation}---a point in the
construction of the program to jump to---and accept a value that the continuation is
resumed with. It is convention to use \kw{unit} when the value does not matter, or will
not be retrievable.

The value supplied to the \kw{return} keyword is used as the value that the function
it is bound to returns when called, as it would be when used regularly as a statement.

\kw{resume}, \kw{cancel}, and \kw{become}, being part of the effect system and notably
harder to define semantics for, are once again left for \S\ref{sec:effects}. Though not
expressed in the grammar here, these three keywords are only permitted to be used or
referenced within the body of a handler, the syntax of which is detailed later on.

The \kw{break}, \kw{continue}, and \kw{next} keywords are similarly only permitted
within loops. In particular, \kw{next} is only permitted within the expression form
of a \kw{for} loop, as detailed in \S\ref{sec:poetry-for}. \kw{break} and \kw{continue}
are for use in any loops (\kw{while} (\S\ref{sec:prose-while}) and statement form
\kw{for} (\S\ref{sec:prose-for})).

\begin{prooftree}
    \AxiomC{$\Gamma\vdash E:\top$}
    \LeftLabel{Break}
    \UnaryInfC{$\Gamma\vdash \kw{break}\ E : \bot$}
\end{prooftree}

\begin{prooftree}
    \AxiomC{$\Gamma\vdash E:\top$}
    \LeftLabel{Continue}
    \UnaryInfC{$\Gamma\vdash \kw{continue}\ E : \bot$}
\end{prooftree}

The \kw{break} keyword is the least interesting; it simply ends the loop that it
is bound to, and the program resumes execution from the point after the end of the
loop. However, if the \kw{break} keyword \emph{escapes} the loop, and is called
a second time, execution once again jumps back to the end of that loop. Honestly, not
the most useful behaviour, but it is consistent with the way continuations work, and
so it remains.

The \kw{continue} keyword is slightly more interesting: its continuation resumes
at the beginning of the next iteration of the loop. While syntactically it is visibly
bound to a loop, it is also semantically bound to a particular \emph{iteration} of
that loop. If the \kw{continue} keyword escapes its iteration and is called,
the loop will start from an iteration that has already occurred. Combined with
\kw{break} it is also now possible to escape a loop, do some things, then go back
into the loop and continue. What this would be useful for is unknown (why didn't
you just put that code in the loop to begin with?) but again, it is consistent with
how continuations work, and so it is how it is.

Unusual for \kw{break} and \kw{continue} is that they both require an expression.
Though this expression is only relevant in some situations, it is required always
for consistency. This situation will become apparent soon.

\begin{prooftree}
    \AxiomC{$\Gamma\vdash E:\tau$}
    \AxiomC{$\Gamma\vdash V:\tau$}
    \BinaryInfC{$\Gamma\vdash \kw{for}\ Q\ \kw{where}\ i\ =\ V \{\ \kw{next}\ E\ \} : \rho$}
    \LeftLabel{Next}
    \UnaryInfC{$\Gamma\vdash \kw{next}\ E : \rho$}
\end{prooftree}

The \kw{next} keyword is new: it moves forwards to the next iteration of the loop
in much the same way as \kw{continue} but, once the loop is complete, the iterations
are reversed and the final value of the loop up to that point is returned from \kw{next}.
In this way, the final value of the loop is not necessarily the final value of the final
iteration, but that value can be manipulated also on the way out. After using \kw{next},
it is typically expected that \kw{break} will also be used to go back outwards, otherwise
the loop will continue inwards again upon reaching the end of the loop body (as reaching
the end of the loop body is much the same as calling \kw{continue} with the final value).

The \kw{break} keyword also does not immediately end the loop, but instead goes back
to the previous iteration in which \kw{next} was called. The value provided to the
\kw{break} keyword being the value returned from \kw{next}.

\kw{continue} in this situation can be considered to be equivalent to the following
syntax transformation, and so hopefully does not need to be explained in particular:

\begin{align*}
    \texttt{\kw{continue} \$e} &\Rightarrow \texttt{\kw{break} \kw{next} \$e}
\end{align*}

Finally, the \kw{return} statement is also interesting in that it is, to some degree,
a realization of the operator \textbf{J}, described by Peter J. Landin\cite{j}.
Passing the return keyword to another function allows for a returned value to skip
multiple layers and return from the function the first class keyword was bound to.
If the \kw{return} escapes its function (e.g.\ by being itself returned), we now have
a way to pass a value ``back in time''. It was shown by John C. Reynolds that \textbf{J}
can be used to implement the call/cc function, much like the one seen in some Lisps\cite{defint}.

\begin{prooftree}
    \AxiomC{$\Gamma\vdash E:\top$}
    \LeftLabel{Return}
    \UnaryInfC{$\Gamma\vdash \kw{return}\ E : \bot$}
\end{prooftree}
