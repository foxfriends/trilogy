\subsubsection{Module Access}
\label{sec:module-access}

The module access ``operator'' (\op{::}) is used to reference an exported member from
another module. While intuitively this may appear to be similar to member access
operator, module access is not truly an operator: right hand operand is not treated
as an expression, nor is this operator something that can be referenced independently
of its being used.

\begin{bnf*}
    \bnfprod{ModuleAccess}{
        \bnfpn{Expression}
        \bnfsp
        \bnfts{\op{::}}
        \bnfsp
        \bnfpn{Identifier}
        \bnfor
    }
\end{bnf*}

The left hand side of the \op{::} is an expression evaluated like any other, but the
right hand side must be an identifier, which is the identifier of the member to look
up from the module that the left hand side evaluates to. If the expression does not
evaluate to a module, it is considered a runtime type error.

\begin{prooftree}
    \AxiomC{$\Gamma\vdash E = \sigma : \ty{Module}$}
    \AxiomC{$\Phi\vdash i : \tau$}
    \AxiomC{$\sigma\ \text{exports}\ i$}
    \LeftLabel{Module Access}
    \TrinaryInfC{$\Gamma\vdash E\ \texttt{::}\ i : \tau$}
\end{prooftree}

Though not strictly module access itself, it makes sense to define \kw{super} in
this section as well.

\begin{bnf*}
    \bnfprod{Super}{\bnfts{\kw{super}}}
\end{bnf*}

The \kw{super} keyword refers to the current module's parent module, allowing access
to specific members (even shadowed ones) from the enclosing scope. Note that this always
refers to the lexically enclosing module so the document-level module may not refer to
\kw{super}.

\begin{prooftree}
    \AxiomC{}
    \LeftLabel{Super}
    \UnaryInfC{$\kw{super} : \ty{Module}$}
\end{prooftree}
