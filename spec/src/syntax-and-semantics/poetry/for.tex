\subsubsection{Fold (For Expression)}
\label{sec:poetry-for}

What may be surprising about the \kw{for} loop in \Trilogy{} is that it may be an
expression, as well as a statement. In the expression form, it evaluates to a value,
and works as you might expect a left-fold function to work. The syntax is similar to
that of the regular \kw{for} loop (\S\ref{sec:prose-for}), but includes the additional
\kw{where} clause to provide the accumulator variable's name and initial value.

\begin{bnf*}
    \bnfprod{Fold}{
        \bnfts{\kw{for}}
        \bnfsp
        \bnfpn{Query}
        \bnfsp
        \bnfts{\kw{where}}
        \bnfsp
        \bnfpn{Identifier}
        \bnfsp
        \bnfts{=}
        \bnfsp
        \bnfpn{Expression}
        \bnfsp
        \bnfpn{Block}
        \bnfor
    }
\end{bnf*}

While the execution of this loop is much the same as a regular \kw{for} loop, the evaluation
is more like a fold: on the first iteration of the loop, the variable identified in the
\kw{where} clause is assigned its default value. On future iterations of the loop, the variable
is assigned the final value of the previous iteration. When there are no more iterations, the
final value of this accumulator variable is the resulting value of the loop.

\begin{prooftree}
    \AxiomC{$\Gamma\vdash E:\tau$}
    \AxiomC{$\Gamma\vdash V:\tau$}
    \LeftLabel{Fold}
    \BinaryInfC{$\Gamma\vdash \kw{for}\ Q\ \kw{where}\ i\ =\ V \{\ E\ \} : \tau$}
\end{prooftree}

More details on the behaviour of this loop are explained in the next section.
