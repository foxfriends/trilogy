\subsubsection{Closures}

Closures in \Trilogy{} come in both function closure and procedure closure forms.
Function closures, I feel, are a very legitimate construct. Procedure closures
on the other hand are even more definitely breaking the illusion of \Poetry{}'s
pureness than even including regular procedure calls, but if you squint really
hard you can almost let it go. Hopefully you can excuse this breach of mathematical
correctness for the matter of practicality supplied by the ability to define
procedure closures, and that this does not cause more issues than its worth
in implementation.

\begin{bnf*}
    \bnfprod{FnClosure}{
        \bnfts{\kw{fn}}
        \bnfsp
        \bnfpn{ParameterList}
        \bnfsp
        \bnfts{.}
        \bnfsp
        \bnfpn{Expr}
    } \\
    \bnfprod{DoClosure}{
        \bnfts{\kw{do}}
        \bnfsp
        \bnfts{(}
        \bnfsp
        \bnfpn{PatternList}
        \bnfsp
        \bnfts{)}
        \bnfsp
        \bnfpn{Block}
        \bnfor
    } \\
    \bnfmore{
        \bnfts{\kw{do}}
        \bnfsp
        \bnfts{(}
        \bnfsp
        \bnfpn{PatternList}
        \bnfsp
        \bnfts{)}
        \bnfsp
        \bnfpn{Expr}
    }
\end{bnf*}

The \kw{fn} desginates a function closure, the \kw{fn} expression evaluating
to a function as any other function definition, only that this function
captures bindings from its environment. These captured bindings may escape
the lifetime of their syntactic scope if the closure escapes that scope.

This function's body is specified as a single expression following the
parameter list. This expression continues until the first end-of-line that
allows the next line to parse were the closure ended. Hopefully this is not
too ambiguous while also making for ergonomic pipelines using the function
application operator (\op{|>}) with on-the-fly function closures. If a
function closure is not working as expected, parentheses will solve the
issue.

As with regular function definitions, \kw{fn} includes support for
multi-argument functions as a syntax transformation into single argument
functions returning further single argument functions.

\begin{align*}
    \texttt{\kw{fn} \$(\$r:pat)+ \$p:pat. \$e:expr} & \Rightarrow \texttt{\kw{fn} \$(\$r)+. \kw{fn} \$p. \$e} %
\end{align*}

Given that, only the semantics of single argument functions need to be specified.

\begin{prooftree}
    \AxiomC{$\Sigma\vdash X : \tau$}
    \AxiomC{$P = X$}
    \AxiomC{$\Gamma,\setof{a\given a \defby P}\vdash E : \rho$}
    \LeftLabel{Function Closure}
    \TrinaryInfC{$\Gamma\vdash \kw{fn}\ P\ \texttt{.}\ E : \tau \rightarrow \rho$}
\end{prooftree}

Procedure closures, desginated by \kw{do}, come in two forms, but work
similarly. In the form accepting a block, that block is used as the body
of the procedure, much the same as any regular procedure definition. The
\kw{return} keyword may be used in this block to return a value, otherwise
the procedure evaluates to \kw{unit} when called.

The form accepting an expression actually works more like a function, where
that expression is evaluated and returned automatically. This is particularly
to cover for the fact that function closures (and indeed functions in general)
may not be nullary, so to be able to defer a computation, a procedure is a
fine substitute.

Similarly to function closures, the procedure closure captures bindings from
its environment, and may allow those captured bindings to escape the lifetime
of their syntactic scopes if the closure escapes that scope.

In both forms of procedure closure, the first class keyword \kw{return} is rebound to
this new procedure. To capture a parent procedure's return statement for use in a
procedure closure, it must first be renamed by assigning it to another variable.

Despite the \kw{do} notation not using an exclamation mark, to eventually call
this procedure does require the exclamation mark, as with any other procedure call.
It is only omitted from the \kw{do} notation because it is ugly.

\begin{prooftree}
    \def\extraVskip{3.5pt}
    \AxiomC{$\Sigma\vdash X_i : \tau_i$}
    \AxiomC{$P_i = X_i$}
    \BinaryInfC{$\Gamma,\setof{a_i\given a_i \defby P_i}\vdash M \Rightarrow \Phi$}
    \AxiomC{$\Phi\vdash N : \rho$}
    \def\extraVskip{2pt}
    \LeftLabel{Procedure Closure}
    \BinaryInfC{$\Gamma\vdash \kw{do}\texttt{(}\ P_{1\hdots n}\ \texttt{)}\ \block{M\ \op{;}\ \kw{return}\ N} : \texttt{!(} \tau_{1\hdots n} \texttt{)} \rightarrow \rho$}
\end{prooftree}

\begin{prooftree}
    \AxiomC{$\Sigma\vdash X_i : \tau_i$}
    \AxiomC{$P_i = X_i$}
    \AxiomC{$\Gamma,\setof{a_i\given a_i \defby P_i}\vdash M : \top$}
    \LeftLabel{Procedure Closure}
    \TrinaryInfC{$\Gamma\vdash \kw{do}\texttt{(}\ P_{1\hdots n}\ \texttt{)}\ \block{M} : \texttt{!(} \tau_{1\hdots n} \texttt{)} \rightarrow \kw{unit}$}
\end{prooftree}

\begin{prooftree}
    \AxiomC{$\Sigma\vdash X_i : \tau_i$}
    \AxiomC{$P_i = X_i$}
    \AxiomC{$\Gamma,\setof{a_i\given a_i \defby P_i}\vdash E : \rho$}
    \LeftLabel{Procedure Closure}
    \TrinaryInfC{$\Gamma\vdash \kw{do}\texttt{(}\ P_{1\hdots n}\ \texttt{)}\ E : \texttt{!(} \tau_{1\hdots n} \texttt{)} \rightarrow \rho$}
\end{prooftree}
