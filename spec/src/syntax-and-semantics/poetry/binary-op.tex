\subsubsection{Binary Operation}
\label{sec:binop}

Binary operators in \Poetry{} are infix operators, and are always written
with both arguments. As previously mentioned, operator precedence is not
handled by the syntax tree but by another pass afterwards.

\begin{bnf*}
    \bnfprod{BinaryOp}{
        \bnfpn{Expression}
        \bnfsp
        \bnfpn{BinaryOperator}
        \bnfsp
        \bnfpn{Expression}
    } \\
    \bnfprod{BinaryOperator}{
        \bnfts{\kw{and}}
        \bnfor
        \bnfts{\kw{or}}
        \bnfor
        \bnfts{\op{+}}
        \bnfor
        \bnfts{\op{-}}
        \bnfor
        \bnfts{\op{*}}
        \bnfor
        \bnfts{\op{/}}
        \bnfor
        \bnfts{\op{\%}}
        \bnfor
        \bnfts{\op{**}}
        \bnfor
        \bnfts{\op{//}}
        \bnfor
    } \\
    \bnfmore{
        \bnfts{\op{==}}
        \bnfor
        \bnfts{\op{===}}
        \bnfor
        \bnfts{\op{<}}
        \bnfor
        \bnfts{\op{>}}
        \bnfor
        \bnfts{\op{>=}}
        \bnfor
        \bnfts{\op{<=}}
        \bnfor
    } \\
    \bnfmore{
        \bnfts{\op{\&}}
        \bnfor
        \bnfts{\op{|}}
        \bnfor
        \bnfts{\op{\textasciicircum}}
        \bnfor
        \bnfts{\op{<\textasciitilde}}
        \bnfor
        \bnfts{\op{\textasciitilde>}}
        \bnfor
        \bnfts{\op{,}}
        \bnfor
        \bnfts{\op{:}}
        \bnfor
        \bnfts{\op{.}}
        \bnfor
    } \\
    \bnfmore{
        \bnfts{\op{<>}}
        \bnfor
        \bnfts{\op{>>}}
        \bnfor
        \bnfts{\op{<<}}
        \bnfor
        \bnfts{\op{|>}}
        \bnfor
        \bnfts{\op{<|}}
    } \\
\end{bnf*}

Now comes the long and tedious process of defining each of these operators.
Worst part is that I know you know how these operators are supposed to work,
they're all the regular old stuff we see in every other language, but since
we are working with the primitive types, there's nothing we can do but to
define these things as internal magic, and so I guess it does require
specification somewhere.

Starting with the Boolean operators \kw{and} and \kw{or}, these are the typical
ones with behaviour that hardly needs explaining, and are short circuiting. If
one of the operands to either of these operators is evaluated and does not result
in a Boolean value, this is a runtime type error (if it does not get evaluated
due to short circuiting, the type error never happens).

\begin{prooftree}
    \AxiomC{$\Gamma\vdash A = \kw{false}$}
    \LeftLabel{And}
    \UnaryInfC{$\Gamma\vdash A\ \kw{and}\ B = \kw{false}$}
\end{prooftree}

\begin{prooftree}
    \AxiomC{$\Gamma\vdash A = \kw{true}$}
    \AxiomC{$\Gamma\vdash B = \kw{false}$}
    \LeftLabel{And}
    \BinaryInfC{$\Gamma\vdash A\ \kw{and}\ B = \kw{false}$}
\end{prooftree}

\begin{prooftree}
    \AxiomC{$\Gamma\vdash A = \kw{true}$}
    \AxiomC{$\Gamma\vdash B = \kw{true}$}
    \LeftLabel{And}
    \BinaryInfC{$\Gamma\vdash A\ \kw{and}\ B = \kw{true}$}
\end{prooftree}

\begin{prooftree}
    \AxiomC{$\Gamma\vdash A = \kw{true}$}
    \LeftLabel{Or}
    \UnaryInfC{$\Gamma\vdash A\ \kw{or}\ B = \kw{true}$}
\end{prooftree}

\begin{prooftree}
    \AxiomC{$\Gamma\vdash A = \kw{false}$}
    \AxiomC{$\Gamma\vdash B = \kw{true}$}
    \LeftLabel{Or}
    \BinaryInfC{$\Gamma\vdash A\ \kw{or}\ B = \kw{true}$}
\end{prooftree}

\begin{prooftree}
    \AxiomC{$\Gamma\vdash A = \kw{false}$}
    \AxiomC{$\Gamma\vdash B = \kw{false}$}
    \LeftLabel{Or}
    \BinaryInfC{$\Gamma\vdash A\ \kw{or}\ B = \kw{false}$}
\end{prooftree}

The mathematical operators are next, and as you might expect they work
as expected. Some details are worth noting however.

The mathematical operators work on values of Number type, if either operand
is not a Number, it causes a runtime type error.

Though we are strict about that, division by zero is not an error but instead
yields the exceptional result \val{'INF}. I am aware that division by zero is
explicitly defined as undefined, and is very definitely not mathematically
infinity, but \val{'INF} is not infinity, it is simply a signal that division
by zero was performed. This is the case for both regular divison (\op{/}) and
integer division (\op{//}), which floors to the nearest integer after dividing.

The \op{\%} operator is the remainder operator, returning the remainder of
integer division. Notably the remainder operator is the integer remainder operator,
not a modulus operator, and is implemented following the same floored division
as with integer division.

The \op{**} operator is the exponentiation operator and, given the fact that
precise rationals are available in \Trilogy{}, it also can be used for computing
square roots and such when given a fractional power. Since \Trilogy{} also
supports complex numbers, the result of something such as \texttt{-1 ** 1/2}
will not be an error, but actually the complex number \val{0i1}.

\begin{prooftree}
    \AxiomC{$\Gamma\vdash A = a : \ty{Number}$}
    \AxiomC{$\Gamma\vdash B = b : \ty{Number}$}
    \LeftLabel{Addition}
    \BinaryInfC{$\Gamma\vdash A\ \op{+}\ B = a + b : \ty{Number}$}
\end{prooftree}

\begin{prooftree}
    \AxiomC{$\Gamma\vdash A = a : \ty{Number}$}
    \AxiomC{$\Gamma\vdash B = b : \ty{Number}$}
    \LeftLabel{Subtraction}
    \BinaryInfC{$\Gamma\vdash A\ \op{-}\ B = a - b : \ty{Number}$}
\end{prooftree}

\begin{prooftree}
    \AxiomC{$\Gamma\vdash A = a : \ty{Number}$}
    \AxiomC{$\Gamma\vdash B = b : \ty{Number}$}
    \LeftLabel{Multiplication}
    \BinaryInfC{$\Gamma\vdash A\ \op{*}\ B = a \times b : \ty{Number}$}
\end{prooftree}

\begin{prooftree}
    \AxiomC{$\Gamma\vdash A = a : \ty{Number}$}
    \AxiomC{$\Gamma\vdash B = 0$}
    \LeftLabel{Division}
    \BinaryInfC{$\Gamma\vdash A\ \op{/}\ B \vartriangle \val{'INF}$}
\end{prooftree}

\begin{prooftree}
    \AxiomC{$\Gamma\vdash A = a : \ty{Number}$}
    \AxiomC{$\Gamma\vdash B = b : \ty{Number}$}
    \LeftLabel{Division}
    \BinaryInfC{$\Gamma\vdash A\ \op{/}\ B = a \div b : \ty{Number}$}
\end{prooftree}

\begin{prooftree}
    \AxiomC{$\Gamma\vdash A = a : \ty{Number}$}
    \AxiomC{$\Gamma\vdash B = 0$}
    \LeftLabel{Integer Division}
    \BinaryInfC{$\Gamma\vdash A\ \op{//}\ B \vartriangle \val{'INF}$}
\end{prooftree}

\begin{prooftree}
    \AxiomC{$\Gamma\vdash A = a : \ty{Number}$}
    \AxiomC{$\Gamma\vdash B = b : \ty{Number}$}
    \LeftLabel{Integer Division}
    \BinaryInfC{$\Gamma\vdash A\ \op{//}\ B = \lfloor a \div b \rfloor : \ty{Number}$}
\end{prooftree}

\begin{prooftree}
    \AxiomC{$\Gamma\vdash A = a : \ty{Number}$}
    \AxiomC{$\Gamma\vdash B = b : \ty{Number}$}
    \LeftLabel{Remainder}
    \BinaryInfC{$\Gamma\vdash A\ \op{\%}\ B = a - b \lfloor a \div b \rfloor : \ty{Number}$}
\end{prooftree}

\begin{prooftree}
    \AxiomC{$\Gamma\vdash A = a : \ty{Number}$}
    \AxiomC{$\Gamma\vdash B = b : \ty{Number}$}
    \LeftLabel{Exponentiation}
    \BinaryInfC{$\Gamma\vdash A\ \op{**}\ B = a^b : \ty{Number}$}
\end{prooftree}

