\subsubsection{Member Access}
\label{sec:member-access}

Some values contain other values within them, and those values are accessed using
the member access syntax. Though this uses the \op{.} operator, as seen in many
other languages, in \Trilogy{} that \op{.} truly is just an operator: the value on
the right side is interpreted as an expression like any other.

\begin{bnf*}
    \bnfprod{MemberAccess}{
        \bnfpn{Expression}
        \bnfsp
        \bnfts{\op{.}}
        \bnfsp
        \bnfpn{Expression}
        \bnfor
    }
\end{bnf*}

The left expression is the ``container'' expression, whiel the right is the
``member'' expression. The exact semantics of this operator depends on the
type of the container expression.

For Arrays, the member expression must evaluate to an integer
which is the index of the element to access, starting from 0 at the left.
If the index is an integer but is not within the bounds of the array,
\val{'MIA} is yielded instead. Meanwhile, if the index expression is
not an integer, it is considered a runtime type error.

\begin{prooftree}
    \AxiomC{$\Gamma\vdash E = \sigma : \ty{Array}\ \tau$}
    \AxiomC{$\Gamma\vdash i : \ty{Integer}$}
    \AxiomC{$0 \leq i < |\sigma|$}
    \LeftLabel{Array Access}
    \TrinaryInfC{$\Gamma\vdash E\ \texttt{.}\ i : \tau$}
\end{prooftree}

\begin{prooftree}
    \AxiomC{$\Gamma\vdash E : \ty{Array}\ \tau$}
    \AxiomC{$\Gamma\vdash i : \ty{Integer}$}
    \AxiomC{$i < 0 \lor |\sigma| < i$}
    \LeftLabel{Array MIA}
    \TrinaryInfC{$\Gamma\vdash E\ \texttt{.}\ i \vartriangle \val{'MIA}$}
\end{prooftree}

For Strings, similar to arrays, the member expression must evaluate
to an integer which is the index of the Character to access, starting from 0
at the left. If the index is an integer but is not within the bounds of the
string, \val{'MIA} is yielded instead. Meanwhile, if the index expression
is not an integer, it is considered a runtime type error.

\begin{prooftree}
    \AxiomC{$\Gamma\vdash E = \sigma : \ty{String}$}
    \AxiomC{$\Gamma\vdash i : \ty{Integer}$}
    \AxiomC{$0 \leq i < |\sigma|$}
    \LeftLabel{String Access}
    \TrinaryInfC{$\Gamma\vdash E\ \texttt{.}\ i : \ty{Char}$}
\end{prooftree}

\begin{prooftree}
    \AxiomC{$\Gamma\vdash E : \ty{String}$}
    \AxiomC{$\Gamma\vdash i : \ty{Integer}$}
    \AxiomC{$i < 0 \lor |\sigma| < i$}
    \LeftLabel{String MIA}
    \TrinaryInfC{$\Gamma\vdash E\ \texttt{.}\ i \vartriangle \val{'MIA}$}
\end{prooftree}

For Bits, the expression again must evaluate to an integer, which
is the index of the bit to access, starting from 0 at the right,
represented as a Boolean. If the index is an integer less than 0,
it is a runtime type error. Bits values are considered treated as
infinite length, so large integers will never be out of range, and
are treated as \kw{false}.

\begin{prooftree}
    \AxiomC{$\Gamma\vdash E = \sigma : \ty{Bits}$}
    \AxiomC{$\Gamma\vdash i : \ty{Integer}$}
    \AxiomC{$0 \leq i$}
    \LeftLabel{Bits Access}
    \TrinaryInfC{$\Gamma\vdash E\ \texttt{.}\ i : \ty{Bool}$}
\end{prooftree}

\begin{prooftree}
    \AxiomC{$\Gamma\vdash E = \sigma : \ty{Bits}$}
    \AxiomC{$\Gamma\vdash i : \ty{Integer}$}
    \AxiomC{$i < 0$}
    \LeftLabel{Bits MIA}
    \TrinaryInfC{$\Gamma\vdash E\ \texttt{.}\ i \vartriangle \val{'MIA}$}
\end{prooftree}

For Records, the member expression may be any value which is
the key of the element to access. If the key is not found in the
record, \val{'MIA} is yielded. Any value is valid as a record key,
so there are no runtime type errors for record access.

\begin{prooftree}
    \AxiomC{$\Gamma\vdash E : \ty{Record}\ \kappa:\nu$}
    \AxiomC{$\Gamma\vdash k : \kappa$}
    \LeftLabel{Record Access}
    \BinaryInfC{$\Gamma\vdash E\ \texttt{.}\ k : \nu$}
\end{prooftree}

If the value on the left is of any other type, it is a runtime type error;
other types of values do not have members which can be accessed.
