\subsubsection{Match}

The \kw{match} expression unifies a value with its cases, and evaluates the
branch associated with the first unification that succeeds. This is much
like the \kw{match} statement of \Prose{} but the branches hold expressions
instead of blocks.

As all expressions are required to evaluate to some value, the \kw{match}
expression requires a specific \kw{else} clause to be run in the situation
that no \kw{case} was selected.

\begin{bnf*}
    \bnfprod{Match}{
        \bnfts{\kw{match}}
        \bnfsp
        \bnfpn{Expression}
        \bnfsp
        \bnfpn{Cases}
        \bnfsp
        \bnfts{\kw{else}}
        \bnfsp
        \bnfpn{ElseCase}
    } \\
    \bnfprod{Cases}{
        \bnfpn{Case}\bnfsp\bnfpn{Cases}\bnfor\bnfpn{Case}
    } \\
    \bnfprod{Case}{
        \bnfts{\kw{case}}
        \bnfsp
        \bnfpn{Pattern}
        \bnfsp
        \bnfts{\kw{then}}
        \bnfsp
        \bnfpn{Expression}
        \bnfor
    } \\
    \bnfmore{
        \bnfts{\kw{case}}
        \bnfsp
        \bnfpn{Pattern}
        \bnfsp
        \bnfts{\kw{if}}
        \bnfsp
        \bnfpn{Expression}
        \bnfsp
        \bnfts{\kw{then}}
        \bnfsp
        \bnfpn{Expression}
        \bnfor
    } \\
    \bnfmore{
        \bnfts{\kw{case}}
        \bnfsp
        \bnfts{\kw{if}}
        \bnfsp
        \bnfpn{Expression}
        \bnfsp
        \bnfts{\kw{then}}
        \bnfsp
        \bnfpn{Expression}
    } \\
    \bnfprod{ElseCase}{
        \bnfts{\kw{else}}
        \bnfsp
        \bnfpn{Binding}
        \bnfsp
        \bnfts{\kw{then}}
        \bnfsp
        \bnfpn{Expression}
        \bnfor
    } \\
    \bnfmore{
        \bnfts{\kw{else}}
        \bnfsp
        \bnfpn{Wildcard}
        \bnfsp
        \bnfts{\kw{then}}
        \bnfsp
        \bnfpn{Expression}
    } \\
\end{bnf*}

The \kw{match} expression is evaluated in much the same way as the \kw{match} statement of
\Prose{} (\S\ref{sec:prose-match}). The only difference is that instead of executing the
matched block, the matched expression is evaluated and that expression's final result is the
evaluation of the \kw{match} expression as a whole.

\begin{prooftree}
    \AxiomC{$\Gamma \vdash E$}
    \AxiomC{$P = E$}
    \AxiomC{$\Gamma,\setof{a\given a \defby P} \vdash C = \kw{true}, B : \tau$}
    \LeftLabel{Match}
    \TrinaryInfC{$\Gamma \vdash \kw{match}\ E\ \kw{case}\ P\ \kw{if}\ C\ \kw{then}\ B : \tau$}
\end{prooftree}

\begin{prooftree}
    \def\extraVskip{3.5pt}
    \AxiomC{$\Gamma \vdash E$}
    \AxiomC{$P = E$}
    \AxiomC{$\Gamma,\setof{a\given a \defby P} \vdash C = \kw{false}$}
    \TrinaryInfC{$R_1 = \kw{case}\ P\ \kw{if}\ C\ \kw{then}\ B$}
    \AxiomC{$\Gamma \vdash \kw{match}\ E\ R_2 \cdots R_n : \tau$}
    \LeftLabel{Match\textsuperscript{\kw{false}}}
    \insertBetweenHyps{\hskip -12pt}
    \BinaryInfC{$\Gamma \vdash \kw{match}\ E\ R_1 R_2 \cdots R_n : \tau$}
\end{prooftree}

\vskip 0.5em

\begin{prooftree}
    \def\extraVskip{3.5pt}
    \AxiomC{$\Gamma \vdash E$}
    \AxiomC{$P \neq E$}
    \BinaryInfC{$R_1 = \kw{case}\ P\ \kw{if}\ C\ \kw{then}\ B$}
    \AxiomC{$\Gamma \vdash \kw{match}\ E\ R_2 \cdots R_n : \tau$}
    \LeftLabel{Match\textsuperscript{$\bot$}}
    \BinaryInfC{$\Gamma \vdash \kw{match}\ E\ R_1 R_2 \cdots R_n : \tau$}
\end{prooftree}

The omitted clauses of the \kw{case} statement are formalized by the same syntax
transformation as before:

\begin{align*}
    \texttt{\kw{case} \$p:pat} &\Rightarrow \texttt{\kw{case} \$p \kw{if} \kw{true}} \\ %
    \texttt{\kw{case} \kw{if} \$c:expr} &\Rightarrow \texttt{\kw{case} \_ \kw{if} \$c} \\ %
    \texttt{\kw{else} \_ \kw{then}} &\Rightarrow \texttt{\kw{case} \_ \kw{then}} \\ %
    \texttt{\kw{else} \$i:id \kw{then}} &\Rightarrow \texttt{\kw{case} \$i \kw{then}} \\ %
\end{align*}
