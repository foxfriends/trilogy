\subsubsection{Templates}

Templates are strings into which other expressions are evaluated and then inserted.
That is, templates are \Trilogy{}'s version of string interpolation.

Templates are not just for string interpolation, however, as they may be \emph{tagged},
allowing them to run user-defined code instead of simple string concatenation,
resulting in non-string results. This idea comes right out of Javascript. A tagged
template is a regular (bare) template followed directly by an identifier, the tag.

The wonky looking template tokens from the scanner pay off here, making parsing
relatively easy for templates.

\begin{bnf*}
    \bnfprod{Template}{
        \bnfpn{BareTemplate}
        \bnfsp
        \bnfpn{Identifier}
        \bnfor
        \bnfpn{BareTemplate}
    } \\
    \bnfprod{BareTemplate}{
        \bnfpn{TemplateStart}
        \bnfsp
        \bnfpn{TemplateMore}
        \bnfor
        \bnfpn{DollarString}
    } \\
    \bnfprod{TemplateMore}{
        \bnfpn{Expr}
        \bnfsp
        \bnfpn{TemplateMiddle}
        \bnfsp
        \bnfpn{TemplateMore}
        \bnfor
    } \\
    \bnfmore{
        \bnfpn{Expr}
        \bnfsp
        \bnfpn{TemplateEnd}
    }
\end{bnf*}

The semantics are quite different depending on whether the template was tagged
or not. In the untagged case, a template evaluates to a string by evaluating each
inner expression and converting it to its string representation, then concatenating
those strings as if with glue.

Unfortunately, templates are a bit hard to represent formally, so excuse these wonky
proofs and do your best to interpret them sanely.

\begin{prooftree}
    \AxiomC{$\Gamma\vdash E = v$}
    \AxiomC{$\Gamma\vdash T = \omega$}
    \AxiomC{$\Gamma\vdash S = \sigma\ \op{<>}\ \text{str}(v)\ \op{<>}\ \omega$}
    \LeftLabel{Template}
    \TrinaryInfC{$\Gamma\vdash \sigma\ E\ T = S$}
\end{prooftree}

A tagged template, on the other hand, does not stringify anything. Instead, the
string fragments and the evaluated values are both passed as the parameters to
the function denoted by the tag. This tag function should accept two parameters,
the first the array of string fragments, the second the array of the evaluated
values.

\begin{prooftree}
    \def\defaultHypSeparation{\hskip 0.1in}
    \AxiomC{$\Gamma\vdash E_i = v_i$}
    \AxiomC{$\Gamma\vdash f : \ty{List String} \rightarrow \ty{List}\ \tau \rightarrow \rho$}
    \UnaryInfC{$\Gamma\vdash f\ \texttt{[}\sigma_0\texttt{,}\ \sigma_1\texttt{,}\ \cdots\texttt{,}\ \sigma_n\texttt{]}\ \texttt{[}v_1\texttt{,}\ \cdots\texttt{,}\ v_n\texttt{]} : \rho$}
    \LeftLabel{Tagged Template}
    \BinaryInfC{$\Gamma\vdash \sigma_0\ (E_i\ \sigma_i)^*\ f : \rho$}
\end{prooftree}
