\subsection{Poetry}

As \Law{} is a language of binding, and \Prose{} a language of control
flow, \Poetry{} is a language of value. As a pure functional programming
language, \Poetry{} is highly expression-based, though given the existence
of the effect system it is possible to trigger side effects, in a way.

\begin{bnf*}
    \bnfprod{FunctionHead}{
        \bnfpn{Identifier}
        \bnfsp
        \bnfpn{ParameterList}
    } \\
    \bnfprod{ParameterList}{
        \bnfpn{Pattern}
        \bnfsp
        \bnfpn{PatternList}
        \bnfor
        \bnfpn{Pattern}
        \bnfor
        \bnfes
    } \\
    \bnfprod{FunctionBody}{\bnfpn{Expression}}
\end{bnf*}

Similar to Haskell, functions in \Poetry{} are defined as equations. Defining
the same name multiple times is allowed, given that its number of arguments
is the same, and the patterns of its parameters are changed, allowing different
cases of a function to be handled by different expressions. When the function is
applied, the first matching definition in source code order is used, and the
others are skipped.

Also similar to Haskell, functions support currying and partial application by
default. In \Trilogy{}, this is defined as a syntax transformation over function
definitions with multiple parameters into unary functions returning a function
that accepts the next parameter.

\begin{align*}
    \texttt{\$f:id \$(\$r:pat)* \$p:pat = \$e:expr} & \Rightarrow \texttt{\$f \$(\$r)* = \kw{fn} \$p = \$e} %
\end{align*}

The parameter list may be empty, defining a nullary (0-argument) function. As
\Poetry{} is lazy, this function will only be evaluated when its result is
needed, so even a nullary function may, for example, yield an effect which will
only be performed when it is required. Note that the lazy evaluation of \Poetry{}
can lead to some code being run in unexpected order, so it is recommended to
think carefully about whether effects in \Poetry{} are required.

Though \Poetry{} is lazy, \Prose{} and \Law{} (except for through Iterators)
are not. When an expression of \Poetry{} appears in \Prose{} or \Law{},
as in an assignment or unification, it is evaluated immediately. This implies
that while \Poetry{} may be used to represent infinite data structures those
data structures must remain in \Poetry{} or else they will be infinitely
evaluated, unless converted to Iterator.

\begin{prooftree}
    \AxiomC{$f\ P\ \texttt{=}\ E$}
    \AxiomC{$P = X$}
    \AxiomC{$\Gamma,\setof{a\given a\defby P}\vdash E = v$}
    \LeftLabel{Function}
    \TrinaryInfC{$\Gamma\vdash f(X) = v$}
\end{prooftree}

\subsubsection{Expressions}
\label{sec:expressions}

Seeing as the entirety of \Poetry{} (and admittedly, much of \Trilogy{} as a whole)
is made up of expressions, there are a lot of cases.

Though the expression grammar does include binary operators and other binary keywords
\Trilogy{} does not encode precedence as part of the grammar but instead resolves that
in a later pass in the interpreter.

\begin{bnf*}
    \bnfprod{Expression}{
        \bnfpn{Expr}
        \bnfor
        \bnfpn{Expr}
        \bnfsp
        \bnfpn{EffectHandlers}
    } \\
    \bnfprod{Expr}{
        \bnfpn{Primitive}
        \bnfor
        \bnfpn{Compound}
        \bnfor
        \bnfpn{Comprehension}
        \bnfor
    } \\
    \bnfmore{
        \bnfpn{Reference}
        \bnfor
        \bnfpn{Application}
        \bnfor
        \bnfpn{ProcedureCall}
        \bnfor
    } \\
    \bnfmore{
        \bnfpn{UnaryOp}
        \bnfor
        \bnfpn{BinOp}
        \bnfor
        \bnfpn{Let}
        \bnfor
        \bnfpn{IfElse}
        \bnfor
        \bnfpn{Match}
        \bnfor
    } \\
    \bnfmore{
        \bnfpn{Yield}
        \bnfor
        \bnfpn{Resume}
        \bnfor
        \bnfpn{Return}
        \bnfor
        \bnfpn{End}
        \bnfor
    } \\
    \bnfmore{
        \bnfpn{FnClosure}
        \bnfor
        \bnfpn{DoClosure}
        \bnfor
        \bnfts{(}
        \bnfsp
        \bnfpn{Expression}
        \bnfsp
        \bnfts{)}
    }
\end{bnf*}


\subsubsection{Literals}
\label{sec:literals}

Starting easy, the simplest expression is a primitive value.

\begin{bnf*}
    \bnfprod{Primitive}{
        \bnfts{\kw{unit}}
        \bnfor
        \bnfts{\kw{false}}
        \bnfor
        \bnfts{\kw{true}}
        \bnfor
        \bnfpn{Numeric}
        \bnfor
    } \\
    \bnfmore{
        \bnfpn{Atom}
        \bnfor
        \bnfpn{String}
        \bnfor
        \bnfpn{Character}
        \bnfor
        \bnfpn{Bits}
    }
\end{bnf*}

The evaluation of a literal primitive value is simply the value it represents.
For completeness, some trivial specifications:

\begin{figure}[H]
    \centering
    \parbox[t]{0.3\linewidth}{
        \begin{prooftree}
            \AxiomC{}
            \LeftLabel{Unit}
            \UnaryInfC{$\kw{unit} : \ty{Unit}$}
        \end{prooftree}
    }
    \parbox[t]{0.3\linewidth}{
        \begin{prooftree}
            \AxiomC{}
            \LeftLabel{Boolean}
            \UnaryInfC{$\kw{false} : \ty{Bool}$}
        \end{prooftree}
    }
    \parbox[t]{0.3\linewidth}{
        \begin{prooftree}
            \AxiomC{}
            \LeftLabel{Boolean}
            \UnaryInfC{$\kw{true} : \ty{Bool}$}
        \end{prooftree}
    }
    \parbox[t]{0.4\linewidth}{
        \begin{prooftree}
            \AxiomC{}
            \LeftLabel{Numeric}
            \UnaryInfC{$\langle\text{Numeric}\rangle : \ty{Number}$}
        \end{prooftree}
    }
    \parbox[t]{0.4\linewidth}{
        \begin{prooftree}
            \AxiomC{}
            \LeftLabel{String}
            \UnaryInfC{$\langle\text{String}\rangle : \ty{String}$}
        \end{prooftree}
    }
    \parbox[t]{0.4\linewidth}{
        \begin{prooftree}
            \AxiomC{}
            \LeftLabel{Atom}
            \UnaryInfC{$\langle\text{Atom}\rangle : \ty{Atom}$}
        \end{prooftree}
    }
    \parbox[t]{0.4\linewidth}{
        \begin{prooftree}
            \AxiomC{}
            \LeftLabel{Character}
            \UnaryInfC{$\langle\text{Character}\rangle : \ty{Char}$}
        \end{prooftree}
    }
    \parbox[t]{0.4\linewidth}{
        \begin{prooftree}
            \AxiomC{}
            \LeftLabel{Bits}
            \UnaryInfC{$\langle\text{Bits}\rangle : \ty{Bits}$}
        \end{prooftree}
    }
\end{figure}

Of similar literal-ness are the literals for compound types. Notably
tuples are missing from this section, as they are implemented in terms
of the cons operator, rather than a literal form.

\begin{bnf*}
    \bnfprod{Compound}{
        \bnfpn{Struct}
        \bnfor
        \bnfpn{Array}
        \bnfor
        \bnfpn{Set}
        \bnfor
        \bnfpn{Record}
    } \\
    \bnfprod{Struct}{
        \bnfpn{Atom}
        \bnfsp
        \bnfts{(}
        \bnfsp
        \bnfpn{Expression}
        \bnfsp
        \bnfts{)}
    } \\
    \bnfprod{Array}{
        \bnfts{[}
        \bnfsp
        \bnfpn{CompoundElements}
        \bnfsp
        \bnfts{]}
    } \\
    \bnfprod{Set}{
        \bnfts{[|}
        \bnfsp
        \bnfpn{CompoundElements}
        \bnfsp
        \bnfts{|]}
    } \\
    \bnfprod{CompoundElements}{
        \bnfpn{CompoundElement}
        \bnfsp
        \bnfts{,}
        \bnfsp
        \bnfpn{CompoundElements}
        \bnfor
    } \\
    \bnfmore{
        \bnfpn{CompoundElement}
        \bnfor
        \bnfes
    } \\
    \bnfprod{CompoundElement}{
        \bnfpn{Expression}
        \bnfor
        \bnfts{\op{..}}
        \bnfsp
        \bnfpn{Expression}
    } \\
    \bnfprod{Record}{
        \bnfts{\{|}
        \bnfsp
        \bnfpn{KeyValueElements}
        \bnfsp
        \bnfts{|\}}
    } \\
    \bnfprod{KeyValueElements}{
        \bnfpn{KeyValueElement}
        \bnfsp
        \bnfts{,}
        \bnfsp
        \bnfpn{KeyValueElements}
        \bnfor
    } \\
    \bnfmore{
        \bnfpn{KeyValueElement}
        \bnfor
        \bnfes
    } \\
    \bnfprod{KeyValueElement}{
        \bnfpn{Expression}
        \bnfsp
        \bnfts{\op{=>}}
        \bnfsp
        \bnfpn{Expression}
        \bnfor
        \bnfts{\op{..}}
        \bnfsp
        \bnfpn{Expression}
    } \\
\end{bnf*}

As you might expect, the evaluation of a compound literal is a value of the
compound type, where the items it contains are the values the evaluate to.
The only thing of particular interest is the use of spread (\op{..}) in
Array, Set, and Record literals, for which the expression must evaluate
to a value of the same compound type, and that value is then inserted
as if its elements were written in place of the spread.

\begin{prooftree}
    \AxiomC{$\Gamma\vdash E : \tau$}
    \LeftLabel{Struct}
    \UnaryInfC{$\Gamma\vdash\texttt{'}l\ \texttt{(}\ E\ \texttt{)} : \texttt{'} l(\tau)$}
\end{prooftree}

\begin{prooftree}
    \AxiomC{}
    \LeftLabel{Array\textsuperscript{$\epsilon$}}
    \UnaryInfC{$\texttt{[]} : \ty{List}\ \tau$}
\end{prooftree}

\begin{prooftree}
    \AxiomC{$\Gamma\vdash X_1 : \tau$}
    \AxiomC{$\Gamma\vdash \texttt{[}X_2\texttt{,}\ \cdots\texttt{,}\ X_n\texttt{]} : \ty{List}\ \tau$}
    \LeftLabel{Array}
    \BinaryInfC{$\Gamma\vdash\texttt{[}X_1\texttt{,}\ X_2\texttt{,}\ \cdots\texttt{,}\ X_n\texttt{]} : \ty{List}\ \tau$}
\end{prooftree}

\begin{prooftree}
    \AxiomC{$\Gamma\vdash X_1 : \ty{List}\ \tau$}
    \AxiomC{$\Gamma\vdash \texttt{[}X_2\texttt{,}\ \cdots\texttt{,}\ X_n\texttt{]} : \ty{List}\ \tau$}
    \LeftLabel{Array\textsuperscript{\op{..}}}
    \BinaryInfC{$\Gamma\vdash\texttt{[}\op{..}X_1\texttt{,}\ X_2\texttt{,}\ \cdots\texttt{,}\ X_n\texttt{]} : \ty{List}\ \tau$}
\end{prooftree}

\begin{prooftree}
    \AxiomC{}
    \LeftLabel{Set\textsuperscript{$\epsilon$}}
    \UnaryInfC{$\texttt{[||]} : \ty{Set}\ \tau$}
\end{prooftree}

\begin{prooftree}
    \AxiomC{$\Gamma\vdash X_1 : \tau$}
    \AxiomC{$\Gamma\vdash \texttt{[|}\ X_2\ \texttt{,}\cdots\texttt{,}\ X_n\ \texttt{|]} : \ty{Set}\ \tau$}
    \LeftLabel{Set}
    \BinaryInfC{$\Gamma\vdash\texttt{[|}\ X_1\ \texttt{,}\ X_2\ \texttt{,}\cdots\texttt{,}\ X_n\ \texttt{|]} : \ty{Set}\ \tau$}
\end{prooftree}

\begin{prooftree}
    \AxiomC{$\Gamma\vdash X_1 : \ty{Set}\ \tau$}
    \AxiomC{$\Gamma\vdash \texttt{[|}\ X_2\ \texttt{,}\cdots\texttt{,}\ X_n\ \texttt{|]} : \ty{Set}\ \tau$}
    \LeftLabel{Set\textsuperscript{\op{..}}}
    \BinaryInfC{$\Gamma\vdash\texttt{[|}\ \op{..}X_1\ \texttt{,}\ X_2\ \texttt{,}\cdots\texttt{,}\ X_n\ \texttt{|]} : \ty{Set}\ \tau$}
\end{prooftree}

\begin{prooftree}
    \AxiomC{}
    \LeftLabel{Record\textsuperscript{$\epsilon$}}
    \UnaryInfC{$\texttt{\{||\}} : \ty{Record}\ \kappa\ \nu$}
\end{prooftree}

\begin{prooftree}
    \def\defaultHypSeparation{\hskip 0in}
    \AxiomC{$\Gamma\vdash K_1 : \kappa$}
    \AxiomC{$\Gamma\vdash V_1 : \nu$}
    \AxiomC{$\Gamma\vdash \texttt{\{|}K_2:V_2\ \texttt{,}\cdots\texttt{,}\ K_n:V_n\texttt{|\}} : \ty{Record}\ \kappa\ \nu$}
    \LeftLabel{Record}
    \TrinaryInfC{$\Gamma\vdash\texttt{\{|}K_1\ \texttt{=>}\ V_1\ \texttt{,}\ K_2\ \texttt{=>}\ V_2\ \texttt{,}\cdots\texttt{,}\ K_n\ \texttt{=>}\ V_n\texttt{|\}} : \ty{Record}\ \kappa\ \nu$}
\end{prooftree}

\begin{prooftree}
    \def\defaultHypSeparation{\hskip 0in}
    \AxiomC{$\Gamma\vdash X : \ty{Record}\ \kappa\ \nu$}
    \AxiomC{$\Gamma\vdash \texttt{\{|}K_2\ \texttt{=>}\ V_2\ \texttt{,}\cdots\texttt{,}\ K_n\ \texttt{=>}\ V_n\texttt{|\}} : \ty{Record}\ \kappa\ \nu$}
    \LeftLabel{Record\textsuperscript{\op{..}}}
    \BinaryInfC{$\Gamma\vdash\texttt{\{|}\op{..}X\ \texttt{,}\ K_2\ \texttt{=>}\ V_2\ \texttt{,}\cdots\texttt{,}\ K_n\ \texttt{=>}\ V_n\texttt{|\}} : \ty{Record}\ \kappa\ \nu$}
\end{prooftree}


\subsubsection{Comprehension}

Comprehension is a syntax by which collections can be iterated, filtered, and updated.
In \Trilogy{}, comprehensions also allow lists to be generated; a special power afforded
by \Law{} that other languages' comprehensions cannot do.

The syntax of comprehension is a bit like a \kw{for} loop in a compound literal:

\begin{bnf*}
    \bnfprod{Comprehension}{
        \bnfpn{ArrayComprehension}
        \bnfor
    } \\
    \bnfmore{
        \bnfpn{SetComprehension}
        \bnfor
    } \\
    \bnfmore{
        \bnfpn{RecordComprehension}
        \bnfor
    } \\
    \bnfmore{
        \bnfpn{IteratorComprehension}
    } \\
    \bnfprod{ArrayComprehension}{
        \bnfts{[}
        \bnfsp
        \bnfpn{Expression}
        \bnfsp
        \bnfts{\kw{for}}
        \bnfsp
        \bnfpn{Query}
        \bnfsp
        \bnfts{]}
    } \\
    \bnfprod{SetComprehension}{
        \bnfts{\{}
        \bnfsp
        \bnfpn{Expression}
        \bnfsp
        \bnfts{\kw{for}}
        \bnfsp
        \bnfpn{Query}
        \bnfsp
        \bnfts{\}}
    } \\
    \bnfprod{RecordComprehension}{
        \bnfts{\{}
        \bnfsp
        \bnfpn{Expression}
        \bnfsp
        \bnfts{:}
        \bnfsp
        \bnfpn{Expression}
        \bnfsp
        \bnfts{\kw{for}}
        \bnfsp
        \bnfpn{Query}
        \bnfsp
        \bnfts{\}}
    } \\
    \bnfprod{IteratorComprehension}{
        \bnfts{\$(} %
        \bnfsp
        \bnfpn{Expression}
        \bnfsp
        \bnfsp
        \bnfts{\kw{for}}
        \bnfsp
        \bnfpn{Query}
        \bnfsp
        \bnfts{)}
    }
\end{bnf*}

The expression and the bindings introduced to that expression are in reverse in
this syntax, one of the only places in \Trilogy{} where that is the case. The
query in each form of comprehension introduces the bindings which may be used
in the expression which generates the collection that is being comprehended.

Where most languages allow iterator terms and filter terms for their comprehensions,
\Trilogy{} requires none of that, instead opting to use \Law{}'s queries to allow
filtering, iterating, and generating all with a familiar syntax.

\begin{prooftree}
    \AxiomC{$\Gamma\vdash U$}
    \AxiomC{$U = Q$}
    \AxiomC{$\Gamma,\setof{a\given a \defby Q}\vdash E : \tau$}
    \LeftLabel{List Comprehension}
    \TrinaryInfC{$\Gamma\vdash \texttt{[}\ E\ \kw{for}\ Q\ \texttt{]} : \ty{List}\ \tau$}
\end{prooftree}

\begin{prooftree}
    \AxiomC{$\Gamma\vdash U$}
    \AxiomC{$U = Q$}
    \AxiomC{$\Gamma,\setof{a\given a \defby Q}\vdash E : \tau$}
    \LeftLabel{Set Comprehension}
    \TrinaryInfC{$\Gamma\vdash \texttt{\{}\ E\ \kw{for}\ Q\ \texttt{\}} : \ty{Set}\ \tau$}
\end{prooftree}

\begin{prooftree}
    \AxiomC{$\Gamma\vdash U$}
    \AxiomC{$U = Q$}
    \AxiomC{$\Gamma,\setof{a\given a \defby Q}\vdash K : \kappa, V : \nu$}
    \LeftLabel{Record Comprehension}
    \TrinaryInfC{$\Gamma\vdash \texttt{\{}\ K\ \texttt{:}\ V\ \kw{for}\ Q\ \texttt{\}} : \ty{Record}\ \kappa:\nu$}
\end{prooftree}

\begin{prooftree}
    \AxiomC{$\Gamma\vdash U$}
    \AxiomC{$U = Q$}
    \AxiomC{$\Gamma,\setof{a\given a \defby Q}\vdash E : \tau$}
    \LeftLabel{Iterator Comprehension}
    \TrinaryInfC{$\Gamma\vdash \texttt{\$(}\ E\ \kw{for}\ Q\ \texttt{)} : \ty{Iter}\ \tau$}
\end{prooftree}


\subsubsection{Reference}

Defined names may be referenced at any time, their naming evaluating to the value
they currently hold. This is a typical variable reference as in any language.
Notably this is \emph{only} for variable references, looking up a name in another
scope is handled by the member access.

As a bit of a special case, the built-in operators may be referenced as if they
were functions by wrapping them in parentheses. This allows operators to be passed
as values and applied as if they were functions to their arguments, occasionally
useful in some functional compositions. The operators themselves are specified
in \S\ref{sec:unaryop} and \S\ref{sec:binop}

\begin{bnf*}
    \bnfprod{Reference}{
        \bnfpn{Identifier}
        \bnfor
        \bnfts{(}
        \bnfsp
        \bnfpn{UnaryOperator}
        \bnfsp
        \bnfts{)}
        \bnfor
        \bnfts{(}
        \bnfsp
        \bnfpn{BinaryOperator}
        \bnfsp
        \bnfts{)}
    }
\end{bnf*}

Admittedly the following semantics look a little wonky due to an ambiguous choice
of notation. Rest assured, in the hypotheses the $x$ refers to the binding in the
context, while in the inference the $x$ refers to the symbol in source code, so
this is not an infinitely recursive deduction.

\begin{prooftree}
    \AxiomC{$\Gamma\vdash x : \tau$}
    \LeftLabel{Reference}
    \UnaryInfC{$\Gamma\vdash x : \tau$}
\end{prooftree}

% TODO: for completeness, include semantics of each parenthesized operator? Or
% is that just a pain...


\subsubsection{Member Access}
\label{sec:member-access}

Some values contain other values within them, and those values are accessed using
the member access syntax. There are a few variations on this syntax, allowing for
some flexibility the structure of your data.

\begin{bnf*}
    \bnfprod{MemberAccess}{
        \bnfpn{Expression}
        \bnfsp
        \bnfts{\op{.}}
        \bnfsp
        \bnfpn{Identifier}
        \bnfor
    } \\
    \bnfmore{
        \bnfpn{Expression}
        \bnfsp
        \bnfts{\op{.[}}
        \bnfsp
        \bnfpn{Expression}
        \bnfsp
        \bnfts{]}
    }
\end{bnf*}

The only form of particular relevance is the second, as the first can be
transformed by interpreting the identifier as an atom and passing that
in the brackets:

\begin{align*}
    \texttt{\$c:expr . \$i:id} & \Rightarrow \texttt{\$c .[ ' \$i ]} %
\end{align*}

After that, the way the lookup works depends on the type of the expression
on the left.

For Arrays, the expression in the brackets must evaluate to an integer
which is the index of the element to access, starting from 0 at the left.
If the index is an integer but is not within the bounds of the array,
\val{'MIA} is yielded instead. Meanwhile, if the index expression is
not an integer, it is considered a runtime type error.

\begin{prooftree}
    \AxiomC{$\Gamma\vdash E = \sigma : \ty{Array}\ \tau$}
    \AxiomC{$\Gamma\vdash i : \ty{Integer}$}
    \AxiomC{$0 \leq i < |\sigma|$}
    \LeftLabel{Array Access}
    \TrinaryInfC{$\Gamma\vdash E\ \texttt{.[}\ i\ \texttt{]} : \tau$}
\end{prooftree}

\begin{prooftree}
    \AxiomC{$\Gamma\vdash E : \ty{Array}\ \tau$}
    \AxiomC{$\Gamma\vdash i : \ty{Integer}$}
    \AxiomC{$i < 0 \lor |\sigma| < i$}
    \LeftLabel{Array MIA}
    \TrinaryInfC{$\Gamma\vdash E\ \texttt{.[}\ i\ \texttt{]} \vartriangle \val{'MIA}$}
\end{prooftree}

For Strings, similar to arrays, the expression in the brackets must evaluate
to an integer which is the index of the Character to access, starting from 0
at the left. If the index is an integer but is not within the bounds of the
string, \val{'MIA} is yielded instead. Meanwhile, if the index expression
is not an integer, it is considered a runtime type error.

\begin{prooftree}
    \AxiomC{$\Gamma\vdash E = \sigma : \ty{String}$}
    \AxiomC{$\Gamma\vdash i : \ty{Integer}$}
    \AxiomC{$0 \leq i < |\sigma|$}
    \LeftLabel{String Access}
    \TrinaryInfC{$\Gamma\vdash E\ \texttt{.[}\ i\ \texttt{]} : \ty{Char}$}
\end{prooftree}

\begin{prooftree}
    \AxiomC{$\Gamma\vdash E : \ty{String}$}
    \AxiomC{$\Gamma\vdash i : \ty{Integer}$}
    \AxiomC{$i < 0 \lor |\sigma| < i$}
    \LeftLabel{String MIA}
    \TrinaryInfC{$\Gamma\vdash E\ \texttt{.[}\ i\ \texttt{]} \vartriangle \val{'MIA}$}
\end{prooftree}

For Bits, the expression again must evaluate to an integer, which
is the index of the bit to access, starting from 0 at the right,
represented as a Boolean. If the index is an integer less than 0,
it is a runtime type error. Bits values are considered treated as
infinite length, so large integers will never be out of range, and
are treated as \kw{false}.

\begin{prooftree}
    \AxiomC{$\Gamma\vdash E = \sigma : \ty{Bits}$}
    \AxiomC{$\Gamma\vdash i : \ty{Integer}$}
    \AxiomC{$0 \leq i$}
    \LeftLabel{Bits Access}
    \TrinaryInfC{$\Gamma\vdash E\ \texttt{.[}\ i\ \texttt{]} : \ty{Bool}$}
\end{prooftree}

\begin{prooftree}
    \AxiomC{$\Gamma\vdash E = \sigma : \ty{Bits}$}
    \AxiomC{$\Gamma\vdash i : \ty{Integer}$}
    \AxiomC{$i < 0$}
    \LeftLabel{Bits MIA}
    \TrinaryInfC{$\Gamma\vdash E\ \texttt{.[}\ i\ \texttt{]} \vartriangle \val{'MIA}$}
\end{prooftree}

For Records, the expression in the brackets may be any value which is
the key of the element to access. If the key is not found in the
record, \val{'MIA} is yielded. Any value is valid as a record key,
so there are no runtime type errors for record access.

\begin{prooftree}
    \AxiomC{$\Gamma\vdash E : \ty{Record}\ \kappa:\nu$}
    \AxiomC{$\Gamma\vdash k : \kappa$}
    \LeftLabel{Record Access}
    \BinaryInfC{$\Gamma\vdash E\ \texttt{.[}\ k\ \texttt{]} : \nu$}
\end{prooftree}

If the value on the left is of any other type, it is a runtime type error;
other types of values do not have members which can be accessed.


\subsubsection{Application}

Functions in \Poetry{} are single-argument, and that argument is supplied
simply by writing it after the function, separated by a space. The function
itself may be an expression, including another application expression, allowing
for multi-argument functions to be simulated by passing further space separated
parameters.

\begin{bnf*}
    \bnfprod{Application}{
        \bnfpn{Expr}
        \bnfsp
        \bnfpn{Expr}
    }
\end{bnf*}

The first expression must evaluate to a value of Function type, otherwise
it is a runtime type error.

\begin{prooftree}
    \AxiomC{$\Gamma\vdash F : \tau \rightarrow \rho$}
    \AxiomC{$\Gamma\vdash E : \tau$}
    \LeftLabel{Application}
    \BinaryInfC{$\Gamma\vdash F\ E : \rho$}
\end{prooftree}

In terms of precedence, the expression \texttt{f~a~b~c} is equivalent to
\texttt{((f~a)~b)~c}. Parentheses and the application operator (\op{|>},
explored in \S\ref{sec:binop}) may be used to manipulate the precedence as
necessary.


\subsubsection{Procedure Call}

Procedures may be called in expressions, even in functions. Though this sort of
breaks the ``pureness'' of \Poetry{}, I think it's also sort of justifiable if you
consider the procedure call to actually be syntax sugar for a \kw{yield} that
corresponds to a handler that calls that procedure and resumes with the result.
Assume that handler is applied automatically by the procedure's definition.

\begin{bnf*}
    \bnfprod{ProcedureCall}{
        \bnfpn{Expression}
        \bnfsp
        \bnfts{!(}
        \bnfsp
        \bnfpn{ArgumentList}
        \bnfsp
        \bnfts{)}
    } \\
    \bnfprod{ArgumentList}{
        \bnfpn{Expression}
        \bnfsp
        \bnfts{,}
        \bnfsp
        \bnfpn{ArgumentList}
        \bnfor
        \bnfpn{Expression}
        \bnfor
        \bnfes
    }
\end{bnf*}

Semantically, a procedure call looks up an identifier in the current scope to
locate a procedure, and then supplies the provided arguments as the parameters
to the procedure. This is much the same as any function call in any other
imperative language.

\begin{prooftree}
    \AxiomC{$\Gamma\vdash p : \texttt{!}(x_{1\hdots n})\rightarrow\tau$}
    \AxiomC{$a_i = x_i$}
    \LeftLabel{Procedure Call}
    \BinaryInfC{$\Gamma\vdash p\texttt{!}(a_{1\hdots n}):\tau$}
\end{prooftree}


\subsubsection{Unary Operation}
\label{sec:unaryop}

There are a few unary operators in \Poetry{}, all of which are prefix operators.
There are no postfix operators in \Trilogy{}, as they end up being a parsing
nightmare for little value (the only commonly seen postfix operators are
ones like \texttt{++}, and we prefer \texttt{+= 1} in general these days).

\begin{bnf*}
    \bnfprod{UnaryOp}{
        \bnfpn{UnaryOperator}
        \bnfsp
        \bnfpn{Expression}
    } \\
    \bnfprod{UnaryOperator}{
        \bnfts{\op{-}}
        \bnfor
        \bnfpn{OnlyUnaryOperator}
    } \\
    \bnfprod{OnlyUnaryOperator}{
        \bnfts{\kw{not}}
        \bnfor
        \bnfts{\op{\textasciitilde}}
        \bnfor
        \bnfts{\kw{yield}}
    }
\end{bnf*}

It turns out, even prefix unary operators cause some parsing ambiguity, particularly
when it comes to application of a function to a negative number. To resolve this
ambiguity, we simply declare unary operators to be the last choice when parsing;
only if there is no other interpretation of the expression but the interpretation
where the operator is used as a unary operator will it be a interpreted unary operation.

To illustrate that point, the following table summarizes some syntax and a
less ambiguous ``formal'' representation. The formal representation is actually
valid \Trilogy{} code as well, showing that \Trilogy{} could be used like a Lisp
if you really wanted to.

\begin{table}[H]
    \centering
    \begin{tabular}{lll}
        \hline
        \textbf{Conventional} & \textbf{Formal} & \textbf{Explanation} \\
        \hline
        \tri{x - 5} & \tri{((-) x 5)} & Infix takes precedence \\
        \tri{x (- 5)} & \tri{(x ((-) 5))} & Explicitly disambiguated \\
        \tri{x (-) 5} &  \tri{((x (-)) 5))} & \tri{(-)} is a reference \\
        \tri{x not true} & \tri{(x ((not) true))} & \tri{not} cannot be infix \\
        \tri{x (not) true} & \tri{((x (not)) true))} & \tri{(not)} is a reference \\
        \tri{x + - 5} & \tri{((+) x ((-) 5))} & One of two operators is unary \\
        \hline
    \end{tabular}
\end{table}

The \kw{not} operator is used to perform Boolean negation. \texttt{\kw{not}~\kw{true}}
evaluates to \kw{false} and \texttt{\kw{not}~\kw{false}} evaluates to \kw{true}, as
you might have expected. If the expression is not a Boolean, it causes a runtime type
error. This is much like the operator \texttt{!} seen in other languages.

\begin{figure}[H]
    \centering
    \parbox[t]{0.4\linewidth}{
        \begin{prooftree}
            \AxiomC{$\Gamma\vdash E=\kw{true}$}
            \LeftLabel{Not}
            \UnaryInfC{$\Gamma\vdash \kw{not}\ E=\kw{false}$}
        \end{prooftree}
    }
    \parbox[t]{0.4\linewidth}{
        \begin{prooftree}
            \AxiomC{$\Gamma\vdash E=\kw{false}$}
            \LeftLabel{Not}
            \UnaryInfC{$\Gamma\vdash \kw{not}\ E=\kw{true}$}
        \end{prooftree}
    }
\end{figure}

The operator \op{-} is the numeric negation operator. In \Trilogy{} there are no
negative numeric literals, so all negative numbers must go through this operator.
Used on a non-Number value, it causes a runtime type error.

\begin{prooftree}
    \AxiomC{$\Gamma\vdash E = n : \ty{Number}$}
    \LeftLabel{Negation}
    \UnaryInfC{$\Gamma\vdash \op{-}\ E = -n$}
\end{prooftree}

The operator \op{\textasciitilde} is the bitwise negation operator. Used on a value
of Bits type, the state of each bit in the value is inverted.
Used on a non-Bits value, it causes a runtime type error.

\begin{prooftree}
    \AxiomC{$\Gamma\vdash E:\ty{Bits}$}
    \LeftLabel{Bitwise Negation}
    \UnaryInfC{$\Gamma\vdash \op{\textasciitilde}\ E:\ty{Bits}$}
\end{prooftree}

The \kw{yield} keyword used in \Poetry{} is actually also an operator, mostly for
convenience. This operator works similarly to how the \kw{yield} statement of \Prose{}
works, but the returned value is accessible. It may return more or less than once if
the effect handler resumes more or less that once. Though it would be nice to explain
it now, once again leave the detailed explanation to \S\ref{sec:effects}.


\subsubsection{Binary Operation}
\label{sec:binop}

Binary operators in \Poetry{} are infix operators, and are always written
with both arguments. As previously mentioned, operator precedence is not
handled by the syntax tree but by another pass afterwards.

\begin{bnf*}
    \bnfprod{BinaryOp}{
        \bnfpn{Expression}
        \bnfsp
        \bnfpn{BinaryOperator}
        \bnfsp
        \bnfpn{Expression}
    } \\
    \bnfprod{BinaryOperator}{
        \bnfts{\kw{and}}
        \bnfor
        \bnfts{\kw{or}}
        \bnfor
        \bnfts{\op{+}}
        \bnfor
        \bnfts{\op{-}}
        \bnfor
        \bnfts{\op{*}}
        \bnfor
        \bnfts{\op{/}}
        \bnfor
        \bnfts{\op{\%}}
        \bnfor
        \bnfts{\op{**}}
        \bnfor
        \bnfts{\op{//}}
        \bnfor
    } \\
    \bnfmore{
        \bnfts{\op{==}}
        \bnfor
        \bnfts{\op{===}}
        \bnfor
        \bnfts{\op{<}}
        \bnfor
        \bnfts{\op{>}}
        \bnfor
        \bnfts{\op{>=}}
        \bnfor
        \bnfts{\op{<=}}
        \bnfor
    } \\
    \bnfmore{
        \bnfts{\op{\&}}
        \bnfor
        \bnfts{\op{|}}
        \bnfor
        \bnfts{\op{\textasciicircum}}
        \bnfor
        \bnfts{\op{<\textasciitilde}}
        \bnfor
        \bnfts{\op{\textasciitilde>}}
        \bnfor
        \bnfts{\op{,}}
        \bnfor
        \bnfts{\op{:}}
        \bnfor
    } \\
    \bnfmore{
        \bnfts{\op{<>}}
        \bnfor
        \bnfts{\op{>>}}
        \bnfor
        \bnfts{\op{<<}}
        \bnfor
        \bnfts{\op{|>}}
        \bnfor
        \bnfts{\op{<|}}
    } \\
\end{bnf*}

Now comes the long and tedious process of defining each of these operators.
Worst part is that I know you know how these operators are supposed to work,
they're all the regular old stuff we see in every other language, but since
we are working with the primitive types, there's nothing we can do but to
define these things as internal magic, and so I guess it does require
specification somewhere.

Starting with the Boolean operators \kw{and} and \kw{or}, these are the typical
ones with behaviour that hardly needs explaining, and are short circuiting. If
one of the operands to either of these operators is evaluated and does not result
in a Boolean value, this is a runtime type error (if it does not get evaluated
due to short circuiting, the type error never happens).

\begin{prooftree}
    \AxiomC{$\Gamma\vdash A = \kw{false}$}
    \LeftLabel{And}
    \UnaryInfC{$\Gamma\vdash A\ \kw{and}\ B = \kw{false}$}
\end{prooftree}

\begin{prooftree}
    \AxiomC{$\Gamma\vdash A = \kw{true}$}
    \AxiomC{$\Gamma\vdash B = \kw{false}$}
    \LeftLabel{And}
    \BinaryInfC{$\Gamma\vdash A\ \kw{and}\ B = \kw{false}$}
\end{prooftree}

\begin{prooftree}
    \AxiomC{$\Gamma\vdash A = \kw{true}$}
    \AxiomC{$\Gamma\vdash B = \kw{true}$}
    \LeftLabel{And}
    \BinaryInfC{$\Gamma\vdash A\ \kw{and}\ B = \kw{true}$}
\end{prooftree}

\begin{prooftree}
    \AxiomC{$\Gamma\vdash A = \kw{true}$}
    \LeftLabel{Or}
    \UnaryInfC{$\Gamma\vdash A\ \kw{or}\ B = \kw{true}$}
\end{prooftree}

\begin{prooftree}
    \AxiomC{$\Gamma\vdash A = \kw{false}$}
    \AxiomC{$\Gamma\vdash B = \kw{true}$}
    \LeftLabel{Or}
    \BinaryInfC{$\Gamma\vdash A\ \kw{or}\ B = \kw{true}$}
\end{prooftree}

\begin{prooftree}
    \AxiomC{$\Gamma\vdash A = \kw{false}$}
    \AxiomC{$\Gamma\vdash B = \kw{false}$}
    \LeftLabel{Or}
    \BinaryInfC{$\Gamma\vdash A\ \kw{or}\ B = \kw{false}$}
\end{prooftree}

The mathematical operators are next, and as you might expect they work
as expected. Some details are worth noting however.

The mathematical operators work on values of Number type, if either operand
is not a Number, it causes a runtime type error.

Though we are strict about that, division by zero is not an error but instead
yields the exceptional result \val{'INF}. I am aware that division by zero is
explicitly defined as undefined, and is very definitely not mathematically
infinity, but \val{'INF} is not infinity, it is simply a signal that division
by zero was performed. This is the case for both regular divison (\op{/}) and
integer division (\op{//}), which floors to the nearest integer after dividing.

The \op{\%} operator is the remainder operator, returning the remainder of
integer division. Notably the remainder operator is the integer remainder operator,
not a modulus operator, and is implemented following the same floored division
as with integer division.

The \op{**} operator is the exponentiation operator and, given the fact that
precise rationals are available in \Trilogy{}, it also can be used for computing
square roots and such when given a fractional power. Since \Trilogy{} also
supports complex numbers, the result of something such as \texttt{-1 ** 1/2}
will not be an error, but actually the complex number \val{0i1}.

\begin{prooftree}
    \AxiomC{$\Gamma\vdash A = a : \ty{Number}$}
    \AxiomC{$\Gamma\vdash B = b : \ty{Number}$}
    \LeftLabel{Addition}
    \BinaryInfC{$\Gamma\vdash A\ \op{+}\ B = a + b : \ty{Number}$}
\end{prooftree}

\begin{prooftree}
    \AxiomC{$\Gamma\vdash A = a : \ty{Number}$}
    \AxiomC{$\Gamma\vdash B = b : \ty{Number}$}
    \LeftLabel{Subtraction}
    \BinaryInfC{$\Gamma\vdash A\ \op{-}\ B = a - b : \ty{Number}$}
\end{prooftree}

\begin{prooftree}
    \AxiomC{$\Gamma\vdash A = a : \ty{Number}$}
    \AxiomC{$\Gamma\vdash B = b : \ty{Number}$}
    \LeftLabel{Multiplication}
    \BinaryInfC{$\Gamma\vdash A\ \op{*}\ B = a \times b : \ty{Number}$}
\end{prooftree}

\begin{prooftree}
    \AxiomC{$\Gamma\vdash A = a : \ty{Number}$}
    \AxiomC{$\Gamma\vdash B = 0$}
    \LeftLabel{Division}
    \BinaryInfC{$\Gamma\vdash A\ \op{/}\ B \vartriangle \val{'INF}$}
\end{prooftree}

\begin{prooftree}
    \AxiomC{$\Gamma\vdash A = a : \ty{Number}$}
    \AxiomC{$\Gamma\vdash B = b : \ty{Number}$}
    \LeftLabel{Division}
    \BinaryInfC{$\Gamma\vdash A\ \op{/}\ B = a \div b : \ty{Number}$}
\end{prooftree}

\begin{prooftree}
    \AxiomC{$\Gamma\vdash A = a : \ty{Number}$}
    \AxiomC{$\Gamma\vdash B = 0$}
    \LeftLabel{Integer Division}
    \BinaryInfC{$\Gamma\vdash A\ \op{//}\ B \vartriangle \val{'INF}$}
\end{prooftree}

\begin{prooftree}
    \AxiomC{$\Gamma\vdash A = a : \ty{Number}$}
    \AxiomC{$\Gamma\vdash B = b : \ty{Number}$}
    \LeftLabel{Integer Division}
    \BinaryInfC{$\Gamma\vdash A\ \op{//}\ B = \lfloor a \div b \rfloor : \ty{Number}$}
\end{prooftree}

\begin{prooftree}
    \AxiomC{$\Gamma\vdash A = a : \ty{Number}$}
    \AxiomC{$\Gamma\vdash B = b : \ty{Number}$}
    \LeftLabel{Remainder}
    \BinaryInfC{$\Gamma\vdash A\ \op{\%}\ B = a - b \lfloor a \div b \rfloor : \ty{Number}$}
\end{prooftree}

\begin{prooftree}
    \AxiomC{$\Gamma\vdash A = a : \ty{Number}$}
    \AxiomC{$\Gamma\vdash B = b : \ty{Number}$}
    \LeftLabel{Exponentiation}
    \BinaryInfC{$\Gamma\vdash A\ \op{**}\ B = a^b : \ty{Number}$}
\end{prooftree}

Comparison operators compare two values and returns a Boolean result depending
on the result of the comparison.

The structural equality operator (\op{==}) compares values of structural type
(\S\ref{sec:structuraltypes}), returning \kw{true} if they are the same value,
and \kw{false} otherwise. Non-structural types are compared by reference
instead, only an exact reference evaluating to \kw{true}. This turns out to
be much like a Boolean valuation of a direct unification, only without option
of comparing two incomplete patterns.

\begin{prooftree}
    \AxiomC{$\Gamma\vdash A$}
    \AxiomC{$\Gamma\vdash B$}
    \AxiomC{$\Gamma\vdash A = B$}
    \LeftLabel{Structural Equality}
    \TrinaryInfC{$\Gamma\vdash A\ \op{==}\ B = \kw{true}$}
\end{prooftree}

\begin{prooftree}
    \AxiomC{$\Gamma\vdash A$}
    \AxiomC{$\Gamma\vdash B$}
    \AxiomC{$\Gamma\vdash A \neq B$}
    \LeftLabel{Structural Equality}
    \TrinaryInfC{$\Gamma\vdash A\ \op{==}\ B = \kw{false}$}
\end{prooftree}

The referential equality operator (\op{===}) compares values of reference type
(\S\ref{sec:referencetypes}), returning \kw{true} if they are the same instance,
and \kw{false} otherwise. Non-reference types are compared structurally instead.
This operator is of a more practical sort, hooking in to implementation detail
rather than logical or mathematical purity.

\begin{prooftree}
    \AxiomC{$\Gamma\vdash A$}
    \AxiomC{$\Gamma\vdash B$}
    \AxiomC{$\Gamma\vdash \text{addr}(A) = \text{addr}(B)$}
    \LeftLabel{Reference Equality}
    \TrinaryInfC{$\Gamma\vdash A\ \op{===}\ B = \kw{true}$}
\end{prooftree}

\begin{prooftree}
    \AxiomC{$\Gamma\vdash A$}
    \AxiomC{$\Gamma\vdash B$}
    \AxiomC{$\Gamma\vdash \text{addr}(A) \neq \text{addr}(B)$}
    \LeftLabel{Reference Equality}
    \TrinaryInfC{$\Gamma\vdash A\ \op{===}\ B = \kw{false}$}
\end{prooftree}

The ordered comparison operators work on any two values of the same type. Not
all types are orderable, however, in which case the comparison returns \kw{unit}
to indicate that an order cannot be determined. This \kw{unit} result is distinct
from indicating that two compared items are the ``same'' in magnitude. When used
on two values of different types, they are treated as unorderable and the comparison
again returns \kw{unit}. This is different from most other type-mismatches in \Trilogy{},
which cause a runtime type error; where those situations have no clear path forward,
the ``unorderable'' option in this case is more flexible and does not make any
assumptions about the behaviour of a program.

Rather than defining the proof trees for every comparable type, this table summarizes
orderings, and is referred to as $\text{cmp}(A \square B)$ in the proof trees below. Note
that in lexicographical order, elements are compared before the lengths of the sequences;
if one value is a prefix of the other, the prefix value is less than the longer value.
In the fallible lexicographical order, if any element-wise comparison is unorderable, the
whole result is treated unorderable.

\begin{table}[H]
    \centering
    \begin{tabular}{ll}
        \hline
        \textbf{Type}          & \textbf{Ordering} \\
        \hline
        Unit                   & Unorderable \\
        Atom                   & Unorderable \\
        Boolean                & \kw{false} then \kw{true} \\
        Real Number            & Number line ordering \\
        Imaginary Number       & Magnitude ordering \\
        Complex Number         & Unorderable \\
        Character              & Unicode code point order \\
        String                 & Lexicographical order of characters \\
        Struct (same tag)      & Ordered by wrapped value order \\
        Struct (different tag) & Unorderable \\
        Bits                   & Ordered by big-endian integer interpretation \\
        Tuple                  & Compare left elements, then right elements \\
        Array                  & Fallible lexicographical order of elements \\
        Set                    & Unorderable\footnotemark[1] \\
        Record                 & Unorderable\footnotemark[1] \\
        Iterator               & Unorderable \\
        Semantic types         & Unorderable \\
        Mismatched types       & Unorderable \\
        \hline
    \end{tabular}
\end{table}

\footnotetext[1]{May be added at a later date, if a reasonable ordering is
determined and found to be useful.}

\begin{prooftree}
    \AxiomC{$\Gamma\vdash A$}
    \AxiomC{$\Gamma\vdash B$}
    \LeftLabel{Less Than}
    \BinaryInfC{$\Gamma\vdash A\ \op{<}\ B = \text{cmp}(A < B)$}
\end{prooftree}

\begin{prooftree}
    \AxiomC{$\Gamma\vdash A$}
    \AxiomC{$\Gamma\vdash B$}
    \LeftLabel{Greater Than}
    \BinaryInfC{$\Gamma\vdash A\ \op{>}\ B = \text{cmp}(A > B)$}
\end{prooftree}

\begin{prooftree}
    \AxiomC{$\Gamma\vdash A$}
    \AxiomC{$\Gamma\vdash B$}
    \LeftLabel{Less Than Or Equal To}
    \BinaryInfC{$\Gamma\vdash A\ \op{<=}\ B = \text{cmp}(A \leq B)$}
\end{prooftree}

\begin{prooftree}
    \AxiomC{$\Gamma\vdash A$}
    \AxiomC{$\Gamma\vdash B$}
    \LeftLabel{Greater Than Or Equal To}
    \BinaryInfC{$\Gamma\vdash A\ \op{>=}\ B = \text{cmp}(A \geq B)$}
\end{prooftree}

The bitwise operators work only on values of Bits type, with the exception
of the shift operators which expect the right-hand operand to be an integer.
Again, if the types are incorrect, it causes is a runtime type error.

The regular bitwise operators extend the shorter operand with 0s on the left
until both are of the same length. The left shift will extend with 0s on the
right, while the right shift removes bits from the right (no bits are added).

\begin{prooftree}
    \AxiomC{$\Gamma\vdash A = a : \ty{Bits}$}
    \AxiomC{$\Gamma\vdash B = b : \ty{Bits}$}
    \LeftLabel{Bitwise And}
    \BinaryInfC{$\Gamma\vdash A\ \op{\&}\ B = a \land b$}
\end{prooftree}

\begin{prooftree}
    \AxiomC{$\Gamma\vdash A = a : \ty{Bits}$}
    \AxiomC{$\Gamma\vdash B = b : \ty{Bits}$}
    \LeftLabel{Bitwise Or}
    \BinaryInfC{$\Gamma\vdash A\ \op{|}\ B = a \lor b$}
\end{prooftree}

\begin{prooftree}
    \AxiomC{$\Gamma\vdash A = a : \ty{Bits}$}
    \AxiomC{$\Gamma\vdash B = b : \ty{Bits}$}
    \LeftLabel{Bitwise Xor}
    \BinaryInfC{$\Gamma\vdash A\ \op{|}\ B = a \oplus b$}
\end{prooftree}

\begin{prooftree}
    \AxiomC{$\Gamma\vdash A = a : \ty{Bits}$}
    \AxiomC{$\Gamma\vdash B = b : \ty{Integer}$}
    \LeftLabel{Left Shift}
    \BinaryInfC{$\Gamma\vdash A\ \op{<\textasciitilde}\ B = a(b\cdot \texttt{0})$}
\end{prooftree}

\begin{prooftree}
    \AxiomC{$\Gamma\vdash A = \rho \sigma : \ty{Bits}$}
    \AxiomC{$\Gamma\vdash B = b : \ty{Integer}$}
    \AxiomC{$|\sigma| = b$}
    \LeftLabel{Right Shift}
    \TrinaryInfC{$\Gamma\vdash A\ \op{\textasciitilde>}\ B = \rho$}
\end{prooftree}


\subsubsection{Is}

The \kw{is} keyword used in \Poetry{} does the opposite of what it does
in \Law{}; that is, it checks if a query has any solutions and converts
that result into a Boolean value.

\begin{bnf*}
    \bnfprod{Is}{
        \bnfts{\kw{is}}
        \bnfsp
        \bnfpn{Query}
    }
\end{bnf*}

\begin{figure}[H]
    \centering
    \parbox[t]{0.4\linewidth}{
        \begin{prooftree}
            \AxiomC{$\Gamma\vdash Q:\bot$}
            \LeftLabel{Is}
            \UnaryInfC{$\Gamma\vdash \kw{is}\ Q=\kw{false}$}
        \end{prooftree}
    }
    \parbox[t]{0.4\linewidth}{
        \begin{prooftree}
            \AxiomC{$\Gamma\vdash Q:\top$}
            \LeftLabel{Is}
            \UnaryInfC{$\Gamma\vdash \kw{is}\ Q=\kw{true}$}
        \end{prooftree}
    }
\end{figure}

