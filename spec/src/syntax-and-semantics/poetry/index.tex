\subsection{Poetry}

As \Law{} is a language of binding, and \Prose{} a language of control
flow, \Poetry{} is a language of value. As a pure functional programming
language, \Poetry{} is highly expression-based, though given the existence
of the effect system it is possible to trigger side effects, in a way.

\begin{bnf*}
    \bnfprod{FunctionHead}{
        \bnfpn{Identifier}
        \bnfsp
        \bnfpn{ParameterList}
    } \\
    \bnfprod{ParameterList}{
        \bnfpn{Pattern}
        \bnfsp
        \bnfpn{PatternList}
        \bnfor
        \bnfpn{Pattern}
    } \\
    \bnfprod{FunctionBody}{\bnfpn{Expression}}
\end{bnf*}

Similar to Haskell, functions in \Poetry{} are defined as equations. Defining
the same name multiple times is allowed, given that its number of arguments
is the same, and the patterns of its parameters are changed, allowing different
cases of a function to be handled by different expressions. When the function is
applied, the first matching definition in source code order is used, and the
others are skipped.

Also similar to Haskell, functions support currying and partial application by
default. In \Trilogy{}, this is defined as a syntax transformation over function
definitions with multiple parameters into unary functions returning a function
that accepts the next parameter.

Unsimilarly to Haskell, \Poetry{} is evaluated eagerly. This means that nullary
(0-argument) functions are pretty useless, so all functions \emph{must} have a
parameter. Nullary procedures may be a simple alternative in the meantime, where
the calling of that procedure is explicit, or a function accepting a single
ignored \kw{unit} parameter, if truly \Poetry{} is preferred.

\begin{align*}
    \texttt{\$f:id \$(\$r:pat)+ \$p:pat = \$e:expr} & \Rightarrow \texttt{\$f \$(\$r)+ = \kw{fn} \$p. \$e} %
\end{align*}

%% TODO: I really can't decide if it's worth making Poetry lazy. Technically it is
%% a pretty neat thing to be able to work in lazy code, but practically... I find
%% it less so. Poetry is meant to be practical.
%%
%% This does eliminate the ability to have nullary functions, but I really don't
%% mind. You can just pass a unit anyway.

% The parameter list may be empty, defining a nullary (0-argument) function. As
% \Poetry{} is lazy, this function will only be evaluated when its result is
% needed, so even a nullary function may, for example, yield an effect which will
% only be performed when it is required. Note that the lazy evaluation of \Poetry{}
% can lead to some code being run in unexpected order, so it is recommended to
% think carefully about whether effects in \Poetry{} are required.

% Though \Poetry{} is lazy, \Prose{} and \Law{} (except for through Iterators)
% are not. When an expression of \Poetry{} appears in \Prose{} or \Law{},
% as in an assignment or unification, it is evaluated immediately. This implies
% that while \Poetry{} may be used to represent infinite data structures those
% data structures must remain in \Poetry{} or else they will be infinitely
% evaluated, unless converted to Iterator.

\begin{prooftree}
    \def\defaultHypSeparation{\hskip 0.1in}
    \AxiomC{$\kw{func}\ f\ P\ \texttt{=}\ E$}
    \AxiomC{$\Sigma\vdash X : \tau$}
    \AxiomC{$P = X$}
    \AxiomC{$\Gamma,\setof{a\given a\defby P}\vdash E : \rho $}
    \LeftLabel{Function}
    \QuaternaryInfC{$\Gamma\vdash f : \tau \rightarrow \rho$}
\end{prooftree}

\subsubsection{Expressions}
\label{sec:expressions}

Seeing as the entirety of \Poetry{} (and admittedly, much of \Trilogy{} as a whole)
is made up of expressions, there are a lot of cases.

Though the expression grammar does include binary operators and other binary keywords
\Trilogy{} does not encode precedence as part of the grammar but instead resolves that
in a later pass in the interpreter (\S\ref{sec:precedence}).

\begin{bnf*}
    \bnfprod{Expression}{
        \bnfpn{Expr}
        \bnfor
        \bnfpn{Expr}
        \bnfsp
        \bnfpn{EffectHandlers}
    } \\
    \bnfprod{Expr}{
        \bnfpn{Primitive}
        \bnfor
        \bnfpn{Compound}
        \bnfor
        \bnfpn{Comprehension}
        \bnfor
    } \\
    \bnfmore{
        \bnfpn{Reference}
        \bnfor
        \bnfpn{MemberAccess}
        \bnfor
        \bnfpn{Application}
        \bnfor
    } \\
    \bnfmore{
        \bnfpn{ProcedureCall}
        \bnfor
        \bnfpn{BinOp}
        \bnfor
        \bnfpn{UnaryOp}
        \bnfor
    } \\
    \bnfmore{
        \bnfpn{Let}
        \bnfor
        \bnfpn{IfElse}
        \bnfor
        \bnfpn{Match}
        \bnfor
        \bnfpn{Is}
        \bnfor
        \bnfpn{End}
        \bnfor
    } \\
    \bnfmore{
        \bnfpn{Resume}
        \bnfor
        \bnfpn{Cancel}
        \bnfor
        \bnfpn{Return}
        \bnfor
        \bnfpn{Break}
        \bnfor
    } \\
    \bnfmore{
        \bnfpn{Continue}
        \bnfor
        \bnfpn{FnClosure}
        \bnfor
        \bnfpn{DoClosure}
        \bnfor
    } \\
    \bnfmore{
        \bnfts{(}
        \bnfsp
        \bnfpn{Expression}
        \bnfsp
        \bnfts{)}
    }
\end{bnf*}


\subsubsection{Literals}
\label{sec:literals}

Starting easy, the simplest expression is a primitive value.

\begin{bnf*}
    \bnfprod{Primitive}{
        \bnfts{\kw{unit}}
        \bnfor
        \bnfts{\kw{false}}
        \bnfor
        \bnfts{\kw{true}}
        \bnfor
        \bnfpn{Numeric}
        \bnfor
    } \\
    \bnfmore{
        \bnfpn{Atom}
        \bnfor
        \bnfpn{String}
        \bnfor
        \bnfpn{Character}
        \bnfor
        \bnfpn{Bits}
    }
\end{bnf*}

The evaluation of a literal primitive value is simply the value it represents.
For completeness, some trivial specifications:

\begin{figure}[H]
    \centering
    \parbox[t]{0.3\linewidth}{
        \begin{prooftree}
            \AxiomC{}
            \LeftLabel{Unit}
            \UnaryInfC{$\kw{unit} : \ty{Unit}$}
        \end{prooftree}
    }
    \parbox[t]{0.3\linewidth}{
        \begin{prooftree}
            \AxiomC{}
            \LeftLabel{Boolean}
            \UnaryInfC{$\kw{false} : \ty{Bool}$}
        \end{prooftree}
    }
    \parbox[t]{0.3\linewidth}{
        \begin{prooftree}
            \AxiomC{}
            \LeftLabel{Boolean}
            \UnaryInfC{$\kw{true} : \ty{Bool}$}
        \end{prooftree}
    }
    \parbox[t]{0.4\linewidth}{
        \begin{prooftree}
            \AxiomC{}
            \LeftLabel{Numeric}
            \UnaryInfC{$\langle\text{Numeric}\rangle : \ty{Number}$}
        \end{prooftree}
    }
    \parbox[t]{0.4\linewidth}{
        \begin{prooftree}
            \AxiomC{}
            \LeftLabel{String}
            \UnaryInfC{$\langle\text{String}\rangle : \ty{String}$}
        \end{prooftree}
    }
    \parbox[t]{0.4\linewidth}{
        \begin{prooftree}
            \AxiomC{}
            \LeftLabel{Atom}
            \UnaryInfC{$\langle\text{Atom}\rangle : \ty{Atom}$}
        \end{prooftree}
    }
    \parbox[t]{0.4\linewidth}{
        \begin{prooftree}
            \AxiomC{}
            \LeftLabel{Character}
            \UnaryInfC{$\langle\text{Character}\rangle : \ty{Char}$}
        \end{prooftree}
    }
    \parbox[t]{0.4\linewidth}{
        \begin{prooftree}
            \AxiomC{}
            \LeftLabel{Bits}
            \UnaryInfC{$\langle\text{Bits}\rangle : \ty{Bits}$}
        \end{prooftree}
    }
\end{figure}

Of similar literal-ness are the literals for compound types. Notably
tuples are missing from this section, as they are implemented in terms
of the cons operator, rather than a literal form.

\begin{bnf*}
    \bnfprod{Compound}{
        \bnfpn{Struct}
        \bnfor
        \bnfpn{Array}
        \bnfor
        \bnfpn{Set}
        \bnfor
        \bnfpn{Record}
    } \\
    \bnfprod{Struct}{
        \bnfpn{Atom}
        \bnfsp
        \bnfts{(}
        \bnfsp
        \bnfpn{Expression}
        \bnfsp
        \bnfts{)}
    } \\
    \bnfprod{Array}{
        \bnfts{[}
        \bnfsp
        \bnfpn{CompoundElements}
        \bnfsp
        \bnfts{]}
    } \\
    \bnfprod{Set}{
        \bnfts{[|}
        \bnfsp
        \bnfpn{CompoundElements}
        \bnfsp
        \bnfts{|]}
    } \\
    \bnfprod{CompoundElements}{
        \bnfpn{CompoundElement}
        \bnfsp
        \bnfts{,}
        \bnfsp
        \bnfpn{CompoundElements}
        \bnfor
    } \\
    \bnfmore{
        \bnfpn{CompoundElement}
        \bnfor
        \bnfes
    } \\
    \bnfprod{CompoundElement}{
        \bnfpn{Expression}
        \bnfor
        \bnfts{\op{..}}
        \bnfsp
        \bnfpn{Expression}
    } \\
    \bnfprod{Record}{
        \bnfts{\{}
        \bnfsp
        \bnfpn{KeyValueElements}
        \bnfsp
        \bnfts{\}}
    } \\
    \bnfprod{KeyValueElements}{
        \bnfpn{KeyValueElement}
        \bnfsp
        \bnfts{,}
        \bnfsp
        \bnfpn{KeyValueElements}
        \bnfor
    } \\
    \bnfmore{
        \bnfpn{KeyValueElement}
        \bnfor
        \bnfes
    } \\
    \bnfprod{KeyValueElement}{
        \bnfpn{Expression}
        \bnfsp
        \bnfts{\op{=>}}
        \bnfsp
        \bnfpn{Expression}
        \bnfor
        \bnfts{\op{..}}
        \bnfsp
        \bnfpn{Expression}
    } \\
\end{bnf*}

As you might expect, the evaluation of a compound literal is a value of the
compound type, where the items it contains are the values the evaluate to.
The only thing of particular interest is the use of spread (\op{..}) in
Array, Set, and Record literals, for which the expression must evaluate
to a value of the same compound type, and that value is then inserted
as if its elements were written in place of the spread.

\begin{prooftree}
    \AxiomC{$\Gamma\vdash E : \tau$}
    \LeftLabel{Struct}
    \UnaryInfC{$\Gamma\vdash\texttt{'}l\ \texttt{(}\ E\ \texttt{)} : \texttt{'} l(\tau)$}
\end{prooftree}

\begin{prooftree}
    \AxiomC{}
    \LeftLabel{Array\textsuperscript{$\epsilon$}}
    \UnaryInfC{$\texttt{[]} : \ty{List}\ \tau$}
\end{prooftree}

\begin{prooftree}
    \AxiomC{$\Gamma\vdash X_1 : \tau$}
    \AxiomC{$\Gamma\vdash \texttt{[}X_2\texttt{,}\ \cdots\texttt{,}\ X_n\texttt{]} : \ty{List}\ \tau$}
    \LeftLabel{Array}
    \BinaryInfC{$\Gamma\vdash\texttt{[}X_1\texttt{,}\ X_2\texttt{,}\ \cdots\texttt{,}\ X_n\texttt{]} : \ty{List}\ \tau$}
\end{prooftree}

\begin{prooftree}
    \AxiomC{$\Gamma\vdash X_1 : \ty{List}\ \tau$}
    \AxiomC{$\Gamma\vdash \texttt{[}X_2\texttt{,}\ \cdots\texttt{,}\ X_n\texttt{]} : \ty{List}\ \tau$}
    \LeftLabel{Array\textsuperscript{\op{..}}}
    \BinaryInfC{$\Gamma\vdash\texttt{[}\op{..}X_1\texttt{,}\ X_2\texttt{,}\ \cdots\texttt{,}\ X_n\texttt{]} : \ty{List}\ \tau$}
\end{prooftree}

\begin{prooftree}
    \AxiomC{}
    \LeftLabel{Set\textsuperscript{$\epsilon$}}
    \UnaryInfC{$\texttt{[||]} : \ty{Set}\ \tau$}
\end{prooftree}

\begin{prooftree}
    \AxiomC{$\Gamma\vdash X_1 : \tau$}
    \AxiomC{$\Gamma\vdash \texttt{[|}\ X_2\ \texttt{,}\cdots\texttt{,}\ X_n\ \texttt{|]} : \ty{Set}\ \tau$}
    \LeftLabel{Set}
    \BinaryInfC{$\Gamma\vdash\texttt{[|}\ X_1\ \texttt{,}\ X_2\ \texttt{,}\cdots\texttt{,}\ X_n\ \texttt{|]} : \ty{Set}\ \tau$}
\end{prooftree}

\begin{prooftree}
    \AxiomC{$\Gamma\vdash X_1 : \ty{Set}\ \tau$}
    \AxiomC{$\Gamma\vdash \texttt{[|}\ X_2\ \texttt{,}\cdots\texttt{,}\ X_n\ \texttt{|]} : \ty{Set}\ \tau$}
    \LeftLabel{Set\textsuperscript{\op{..}}}
    \BinaryInfC{$\Gamma\vdash\texttt{[|}\ \op{..}X_1\ \texttt{,}\ X_2\ \texttt{,}\cdots\texttt{,}\ X_n\ \texttt{|]} : \ty{Set}\ \tau$}
\end{prooftree}

\begin{prooftree}
    \AxiomC{}
    \LeftLabel{Record\textsuperscript{$\epsilon$}}
    \UnaryInfC{$\texttt{\{\}} : \ty{Record}\ \kappa\ \nu$}
\end{prooftree}

\begin{prooftree}
    \def\defaultHypSeparation{\hskip 0in}
    \AxiomC{$\Gamma\vdash K_1 : \kappa$}
    \AxiomC{$\Gamma\vdash V_1 : \nu$}
    \AxiomC{$\Gamma\vdash \texttt{\{}K_2:V_2\ \texttt{,}\cdots\texttt{,}\ K_n:V_n\texttt{\}} : \ty{Record}\ \kappa\ \nu$}
    \LeftLabel{Record}
    \TrinaryInfC{$\Gamma\vdash\texttt{\{}K_1\ \texttt{=>}\ V_1\ \texttt{,}\ K_2\ \texttt{=>}\ V_2\ \texttt{,}\cdots\texttt{,}\ K_n\ \texttt{=>}\ V_n\texttt{\}} : \ty{Record}\ \kappa\ \nu$}
\end{prooftree}

\begin{prooftree}
    \def\defaultHypSeparation{\hskip 0in}
    \AxiomC{$\Gamma\vdash X : \ty{Record}\ \kappa\ \nu$}
    \AxiomC{$\Gamma\vdash \texttt{\{}K_2\ \texttt{=>}\ V_2\ \texttt{,}\cdots\texttt{,}\ K_n\ \texttt{=>}\ V_n\texttt{\}} : \ty{Record}\ \kappa\ \nu$}
    \LeftLabel{Record\textsuperscript{\op{..}}}
    \BinaryInfC{$\Gamma\vdash\texttt{\{}\op{..}X\ \texttt{,}\ K_2\ \texttt{=>}\ V_2\ \texttt{,}\cdots\texttt{,}\ K_n\ \texttt{=>}\ V_n\texttt{\}} : \ty{Record}\ \kappa\ \nu$}
\end{prooftree}


\subsubsection{Comprehension}

Comprehension is a syntax by which collections can be iterated, filtered, and updated.
In \Trilogy{}, comprehensions also allow lists to be generated; a special power afforded
by \Law{} that other languages' comprehensions cannot do.

The syntax of comprehension is a bit like a \kw{for} loop in a compound literal:

\begin{bnf*}
    \bnfprod{Comprehension}{
        \bnfpn{ArrayComprehension}
        \bnfor
    } \\
    \bnfmore{
        \bnfpn{SetComprehension}
        \bnfor
    } \\
    \bnfmore{
        \bnfpn{RecordComprehension}
        \bnfor
    } \\
    \bnfmore{
        \bnfpn{IteratorComprehension}
    } \\
    \bnfprod{ArrayComprehension}{
        \bnfts{[}
        \bnfsp
        \bnfpn{Expr}
        \bnfsp
        \bnfts{\kw{for}}
        \bnfsp
        \bnfpn{Query}
        \bnfsp
        \bnfts{]}
    } \\
    \bnfprod{SetComprehension}{
        \bnfts{\{|}
        \bnfsp
        \bnfpn{Expr}
        \bnfsp
        \bnfts{\kw{for}}
        \bnfsp
        \bnfpn{Query}
        \bnfsp
        \bnfts{|\}}
    } \\
    \bnfprod{RecordComprehension}{
        \bnfts{\{}
        \bnfsp
        \bnfpn{Expr}
        \bnfsp
        \bnfts{:}
        \bnfsp
        \bnfpn{Expr}
        \bnfsp
        \bnfts{\kw{for}}
        \bnfsp
        \bnfpn{Query}
        \bnfsp
        \bnfts{\}}
    } \\
    \bnfprod{IteratorComprehension}{
        \bnfts{\$(} %
        \bnfsp
        \bnfpn{Expr}
        \bnfsp
        \bnfsp
        \bnfts{\kw{for}}
        \bnfsp
        \bnfpn{Query}
        \bnfsp
        \bnfts{)}
    }
\end{bnf*}

The expression and the bindings introduced to that expression are in reverse in
this syntax, one of the only places in \Trilogy{} where that is the case. The
query in each form of comprehension introduces the bindings which may be used
in the expression which generates the collection that is being comprehended.

Where most languages allow iterator terms and filter terms for their comprehensions,
\Trilogy{} requires none of that, instead opting to use \Law{}'s queries to allow
filtering, iterating, and generating all with a familiar syntax.

\begin{prooftree}
    \AxiomC{$\Gamma\vdash U$}
    \AxiomC{$U = Q$}
    \AxiomC{$\Gamma,\setof{a\given a \defby Q}\vdash E : \tau$}
    \LeftLabel{List Comprehension}
    \TrinaryInfC{$\Gamma\vdash \texttt{[}\ E\ \kw{for}\ Q\ \texttt{]} : \ty{List}\ \tau$}
\end{prooftree}

\begin{prooftree}
    \AxiomC{$\Gamma\vdash U$}
    \AxiomC{$U = Q$}
    \AxiomC{$\Gamma,\setof{a\given a \defby Q}\vdash E : \tau$}
    \LeftLabel{Set Comprehension}
    \TrinaryInfC{$\Gamma\vdash \texttt{\{|}\ E\ \kw{for}\ Q\ \texttt{|\}} : \ty{Set}\ \tau$}
\end{prooftree}

\begin{prooftree}
    \AxiomC{$\Gamma\vdash U$}
    \AxiomC{$U = Q$}
    \AxiomC{$\Gamma,\setof{a\given a \defby Q}\vdash K : \kappa, V : \nu$}
    \LeftLabel{Record Comprehension}
    \TrinaryInfC{$\Gamma\vdash \texttt{\{}\ K\ \texttt{:}\ V\ \kw{for}\ Q\ \texttt{\}} : \ty{Record}\ \kappa:\nu$}
\end{prooftree}

\begin{prooftree}
    \AxiomC{$\Gamma\vdash U$}
    \AxiomC{$U = Q$}
    \AxiomC{$\Gamma,\setof{a\given a \defby Q}\vdash E : \tau$}
    \LeftLabel{Iterator Comprehension}
    \TrinaryInfC{$\Gamma\vdash \texttt{\$(}\ E\ \kw{for}\ Q\ \texttt{)} : \ty{Iter}\ \tau$}
\end{prooftree}


\subsubsection{Reference}

Defined names may be referenced at any time, their naming evaluating to the value
they currently hold. This is a typical variable reference as in any language.
Notably this is \emph{only} for local references, looking up a name in another
scope is handled by the module access operator.

As a bit of a special case, the built-in operators may be referenced as if they
were functions by wrapping them in parentheses. This allows operators to be passed
as values and applied as if they were functions to their arguments, occasionally
useful in some functional compositions. The operators themselves are specified
in \S\ref{sec:binop} and \S\ref{sec:unaryop} (the member access operator is included
as well, but since it is described separately in \S\ref{sec:member-access}, it is
also listed separately here).

The one odd bit here is that the \op{-} symbol is always referenced as the
subtraction operator, and never the unary negation operator. Unary negation
can be constructed from subtraction by pre-supplying the zero as in \tri{((-) 0)}.

It is here also that first class keywords are converted into values.
First class keywords are described in \S\ref{sec:first-class-keyword}.

\begin{bnf*}
    \bnfprod{Reference}{
        \bnfpn{Identifier}
        \bnfor
        \bnfts{(}
        \bnfsp
        \bnfpn{OnlyUnaryOperator}
        \bnfsp
        \bnfts{)}
        \bnfor
        \bnfts{(}
        \bnfsp
        \bnfts{.}
        \bnfsp
        \bnfts{)}
        \bnfor
    } \\
    \bnfmore{
        \bnfts{(}
        \bnfsp
        \bnfpn{BinaryOperator}
        \bnfsp
        \bnfts{)}
        \bnfor
        \bnfts{(}
        \bnfsp
        \bnfpn{FirstClassKeyword}
        \bnfsp
        \bnfts{)}
    }
\end{bnf*}

Admittedly the following semantics look a little wonky due to an ambiguous choice
of notation. Rest assured, in the hypotheses the $x$ refers to the binding in the
context, while in the inference the $x$ refers to the symbol in source code, so
this is not an infinitely recursive deduction.

\begin{prooftree}
    \AxiomC{$\Gamma\vdash x : \tau$}
    \LeftLabel{Reference}
    \UnaryInfC{$\Gamma\vdash x : \tau$}
\end{prooftree}

For brevity, the semantics of the operator and keyword references are condensed
into this more symbolic form. Repeating the same tree for every operator would
be a real pain for everyone.

\begin{prooftree}
    \AxiomC{$\Gamma\vdash E : \tau$}
    \AxiomC{$\Gamma\vdash \langle \text{UnaryOperator} \rangle\ E : \rho$}
    \LeftLabel{Reference}
    \BinaryInfC{$\Gamma\vdash \texttt{(}\ \langle \text{UnaryOperator} \rangle\ \texttt{)}\ : \tau \rightarrow \rho$}
\end{prooftree}

\begin{prooftree}
    \AxiomC{$\Gamma\vdash L : \tau$}
    \AxiomC{$\Gamma\vdash R : \rho$}
    \AxiomC{$\Gamma\vdash L\ \langle \text{BinaryOperator} \rangle\ R : \phi$}
    \LeftLabel{Reference}
    \TrinaryInfC{$\Gamma\vdash \texttt{(}\ \langle \text{BinaryOperator} \rangle\ \texttt{)}\ : \tau \rightarrow \rho \rightarrow \phi$}
\end{prooftree}

\begin{prooftree}
    \AxiomC{$\Gamma\vdash E : \tau$}
    \AxiomC{$\Gamma\vdash \langle \text{FirstClassKeyword} \rangle\ E : \rho$}
    \LeftLabel{Reference}
    \BinaryInfC{$\Gamma\vdash \texttt{(}\ \langle \text{FirstClassKeyword} \rangle\ \texttt{)}\ : \tau \rightarrow \rho$}
\end{prooftree}


\subsubsection{Member Access}
\label{sec:member-access}

Some values contain other values within them, and those values are accessed using
the member access syntax. There are a few variations on this syntax, allowing for
some flexibility the structure of your data.

\begin{bnf*}
    \bnfprod{MemberAccess}{
        \bnfpn{Expression}
        \bnfsp
        \bnfts{\op{.}}
        \bnfsp
        \bnfpn{Identifier}
        \bnfor
    } \\
    \bnfmore{
        \bnfpn{Expression}
        \bnfsp
        \bnfts{\op{.[}}
        \bnfsp
        \bnfpn{Expression}
        \bnfsp
        \bnfts{]}
    }
\end{bnf*}

The only form of particular relevance is the second, as the first can be
transformed by interpreting the identifier as an atom and passing that
in the brackets:

\begin{align*}
    \texttt{\$c:expr . \$i:id} & \Rightarrow \texttt{\$c .[ ' \$i ]} %
\end{align*}

After that, the way the lookup works depends on the type of the expression
on the left.

For Arrays, the expression in the brackets must evaluate to an integer
which is the index of the element to access, starting from 0 at the left.
If the index is an integer but is not within the bounds of the array,
\val{'MIA} is yielded instead. Meanwhile, if the index expression is
not an integer, it is considered a runtime type error.

\begin{prooftree}
    \AxiomC{$\Gamma\vdash E = \sigma : \ty{Array}\ \tau$}
    \AxiomC{$\Gamma\vdash i : \ty{Integer}$}
    \AxiomC{$0 \leq i < |\sigma|$}
    \LeftLabel{Array Access}
    \TrinaryInfC{$\Gamma\vdash E\ \texttt{.[}\ i\ \texttt{]} : \tau$}
\end{prooftree}

\begin{prooftree}
    \AxiomC{$\Gamma\vdash E : \ty{Array}\ \tau$}
    \AxiomC{$\Gamma\vdash i : \ty{Integer}$}
    \AxiomC{$i < 0 \lor |\sigma| < i$}
    \LeftLabel{Array MIA}
    \TrinaryInfC{$\Gamma\vdash E\ \texttt{.[}\ i\ \texttt{]} \vartriangle \val{'MIA}$}
\end{prooftree}

For Strings, similar to arrays, the expression in the brackets must evaluate
to an integer which is the index of the Character to access, starting from 0
at the left. If the index is an integer but is not within the bounds of the
string, \val{'MIA} is yielded instead. Meanwhile, if the index expression
is not an integer, it is considered a runtime type error.

\begin{prooftree}
    \AxiomC{$\Gamma\vdash E = \sigma : \ty{String}$}
    \AxiomC{$\Gamma\vdash i : \ty{Integer}$}
    \AxiomC{$0 \leq i < |\sigma|$}
    \LeftLabel{String Access}
    \TrinaryInfC{$\Gamma\vdash E\ \texttt{.[}\ i\ \texttt{]} : \ty{Char}$}
\end{prooftree}

\begin{prooftree}
    \AxiomC{$\Gamma\vdash E : \ty{String}$}
    \AxiomC{$\Gamma\vdash i : \ty{Integer}$}
    \AxiomC{$i < 0 \lor |\sigma| < i$}
    \LeftLabel{String MIA}
    \TrinaryInfC{$\Gamma\vdash E\ \texttt{.[}\ i\ \texttt{]} \vartriangle \val{'MIA}$}
\end{prooftree}

For Bits, the expression again must evaluate to an integer, which
is the index of the bit to access, starting from 0 at the right,
represented as a Boolean. If the index is an integer less than 0,
it is a runtime type error. Bits values are considered treated as
infinite length, so large integers will never be out of range, and
are treated as \kw{false}.

\begin{prooftree}
    \AxiomC{$\Gamma\vdash E = \sigma : \ty{Bits}$}
    \AxiomC{$\Gamma\vdash i : \ty{Integer}$}
    \AxiomC{$0 \leq i$}
    \LeftLabel{Bits Access}
    \TrinaryInfC{$\Gamma\vdash E\ \texttt{.[}\ i\ \texttt{]} : \ty{Bool}$}
\end{prooftree}

\begin{prooftree}
    \AxiomC{$\Gamma\vdash E = \sigma : \ty{Bits}$}
    \AxiomC{$\Gamma\vdash i : \ty{Integer}$}
    \AxiomC{$i < 0$}
    \LeftLabel{Bits MIA}
    \TrinaryInfC{$\Gamma\vdash E\ \texttt{.[}\ i\ \texttt{]} \vartriangle \val{'MIA}$}
\end{prooftree}

For Records, the expression in the brackets may be any value which is
the key of the element to access. If the key is not found in the
record, \val{'MIA} is yielded. Any value is valid as a record key,
so there are no runtime type errors for record access.

\begin{prooftree}
    \AxiomC{$\Gamma\vdash E : \ty{Record}\ \kappa:\nu$}
    \AxiomC{$\Gamma\vdash k : \kappa$}
    \LeftLabel{Record Access}
    \BinaryInfC{$\Gamma\vdash E\ \texttt{.[}\ k\ \texttt{]} : \nu$}
\end{prooftree}

If the value on the left is of any other type, it is a runtime type error;
other types of values do not have members which can be accessed.


\subsubsection{Application}

Functions in \Poetry{} are single-argument, and that argument is supplied
simply by writing it after the function, separated by a space. The function
itself may be an expression, including another application expression, allowing
for multi-argument functions to be simulated by passing further space separated
parameters.

\begin{bnf*}
    \bnfprod{Application}{
        \bnfpn{Expr}
        \bnfsp
        \bnfpn{Expr}
    }
\end{bnf*}

The first expression must evaluate to a value of Function type, otherwise
it is a runtime type error.

\begin{prooftree}
    \AxiomC{$\Gamma\vdash F : \tau \rightarrow \rho$}
    \AxiomC{$\Gamma\vdash E : \tau$}
    \LeftLabel{Application}
    \BinaryInfC{$\Gamma\vdash F\ E : \rho$}
\end{prooftree}

In terms of precedence, the expression \texttt{f~a~b~c} is equivalent to
\texttt{((f~a)~b)~c}. Parentheses and the application operator (\op{|>},
explored in \S\ref{sec:binop}) may be used to manipulate the precedence as
necessary.


\subsubsection{Procedure Call}

Much like a procedure call in \Prose{}, procedures may be called in \Poetry{}.
Though this sort of breaks the ``pureness'' of \Poetry{}, I think it's also
sort of justifiable if you consider the procedure call to actually be syntax
sugar for a \kw{yield} that corresponds to a handler that calls that procedure
and resumes with the result. Assume that handler is applied automatically by
the procedure's definition.

There's no need to define the syntax of a procedure call in this situation,
as it is exactly the same as the syntax of a procedure call in \Prose{}
(\S\ref{sec:prose-procedure-call}). The semantics however are slightly
different when called in an expression context, adding the relevance of
the procedure call's returned value. Honestly, the definition earlier
could have just had this type to begin with, and ignored it.

\begin{prooftree}
    \AxiomC{$\Gamma\vdash p : \texttt{!}(x_{1\hdots n})\rightarrow\tau$}
    \AxiomC{$a_i = x_i$}
    \LeftLabel{Procedure Call}
    \BinaryInfC{$\Gamma\vdash p\texttt{!}(a_{1\hdots n}):\tau$}
\end{prooftree}


\subsubsection{Binary Operation}
\label{sec:binop}

Binary operators in \Poetry{} are infix operators, and are always written
with both arguments. As previously mentioned, operator precedence is not
handled by the syntax tree but by another pass afterwards.

\begin{bnf*}
    \bnfprod{BinaryOp}{
        \bnfpn{Expression}
        \bnfsp
        \bnfpn{BinaryOperator}
        \bnfsp
        \bnfpn{Expression}
    } \\
    \bnfprod{BinaryOperator}{
        \bnfts{\kw{and}}
        \bnfor
        \bnfts{\kw{or}}
        \bnfor
        \bnfts{\op{+}}
        \bnfor
        \bnfts{\op{-}}
        \bnfor
        \bnfts{\op{*}}
        \bnfor
        \bnfts{\op{/}}
        \bnfor
        \bnfts{\op{\%}}
        \bnfor
        \bnfts{\op{**}}
        \bnfor
        \bnfts{\op{//}}
        \bnfor
    } \\
    \bnfmore{
        \bnfts{\op{==}}
        \bnfor
        \bnfts{\op{===}}
        \bnfor
        \bnfts{\op{<}}
        \bnfor
        \bnfts{\op{>}}
        \bnfor
        \bnfts{\op{>=}}
        \bnfor
        \bnfts{\op{<=}}
        \bnfor
    } \\
    \bnfmore{
        \bnfts{\op{\&}}
        \bnfor
        \bnfts{\op{|}}
        \bnfor
        \bnfts{\op{\textasciicircum}}
        \bnfor
        \bnfts{\op{<\textasciitilde}}
        \bnfor
        \bnfts{\op{\textasciitilde>}}
        \bnfor
        \bnfts{\op{,}}
        \bnfor
        \bnfts{\op{:}}
        \bnfor
        \bnfts{\op{.}}
        \bnfor
    } \\
    \bnfmore{
        \bnfts{\op{<>}}
        \bnfor
        \bnfts{\op{>>}}
        \bnfor
        \bnfts{\op{<<}}
        \bnfor
        \bnfts{\op{|>}}
        \bnfor
        \bnfts{\op{<|}}
    } \\
\end{bnf*}


\subsubsection{Unary Operation}
\label{sec:unaryop}

There are a few unary operators in \Poetry{}, all of which are prefix operators.
There are no postfix operators in \Trilogy{}, as they end up being a parsing
nightmare for little value (the only commonly seen postfix operators are
ones like \texttt{++}, and we prefer \texttt{+= 1} in general these days).

\begin{bnf*}
    \bnfprod{UnaryOp}{
        \bnfpn{UnaryOperator}
        \bnfsp
        \bnfpn{Expression}
    } \\
    \bnfprod{UnaryOperator}{
        \bnfts{\op{-}}
        \bnfor
        \bnfpn{OnlyUnaryOperator}
    } \\
    \bnfprod{OnlyUnaryOperator}{
        \bnfts{\kw{not}}
        \bnfor
        \bnfts{\op{\textasciitilde}}
        \bnfor
        \bnfts{\kw{yield}}
    }
\end{bnf*}

It turns out, even prefix unary operators cause some parsing ambiguity, particularly
when it comes to application of a function to a negative number. To resolve this
ambiguity, we simply declare unary operators to be the last choice when parsing;
only if there is no other interpretation of the expression but the interpretation
where the operator is used as a unary operator will it be a interpreted unary operation.

To illustrate that point, the following table summarizes some syntax and a
less ambiguous ``formal'' representation. The formal representation is actually
valid \Trilogy{} code as well, showing that \Trilogy{} could be used like a Lisp
if you really wanted to.

\begin{table}[H]
    \centering
    \begin{tabular}{lll}
        \hline
        \textbf{Conventional} & \textbf{Formal} & \textbf{Explanation} \\
        \hline
        \tri{x - 5} & \tri{((-) x 5)} & Infix takes precedence \\
        \tri{x (- 5)} & \tri{(x ((-) 5))} & Explicitly disambiguated \\
        \tri{x (-) 5} &  \tri{((x (-)) 5))} & \tri{(-)} is a reference \\
        \tri{x not true} & \tri{(x ((not) true))} & \tri{not} cannot be infix \\
        \tri{x (not) true} & \tri{((x (not)) true))} & \tri{(not)} is a reference \\
        \tri{x + - 5} & \tri{((+) x ((-) 5))} & One of two operators is unary \\
        \hline
    \end{tabular}
\end{table}

The \kw{not} operator is used to perform Boolean negation. \texttt{\kw{not}~\kw{true}}
evaluates to \kw{false} and \texttt{\kw{not}~\kw{false}} evaluates to \kw{true}, as
you might have expected. If the expression is not a Boolean, it causes a runtime type
error. This is much like the operator \texttt{!} seen in other languages.

\begin{figure}[H]
    \centering
    \parbox[t]{0.4\linewidth}{
        \begin{prooftree}
            \AxiomC{$\Gamma\vdash E=\kw{true}$}
            \LeftLabel{Not}
            \UnaryInfC{$\Gamma\vdash \kw{not}\ E=\kw{false}$}
        \end{prooftree}
    }
    \parbox[t]{0.4\linewidth}{
        \begin{prooftree}
            \AxiomC{$\Gamma\vdash E=\kw{false}$}
            \LeftLabel{Not}
            \UnaryInfC{$\Gamma\vdash \kw{not}\ E=\kw{true}$}
        \end{prooftree}
    }
\end{figure}

The operator \op{-} is the numeric negation operator. In \Trilogy{} there are no
negative numeric literals, so all negative numbers must go through this operator.
Used on a non-Number value, it causes a runtime type error.

\begin{prooftree}
    \AxiomC{$\Gamma\vdash E = n : \ty{Number}$}
    \LeftLabel{Negation}
    \UnaryInfC{$\Gamma\vdash \op{-}\ E = -n$}
\end{prooftree}

The operator \op{\textasciitilde} is the bitwise negation operator. Used on a value
of Bits type, the state of each bit in the value is inverted.
Used on a non-Bits value, it causes a runtime type error.

\begin{prooftree}
    \AxiomC{$\Gamma\vdash E:\ty{Bits}$}
    \LeftLabel{Bitwise Negation}
    \UnaryInfC{$\Gamma\vdash \op{\textasciitilde}\ E:\ty{Bits}$}
\end{prooftree}

The \kw{yield} keyword used in \Poetry{} is actually also an operator, mostly for
convenience. This operator works similarly to how the \kw{yield} statement of \Prose{}
works, but the returned value is accessible. It may return more or less than once if
the effect handler resumes more or less that once. Though it would be nice to explain
it now, once again leave the detailed explanation to \S\ref{sec:effects}.


\subsubsection{Let}

The \kw{let} statement in \Prose{} exposes a ``binding context'' in which
bindings may be introduced, similar to that of \Law{}. Depending on how many times
the binding pattern matches, this may introduce a branch or cause a fizzle.
Most often, however, such bindings are done via a direct single unification
which is determinisitic, and so control flow will progress intuitively.

\begin{bnf*}
    \bnfprod{SLet}{
        \bnfts{\kw{let}}
        \bnfsp
        \bnfpn{Query}
    } \\
\end{bnf*}

Semantically, a \kw{let} statement introduces an implicit scope from its location
in the source code until the end of the nearest explicit scope (e.g.\ the end of a block).
The bindings declared in this statement are available within that scope.

What is more unique to \Trilogy{} is the branching and fizzling capability of the
\kw{let} statement when provided a query via a more complex rule of \Law{}.
Depending on how many possible bindings there are for the provided rule, a \kw{let}
statement may lead to multiple executions of the program running in parallel, or none.

\begin{prooftree}
    \AxiomC{$\Gamma \vdash Q \Rightarrow \Phi$}
    \LeftLabel{Branch}
    \UnaryInfC{$\Gamma \vdash \kw{let}\ Q \Rightarrow \Phi$}
\end{prooftree}


\subsubsection{Conditionals}

\Poetry{}'s take on the conditional uses slightly modified syntax when
compared to that of \Prose{}. Rather than using blocks as the bodies of
the branches, the bodies are expressions. To separate the condition from
the body, the \kw{then} is used.

Also worth noting is that there must always be an \kw{else} clause, as
the expression must always evaluate to some value.

\begin{bnf*}
    \bnfprod{IfElse}{
        \bnfts{\kw{if}}
        \bnfsp
        \bnfpn{Expression}
        \bnfsp
        \bnfts{\kw{then}}
        \bnfsp
        \bnfpn{Expression}
        \bnfsp
        \bnfts{\kw{else}}
        \bnfsp
        \bnfpn{Expression}
    }
\end{bnf*}

This rigid form makes for a relatively simple semantics. Similar to the
conditional statement, the condition is an expression that must evaluate
to a Boolean value. When \kw{true}, the \kw{then} block is evaluated.
Whe \kw{false}, the \kw{else} block is evaluated. When not a Boolean value,
then it is considered a runtime type error.

\begin{prooftree}
    \AxiomC{$\Gamma\vdash E = \kw{true}$}
    \AxiomC{$\Gamma\vdash T : \tau$}
    \LeftLabel{If}
    \BinaryInfC{$\Gamma\vdash \kw{if}\ E\ \kw{then}\ T\ \kw{else}\ F : \tau$}
\end{prooftree}

\begin{prooftree}
    \AxiomC{$\Gamma\vdash E = \kw{false}$}
    \AxiomC{$\Gamma\vdash F : \tau$}
    \LeftLabel{Else}
    \BinaryInfC{$\Gamma\vdash \kw{if}\ E\ \kw{then}\ T\ \kw{else}\ F : \tau$}
\end{prooftree}


\subsubsection{Match-Else}

The \kw{match}-\kw{else} expression unifies a value with its cases, and evaluates the
branch associated with the first unification that succeeds. This is much
like the \kw{match} statement of \Prose{}, but as all expressions are required
to evaluate to some value, the \kw{match}-\kw{else} expression requires a specific
\kw{else} clause to be run in the situation that no \kw{case} was selected.

Repeated here are the same grammar for Cases as seen in \S\ref{sec:prose-match}.

\begin{bnf*}
    \bnfprod{MatchElse}{
        \bnfts{\kw{match}}
        \bnfsp
        \bnfpn{Expression}
        \bnfsp
        \bnfpn{Cases}
        \bnfsp
        \bnfts{\kw{else}}
        \bnfsp
        \bnfpn{ElseCase}
    } \\
    \bnfprod{Cases}{
        \bnfpn{Case}
        \bnfsp
        \bnfpn{Cases}
        \bnfor
        \bnfpn{Case}
    } \\
    \bnfprod{Case}{
        \bnfts{\kw{case}}
        \bnfsp
        \bnfpn{Pattern}
        \bnfsp
        \bnfpn{CaseGuard}
        \bnfsp
        \bnfpn{CaseBody}
    } \\
    \bnfprod{CaseGuard}{
        \bnfts{\kw{if}}
        \bnfsp
        \bnfpn{Expression}
        \bnfor
        \bnfes
    } \\
    \bnfprod{CaseBody}{
        \bnfts{\kw{then}}
        \bnfsp
        \bnfpn{Expression}
        \bnfor
        \bnfpn{Block}
    } \\
    \bnfprod{ElseCase}{
        \bnfts{\kw{else}}
        \bnfsp
        \bnfpn{Binding}
        \bnfsp
        \bnfpn{CaseBody}
        \bnfor
    } \\
    \bnfmore{
        \bnfts{\kw{else}}
        \bnfsp
        \bnfpn{Wildcard}
        \bnfsp
        \bnfpn{CaseBody}
    } \\
\end{bnf*}

The \kw{match}-\kw{else} expression is evaluated in much the same way as the \kw{match}
statement of \Prose{} (\S\ref{sec:prose-match}). The only addition is the inclusion of
the \kw{else} case, and the fact that the result of evaluating the matched case is used
as the resulting value of the whole \kw{match}-\kw{else} expression.

\begin{prooftree}
    \AxiomC{$\Gamma \vdash E$}
    \AxiomC{$P = E$}
    \AxiomC{$\Gamma,\setof{a\given a \defby P} \vdash C = \kw{true}, B : \tau$}
    \LeftLabel{Match}
    \TrinaryInfC{$\Gamma \vdash \kw{match}\ E\ \kw{case}\ P\ \kw{if}\ C\ \kw{then}\ B : \tau$}
\end{prooftree}

\begin{prooftree}
    \def\extraVskip{3.5pt}
    \AxiomC{$\Gamma \vdash E$}
    \AxiomC{$P = E$}
    \AxiomC{$\Gamma,\setof{a\given a \defby P} \vdash C = \kw{false}$}
    \TrinaryInfC{$R_1 = \kw{case}\ P\ \kw{if}\ C\ \kw{then}\ B$}
    \AxiomC{$\Gamma \vdash \kw{match}\ E\ R_2 \cdots R_n : \tau$}
    \LeftLabel{Match\textsuperscript{\kw{false}}}
    \insertBetweenHyps{\hskip -12pt}
    \BinaryInfC{$\Gamma \vdash \kw{match}\ E\ R_1 R_2 \cdots R_n : \tau$}
\end{prooftree}

\vskip 0.5em

\begin{prooftree}
    \def\extraVskip{3.5pt}
    \AxiomC{$\Gamma \vdash E$}
    \AxiomC{$P \neq E$}
    \BinaryInfC{$R_1 = \kw{case}\ P\ \kw{if}\ C\ \kw{then}\ B$}
    \AxiomC{$\Gamma \vdash \kw{match}\ E\ R_2 \cdots R_n : \tau$}
    \LeftLabel{Match\textsuperscript{$\bot$}}
    \BinaryInfC{$\Gamma \vdash \kw{match}\ E\ R_1 R_2 \cdots R_n : \tau$}
\end{prooftree}

The omitted clauses of the \kw{case} statement are formalized by the same syntax
transformation as before, as well as the additional \kw{else} case:

\begin{align*}
    \texttt{\kw{case} \$p:pat} &\Rightarrow \texttt{\kw{case} \$p \kw{if} \kw{true}} \\ %
    \texttt{\kw{case} \kw{if} \$c:expr} &\Rightarrow \texttt{\kw{case} \_ \kw{if} \$c} \\ %
    \texttt{\kw{else} \_ \kw{then}} &\Rightarrow \texttt{\kw{case} \_ \kw{then}} \\ %
    \texttt{\kw{else} \$i:id \kw{then}} &\Rightarrow \texttt{\kw{case} \$i \kw{then}} \\ %
\end{align*}


\subsubsection{Is}

The \kw{is} keyword used in \Poetry{} does the opposite of what it does
in \Law{}; that is, it checks if a query has any solutions and converts
that result into a Boolean value.

\begin{bnf*}
    \bnfprod{Is}{
        \bnfts{\kw{is}}
        \bnfsp
        \bnfpn{Query}
    }
\end{bnf*}

\begin{figure}[H]
    \centering
    \parbox[t]{0.4\linewidth}{
        \begin{prooftree}
            \AxiomC{$\Gamma\vdash Q:\bot$}
            \LeftLabel{Is}
            \UnaryInfC{$\Gamma\vdash \kw{is}\ Q=\kw{false}$}
        \end{prooftree}
    }
    \parbox[t]{0.4\linewidth}{
        \begin{prooftree}
            \AxiomC{$\Gamma\vdash Q:\top$}
            \LeftLabel{Is}
            \UnaryInfC{$\Gamma\vdash \kw{is}\ Q=\kw{true}$}
        \end{prooftree}
    }
\end{figure}


\subsubsection{End}

The \kw{end} keyword can be used in \Poetry{}, with much the same
behaviour as in \Prose{}: called without a value it triggers a fizzle,
with a value it ends the program using that value as the exit code.
See \S\ref{sec:prose-end} for slightly more detail, but not really.

\begin{bnf*}
    \bnfprod{End}{
        \bnfts{\kw{end}}
        \bnfor
        \bnfts{\kw{end}}
        \bnfsp
        \bnfpn{Expr}
    }
\end{bnf*}

In fact, the syntax and semantics are exactly the same. There is truly no
reason why this needs to be distinct from the statement form, but that might
just be a coincidence so it remains a distinct piece of syntax for no reason
other than formality.

\begin{prooftree}
    \AxiomC{}
    \LeftLabel{End}
    \UnaryInfC{$\Gamma\vdash\kw{end}\ : \bot$}
\end{prooftree}

\begin{prooftree}
    \AxiomC{$\Gamma\vdash N:\top$}
    \LeftLabel{End}
    \UnaryInfC{$\Gamma\vdash\kw{end}\ N : \bot$}
\end{prooftree}


\subsubsection{First Class Keywords}
\label{sec:first-class-keyword}

The four keywords \kw{resume}, \kw{return}, \kw{break}, and \kw{continue} are
considered ``first class keywords'', meaning they may be used in expressions,
stored in variables, and passed to functions to be called later.

These four keywords are bound syntactically to their respective structures:
\kw{resume} to the nearest enclosing effect handler, \kw{break} and \kw{continue}
to the nearest enclosing loop, and \kw{return} to the nearest enclosing procedure.
When used in an expression, no matter how deeply nested in other structures or even
closures, they continue to refer to these scopes (though do note that a new \kw{return}
is introduced for nested procedure closures).

The syntactic binding becomes more important when the keyword is assigned to variables:
even when passed as a value to another function or stored in a variable, the keyword
continues to refer to the structure that syntactically enclosed it.

\begin{bnf*}
    \bnfprod{FirstClassKeyword}{
        \bnfts{\kw{resume}}
        \bnfor
        \bnfts{\kw{return}}
        \bnfor
        \bnfts{\kw{break}}
        \bnfor
        \bnfts{\kw{continue}}
    } \\
    \bnfprod{Resume}{
        \bnfts{\kw{resume}}
        \bnfsp
        \bnfpn{Expression}
    } \\
    \bnfprod{Return}{
        \bnfts{\kw{return}}
        \bnfsp
        \bnfpn{Expression}
    } \\
    \bnfprod{Break}{
        \bnfts{\kw{break}}
        \bnfsp
        \bnfpn{Expression}
    } \\
    \bnfprod{Continue}{
        \bnfts{\kw{continue}}
        \bnfsp
        \bnfpn{Expression}
    } \\
\end{bnf*}

In each case, these keywords represent a \emph{continuation}---a point in the
construction of the program to jump to---and accept a value that the continuation is
resumed with. Note that for \kw{break} and \kw{return}, the value is simply lost as
there is no way to receive that value at this time. It is convention to use \kw{unit}
in this situation to acknowledge that the value will not be received.

The value supplied to the \kw{return} keyword is used as the value that the function
it is bound to returns when called, as it would be when used regularly as a statement.

Semantically, none of these are particularly interesting, though \kw{resume}, being
part of the effect system and notably harder to define semantics for, is once again
left for \S\ref{sec:effects}.

\begin{prooftree}
    \AxiomC{$\Gamma\vdash E:\top$}
    \LeftLabel{Break}
    \UnaryInfC{$\Gamma\vdash \kw{break}\ E : \bot$}
\end{prooftree}

\begin{prooftree}
    \AxiomC{$\Gamma\vdash E:\top$}
    \LeftLabel{Continue}
    \UnaryInfC{$\Gamma\vdash \kw{continue}\ E : \bot$}
\end{prooftree}

\begin{prooftree}
    \AxiomC{$\Gamma\vdash E:\top$}
    \LeftLabel{Return}
    \UnaryInfC{$\Gamma\vdash \kw{return}\ E : \bot$}
\end{prooftree}

More interesting than the formal semantics of these statements is a discussion on
the practical semantics, which I believe are quietly hidden in the formal semantics,
but also I am no expert and so maybe they are overlooked.

The \kw{break} keyword is the least interesting; it simply ends the loop that it
is bound to, and the program resumes execution from the point after the end of the
loop. However, if the \kw{break} keyword \emph{escapes} the loop, and is called
a second time, execution once again jumps back to the end of that loop. Honestly, not
the most useful behaviour, but it is consistent with the way continuations work, and
so it remains.

The \kw{continue} keyword is slightly more interesting: it's continuation resumes
at the beginning of the next iteration of the loop. While syntactically it is visibly
bound to a loop, it is also semantically bound to a particular \emph{iteration} of
that loop. If the \kw{continue} keyword escapes its iteration and is called,
the loop will start from an iteration that has already occurred. Combined with
\kw{break} it is also now possible to escape a loop, do some things, then go back
into the loop and continue. What this would be useful for is unknown (why didn't
you just put that code in the loop to begin with?) but again, it is consistent with
how continuations work, and so it is how it is.

Finally, the \kw{return} statement is also interesting in that it is, to some degree,
a realization of the operator \textbf{J}, described by Peter J. Landin\cite{j}.
Passing the return value to another function allows for a returned value to skip
multiple layers and return from the function the first class keyword was bound to.
If the \kw{return} escapes its function (e.g. by being itself returned), we now have
a way to pass a value ``back in time''. It was shown by John C. Reynolds that \textbf{J}
can be used to implement the call/cc function, much like the one seen in some Lisps\cite{defint}.


\subsubsection{Closures}

Closures in \Trilogy{} come in both function closure and procedure closure forms.
Function closures, I feel, are a very legitimate construct. Procedure closures
on the other hand are even more definitely breaking the illusion of \Poetry{}'s
pureness than even including regular procedure calls, but if you squint really
hard you can almost let it go. Hopefully you can excuse this breach of mathematical
correctness for the matter of practicality supplied by the ability to define
procedure closures, and that this does not cause more issues than its worth
in implementation.

\begin{bnf*}
    \bnfprod{FnClosure}{
        \bnfts{\kw{fn}}
        \bnfsp
        \bnfpn{ParameterList}
        \bnfsp
        \bnfts{.}
        \bnfsp
        \bnfpn{Expression}
    } \\
    \bnfprod{DoClosure}{
        \bnfts{\kw{do}}
        \bnfsp
        \bnfts{(}
        \bnfsp
        \bnfpn{PatternList}
        \bnfsp
        \bnfts{)}
        \bnfsp
        \bnfpn{Block}
        \bnfor
    } \\
    \bnfmore{
        \bnfts{\kw{do}}
        \bnfsp
        \bnfts{(}
        \bnfsp
        \bnfpn{PatternList}
        \bnfsp
        \bnfts{)}
        \bnfsp
        \bnfpn{Expression}
    } \\
    \bnfprod{QyClosure}{
        \bnfts{\kw{qy}}
        \bnfsp
        \bnfts{(}
        \bnfsp
        \bnfpn{PatternList}
        \bnfsp
        \bnfts{)}
        \bnfsp
        \bnfts{<-}
        \bnfsp
        \bnfpn{Query}
    }
\end{bnf*}

The \kw{fn} desginates a function closure, the \kw{fn} expression evaluating
to a function as any other function definition, only that this function
captures bindings from its environment. These captured bindings may escape
the lifetime of their syntactic scope if the closure escapes that scope.

As with regular function definitions, \kw{fn} includes support for
multi-argument functions as a syntax transformation into single argument
functions returning further single argument functions.

\begin{align*}
    \texttt{\kw{fn} \$(\$r:pat)+ \$p:pat. \$e:expr} & \Rightarrow \texttt{\kw{fn} \$(\$r)+. \kw{fn} \$p. \$e} %
\end{align*}

Given that, only the semantics of single argument functions need to be specified.

\begin{prooftree}
    \AxiomC{$\Sigma\vdash X : \tau$}
    \AxiomC{$P = X$}
    \AxiomC{$\Gamma,\setof{a\given a \defby P}\vdash E : \rho$}
    \LeftLabel{Function Closure}
    \TrinaryInfC{$\Gamma\vdash \kw{fn}\ P\ \texttt{.}\ E : \tau \rightarrow \rho$}
\end{prooftree}

Procedure closures, desginated by \kw{do}, come in two forms, but work
similarly. In the form accepting a block, that block is used as the body
of the procedure, much the same as any regular procedure definition. The
\kw{return} keyword may be used in this block to return a value, otherwise
the procedure evaluates to \kw{unit} when called.

The form accepting an expression actually works more like a function, where
that expression is evaluated and returned automatically. This is particularly
to cover for the fact that function closures (and indeed functions in general)
may not be nullary so, to be able to defer a computation, a procedure is a
fine substitute.

Similarly to function closures, the procedure closure captures bindings from
its environment, and may allow those captured bindings to escape the lifetime
of their syntactic scopes if the closure escapes that scope.

In both forms of procedure closure, the first class keyword \kw{return} is rebound to
this new procedure. To capture a parent procedure's return statement for use in a
procedure closure, it must first be renamed by assigning it to another variable.

Despite the \kw{do} notation not using an exclamation mark, to eventually call
this procedure does require the exclamation mark, as with any other procedure call.
It is only omitted from the \kw{do} notation because it is ugly.

\begin{prooftree}
    \def\extraVskip{3.5pt}
    \AxiomC{$\Sigma\vdash X_i : \tau_i$}
    \AxiomC{$P_i = X_i$}
    \BinaryInfC{$\Gamma,\setof{a_i\given a_i \defby P_i}\vdash M \Rightarrow \Phi$}
    \AxiomC{$\Phi\vdash N : \rho$}
    \def\extraVskip{2pt}
    \LeftLabel{Procedure Closure}
    \BinaryInfC{$\Gamma\vdash \kw{do}\texttt{(}\ P_{1\hdots n}\ \texttt{)}\ \block{M\ \op{;}\ \kw{return}\ N} : \texttt{!(} \tau_{1\hdots n} \texttt{)} \rightarrow \rho$}
\end{prooftree}

\begin{prooftree}
    \AxiomC{$\Sigma\vdash X_i : \tau_i$}
    \AxiomC{$P_i = X_i$}
    \AxiomC{$\Gamma,\setof{a_i\given a_i \defby P_i}\vdash M : \top$}
    \LeftLabel{Procedure Closure}
    \TrinaryInfC{$\Gamma\vdash \kw{do}\texttt{(}\ P_{1\hdots n}\ \texttt{)}\ \block{M} : \texttt{!(} \tau_{1\hdots n} \texttt{)} \rightarrow \kw{unit}$}
\end{prooftree}

\begin{prooftree}
    \AxiomC{$\Sigma\vdash X_i : \tau_i$}
    \AxiomC{$P_i = X_i$}
    \AxiomC{$\Gamma,\setof{a_i\given a_i \defby P_i}\vdash E : \rho$}
    \LeftLabel{Procedure Closure}
    \TrinaryInfC{$\Gamma\vdash \kw{do}\texttt{(}\ P_{1\hdots n}\ \texttt{)}\ E : \texttt{!(} \tau_{1\hdots n} \texttt{)} \rightarrow \rho$}
\end{prooftree}

Query closures, designated by \kw{qy}, create closures similar to rules. Such queries
can be used in \kw{for} loops, comprehensions, or passed to other higher-order rules,
functions, or procedures.

\begin{prooftree}
    \AxiomC{$\Sigma\vdash X_i : \tau_i$}
    \AxiomC{$P_i = X_i$}
    \AxiomC{$\Gamma,\setof{a_i\given a_i \defby P_i}\vdash Q \Rightarrow \Phi$}
    \LeftLabel{Query Closure}
    \TrinaryInfC{$\Gamma\vdash \kw{qy}\texttt{(}\ P_{1\hdots n}\ \texttt{)}\ \texttt{<-}\ Q : \texttt{?(} \tau_{1\hdots n} \texttt{)} \Rightarrow \Phi$}
\end{prooftree}


\subsubsection{Templates}

Templates are strings into which other expressions are evaluated and then inserted.
That is, templates are \Trilogy{}'s version of string interpolation.

Templates are not just for string interpolation, however, as they may be \emph{tagged},
allowing them to run user-defined code instead of simple string concatenation,
resulting in non-string results. This idea comes right out of Javascript. A tagged
template is a regular (bare) template followed directly by an identifier, the tag.

The wonky looking template tokens from the scanner pay off here, making parsing
relatively easy for templates.

\begin{bnf*}
    \bnfprod{Template}{
        \bnfpn{BareTemplate}
        \bnfsp
        \bnfpn{Identifier}
        \bnfor
        \bnfpn{BareTemplate}
    } \\
    \bnfprod{BareTemplate}{
        \bnfpn{TemplateStart}
        \bnfsp
        \bnfpn{TemplateMore}
    } \\
    \bnfprod{TemplateMore}{
        \bnfpn{Expression}
        \bnfsp
        \bnfpn{TemplateMiddle}
        \bnfsp
        \bnfpn{TemplateMore}
        \bnfor
    } \\
    \bnfmore{
        \bnfpn{Expression}
        \bnfsp
        \bnfpn{TemplateEnd}
    }
\end{bnf*}

The semantics are quite different depending on whether the template was tagged
or not. In the untagged case, a template evaluates to a string by evaluating each
inner expression and converting it to its string representation, then concatenating
those strings as if with glue.

Unfortunately, templates are a bit hard to represent formally, so excuse these wonky
proofs and do your best to interpret them sanely.

\begin{prooftree}
    \AxiomC{$\Gamma\vdash E = v$}
    \AxiomC{$\Gamma\vdash T = \omega$}
    \AxiomC{$\Gamma\vdash S = \sigma\ \op{<>}\ \text{str}(v)\ \op{<>}\ \omega$}
    \LeftLabel{Template}
    \TrinaryInfC{$\Gamma\vdash \sigma\ E\ T = S$}
\end{prooftree}

A tagged template, on the other hand, does not stringify anything. Instead, the
string fragments and the evaluated values are both passed as the parameters to
the function denoted by the tag. This tag function should accept two parameters,
the first the array of string fragments, the second the array of the evaluated
values.

\begin{prooftree}
    \def\defaultHypSeparation{\hskip 0.1in}
    \AxiomC{$\Gamma\vdash E_i = v_i$}
    \AxiomC{$\Gamma\vdash f : \ty{List String} \rightarrow \ty{List}\ \tau \rightarrow \rho$}
    \UnaryInfC{$\Gamma\vdash f\ \texttt{[}\sigma_0\texttt{,}\ \sigma_1\texttt{,}\ \cdots\texttt{,}\ \sigma_n\texttt{]}\ \texttt{[}v_1\texttt{,}\ \cdots\texttt{,}\ v_n\texttt{]} : \rho$}
    \LeftLabel{Tagged Template}
    \BinaryInfC{$\Gamma\vdash \sigma_0\ (E_i\ \sigma_i)^*\ f : \rho$}
\end{prooftree}


\subsubsection{Expression Precedence}
\label{sec:precedence}

Expressions in \Trilogy{}, as in many other languages, are assigned precedence values to
allow the removal of cluttering parentheses. These precedences are not handled directly
in the grammar, but only by word of mouth in this section. The reasoning behind this is
that, however unlikely, future versions of \Trilogy{} may allow the creation of custom operators
or the adjustment of existing operator precedence. To avoid having to rewrite the
grammar in that far and uncertain future, we'll define a precedence table instead.

One unique choice in the design of \Trilogy{} is that, though whitespace is typically
not considered to be significant, a line break can change the interpretation of what
might have been otherwise parsed as an expression. In general,
expressions do not span multiple lines---if there is a line break, the expression ends,
and the next line begins a new sequenced expression as if the sequencing operator (\op{,})
was used instead. The only exception to this rule is if the token at the end of the
current line cannot end an expression (e.g.\ a keyword such as \kw{if}), or if the token
at the beginning of the next line cannot start an expression (e.g.\ an infix operator),
then the line break is ignored.

The observant may note that the \op{-} operator may be used in both prefix or infix position.
Recall (\S\ref{sec:unaryop}) that \emph{an operator is only treated as prefix if it could not possibly
be treated as infix}. Therefore, when the \op{-} operator starts a line following another expression,
it is considered to be the infix operator. Given that it has locked in an infix operator,
there is no way that a line can begin with an infix operator, so it must be a continuation
of the expression on the line before. While this certainly could lead to some confusing
situations, the chance that an expression starting a line meaningfully begins with a negation
is so low that this should never cause issue in practice, meanwhile splitting a long subtraction
onto multiple lines seems fairly common.

Table~\ref{tab:prec} lists expression forms in order of decreasing precedence (that is,
the top rows bind more tightly than the bottom). For many types of expression, this
order could be determined by analyzing the syntax tree, but for easy consumption they
are included in the table.

Listed also is associativity, when relevant: left to right, right to left, or indeterminate.
If associativity is indeterminate, explicit disambiguation with parentheses is required.

\begin{table}[h]
    \centering
    \begin{tabular}{ll}
        \hline
        \textbf{Operator} & \textbf{Associativity} \\
        \hline
        Member access\quad\op{.} & Left to right \\
        Procedure call\quad\op{!()} & Left to right \\
        Unary\quad\kw{not}\quad\op{-}\quad\op{\textasciitilde} & \\
        Rule check\quad\kw{is} & \\
        Function application & Left to right \\
        Paths\quad\op{::} & Left to right \\
        \op{>>} & Left to right \\
        \op{<<} & Left to right \\
        \op{**} & Right to left \\
        \op{\%}\quad\op{/}\quad\op{//}\quad\op{*} & Left to right \\
        \op{+}\quad\op{-} & Left to right \\
        \op{\&} & Left to right \\
        \op{<\textasciitilde}\quad\op{\textasciitilde>} & Left to right \\
        \op{\textasciicircum} & Left to right \\
        \op{|} & Left to right \\
        \op{<>} & Left to right \\
        \op{:} & Right to left \\
        \op{<=}\quad\op{>=}\quad\op{<}\quad\op{>} & Indeterminate \\
        \op{==}\quad\op{===} & Indeterminate \\
        \op{\&\&} & Left to right \\
        \op{||} & Left to right \\
        \op{|>} & Left to right \\
        \op{<|} & Right to left \\
        \hline
        \kw{if}/\kw{match}/\kw{with}\ \kw{else} & \\
        \kw{yield}\quad\kw{resume} & \\
        \kw{end}\quad\kw{exit}\quad\kw{return} & \\
        \kw{cancel}\quad\kw{break}\quad\kw{continue} & \\
        \kw{do}\texttt{()}\quad\kw{fn}\op{.} & \\
        \hline
        \kw{let}\ \op{,} & Right to left \\
        \op{,} & Right to left \\
        \hline
        \kw{when}\quad\kw{given} & Left to right \\
        \hline
    \end{tabular}
    \caption{\label{tab:prec}Expression Precedence}
\end{table}

\FloatBarrier

