\subsubsection{Expression Precedence}
\label{sec:precedence}

Expressions in \Trilogy{}, as in many other languages, are assigned precedence values to
allow the removal of cluttering parentheses. These precedences are not handled directly
in the grammar, but only by word of mouth in this section. The reasoning behind this is
that, however unlikely, future versions of \Trilogy{} may allow the creation of custom operators
or the adjustment of existing operator precedence. To avoid having to rewrite the
grammar in that far and uncertain future, we'll define a precedence table instead.

One unique choice in the design of \Trilogy{} is that, though whitespace is typically
not considered to be significant, a line break can change the interpretation of what
might have been otherwise parsed as an expression. In general,
expressions do not span multiple lines---if there is a line break, the expression ends,
and the next line begins a new sequenced expression as if the sequencing operator (\op{,})
was used instead. The only exception to this rule is if the token at the end of the
current line cannot end an expression (e.g.\ a keyword such as \kw{if}), or if the token
at the beginning of the next line cannot start an expression (e.g.\ an infix operator),
then the line break is ignored.

The observant may note that the \op{-} operator may be used in both prefix or infix position.
Recall (\S\ref{sec:unaryop}) that \emph{an operator is only treated as prefix if it could not possibly
be treated as infix}. Therefore, when the \op{-} operator starts a line following another expression,
it is considered to be the infix operator. Given that it has locked in an infix operator,
there is no way that a line can begin with an infix operator, so it must be a continuation
of the expression on the line before. While this certainly could lead to some confusing
situations, the chance that an expression starting a line meaningfully begins with a negation
is so low that this should never cause issue in practice, meanwhile splitting a long subtraction
onto multiple lines seems fairly common.

Table~\ref{tab:prec} lists expression forms in order of decreasing precedence (that is,
the top rows bind more tightly than the bottom). For many types of expression, this
order could be determined by analyzing the syntax tree, but for easy consumption they
are included in the table.

Listed also is associativity, when relevant: left to right, right to left, or indeterminate.
If associativity is indeterminate, explicit disambiguation with parentheses is required.

\begin{table}[h]
    \centering
    \begin{tabular}{ll}
        \hline
        \textbf{Operator} & \textbf{Associativity} \\
        \hline
        Member access\quad\op{.} & Left to right \\
        Procedure call\quad\op{!()} & Left to right \\
        Unary\quad\kw{not}\quad\op{-}\quad\op{\textasciitilde} & \\
        Rule check\quad\kw{is} & \\
        Function application & Left to right \\
        Paths\quad\op{::} & Left to right \\
        \op{>>} & Left to right \\
        \op{<<} & Left to right \\
        \op{**} & Right to left \\
        \op{\%}\quad\op{/}\quad\op{//}\quad\op{*} & Left to right \\
        \op{+}\quad\op{-} & Left to right \\
        \op{\&} & Left to right \\
        \op{<\textasciitilde}\quad\op{\textasciitilde>} & Left to right \\
        \op{\textasciicircum} & Left to right \\
        \op{|} & Left to right \\
        \op{<>} & Left to right \\
        \op{:} & Right to left \\
        \op{<=}\quad\op{>=}\quad\op{<}\quad\op{>} & Indeterminate \\
        \op{==}\quad\op{===} & Indeterminate \\
        \op{\&\&} & Left to right \\
        \op{||} & Left to right \\
        \op{|>} & Left to right \\
        \op{<|} & Right to left \\
        \hline
        \kw{if}/\kw{match}/\kw{with}\ \kw{else} & \\
        \kw{yield}\quad\kw{resume} & \\
        \kw{end}\quad\kw{exit}\quad\kw{return} & \\
        \kw{cancel}\quad\kw{break}\quad\kw{continue} & \\
        \kw{do}\texttt{()}\quad\kw{fn}\op{.} & \\
        \hline
        \kw{let}\ \op{,} & Right to left \\
        \op{,} & Right to left \\
        \hline
        \kw{when}\quad\kw{given} & Left to right \\
        \hline
    \end{tabular}
    \caption{\label{tab:prec}Expression Precedence}
\end{table}

\FloatBarrier
