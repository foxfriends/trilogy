\subsubsection{Let}

The \kw{let} expression performs a binding for the scope of its body. As with
the \kw{let} statement of \Prose{}, the binding may bind more or less than once,
leading to branching or fizzling behaviour which may be unexpected if not careful.

Interesting about the \kw{let} expression is that the line break is required to
separate the binding from the body, as opposed to an explicit separator keyword
as in languages such as Haskell\footnote{A radical choice, which may be revised later
with more experience. Apologies to anyone who uses this expression if it changes. I
just never liked the way that \texttt{let}\dots\texttt{in} becomes indented.}.

\begin{bnf*}
    \bnfprod{Let}{
        \bnfts{\kw{let}}
        \bnfsp
        \bnfpn{Query}
        \bnfsp
        \bnfpn{EOL}
        \bnfsp
        \bnfpn{Expression}
    }
\end{bnf*}

In a \kw{let} expression, the body is evaluated with the newly introduced bindings,
and the whole expression evaluates to the result of that body's evaluation.

\begin{prooftree}
    \AxiomC{$\Gamma\vdash Q \Rightarrow \Phi$}
    \AxiomC{$\Phi\vdash B : \tau$}
    \LeftLabel{Let}
    \BinaryInfC{$\Gamma\vdash \kw{let}\ Q\ B : \tau$}
\end{prooftree}
