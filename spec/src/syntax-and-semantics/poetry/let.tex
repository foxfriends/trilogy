\subsubsection{Let}

The \kw{let} expression performs a binding for the scope of its body. As with
the \kw{let} statement of \Prose{}, the binding may bind more or less than once,
leading to branching or fizzling behaviour which may be unexpected if not careful.

The use of the comma (\op{,}) as seen in the \kw{let} expression is not actually
the sequencing operator, but just a separator. It is chosen as separator due to the
very similar behaviour of \tri{let _ = x, y} and \tri{x, y}.

\begin{bnf*}
    \bnfprod{Let}{
        \bnfts{\kw{let}}
        \bnfsp
        \bnfpn{Query}
        \bnfsp
        \bnfts{\op{,}}
        \bnfsp
        \bnfpn{Expression}
    }
\end{bnf*}

In a \kw{let} expression, the body is evaluated with the newly introduced bindings,
and the whole expression evaluates to the result of that body's evaluation.

\begin{prooftree}
    \AxiomC{$\Gamma\vdash Q \Rightarrow \Phi$}
    \AxiomC{$\Phi\vdash B : \tau$}
    \LeftLabel{Let}
    \BinaryInfC{$\Gamma\vdash \kw{let}\ Q\ \op{,}\ B : \tau$}
\end{prooftree}
