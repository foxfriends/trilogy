\subsubsection{Structural Patterns}

Structural patterns are patterns that unify based on the shape or structure of
some value using a syntax very similar to the value's literal syntax. In fact,
for the primitive types, the syntax of their structural patterns really are
the same as their literal syntax, which is covered in \S\ref{sec:literals}.
For compound types, the syntax is slightly adjusted to account for some of the
limitations of patterns versus expressions.

\begin{bnf*}
    \bnfprod{StructuralPattern}{
        \bnfpn{Primitive}
        \bnfor
        \bnfpn{Wildcard}
        \bnfor
    } \\
    \bnfmore{
        \bnfpn{NegativePattern}
        \bnfor
        \bnfpn{GluePattern}
        \bnfor
    } \\
    \bnfmore{
        \bnfpn{StructPattern}
        \bnfor
        \bnfpn{TuplePattern}
        \bnfor
    } \\
    \bnfmore{
        \bnfpn{ArrayPattern}
        \bnfor
        \bnfpn{SetPattern}
        \bnfor
    } \\
    \bnfmore{
        \bnfpn{RecordPattern}
        \bnfor
    } \\
    \bnfmore{
        \bnfpn{PinnedIdentifier}
        \bnfor
        \bnfpn{Binding}
    }
\end{bnf*}

For the primitives, they match if the value represented by the literal
matches the value represented by the corresponding expression. A proof
of this matching looks unremarkable.

\begin{prooftree}
    \AxiomC{}
    \LeftLabel{Literal Match}
    \UnaryInfC{$v = v$}
\end{prooftree}

Also very unremarkable is the wildcard pattern, which always unifies with any
value and produces no bindings. This comes in handy for discarding parts of
values that are not needed.

\begin{prooftree}
    \AxiomC{}
    \LeftLabel{Wildcard Match}
    \UnaryInfC{$\texttt{\_} = v$}
\end{prooftree}

As a bit of a special case, \Trilogy{} includes some value matching utilities
at the pattern level.

\begin{bnf*}
    \bnfprod{NegativePattern}{
        \bnfts{\op{-}}
        \bnfsp
        \bnfpn{Pattern}
    }
\end{bnf*}

For numbers, the \op{-} sign may be used in patterns to match a negative number.
This is included largely as a concession for not including negative number
literals in the language, but has the nice property that it may be used on
other patterns as well, particularly identifiers or pinned identifiers, to
match the negation of those values.

\begin{prooftree}
    \AxiomC{$v : \ty{Number}$}
    \AxiomC{$P = \op{-}\ v$}
    \LeftLabel{Negative Match}
    \BinaryInfC{$\op{-}\ P = v$}
\end{prooftree}

For strings, the glue operator (\op{<>}) can be used in patterns to match
a literal prefix or suffix (or both). The literal must be a prefix or
suffix though, a literal ``middle'' portion is not permitted.

\begin{bnf*}
    \bnfprod{GluePattern}{
        \bnfpn{String}
        \bnfsp
        \bnfts{\op{<>}}
        \bnfsp
        \bnfpn{Pattern}
        \bnfsp
        \bnfts{\op{<>}}
        \bnfsp
        \bnfpn{String}
        \bnfor
    } \\
    \bnfmore{
        \bnfpn{String}
        \bnfsp
        \bnfts{\op{<>}}
        \bnfsp
        \bnfpn{Pattern}
        \bnfor
        \bnfpn{Pattern}
        \bnfsp
        \bnfts{\op{<>}}
        \bnfsp
        \bnfpn{String}
    }
\end{bnf*}

Treat the single-sided glue patterns as if they had an empty string
glued on the other side.

\begin{prooftree}
    \AxiomC{$v = \alpha P \omega : \ty{String}$}
    \LeftLabel{Glue Match}
    \UnaryInfC{$\alpha\ \op{<>}\ P\ \op{<>}\ \omega = v$}
\end{prooftree}

Compound types have patterns based on their literal syntax as well.
Unification occurs recursively, succeeding if the nested unifications
succeed.

\begin{bnf*}
    \bnfprod{StructPattern}{
        \bnfpn{Atom}
        \bnfsp
        \bnfts{(}
        \bnfsp
        \bnfpn{Pattern}
        \bnfsp
        \bnfts{)}
    } \\
    \bnfprod{TuplePattern}{
        \bnfpn{Pattern}
        \bnfsp
        \bnfts{\op{:}}
        \bnfsp
        \bnfpn{Pattern}
    } \\
\end{bnf*}

Structs and tuples are easy, they simply unify their one and two values with
the two patterns.

The tuple pattern is special in that it can also be used to unify against
an iterator, in which case the iterator is automatically stepped forwards
so that the unification is working entirely on tuples. This may make it
appear as though the pattern has caused the iterator to take a step, but it
is actually more accurately that the iterator was stepped through as part
of the evaluation of the value with which the pattern is being unified.

\begin{prooftree}
    \AxiomC{$P = v$}
    \LeftLabel{Struct Match}
    \UnaryInfC{$\texttt{'}l\texttt{(}\ P\ \texttt{)} = \texttt{'}l\texttt{(}\ v \ \texttt{)}$}
\end{prooftree}

\begin{prooftree}
    \AxiomC{$L = x$}
    \AxiomC{$R = y$}
    \LeftLabel{Tuple Match}
    \BinaryInfC{$L\ \op{:}\ R = x\ \op{:}\ y$}
\end{prooftree}

Array patterns may include exactly one ``rest'' pattern, which is used to
contain any values not explicitly included in the pattern, allowing for
patterns to match variable length arrays. This complicates the grammar
significantly.

\begin{bnf*}
    \bnfprod{ArrayPattern}{
        \bnfts{[}
        \bnfsp
        \bnfpn{ArrayContents}
        \bnfsp
        \bnfts{]}
    } \\
    \bnfprod{ArrayContents}{
        \bnfpn{PatternListN}
        \bnfsp
        \bnfts{,}
        \bnfsp
        \bnfpn{RestPattern}
        \bnfsp
        \bnfts{,}
        \bnfsp
        \bnfpn{PatternList}
        \bnfor
    } \\
    \bnfmore{
        \bnfpn{PatternListN}
        \bnfsp
        \bnfts{,}
        \bnfsp
        \bnfpn{RestPattern}
        \bnfor
    } \\
    \bnfmore{
        \bnfpn{RestPattern}
        \bnfsp
        \bnfts{,}
        \bnfsp
        \bnfpn{PatternList}
        \bnfor
    } \\
    \bnfmore{
        \bnfpn{RestPattern}
        \bnfor
        \bnfpn{PatternList}
    } \\
    \bnfprod{RestPattern}{
        \bnfts{..}
        \bnfsp
        \bnfpn{Pattern}
    } \\
    \bnfprod{PatternListN}{
        \bnfpn{Pattern}
        \bnfsp
        \bnfts{,}
        \bnfsp
        \bnfpn{PatternListN}
        \bnfor
        \bnfpn{Pattern}
    }
\end{bnf*}

Semantically, array elements are matched pairwise with the element patterns
of the array pattern, starting from the ends. After all explicitly listed
patterns have been unified successfully, the rest pattern (if any) is assigned
to an array containing the remaining unmatched elements of the value, or an
empty array if there were none more.

\begin{prooftree}
    \AxiomC{$v : \ty{Array}\ \tau$}
    \AxiomC{$R = v$}
    \LeftLabel{Array Rest}
    \BinaryInfC{$\texttt{[}\texttt{..}R\texttt{]} = v$}
\end{prooftree}

\begin{prooftree}
    \AxiomC{$v = \texttt{[}X_1,X_2,\cdots,X_n\texttt{]}$}
    \AxiomC{$P = X_1$}
    \AxiomC{$R = \texttt{[}X_2,\cdots,X_n\texttt{]}$}
    \LeftLabel{Array Prefix}
    \TrinaryInfC{$\texttt{[}P\texttt{,}\ \texttt{..}R\texttt{]} = v$}
\end{prooftree}

\begin{prooftree}
    \AxiomC{$v = \texttt{[}X_1,\cdots,X_{n-1},X_n\texttt{]}$}
    \AxiomC{$P = X_n$}
    \AxiomC{$R = \texttt{[}X_1,\cdots,X_{n-1}\texttt{]}$}
    \LeftLabel{Array Suffix}
    \TrinaryInfC{$\texttt{[}\texttt{..}R\texttt{,}\ P\texttt{]} = v$}
\end{prooftree}

Like arrays, set patterns may also contain a single rest pattern, but since sets
are unordered, this rest pattern is required to be at the end of the set pattern.
A trailing comma is not permitted after the rest pattern, as no elements may follow.

\begin{bnf*}
    \bnfprod{SetPattern}{
        \bnfts{\{|}
        \bnfsp
        \bnfpn{SetContents}
        \bnfsp
        \bnfts{|\}}
    } \\
    \bnfprod{SetContents}{
        \bnfpn{PatternListN}
        \bnfsp
        \bnfts{,}
        \bnfsp
        \bnfpn{RestPattern}
        \bnfor
    } \\
    \bnfmore{
        \bnfpn{RestPattern}
        \bnfor
        \bnfpn{PatternList}
    }
\end{bnf*}

Unifying the elements of set patterns is a bit tricky, as there is no requirement
as to which element each pattern corresponds to. To get around this complexity for
now, as set unification has been proven to be NP-complete\cite{setunif}, we simply
forbid set unification patterns to include patterns which are not single-valued.
This forbidding is not handled in the parser, but by a later pass, so is not reflected
in the grammar.

Single valued patterns include literal patterns, pinned identifiers, and compound
patterns whose sub-patterns consist entirely of single valued patterns.

\begin{prooftree}
    \AxiomC{$x \in v$}
    \AxiomC{$x = P_1$}
    \AxiomC{$\texttt{\{|}\ P_2\texttt{,}\ \cdots\texttt{,}\ P_n\texttt{,}\ \ \texttt{..}R\ \texttt{|\}} = v\setminus\setof{x}$}
    \LeftLabel{Set Match}
    \TrinaryInfC{$\texttt{\{|}\ P_1\texttt{,}\ P_2\texttt{,}\ \cdots\texttt{,}\ P_n\texttt{,}\ \texttt{..}R\ \texttt{|\}} = v$}
\end{prooftree}

\begin{prooftree}
    \AxiomC{$v : \ty{Set}\ \tau$}
    \AxiomC{$R = v$}
    \LeftLabel{Set Rest}
    \BinaryInfC{$\texttt{\{|}\ \texttt{..}R\ \texttt{|\}} = v$}
\end{prooftree}

Record patterns simultaneously match keys and values for each entry in
the record. As with sets, though a record may contain a single rest pattern,
that rest pattern must come at the end and may not have a trailing comma.

\begin{bnf*}
    \bnfprod{RecordPattern}{
        \bnfts{\{}
        \bnfsp
        \bnfpn{RecordContents}
        \bnfsp
        \bnfts{\}}
    } \\
    \bnfprod{RecordContents}{
        \bnfpn{KeyValueListN}
        \bnfsp
        \bnfts{,}
        \bnfsp
        \bnfpn{RestPattern}
        \bnfor
    } \\
    \bnfmore{
        \bnfpn{RestPattern}
        \bnfor
        \bnfpn{KeyValueList}
    } \\
    \bnfprod{KeyValueList}{
        \bnfpn{Pattern}
        \bnfsp
        \bnfts{:}
        \bnfsp
        \bnfpn{Pattern}
        \bnfsp
        \bnfts{,}
        \bnfsp
        \bnfpn{KeyValueList}
        \bnfor
    } \\
    \bnfmore{
        \bnfpn{Pattern}
        \bnfsp
        \bnfts{:}
        \bnfsp
        \bnfpn{Pattern}
        \bnfor
        \bnfes
    } \\
    \bnfprod{KeyValueListN}{
        \bnfpn{Pattern}
        \bnfsp
        \bnfts{:}
        \bnfsp
        \bnfpn{Pattern}
        \bnfsp
        \bnfts{,}
        \bnfsp
        \bnfpn{KeyValueListN}
        \bnfor
    } \\
    \bnfmore{
        \bnfpn{Pattern}
        \bnfsp
        \bnfts{:}
        \bnfsp
        \bnfpn{Pattern}
    }
\end{bnf*}

As records (ignoring values) are the same as sets, the same problem applies for
unification, so the patterns for the keys of a record must be single valued patterns.
The patterns associated with those keys may be any pattern, making it possible to
extract a value at a key from a record, which is mostly what you would want to do
with record patterns anyway.

\begin{prooftree}
    \AxiomC{$k : v \in r$}
    \AxiomC{$x = K_1$}
    \AxiomC{$v = V_1$}
    \noLine
    \TrinaryInfC{$\texttt{\{}\ K_2:V_2\texttt{,}\ \cdots\texttt{,}\ K_n:V_n\texttt{,}\ \ \texttt{..}R\ \texttt{\}} = r\setminus\setof{k:v}$}
    \LeftLabel{Record Match}
    \UnaryInfC{$\texttt{\{}\ K_1:V_1\texttt{,}\ K_2:V_2\texttt{,}\ \cdots\texttt{,}\ K_n:V_n\texttt{,}\ \texttt{..}R\ \texttt{\}} = r$}
\end{prooftree}

\begin{prooftree}
    \AxiomC{$v : \ty{Record}\ \kappa\ : \tau$}
    \AxiomC{$R = r$}
    \LeftLabel{Record Rest}
    \BinaryInfC{$\texttt{\{}\ \texttt{..}R\ \texttt{\}} = r$}
\end{prooftree}

A pinned identifier provides a way to create dynamic patterns: the value of the
pinned identifier is used as if it were a literal pattern. This idea could be
extended to arbitrary pinned expressions in future, but for now just pinned
identifiers will suffice.

\begin{bnf*}
    \bnfprod{PinnedIdentifier}{
        \bnfts{\textasciicircum}
        \bnfsp
        \bnfpn{Identifier}
    }
\end{bnf*}

\begin{prooftree}
    \AxiomC{$\Gamma\vdash I = x$}
    \AxiomC{$x = v$}
    \LeftLabel{Pinned Match}
    \BinaryInfC{$\Gamma\vdash\texttt{\textasciicircum}I = v$}
\end{prooftree}
