\subsubsection{Unification}

One of the most fundamental parts of \Law{} and indeed of \Trilogy{} itself
is unification. All assignment and binding in \Trilogy{} occurs through
unification of values and patterns.

\begin{bnf*}
    \bnfprod{Unification}{
        \bnfpn{Pattern}
        \bnfsp
        \bnfts{\op{=}}
        \bnfsp
        \bnfpn{Expression}
        \bnfor
    } \\
    \bnfmore{
        \bnfpn{Pattern}
        \bnfsp
        \bnfts{\kw{in}}
        \bnfsp
        \bnfpn{Expression}
        \bnfor
    } \\
    \bnfmore{
        \bnfts{(}
        \bnfsp
        \bnfpn{Query}
        \bnfsp
        \bnfts{)}
        \bnfor
        \bnfpn{Lookup}
        \bnfor
    } \\
    \bnfmore{
        \bnfts{\kw{pass}}
        \bnfor
        \bnfts{\kw{end}}
        \bnfor
        \bnfts{\kw{is}}
        \bnfsp
        \bnfpn{Expression}
        \bnfor
    } \\
    \bnfmore{
        \kw{not}\bnfsp\bnfpn{Unification}
    }
\end{bnf*}

While most unifications are performed by ``looking up'' (in the database)
a rule and unifying its definition with the inputs, a few special forms are
included for convenience.

The \op{=} ``operator'' is used to unify an expression with a pattern directly. The
expression is a syntax of \Poetry{}, rather than \Law{}, so its details are covered
later. This does break the purely logical aspect of \Law{}; a direct unification in
this way cannot be ``reversed'' the same way pure \Law{} unifications may be called
in reverse.

\begin{prooftree}
    \AxiomC{$\Gamma\vdash E$}
    \AxiomC{$P = E$}
    \AxiomC{$\Gamma,\setof{a \given a \defby P}\extends\Phi$}
    \LeftLabel{Unification}
    \TrinaryInfC{$\Gamma\vdash P\ \op{=}\ E \Rightarrow\Phi$}
\end{prooftree}

The \kw{in} keyword is used to unify an expression with each element of a collection,
binding each element in sequence and continuing the program. Once again, the collection
is specified as an expression in \Poetry{}, so this unification may not be reversed.
The collection may be a Set, Array, List, Record, or Iterator, as described in
Compound Types (\S\ref{sec:compound-types}).

\begin{prooftree}
    \AxiomC{$\Gamma\vdash E$}
    \AxiomC{$e\in E$}
    \AxiomC{$P = e$}
    \AxiomC{$\setof{a \given a \defby P}\extends\Phi$}
    \LeftLabel{Iteration}
    \QuaternaryInfC{$\Gamma\vdash P\ \kw{in}\ E\Rightarrow\Phi$}
\end{prooftree}

Parentheses may be used for grouping, as one might expect. The grouped query works
as if it were a single unification, all bindings made within the group
persisting afterwards.

\begin{prooftree}
    \AxiomC{$\Gamma\vdash P\Rightarrow \Phi$}
    \LeftLabel{Unwrap}
    \UnaryInfC{$\Gamma\vdash \texttt{(}\ P\ \texttt{)} \Rightarrow \Phi$}
\end{prooftree}

The two keywords \kw{pass} and \kw{end} may be used as a unification on
their own as well. \kw{pass} is treated as a unification that succeeds once,
while \kw{end} is treated as a unification that never succeeds.

\begin{figure}[H]
    \centering
    \parbox[t]{0.3\linewidth}{
        \begin{prooftree}
            \AxiomC{}
            \LeftLabel{$\top$}
            \UnaryInfC{$\Gamma \vdash \kw{pass} : \top$}
        \end{prooftree}
    }
    \parbox[t]{0.3\linewidth}{
        \begin{prooftree}
            \AxiomC{}
            \LeftLabel{$\bot$}
            \UnaryInfC{$\Gamma \vdash \kw{end} : \bot$}
        \end{prooftree}
    }
\end{figure}

The keyword \kw{is} may be used to evaluate an expression and convert its Boolean
result into logical success or failure. As this form of unification includes an
expression, it is not reversible. If the expression does not evaluate to a Boolean
value, it is treated as a runtime type error. No bindings are created in this form.

\begin{figure}[H]
    \centering
    \parbox[t]{0.4\linewidth}{
        \begin{prooftree}
            \AxiomC{$\Gamma\vdash E = \kw{true}$}
            \LeftLabel{Is $\top$}
            \UnaryInfC{$\Gamma\vdash \kw{is}\ E : \top$}
        \end{prooftree}
    }
    \parbox[t]{0.4\linewidth}{
        \begin{prooftree}
            \AxiomC{$\Gamma\vdash E = \kw{false}$}
            \LeftLabel{Is $\bot$}
            \UnaryInfC{$\Gamma\vdash \kw{is}\ E : \bot$}
        \end{prooftree}
    }
\end{figure}

The keyword \kw{not} may be used to negate a unification. A negated unification
produces no bindings, but may cause the execution to continue or fizzle
depending on whether its unification succeeds at least once.

\begin{figure}[H]
    \centering
    \parbox[t]{0.45\linewidth}{
        \begin{prooftree}
            \AxiomC{$\Gamma\vdash U : \top$}
            \LeftLabel{Negation}
            \UnaryInfC{$\Gamma\vdash \kw{not}\ U : \bot$}
        \end{prooftree}
    }
    \parbox[t]{0.45\linewidth}{
        \begin{prooftree}
            \AxiomC{$\Gamma\vdash U : \bot$}
            \LeftLabel{Negation}
            \UnaryInfC{$\Gamma\vdash \kw{not}\ U : \top$}
        \end{prooftree}
    }
\end{figure}
