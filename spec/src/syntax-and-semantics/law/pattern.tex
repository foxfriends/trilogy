\subsubsection{Patterns}

Referenced in pretty much every form of unifications seen so far, the pattern is
the way unifications interact with the built-in data types of \Trilogy{}. Values
unify with patterns, binding free variables in the pattern with the corresponding
parts of a value that has the same structure as the pattern expects.

\begin{bnf*}
    \bnfprod{Pattern}{
        \bnfpn{Pattern}
        \bnfsp
        \bnfts{\kw{or}}
        \bnfsp
        \bnfpn{Pattern}
        \bnfor
    } \\
    \bnfmore{
        \bnfpn{Pattern}
        \bnfsp
        \bnfts{\kw{and}}
        \bnfsp
        \bnfpn{Pattern}
        \bnfor
    } \\
    \bnfmore{
        \bnfpn{StructuralPattern}
    } \\
\end{bnf*}

All patterns contain a \emph{structural} portion, and optionally a \emph{type}
portion as well. The structural portion checks the structure of the data with
the ability to bind parts of that data to identifiers, while the type portion
allows describing a domain in which the expected value of a pattern should lie.

Since a pattern does not lead to any value or execution on its own, the semantics for
patterns and their successful binding imply a successful unification, and sometimes
bind variables.

There are two special forms of patterns, using the \kw{or} and \kw{and} keywords.
The \kw{and} pattern requires that both of its sub-patterns are satisfied.

\begin{prooftree}
    \AxiomC{$L = v$}
    \AxiomC{$R = v$}
    \LeftLabel{Both}
    \BinaryInfC{$L\ \kw{and}\ R = v$}
\end{prooftree}

The \kw{or} pattern requires that only one of its sub-patterns are satisfied.
In the case that both of patterns match (unlikely in practice, but there are
some patterns where this is possible) then this pattern may cause a branch.

One special requirement of this pattern not reflected in the syntax or
semantics is that both sides of the \kw{or} pattern must bind the same
identifiers. \Trilogy{} does not allow unbound identifiers in expressions,
so after any pattern unification, all of its identifiers must be bound.

\begin{figure}[H]
    \centering
    \parbox[t]{0.4\linewidth}{
        \begin{prooftree}
            \AxiomC{$L = v$}
            \LeftLabel{Either\textsuperscript{L}}
            \UnaryInfC{$L\ \kw{or}\ R = v$}
        \end{prooftree}
    }
    \parbox[t]{0.4\linewidth}{
        \begin{prooftree}
            \AxiomC{$R = v$}
            \LeftLabel{Either\textsuperscript{R}}
            \UnaryInfC{$L\ \kw{or}\ R = v$}
        \end{prooftree}
    }
\end{figure}

Meanwhile, the parenthesized pattern is included for grouping purposes,
typically in conjunction with the \kw{or} and \kw{and} keywords. The
\kw{and} keyword has higher precedence than \kw{or}, so to represent
\tri{(A or B) and (C or D)} requires the parentheses.

\begin{prooftree}
    \AxiomC{$P = v$}
    \LeftLabel{Unwrap}
    \UnaryInfC{$\texttt{(}\ P\ \texttt{)} = v$}
\end{prooftree}
