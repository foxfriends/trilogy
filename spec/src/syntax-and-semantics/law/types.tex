\subsubsection{Types}

% TODO: this section is not included in the document. After brief thought, I have
% decided to omit types from the first draft of Trilogy; leave them out until there
% is a clear way forward on their implementation.
%
% Types will not be added as an afterthought, but they will be though about after!

Being a dynamically typed language, types in \Trilogy{} are a loose concept,
and are checked primarily at runtime. They are checked though, unlike some
dynamically typed languages which have type annotations that are
ignored at runtime, and only loosely examined statically\footnote{Looking at TypeScript and Python here, to name a few.}.

While type-checking for dynamically typed languages using some crazy
type systems is something I do find interesting, \Trilogy{} takes a
conservative approach to type checking, indeed to the point of almost
being useless. Exploring a fully powerful type system for \Trilogy{}
is left as an experiment for the future.

\begin{bnf*}
    \bnfprod{TypeAnnotation}{
        \bnfts{::}
        \bnfsp
        \bnfpn{TypePattern}
    } \\
    \bnfprod{TypePattern}{
        \bnfpn{Literal}
        \bnfor
        \bnfpn{PrimitiveType}
        \bnfor
        \bnfpn{TupleType}
        \bnfor
    } \\
    \bnfmore{
        \bnfpn{StructType}
        \bnfor
        \bnfpn{ArrayType}
        \bnfor
        \bnfpn{SetType}
        \bnfor
    } \\
    \bnfmore{
        \bnfpn{RecordType}
        \bnfor
        \bnfpn{Identifier}
    } \\
    \bnfprod{PrimitiveType}{
        \bnfts{Number}
        \bnfor
        \bnfts{String}
        \bnfor
        \bnfts{Atom}
        \bnfor
        \bnfts{Bits}
        \bnfor
        \bnfts{Char}
        \bnfor
        \bnfts{Bool}
    } \\
    \bnfprod{TupleType}{
        \bnfpn{TypePattern}
        \bnfsp
        \bnfts{:}
        \bnfsp
        \bnfpn{TypePattern}
    } \\
    \bnfprod{StructType}{
        \bnfpn{Atom}
        \bnfsp
        \bnfts{(}
        \bnfsp
        \bnfpn{TypePattern}
        \bnfsp
        \bnfts{)}
    } \\
    \bnfprod{ArrayType}{
        \bnfts{Array}
        \bnfsp
        \bnfpn{TypePattern}
    } \\
    \bnfprod{SetType}{
        \bnfts{Set}
        \bnfsp
        \bnfpn{TypePattern}
    } \\
    \bnfprod{RecordType}{
        \bnfts{Record}
        \bnfsp
        \bnfpn{TypePattern}
        \bnfsp
        \bnfts{:}
        \bnfsp
        \bnfpn{TypePattern}
    } \\
\end{bnf*}

\Trilogy{}'s type system includes only a few built-in types, type variables,
and the option to use values as types. All of these are special cased uses
of identifiers in type position that just happen to be handled by the parser.
So lacking is this type system that iterators and semantic types cannot even
be expressed, as they would require static analysis which is not included
in the language at this time.

\begin{prooftree}
    \AxiomC{$P = v$}
    \AxiomC{$v : \tau$}
    \LeftLabel{Type Check}
    \BinaryInfC{$P\ \texttt{::}\ \tau=v$}
\end{prooftree}

If a type pattern is not satisfied, the unification fails as any other
failed unification, causing only fizzle. The mechanism by which some
built in constructs check types and cause runtime type errors is a different
one; user annotated types do not cause runtime type errors.

There's actually not all that much else that needs to be written about the
semantics of this type system as type proofs are included in the
evaluation semantics throughout. Especially given that \Trilogy{}'s
type system is so underdeveloped, this section remains vague and
incomplete until a future date.
