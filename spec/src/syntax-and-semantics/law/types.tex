\subsubsection{Types}

% TODO: this section is not included in the document. After brief thought, I have
% decided to omit types from the first draft of Trilogy; leave them out until there
% is a clear way forward on their implementation.
%
% Types will not be added as an afterthought, but they will be though about after!

Being a dynamically typed language, types in \Trilogy{} are a loose concept,
and are checked primarily at runtime. They are checked though, unlike some
dynamically typed languages which have type annotations that are
ignored at runtime, and only loosely examined statically\footnote{Looking at TypeScript and Python here, to name a few.}.

In \Trilogy{}, types are defined using \Law{} via a combination of a pattern
and a query. The pattern describes the shape of the type, and the query
enforces certain rules about how that type's fields are related. \Trilogy{},
however, remains a fully dynamically typed language: there are no ``constructors''
or ``methods'' for these types, only regular functions that construct or act on
values that satisfy these types.

\begin{bnf*}
    \bnfprod{TypeDefinition}{
        \bnfpn{TypeHead}
        \bnfsp
        \bnfts{\kw{is}}
        \bnfsp
        \bnfpn{TypeBody}
    } \\
    \bnfprod{TypeHead}{
        \bnfts{\kw{type}}
        \bnfsp
        \bnfpn{Identifier}
        \bnfsp
        \bnfpn{TypeArguments}
    } \\
    \bnfprod{TypeBody}{
        \bnfpn{Pattern}
        \bnfor
        \bnfpn{Pattern}
        \bnfsp
        \bnfts{\kw{where}}
        \bnfsp
        \bnfpn{Query}
    } \\
    \bnfprod{TypeArguments}{
        \bnfpn{Identifier}
        \bnfsp
        \bnfpn{TypeArguments}
        \bnfor
        \bnfes
    } \\
    \bnfprod{Type}{
        \bnfpn{Identifier}
        \bnfor
        \bnfpn{Type}
        \bnfsp
        \bnfpn{Type}
        \bnfor
        \bnfts{(}
        \bnfsp
        \bnfpn{Type}
        \bnfsp
        \bnfts{)}
    } \\
    \bnfprod{TypePattern}{
        \bnfts{\kw{is}}
        \bnfsp
        \bnfpn{Type}
    }
\end{bnf*}

As these types only work on values with shapes, notably function types and
other values of ``semantic'' types cannot be described directly. This is largely
a side effect of the language being dynamically typed; with little to no type
information held statically during compilation, there is no way to ensure that
the types are accurate when checking a function.

Future versions of \Trilogy{} (or maybe, just future projects) will explore the
opportunity to check these types statically at compilation time.

\begin{prooftree}
    \AxiomC{$\kw{type}\ \tau\ \kw{is}\ P\ \kw{where}\ Q$}
    \AxiomC{$\Gamma\vdash X = P$}
    \UnaryInfC{$\Gamma,\setof{x \given x \defby P}\vdash Q : \top$}
    \LeftLabel{Type Definition}
    \BinaryInfC{$\Gamma\vdash X : \tau$}
\end{prooftree}

\begin{prooftree}
    \AxiomC{$P = v$}
    \AxiomC{$v : \tau$}
    \LeftLabel{Type Check}
    \BinaryInfC{$P\ \kw{is}\ \tau=v$}
\end{prooftree}

If a type pattern is not satisfied, the unification fails as any other
failed unification, causing only fizzle. The mechanism by which some
built in constructs check types and cause runtime type errors is a different
one; user annotated types do not cause runtime type errors.

There's actually not all that much else that needs to be written about the
semantics of this type system as type proofs are included in the
evaluation semantics throughout.
