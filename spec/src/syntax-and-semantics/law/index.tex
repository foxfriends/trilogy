\subsection{Law}

We begin with \Law{}, as the semantics of \Law{} are the most fundamental to
the whole of \Trilogy{}. \Law{} is also the most isolated of the three
sublanguages, and so easiest to describe without much knowledge beyond
the intuitive understanding of the other two.

For those familiar with logical languages such as Prolog, \Law{} may be
familiar, though with different syntax and less logical impurity. \Law{} can
afford to give up the ability to express certain actions (such as side effects
or computations) given the existence of \Prose{} and \Poetry{}.

\begin{bnf*}
    \bnfprod{RuleHead}{
        \bnfts{\texttt{Identifier}}
        \bnfsp
        \bnfts{\texttt{(}}
        \bnfsp
        \bnfpn{PatternList}
        \bnfsp
        \bnfts{\texttt{)}}
    } \\
    \bnfprod{PatternList}{
        \bnfpn{Pattern}
        \bnfsp
        \bnfts{\texttt{,}}
        \bnfsp
        \bnfpn{PatternList}
        \bnfor
        \bnfpn{Pattern}
        \bnfor
        \bnfes
    } \\
    \bnfprod{RuleBody}{\bnfpn{Query}}
\end{bnf*}

\subsubsection{Query}

A query is a way to chain together multiple unifications using different
logical operations.

\begin{bnf*}
    \bnfprod{Query}{
        \bnfpn{Unification}
        \bnfsp
        \bnfts{\kw{and}}
        \bnfsp
        \bnfpn{Query}
        \bnfor
        \bnfpn{Unification}
        \bnfsp
        \bnfts{\kw{or}}
        \bnfsp
        \bnfpn{Query}
        \bnfor
    } \\
    \bnfmore{
        \bnfts{\kw{if}}
        \bnfsp
        \bnfpn{Unification}
        \bnfsp
        \bnfts{\kw{then}}
        \bnfsp
        \bnfpn{Query}
        \bnfor
        \bnfpn{Unification}
        \bnfor
        \bnfts{\texttt{(}}
        \bnfsp
        \bnfpn{Query}
        \bnfsp
        \bnfts{\texttt{)}}
    }
\end{bnf*}

\subsubsection{Unification}

One of the most fundamental parts of \Law{} and indeed of \Trilogy{} itself
is unification. All assignment and binding in \Trilogy{} occurs through
unification of values and patterns.

\begin{bnf*}
    \bnfprod{Unification}{
        \bnfpn{Pattern}
        \bnfsp
        \bnfts{\op{=}}
        \bnfsp
        \bnfpn{Expression}
        \bnfor
    } \\
    \bnfmore{
        \bnfpn{Pattern}
        \bnfsp
        \bnfts{\kw{in}}
        \bnfsp
        \bnfpn{Expression}
        \bnfor
    } \\
    \bnfmore{\bnfts{\kw{true}} \bnfor\bnfts{\kw{false}}\bnfor \bnfpn{Lookup}}
\end{bnf*}

\subsubsection{Lookup}
