\subsection{Law}

The semantics of \Law{} are the most fundamental to the whole of \Trilogy{}, and
\Law{} is also syntactically the least complex. \Law{} is also the most isolated
of the three sublanguages, and so easiest to describe without much knowledge beyond
the intuitive understanding of the other two.


For those familiar with logical languages such as Prolog, \Law{} may be
familiar, though with different syntax and less logical impurity. \Law{} can
afford to give up the ability to express certain actions (such as side effects
or computations) given the existence of \Prose{} and \Poetry{}.

Rule~\ref{bnf:patternlist} is marked for reference later, as it will come up
repeatedly.

\begin{bnf}
    \bnfprod*{RuleHead}{
        \bnfts{\texttt{Identifier}}
        \bnfsp
        \bnfts{\texttt{(}}
        \bnfsp
        \bnfpn{PatternList}
        \bnfsp
        \bnfts{\texttt{)}}
    } \\
    \bnfprod{PatternList}{
        \label{bnf:patternlist}
        \bnfpn{Pattern}
        \bnfsp
        \bnfts{\texttt{,}}
        \bnfsp
        \bnfpn{PatternList}
        \bnfor
        \bnfpn{Pattern}
        \bnfor
        \bnfes
    } \\
    \bnfprod*{RuleBody}{\bnfpn{Query}}
\end{bnf}

Definitions in \Law{} must have a head, and may optionally include a body. While
definitions in other logic languages such as Prolog may be defined with no parameters,
omitting the parentheses entirely, \Law{} does not permit this. All rules must include
a parameter list. The parameter list may be empty, but the usefulness of such rules
is not certain.

\begin{prooftree}
    \AxiomC{$\kw{rule}\ R\ \texttt{(}\ P_{1\hdots n}\ \texttt{)}$}
    \AxiomC{$A_i = P_i$}
    \AxiomC{$\Gamma,\{a\ |\ a := A_i\}:>\Phi$}
    \LeftLabel{Fact}
    \TrinaryInfC{$R(A_{1\hdots n}) \Rightarrow \Phi \in \Gamma$}
\end{prooftree}

\begin{prooftree}
    \def\extraVskip{3.5pt}
    \AxiomC{$\kw{rule}\ R\ \texttt{(}\ P_{1\hdots n}\ \texttt{) <-}\ B$}
    \AxiomC{$A_i = P_i$}
    \BinaryInfC{$\Gamma,\{a\ |\ a := A_i\}:>\Phi$}
    \AxiomC{$\Gamma,\{p\ |\ p := P_i\}:>\Sigma$}
    \AxiomC{$\Sigma\vdash B:\top$}
    \LeftLabel{Rule}
    \TrinaryInfC{$R(A_{1\hdots n}) \Rightarrow \Phi\in\Gamma$}
\end{prooftree}

\subsubsection{Query}

A query is a way to chain together multiple unifications using different
logical operations.

\begin{bnf*}
    \bnfprod{Query}{
        \bnfpn{Unification}
        \bnfsp
        \bnfts{\kw{and}}
        \bnfsp
        \bnfpn{Query}
        \bnfor
        \bnfpn{Unification}
        \bnfsp
        \bnfts{\kw{or}}
        \bnfsp
        \bnfpn{Query}
        \bnfor
    } \\
    \bnfmore{
        \bnfts{\kw{if}}
        \bnfsp
        \bnfpn{Unification}
        \bnfsp
        \bnfts{\kw{then}}
        \bnfsp
        \bnfpn{Query}
        \bnfor
        \bnfpn{Unification}
    }
\end{bnf*}

While logic is orderless, \Law{} is constraint by its being a programming language,
and so unifications in \Law{} do occur in an order; the order in which they are
written. After a unification, the free variables of each pattern are bound and
control continues to the next unification depending on which connective was used.

\begin{prooftree}
    \AxiomC{$\Gamma\vdash P\Rightarrow\Phi$}
    \AxiomC{$\Phi\vdash Q\Rightarrow\Sigma$}
    \LeftLabel{Conjunction}
    \BinaryInfC{$\Gamma\vdash P\ \kw{and}\ Q\Rightarrow\Sigma$}
\end{prooftree}

Conjunction is straight forward, and will be the connective used in most situations.

\begin{center}
    \parbox[t]{0.45\linewidth}{
    \begin{prooftree}
        \AxiomC{$\Gamma\vdash P\Rightarrow\Phi$}
        \LeftLabel{Disjunction\textsuperscript{L}}
        \UnaryInfC{$\Gamma\vdash P\ \kw{or}\ Q\Rightarrow\Phi$}
    \end{prooftree}
    }
    \quad
    \parbox[t]{0.45\linewidth}{
    \begin{prooftree}
        \AxiomC{$\Gamma\vdash Q\Rightarrow\Phi$}
        \LeftLabel{Disjunction\textsuperscript{R}}
        \UnaryInfC{$\Gamma\vdash P\ \kw{or}\ Q\Rightarrow\Phi$}
    \end{prooftree}
}
\end{center}

Disjunction is a bit trickier to get right. Notably, this is not a mutually
exclusive disjunction: both branches will be chosen, given that their unifications
are successful, before moving on to the next branch.

The precedence of \kw{or} is low such that \texttt{A \kw{and} B \kw{or} C \kw{and} D}
is equivalent to \texttt{(A \kw{and} B) \kw{or} (C \kw{and} D)}.

\begin{prooftree}
    \AxiomC{$\Gamma\vdash P\Rightarrow\Phi$}
    \AxiomC{$\Phi\vdash Q\Rightarrow\Sigma$}
    \LeftLabel{Implication\textsuperscript{$\top$}}
    \BinaryInfC{$\Gamma\vdash \kw{if}\ P\ \kw{then}\ Q\Rightarrow\Sigma$}
\end{prooftree}

\begin{prooftree}
    \AxiomC{$\Gamma\vdash P:\bot$}
    \LeftLabel{Implication\textsuperscript{$\top$}}
    \UnaryInfC{$\Gamma\vdash \kw{if}\ P\ \kw{then}\ Q:\top$}
\end{prooftree}

Implication is also a bit tricky to use correctly, particular as it is most likely
to be used in conjunction with \kw{or} in order to set up conditional branches.
With multiple \kw{if}s connected together with \kw{or}, all matching branches
will be chosen, rather than the first matching branch as the \kw{if} statement
of most languages would work.

The precedence of \kw{if} is between that of \kw{and} and \kw{or} such that

\begin{center}
\texttt{\kw{if} A \kw{then} B \kw{or} \kw{if} C \kw{then} D \kw{and} \kw{if} E \kw{then} F \kw{or} G}
\end{center}

is equivalent to

\begin{center}
\texttt{(\kw{if} A \kw{then} B) \kw{or} (\kw{if} C \kw{then} (D \kw{and} \kw{if} E \kw{then} F)) \kw{or} G}
\end{center}

\subsubsection{Unification}

One of the most fundamental parts of \Law{} and indeed of \Trilogy{} itself
is unification. All assignment and binding in \Trilogy{} occurs through
unification of values and patterns.

\begin{bnf*}
    \bnfprod{Unification}{
        \bnfpn{Pattern}
        \bnfsp
        \bnfts{\op{=}}
        \bnfsp
        \bnfpn{Expression}
        \bnfor
    } \\
    \bnfmore{
        \bnfpn{Pattern}
        \bnfsp
        \bnfts{\kw{in}}
        \bnfsp
        \bnfpn{Expression}
        \bnfor
    } \\
    \bnfmore{
        \bnfts{\texttt{(}}
        \bnfsp
        \bnfpn{Query}
        \bnfsp
        \bnfts{\texttt{)}}
        \bnfor
        \bnfpn{Lookup}
        \bnfor
    } \\
    \bnfmore{\bnfts{\kw{true}} \bnfor\bnfts{\kw{false}}\bnfor\kw{not}\bnfsp\bnfpn{Unification}}
\end{bnf*}

While most unifications are performed by ``looking up'' (in the database)
a rule and unifying its definition with the inputs, a few special forms are
included for convenience.

The \op{=} ``operator'' is used to unify an expression with a pattern directly. The
expression is a syntax of \Poetry{}, rather than \Law{}, so its details are covered
later. This does break the purely logical aspect of \Law{}; a direct unification in
this way cannot be ``reversed'' the same way pure \Law{} unifications may be called
in reverse.

\begin{prooftree}
    \AxiomC{$\Gamma\vdash E$}
    \AxiomC{$P = E$}
    \AxiomC{$\Gamma,\{a\ |\ a := P\}:>\Phi$}
    \LeftLabel{Unification}
    \TrinaryInfC{$\Gamma\vdash P\ \op{=}\ E \Rightarrow\Phi$}
\end{prooftree}

The \kw{in} keyword is used to unify an expression with each element of a collection,
binding each element in sequence and continuing the program. Once again, the collection
is specified as an expression in \Poetry{}, so this unification may not be reversed.
The collection may be a Set, Array, List, Record, or Iterator, as described in
Compound Types (\S\ref{sec:compound-types}).

\begin{prooftree}
    \AxiomC{$\Gamma\vdash E$}
    \AxiomC{$e\in E$}
    \AxiomC{$P = e$}
    \AxiomC{$\{a\ |\ a := P\}:>\Phi$}
    \LeftLabel{Iteration}
    \QuaternaryInfC{$\Gamma\vdash P\ \kw{in}\ E\Rightarrow\Phi$}
\end{prooftree}

Parentheses may be used for grouping, as one might expect. The grouped query works
as if it were a single unification, all bindings made within the group
persisting afterwards.

\begin{prooftree}
    \AxiomC{$\Gamma\vdash P\Rightarrow \Phi$}
    \LeftLabel{Unwrap}
    \UnaryInfC{$\Gamma\vdash \texttt{(}\ P\ \texttt{)} \Rightarrow \Phi$}
\end{prooftree}

The two keywords \kw{true} and \kw{false} may be used as a unification on
their own as well. \kw{true} is treated as a unification that succeeds once,
while \kw{false} is treated as a unification that never succeeds. Though
the symbol used is the same, \kw{true} and \kw{false} are not to be interpreted
as Boolean values in this situation.

\begin{figure}[H]
    \centering
    \parbox[t]{0.3\linewidth}{
        \begin{prooftree}
            \AxiomC{}
            \LeftLabel{$\top$}
            \UnaryInfC{$\Gamma \vdash \kw{true} : \top$}
        \end{prooftree}
    }
    \parbox[t]{0.3\linewidth}{
        \begin{prooftree}
            \AxiomC{}
            \LeftLabel{$\bot$}
            \UnaryInfC{$\Gamma \vdash \kw{false} : \bot$}
        \end{prooftree}
    }
\end{figure}

The keyword \kw{not} may be used to negate a unification. A negated unification
can produces no bindings, but may cause the execution to continue or fizzle
depending on whether its unification succeeds at least once.

\begin{figure}[H]
    \centering
    \parbox[t]{0.45\linewidth}{
        \begin{prooftree}
            \AxiomC{$\Gamma\vdash U : \top$}
            \LeftLabel{Negation}
            \UnaryInfC{$\Gamma\vdash \kw{not}\ U : \bot$}
        \end{prooftree}
    }
    \parbox[t]{0.45\linewidth}{
        \begin{prooftree}
            \AxiomC{$\Gamma\vdash U : \bot$}
            \LeftLabel{Negation}
            \UnaryInfC{$\Gamma\vdash \kw{not}\ U : \top$}
        \end{prooftree}
    }
\end{figure}

\subsubsection{Lookup}

The lookup case is the most complex, and most powerful of the unifications,
where a rule is looked up in the current scope and its definitions are applied
by unifying each parameter with the arguments provided.

For all its flexibility, the syntax is very simple. Recall Rule~\ref{bnf:patternlist}
from earlier.

\begin{bnf*}
    \bnfprod{Path}{
        \bnfts{\texttt{Identifier}}
        \bnfsp
        \bnfts{\op{.}}
        \bnfsp
        \bnfpn{Path}
        \bnfor
        \bnfts{\texttt{Identifier}}
    } \\
    \bnfprod{Lookup}{
        \bnfpn{Path}
        \bnfsp
        \bnfts{\texttt{(}}
        \bnfsp
        \bnfpn{PatternList}
        \bnfsp
        \bnfts{\texttt{)}}
    }
\end{bnf*}

The availability of a rule depends on the module scope in which the rule was defined.
Imported names, locally defined rules, and paths through available modules to rules
may be used to look up the rule.

% TODO: This is not right yet?

\begin{prooftree}
    \AxiomC{$R(P_{1\hdots n})\in\Gamma$}
    \AxiomC{$\Gamma \vdash P_i = A_i$}
    \AxiomC{$\Sigma,A_i\vdash R(A_{1\hdots n})\Rightarrow\Phi$}
    \LeftLabel{Lookup}
    \TrinaryInfC{$\Gamma\vdash R\ \texttt{(}\ A_i\ \texttt{)} \Rightarrow\Phi$}
\end{prooftree}
