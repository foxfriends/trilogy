\subsubsection{Query}

A query is a way to chain together multiple unifications using different
logical operations.

\begin{bnf*}
    \bnfprod{Query}{
        \bnfpn{Query}
        \bnfsp
        \bnfts{\kw{and}}
        \bnfsp
        \bnfpn{Query}
        \bnfor
        \bnfpn{Query}
        \bnfsp
        \bnfts{\kw{or}}
        \bnfsp
        \bnfpn{Query}
        \bnfor
    } \\
    \bnfmore{
        \bnfts{\kw{if}}
        \bnfsp
        \bnfpn{Query}
        \bnfsp
        \bnfts{\kw{then}}
        \bnfsp
        \bnfpn{Query}
        \bnfor
        \bnfpn{Query}
        \bnfsp
        \bnfts{\kw{else}}
        \bnfsp
        \bnfpn{Query}
        \bnfor
    } \\
    \bnfmore{
        \bnfpn{Unification}
    }
\end{bnf*}

While logic is orderless, \Law{} is constrained by its being a programming language,
and so unifications in \Law{} do occur in an order; the order in which they are
written. After a unification, the free variables of each pattern are bound and
control continues to the next unification depending on which connective was used.

Similarly, when a rule has multiple definitions, they will be evaluated fully in
source code order, as if they were connected by \kw{or}.

\begin{prooftree}
    \AxiomC{$\Gamma\vdash P\Rightarrow\Phi$}
    \AxiomC{$\Phi\vdash Q\Rightarrow\Sigma$}
    \LeftLabel{Conjunction}
    \BinaryInfC{$\Gamma\vdash P\ \kw{and}\ Q\Rightarrow\Sigma$}
\end{prooftree}

Conjunction is straight forward, acting much like sequencing in other languages,
and will likely be the connective used in most situations.

\begin{center}
    \parbox[t]{0.45\linewidth}{
    \begin{prooftree}
        \AxiomC{$\Gamma\vdash P\Rightarrow\Phi$}
        \LeftLabel{Disjunction\textsuperscript{L}}
        \UnaryInfC{$\Gamma\vdash P\ \kw{or}\ Q\Rightarrow\Phi$}
    \end{prooftree}
    }
    \quad
    \parbox[t]{0.45\linewidth}{
    \begin{prooftree}
        \AxiomC{$\Gamma\vdash Q\Rightarrow\Phi$}
        \LeftLabel{Disjunction\textsuperscript{R}}
        \UnaryInfC{$\Gamma\vdash P\ \kw{or}\ Q\Rightarrow\Phi$}
    \end{prooftree}
}
\end{center}

Disjunction is a bit trickier to get right. Notably, this is not a mutually
exclusive disjunction: both branches will be chosen, given that their unifications
are successful, before moving on.

The precedence of \kw{or} is low such that \tri{A and B or C and D}
is equivalent to \tri{(A and B) or (C and D)}.

\begin{prooftree}
    \AxiomC{$\Gamma\vdash P\Rightarrow\Phi$}
    \LeftLabel{Alternative\textsuperscript{L}}
    \UnaryInfC{$\Gamma\vdash P\ \kw{else}\ Q\Rightarrow\Phi$}
\end{prooftree}

\begin{prooftree}
    \AxiomC{$\Gamma\vdash P : \bot$}
    \AxiomC{$\Gamma\vdash Q\Rightarrow\Phi$}
    \LeftLabel{Alternative\textsuperscript{R}}
    \BinaryInfC{$\Gamma\vdash P\ \kw{else}\ Q\Rightarrow\Phi$}
\end{prooftree}

Meanwhile the \kw{else} keyword acts as a means of providing a true alternative.
In contrast to \kw{or}, only one sub-query of an \kw{else} will be chosen, the
first if it has any results, otherwise the second.

The precedence of \kw{else} is the same as \kw{or}, where the two are each left
associative.

\begin{prooftree}
    \AxiomC{$\Gamma\vdash P\Rightarrow\Phi$}
    \AxiomC{$\Phi\vdash Q\Rightarrow\Sigma$}
    \LeftLabel{Implication\textsuperscript{$\top$}}
    \BinaryInfC{$\Gamma\vdash \kw{if}\ P\ \kw{then}\ Q\Rightarrow\Sigma$}
\end{prooftree}

\begin{prooftree}
    \AxiomC{$\Gamma\vdash P:\bot$}
    \LeftLabel{Implication\textsuperscript{$\top$}}
    \UnaryInfC{$\Gamma\vdash \kw{if}\ P\ \kw{then}\ Q:\top$}
\end{prooftree}

Implication is also a bit tricky to use correctly, particular as it is most likely
to be used in conjunction with \kw{or} or \kw{else} in order to set up conditional
branches.

With multiple \kw{if}s connected together with \kw{or}, all matching branches will
be chosen, rather than the first matching branch, while with \kw{else} it works
somewhat similarly to how you might expect in the ``regular'' way.

The precedence of \kw{if} is between that of \kw{and} and \kw{or} such that

\begin{center}
\tri{if A then B or if C then D and if E then F or G}
\end{center}

is equivalent to

\begin{center}
\tri{(if A then B) or (if C then (D and if E then F)) or G}
\end{center}
