\subsubsection{Bindings}

The final and most fundamental part of any pattern is the binding. Binding
patterns are what defines all variables in \Trilogy{}, and allow values to
be passed from one place to another.

Bindings by default are immutable---once defined, they cannot be changed.
The mutable binding, denoted with \kw{mut} is used to explicitly mark a
binding as reassignable. Of course, the only language in which assignment
is possible is \Prose{}, so it's not all that likely to use a mutable binding
in any other situation.

\begin{bnf*}
    \bnfprod{Binding}{
        \bnfpn{Identifier}
        \bnfor
        \bnfts{\kw{mut}}
        \bnfsp
        \bnfpn{Identifier}
    }
\end{bnf*}

A binding will unify with any value, but if the same identifier is used
multiple times in a single pattern, a few restrictions must be followed.

First, an identifier must be consistently marked as mutable or immutable
throughout the pattern. If there is any inconsistency, it is considered
a parse error.

Second, once an identifier has been bound to a value, all other references
to that identifier will simply be matched as if pinned. While for structural
types this is not much of an issue, it is important to consider the implications
of this rule on reference types. Since reference types are mutable containers
even when declared via an immutable binding, a mutation to the value through
this binding will only affect one of the two bindings. This may be more easily
visualized through example:

\begin{lstlisting}[language=Trilogy]
    let a = [1, 2]
    let b = [1, 2]
    let x : x = a : b
    x push= 3
    print!($"${a} | ${b}")
\end{lstlisting} %

In the statement \tri{let x : x = a : b}, first \texttt{x} will be bound with
\texttt{a}, then unified structurally with be \texttt{b} as if pinned; equivalently
\tri{let x : ^x = a : b}. After \tri{x push= 3}, the final \fn{print} will
result in \texttt{[1, 2, 3] | [1, 2]}.

To help prevent such ambiguous situations, it is recommended that each
identifier be used unbound only once and be explicitly pinned in all
other locations, when possible. This also helps reduce the opportunity
to have inconsistent mutability. Unfortunately, it may not always be
possible if your data is of less certain structure, or when dealing with
non-deterministic bindings as is often the case in \Law{}.

\begin{prooftree}
    \def\extraVskip{3.5pt}
    \AxiomC{}
    \LeftLabel{Binding Unification}
    \UnaryInfC{$I = v$}
    \def\extraVskip{3.5pt}
    \LeftLabel{Binding Definition}
    \UnaryInfC{$I := v$}
\end{prooftree}
