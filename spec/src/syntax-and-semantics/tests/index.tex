\subsection{Tests}
\label{sec:tests}

There is one last form of definition available in \Trilogy{}, the \kw{test}
definition, used for defining tests (as in unit tests). Though tests are
always parsed (and must parse successfully) they are ignored when \Trilogy{}
is not run in test mode.

\begin{bnf*}
    \bnfprod{Test}{
        \bnfts{\kw{test}}
        \bnfsp
        \bnfpn{String}
        \bnfsp
        \bnfpn{Block}
    }
\end{bnf*}

A \kw{test} declaration includes a string, the test's name, and a block, the
test's body. The name is meaningless to the program, but is printed in the
test runner's output for you to identify. Tests with duplicate names are permitted,
but recommended against because it leads to confusion.

The body of the test is a block which is run much like a procedure, only it
does not have parameters, and may not be referenced by other code. If execution
runs off the end of the block successfully, the test is considered successful.
If execution fails due to fizzling or a runtime error, the test is considered
a failure.

Since tests have no semantic meaning to a program, no specification of semantics
is required.
