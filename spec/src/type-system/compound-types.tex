\subsection{Compound Types}
\label{sec:compound-types}

Compound types are constructions that can be made up of the primitive
types. In \Trilogy{}, all compound types are heterogeneous - they may
contain many values of different types.

Similarly to primitives, compound types have string and source
representations. The string representation is computed by recursively
converting component values into their string representations. The
source representation is once again the same as you would write it
in source code.

\subsubsection{Struct}

A Struct wraps a single other value with a label. The label looks a
lot like an Atom, and the value can be any single value (including
a compound value). The Struct type is a structural and non-reference
type.

Each distinct label for a Struct actually acts more like its own ``type'' from
a type-system perspective, since labels cannot be referenced dynamically, only
in literal patterns. In a pattern matching context, you could think of them as
sum-type variants, but without defining them as a members of a specific list
of variants.

Given that, in documentation we rarely refer to the Struct type as ``Struct'',
but instead using the exact label that is expected, as in \texttt{'label(T)}

The string representation of a Struct is the label (without the single quote)
followed by the string representation of the value in parentheses.

\begin{table}[H]
    \centering
    \begin{tabular}{ll}
        \hline
        \textbf{Source} & \textbf{String} \\
        \hline
        \texttt{'hello("world")} & \texttt{hello(world)} \\
        \texttt{'number(3)} & \texttt{number(3)} \\
        \texttt{'compound([1, 2, 3])} & \texttt{compound([1, 2, 3])} \\
        \hline
    \end{tabular}
\end{table}

\subsubsection{Tuple}

A Tuple attaches a pair of values to make a single compound value, and is written
with the colon or ``cons'' operator (\op{:}). The Tuple type is a structural and
reference type. In type annotation, the Tuple type is written as in \texttt{T : U}.

While a Tuple can only contain two values, one of those values may be another
Tuple to emulate larger collections. When chained together, the cons operator
associates right.

While finite Tuples are useful for representing structured data, recursive Tuples
are useful as a representation of (linked) lists. In this interpretation, we use
\kw{unit} to indicate the empty List, and any value consed with a List on the
right side to be a List. Lists may be consumed recursively. We use the name
List (and type annotation \ty{List T}) to refer to constructions of Tuples in
this way, but List is not truly a distinct ``type''; it's simply a special case
of the Tuple type.

The string representation of a Tuple consists of the string representations of its
two parts, separated by a colon much like its source representation.

\begin{table}[H]
    \centering
    \begin{tabular}{ll}
        \hline
        \textbf{Source} & \textbf{String} \\
        \hline
        \texttt{"hello" : "world"} & \texttt{hello : world} \\
        \texttt{3 : true} & \texttt{3 : true} \\
        \texttt{1 : 2 : 3 : unit} & \texttt{1 : 2 : 3 : unit} \\
        \hline
    \end{tabular}
\end{table}

\subsubsection{Array}

An Array is a contiguous, ordered sequence of values, backed by an array of
the implementation language. In terms of function and capabilities, Arrays
and Lists are pretty much equivalent, but the performance characteristics
of the two are quite distinct, particularly when the list will be resizing
frequently, versus having frequent random access. The Array type is a structural
and reference type. In type annotation, the Array type is written \ty{Array T}.

Array literals are written with square brackets, and items separated by
comma. In Array construction, the spread operator (\op{..}) can be used
to flatten one Array into the new one. In Array decconstruction, the spread
operator may be used \emph{at most once} to refer to the rest of the elements
not already matched explicitly by the pattern.

The string representation of an Array is simply the string representation of
each of its elements, separated by commas within square brackets.

\begin{table}[H]
    \centering
    \begin{tabular}{ll}
        \hline
        \textbf{Source} & \textbf{String} \\
        \hline
        \texttt{[1, 2, 3]} & \texttt{[1, 2, 3]} \\
        \texttt{["hello", 3, "world"]} & \texttt{[hello, 3, world]} \\
        \texttt{[1, 2, ..[3, 4], 5, 6]} & \texttt{[1, 2, 3, 4, 5, 6]} \\
        \hline
    \end{tabular}
\end{table}

\subsubsection{Set}

A Set is an unordered collection of values containing no duplicates. Though
values of any type may be contained in a Set, duplicates are compared as if
by the referential equality operator (\op{===}). The Set type is a structural
and reference type. In type annotation, the Set type is written \ty{Set T}

Set literals are written with brackets and pipes, with items separated by comma.
In pattern matching, the spread operator may be used to refer to the elements
not already matched explicitly by the pattern. In Set construction, the
spread operator is not supported; prefer to use the Set operators instead.

The string representation of a Set is the string representation of
each of its elements, separated by commas within curly braces (reflecting the
mathematical representation, rather than the \Trilogy{} representation).

\begin{table}[H]
    \centering
    \begin{tabular}{ll}
        \hline
        \textbf{Source} & \textbf{String} \\
        \hline
        \texttt{[|1, 2, 3|]} & \texttt{\{1, 2, 3\}} \\
        \texttt{[|"hello", "hello", "world"|]} & \texttt{\{hello, world\}} \\
        \hline
    \end{tabular}
\end{table}

\subsubsection{Record}

A Record is an unordered collection of key-value pairs, often also called
a dictionary, map, or object. Record keys are unique. The Record type is a
structural and reference type.

In type annotation, the record type is written \ty{Record K V} where keys of
type \ty{K} refer to values of type \ty{V}.

The main operation of a Record is the indexing operator. The keys of a
Record may be any value; like with Sets, the keys are compared as if by
the referential equality operator (\op{===}).

A Record literal is written as comma separated \texttt{key: value} pairs
within curly braces and pipes. In Record construction, the spread operator (\op{..})
may be used to combine two or more Records. In pattern matching, the spread
operator may be used once to refer to the rest of the entries not already
matched explicitly by the pattern.

Though Records are unordered, keys specified ``later'' in the source file
(whether literally or via a spread) take precedence over keys specified
earlier.

The string representation of a Record is its stringified \texttt{key => value}
pairs in curly braces. The pipes are omitted in the string representation.

\begin{table}[H]
    \centering
    \begin{tabular}{ll}
        \hline
        \textbf{Source} & \textbf{String} \\
        \hline
        \texttt{\{|1 => "one", 2 => "two"|\}} & \texttt{\{1 => one, 2 => two\}} \\
        \texttt{\{|"a" => 1, "a" => 2, "b" => 3|\}} & \texttt{\{a => 2, b => 3\}} \\
        \hline
    \end{tabular}
\end{table}
