\subsection{Semantic Types}

The remaining types are called ``semantic'' types, referring to how values of these
types are what make up the semantic meaning of the program.

All semantic types are non-structural and reference types. Semantic types do
have string and source representations, but in versions prior to the 1.0
release of \Trilogy{}, converting to source representation of these types will
not be officially defined or supported.

\subsubsection{Procedures}

The first semantic type is the Procedure type, which is the type of both
defined procedures and procedure closures.

In type annotation, the Procedure type is written as in \texttt{!(A, B, C) -> T}
where, in this example, \texttt{A}, \texttt{B}, and \texttt{C} are the types of
the parameters and \texttt{T} the type of the return value. The number of parameters
may change depending on the actual signature of the procedure in question.

Values of the Procedure type may be called using the procedure call syntax \texttt{!()},
passing arguments to all of the defined parameters within the parentheses.

Values of Procedure type are created at the top level by \kw{proc} definitions, or as
closures using the keyword \kw{do}.

The string representation of a defined (named) procedure is its name; the name does
not include the \texttt{!} symbol. Meanwhile, the string representation of an anonymous
procedure closure is defined to be the literal string \texttt{<do>}.

\subsubsection{Functions}

The Function type is a semantic type, which is the type of defined functions and
function closures.

In type annotation, the Function type is written as \texttt{T -> U}. Notably, the
function type only has space for a single parameter type: because \Poetry{} supports
currying, all functions actually only accept a single parameter and ``multi-parameter''
functions are syntax sugar for single parameter functions returning a closure over
the argument value and accepting the next parameter.

A value of the Function type may be applied by providing its argument directly after,
separated only by a space.

The string representation of a defined (named) function is its name. Meanwhile, the
string representation of an anonymous function closure is defined to be the literal
string \texttt{<fn>}.

\subsubsection{Continuations}

Very similar to the Function type is the Continuation type. Conceptually, a continuation
may actually just \emph{be} a function, and indeed when written as a type annotation, it
is written the same as a function (e.g. \texttt{T -> U}).

The difference is mostly one of implementation, as a continuation is not constructed in
the same way as a function, but can only be acquired through an effect handler in the
form of the \kw{resume} keyword which, though technically a keyword, works much like a
function in that it can be captured by closures or passed to another function as a
parameter.

Continuations are applied much like functions, and evaluate to some value just as a
function application does.

The string representation of a continuation is the literal string \texttt{<resume>}.

\subsubsection{Modules and Module Functions}

Modules and Module Functions in \Trilogy{} are first class objects which behave much
like values of Record or Function type. While they may be passed as arguments
to other procedures and functions, Module Functions are restricted to accepting other
Modules (and Module Functions) as parameters, ensuring that all module lookups can be
performed statically.

At this time, a Module allows you to implement the closest thing to what you might
consider an abstract data type in another language. A Module Function could be considered
as a generic data type, where the expected signatures of its parameters are like interfaces that
other modules may implement. None of that requires particular handling in the type system
however, as modules are restricted to being applied statically.

In type annotation, the module type is written as \texttt{Module}, and a module function
as \texttt{Module -> Module}. This is not exactly the most strongly typed of descriptors,
but suffices for the purposes of \Trilogy{} as a dynamically typed language. While it may
be useful at times to be able to describe the signature of these modules, in practice it
is unnecessary as the interpreter will be able to determine statically whether a module
call is valid or not.

That said, modules can still be converted to string representation, which is the
of the module followed by the names of its parameter modules, if any, in parentheses,
such as \texttt{ModuleA(ModuleB, ModuleC(ModuleD))}.
