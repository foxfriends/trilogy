\section{Type System}

\Trilogy{} is (currently) a dynamically typed language; all types are determined
and checked at runtime. Recall that ``dynamically typed'' is not the absence of
a type system, but the presence of a type system so complex that it cannot be
fully statically analyzed.

Despite the lack of static analysis, we attempt to fully specify the type system
of \Trilogy{} in hopes that someday it can be statically analyzed (an attempt at
(potentially optional) static (higher-kinded? dependent?) typing is on the long
term roadmap).

Worth mentioning is that the type system of \Trilogy{} spans all sub-languages
consistently. The syntax and behaviour of all data types are the same, no matter
where they are being used. The types are summarized below, with exact specifications
of the operators come later, in Syntax and Semantics (\S\ref{sec:syntax})

\subsection{Built-In Types}

There are 5 primitive data types in \Trilogy{}:

\begin{itemize}
    \item Never
    \item Unit
    \item Boolean
    \item Atom
    \item Character
    \item String
    \item Number
\end{itemize}

\noindent
There are also 5 compound types, which may be constructed out of primitives
and other types:

\begin{itemize}
    \item Labelled
    \item Tuple
    \item Array
    \item Set
    \item Record
    \item Iterator
\end{itemize}

\noindent
Defined items (and their locally defined counterparts) are also values which
have types:

\begin{itemize}
    \item Procedure
    \item Procedure Closure
    \item Function
    \item Function Closure
    \item Rule
    \item Ruleset
    \item Continuation
    \item Module
    \item Module Function
\end{itemize}

\noindent
There are no explicit user defined types, only combinations of the built in
types, so this is a complete list of all types to be seen in a \Trilogy{}
program.

By no means, however, is this the end of the story for \Trilogy{}'s type
system. Custom data types, as well as further built-in primitive types are,
are on the list of things to implement once the language is in a functioning
state.

\subsubsection{Structural Types}

Structural types are types which have a ``physical'' structure, which may be
interpreted intuitively. Structural types may be destructured using pattern
matching, and are compared structurally (deeply) using the structural equality
operator (\kw{is}).

\subsubsection{Reference Types}

Reference types are those which are stored and passed by \emph{reference}. When
a value of a reference type is passed as an argument or a assigned to a variable, both
variables refer to the same \emph{instance} of that value. When dealing with
mutation of an instance with multiple references, modifications made to any
reference to that instances are reflected in all other references. This should be
intuitive to users of Javascript, which follows similar reference semantics.

When comparing reference types, there are two options. The referential equality
operator (\op{==)} will compare the values by reference first; that is, if
both values are of a reference type, they are considered equal if and only
if they are references to the same instance. If either value is not a reference
type, they will be compared structurally. In contrast, the structural equality
operator (\kw{is}) will compare structurally first: if both values are structural
types, they will be compared structurally, otherwise they will be compared by
reference.

Some types are both structural and reference types.

\subsection{Primitive Types}

There's not much interesting to be said about primitives. These types are
backed directly by counterparts in the interpreter's implementation language,
and so cannot be further broken down in exchange for performance closer to
that of native code on their basic operations.

All primitive types are structural types, no primitive types are reference
types.

\subsubsection{Never}

Never is a type with no values. While not particularly useful as a value,
Never is the type of a \kw{return} statement, as well as the type of
a \kw{yield} that is not \kw{resume}d, and of actions which terminate an
execution, such as the \kw{end} statement, among other things.

In type annotation or documentation, the Never type is written \ty{!}.
In the formal semantics throughout this document, Never is typically not
mentioned, but is similar to the usage of $\bot$ to represent the code
that never completes execution.

\subsubsection{Unit}

Unit is a type with a single-value, the literal \kw{unit}. Unit on its own
has no particular meaning, but it can be used to signal the lack of value.
In type annotation, the Unit type is written \ty{Unit}.

This is the type returned by functions or procedures which do not return
any other value. Within the standard library, Unit is used as the sentinel
value for the end of a list.

The string representation of Unit is \texttt{unit}.

\subsubsection{Boolean}

Boolean is a type with two values, the literals \kw{true} and \kw{false},
which act much like they do in every language I can think of.
In type annotation, the Boolean type is written \ty{Bool}.

Boolean values are the result of many comparisons (such as \op{lt} and
\op{gt}), can be manipulated by the Boolean operators (\op{\&\&}, \op{||},
and \op{!}) and are used by control flow constructs (\kw{if} and
\kw{while}).

The string representation of the Boolean values
\kw{true} and \kw{false} are \texttt{true} and \texttt{false},
respectively.

\subsubsection{Atom}

Atom is a type which typically exist in finite quantities. Any individual
Atom has no value on its own, but its literal form may hint to a reader at
what its value is meant to signal. The only real operation available for
Atom values is equality, which is performed in $O(1)$ time.

An Atom literal is an identifier prefixed with a single quote (\texttt{'}).
Within the standard library, some Atoms yielded as effects to signal
exceptional situations:

In type annotation, the Atom type is written \ty{Atom} when any Atom
is expected, or using the exact value of the atom (e.g. \ty{'NAN}) if
only that exact value is expected.

\begin{itemize}
    \item \val{'NAN}: yielded by Number parsing functions when the value
    cannot be parsed as a Number.
    \item \val{'EOF}: yielded by input procedures when the end of input
    has been reached.
\end{itemize}

The string representation of an Atom is its symbol without the single quote
(i.e. \val{'NAN} as a string is \texttt{NAN}).

\subsubsection{Character}

The Character type is used to represent individual Unicode code points,
a Character value representing exactly one.
In type annotation, the Character type is written \ty{Char}.

Internally, Characters are represented using their UTF-8 encodings. Characters
can safely be converted to Numbers, as every Character's UTF-8 encoding is a valid
Number. Meanwhile, not every Number may be safely converted to a Character,
as not every binary sequence represents a valid UTF-8 codepoint.

Character literals are specified as a single UTF-8 code point between
single quotes. Alternatively, an escape sequence (\S\ref{sec:escapesequences})
may be used to represent an ASCII or Unicode character code by its hexadecimal
representation, or one of the specially supported common escape sequences.

The string representation of a Character is a String containing that one
Character. Some examples:

\begin{table}[H]
    \centering
    \begin{tabular}{ll}
        \hline
        \textbf{Source} & \textbf{String} \\
        \hline
        \val{'c'} & \texttt{c} \\
        \val{'\textbackslash t'} & \emph{Line break}\\
        \val{'\textbackslash x65'} & \texttt{e} \\
        \val{'\textbackslash u\{03BB\}} & \texttt{$\lambda$} \\
        \hline
    \end{tabular}
\end{table}

\subsubsection{String}

The String type is a sequence of Characters. This sounds somewhat like
a compound type, but since a String can only contain Characters and has
its own literal format, it is practically more like a primitive. In type
annotation, the String type is written \ty{String}.

Strings may be manipulated using the push operator (\op{+}) to attach a
Character to a String, and the glue operator (\op{<>}) to attach two
Strings together. The member access operator (\op{.}) may also be used to
access a Character at a particular index.

String literals are represented using double quotes, and may contain
any characters or escape sequences. Untagged template literals produce
Strings by inserting the string representation of the interpolated values
(tagged literals don't actually produce Strings).

The string representation of a String is the string itself.

\subsubsection{Number}

The Number type is used to represent numbers, on which mathematical
operations can be performed. In \Trilogy{}, numbers go beyond integers
and floating point numbers to also include any rational or complex
numbers as well. In type annotation, the Number type is written \ty{Number}.
Additonally, the names \ty{Int}, \ty{Float}, and \ty{Rational} refer to
different subsets of the full range of the \ty{Number} type.

These numbers support all the normal operations (addition \op{+},
subtraction \op{-}, multiplication \op{*}, division \op{/},
modulus \op{\%}, integer division \op{//}, power \op{**}), but
division is perfect (\texttt{1 / 3 == 1/3}), and operations that
require complex numbers (\texttt{(-1) ** 1/2 == 0i1}) will succeed.

The string representation of a Number depends on the value of the number,
and is most conveniently described by example:

\begin{table}[H]
    \centering
    \begin{tabular}{rr}
        \hline
        \textbf{Source} & \textbf{String} \\
        \hline
        \val{0} & \texttt{0} \\
        \val{123} & \texttt{123} \\
        \val{-123} & \texttt{-123} \\
        \val{123.456} & \texttt{123.456} \\
        \val{-123.456} & \texttt{-123.456} \\
        \val{1/3} & \texttt{1/3} \\
        \val{-1/3} & \texttt{-1/3} \\
        \val{123i456} & \texttt{123 + 456i} \\
        \val{123 - 0i456} & \texttt{123 - 456i} \\
        \val{123.5i5.4} & \texttt{123.5 + 5.4i} \\
        \hline
    \end{tabular}
\end{table}

\noindent
There are some cases missing, but hopefully they are easy to extrapolate.

\subsubsection{Bits}

The Bits type is used to represent binary numbers as sequences of bits.
While many languages use their number type for this purpose, \Trilogy{}
differentiates Bits from Numbers due to the fact that Numbers are represented
in arbitrary precision and so bitwise operations (and \op{\&}, or \op{|},
xor \op{\textasciicircum}, not \op{\textasciitilde}, shift \op{\textasciitilde>}
and \op{<\textasciitilde}) do not work all that well on them.
In type annotation, the Bits type is written \ty{Bits}.

The string representation of Bits is the literal value of those bits in full
represented using 1s and 0s, as in \texttt{010101}. The source representation
is as the bits literal when written with prefix \texttt{0bb} as in as in
\texttt{0bb010101}


\subsection{Compound Types}

Compound types are constructions that can be made up of the primitive
types. In \Trilogy{}, all compound types are heterogeneous - they may
contain many values of different types.

Similarly to primitives, compound types have string and source
representations. The string representation is computed by recursively
converting component values into their string representations. The
source representation is once again the same as you would write it
in source code.

\subsubsection{Struct}

A Struct wraps a single other value with a label. The label looks a
lot like an Atom, and the value can be any single value (including
a composite value). The Struct type is a structural and non-reference
type.

Each distinct label for a Struct actually acts more like its own ``type'' from
a type-system perspective, since labels cannot be referenced dynamically, only
in literal patterns. In a pattern matching context, you could think of them as
sum-type variants, but without defining them as a members of a specific list
of variants.

Given that, in documentation we rarely refer to the Struct type as ``Struct'',
but instead using the exact label that is expected, as in \texttt{'label(T)}

The string representation of a Struct is the label (without the single quote)
followed by the string representation of the value in parentheses.

\begin{table}[H]
    \centering
    \begin{tabular}{ll}
        \hline
        \textbf{Source} & \textbf{String} \\
        \hline
        \texttt{'hello("world")} & \texttt{hello(world)} \\
        \texttt{'number(3)} & \texttt{number(3)} \\
        \texttt{'compound([1, 2, 3])} & \texttt{compound([1, 2, 3])} \\
        \hline
    \end{tabular}
\end{table}

\subsubsection{Tuple}

A Tuple attaches a pair of values to make a single compound value, and is written
with the colon or ``cons'' operator (\op{:}). The Tuple type is a structural and
reference type. In type annotation, the Tuple type is written as in \texttt{T : U}.

While a Tuple can only contain two values, one of those values may be another
Tuple to emulate larger collections. When chained together, the cons operator
associates right.

While finite Tuples are useful for representing structured data, recursive Tuples
are useful as a representation of (linked) lists. In this interpretation, we use
\kw{unit} to indicate the empty List, and any value consed with a List on the
right side to be a List. Lists may be consumed recursively. We use the name
List (and type annotation \ty{List T}) to refer to constructions of Tuples in
this way, but List is not truly a distinct ``type''; it's simply a special case
of the Tuple type.

The string representation of a Tuple consists of the string representations of its
two parts, separated by a colon much like its source representation.

\begin{table}[H]
    \centering
    \begin{tabular}{ll}
        \hline
        \textbf{Source} & \textbf{String} \\
        \hline
        \texttt{"hello" : "world"} & \texttt{hello : world} \\
        \texttt{3 : true} & \texttt{3 : true} \\
        \texttt{1 : 2 : 3 : unit} & \texttt{1 : 2 : 3 : unit} \\
        \hline
    \end{tabular}
\end{table}

\subsubsection{Array}

An Array is a contiguous, ordered sequence of values, backed by an array of
the implementation language. In terms of function and capabilities, Arrays
and Lists are pretty much equivalent, but the performance characteristics
of the two are quite distinct, particularly when the list will be resizing
frequently, versus having frequent random access. The Array type is a structural
and reference type. In type annotation, the Array type is written \ty{Array T}.

Array literals are written with square brackets, and items separated by
comma. In Array construction, the spread operator (\op{..}) can be used
to flatten one Array into the new one. In Array decconstruction, the spread
operator may be used \emph{at most once} to refer to the rest of the elements
not already matched explicitly by the pattern.

Typical array operations include direct access using the indexing operator,
push (\texttt{+}), and concatenation or glue (\texttt{<>}).

The string representation of an Array is simply the string representation of
each of its elements, separated by commas within square brackets.

\begin{table}[H]
    \centering
    \begin{tabular}{ll}
        \hline
        \textbf{Source} & \textbf{String} \\
        \hline
        \texttt{[1, 2, 3]} & \texttt{[1, 2, 3]} \\
        \texttt{["hello", 3, "world"]} & \texttt{[hello, 3, world]} \\
        \texttt{[1, 2, ..[3, 4], 5, 6]} & \texttt{[1, 2, 3, 4, 5, 6]} \\
        \hline
    \end{tabular}
\end{table}

\subsubsection{Set}

A Set is an unordered collection of values containing no duplicates. Though
values of any type may be contained in a Set, duplicates are compared as if
by the structural equality operator (\op{eq}). The Set type is a structural
and reference type. In type annotation, the Set type is written \ty{Set T}

Sets come with useful operations union (\op{|}), intersection (\op{\&}),
add (\op{+}), remove (\op{-}), difference (\op{/}), and symmetric
difference (\op{\textasciicircum}).

Set literals are written with curly braces and items separated by comma.
In pattern matching, the spread operator may be used to refer to the elements
not already matched explicitly by the pattern. In Set construction, the
spread operator is not supported; prefer to use the Set operators instead.

The string representation of a Set is the string representation of
each of its elements, separated by commas within curly braces.

\begin{table}[H]
    \centering
    \begin{tabular}{ll}
        \hline
        \textbf{Source} & \textbf{String} \\
        \hline
        \texttt{\{1, 2, 3\}} & \texttt{\{1, 2, 3\}} \\
        \texttt{\{"hello", "hello", "world"\}} & \texttt{\{hello, world\}} \\
        \hline
    \end{tabular}
\end{table}

\subsubsection{Record}

A Record is an unordered collection of key-value pairs, often also called
a dictionary, map, or object. Record keys are unique. The Record type is a
structural and reference type.

In type annotation, the record type is written \ty{Record K V} where keys of
type \ty{K} refer to values of type \ty{V}. As a special case, combining
multiple Record types using the type product operator (\op{\&}) allows a Record
to specify specific value types for different key types, e.g.
\ty{Record K V \& Record L W}.

The main operation of a Record is the indexing operator. The keys of a
Record may be any value; like with Sets, the keys are compared as if by
the structural equality operator (\op{==})

A Record literal is written as comma separated \texttt{key: value} pairs
within curly braces. In Record construction, the spread operator (\op{..})
may be used to combine two or more Records. In pattern matching, the spread
operator may be used once to refer to the rest of the entries not already
matched explicitly by the pattern.

Though Records are unordered, keys specified ``later'' in the source file
(whether literally or via a spread) take precedence over keys specified
earlier.

The string representation of a Record is its stringified \texttt{key: value}
pairs in curly braces.

\begin{table}[H]
    \centering
    \begin{tabular}{ll}
        \hline
        \textbf{Source} & \textbf{String} \\
        \hline
        \texttt{\{1: "one", 2: "two"\}} & \texttt{\{1: one, 2: two\}} \\
        \texttt{\{"hello": 1, "hello": 2, "world": 3\}} & \texttt{\{hello: 2, world: 3\}} \\
        \hline
    \end{tabular}
\end{table}

\subsubsection{Iterator}

Iterators are the result of certain functions; it cannot be specified literally.
Iterators act as ordered sequences that can be ``iterated'' over, yielding its
elements one by one into a pattern. Iteration is performed lazily, therefore
infinite iterators are possible. The Iterator type is a non-structural and
reference type. In type annotation, the Iterator type is written \ty{Iter T}.

Iterators in \Trilogy{} are \emph{single use}, and are typically interacted
with via an iterative context (e.g. a \kw{for} loop or comprehension), but
may also be manually stepped through using the \kw{next} keyword.

As there is currently no way to specify a literal Iterator; there is no
source representation of the Iterator type. Additionally, since an Iterator
is single use and may be infinite, its string representation is simply defined
to be the literal string \texttt{<iterator>}.

