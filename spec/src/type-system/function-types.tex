\subsection{Function Types}

The remaining types are called ``function'' types for lack of a better name,
despite the fact that they are refer to more than just functions in terms of
language constructs.

All function types are non-structural and reference types. Function types do
have string and source representations, but in versions prior to the 1.0
release of \Trilogy{}, converting to source representation of these types will
not be officially defined or supported.

\subsubsection{Procedures}

The first ``function'' type is the Procedure type, which is the type of both
defined procedures and procedure closures.

In type annotation, the Procedure type is written as in \texttt{!(A, B, C) -> T}
where, in this example, \texttt{A}, \texttt{B}, and \texttt{C} are the types of
the parameters and \texttt{T} the type of the return value. The number of parameters
may change depending on the actual signature of the procedure in question.

Values of the Procedure type may be called using the procedure call syntax \texttt{!()},
passing arguments to all of the defined parameters within the parentheses.

Values of Procedure type are created at the top level by \kw{proc} definitions, or as
closures using the keyword \kw{do}.

The string representation of a defined (named) procedure is its name; the name does
not include the \texttt{!} symbol. Meanwhile, the string representation of an anonymous
procedure closure is defined to be the literal string \texttt{<do>}.

\subsubsection{Functions}

Somewhat confusingly, the Function type is a function type, which is the type of
defined functions and function closures.

In type annotation, the Function type is written as \texttt{T -> U}. Notably, the
function type only has space for a single parameter type: because \Poetry{} supports
currying, all functions actually only accept a single parameter and ``multi-parameter''
functions are syntax sugar for single parameter functions returning a closure over
the argument value and accepting the next parameter.

A value of the Function type may be applied by providing its argument directly after,
separated only by a space.

The string representation of a defined (named) function is its name. Meanwhile, the
string representation of an anonymous function closure is defined to be the literal
string \texttt{<fn>}.

\subsubsection{Rules}

The Rule type is the type of defined rules.
