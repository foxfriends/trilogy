\section{Lexical Structure}

Grammars in this document are typically
\href{https://en.wikipedia.org/wiki/Parsing_expression_grammar}{PEGs},
despite maybe looking like BNF (though, really, do they look any different?).
I did not find a particularly good grammar-typesetting package in a brief
Google search. Excuse any typographic inconsistencies within the grammars
at this time.

\subsection{Input Format}

\Trilogy{} input is interpreted as UTF-8 text. Any sequence of bytes which
does not constitute a valid UTF-8 codepoint sequence is not valid \Trilogy{}
code.

Including the Unicode \href{https://en.wikipedia.org/wiki/Byte_order_mark}{byte-order mark}
(BOM \texttt{U+FEFF}) is \emph{strongly recommended against}, but \Trilogy{}
implementations may accept its existence (and ignore it) with a warning.

A valid \Trilogy{} file should end with a new line character (\texttt{\textbackslash n}),
but \Trilogy{} implementations may accept a file that does not have
a trailing new line by implying the existence of one, and emitting a
warning.

\subsection{Keywords}

Keywords in \Trilogy{} have different meanings in the different
sublanguages, but \emph{all} keywords are considered ``reserved''
words in all contexts. That is, they may only be used as keywords,
even in sublanguages which do not have a use for that keyword.
Usage as variable or function identifiers, module names, or as
bare record keys is prohibited.

The keywords are as follows:

\begin{table}[H]
\parbox[t]{0.45\linewidth}{
    \centering
    \begin{tabular}[t]{|ll|}
        \hline
        \textbf{Token Type} & \textbf{Keyword} \\
        \hline
        \texttt{KwAnd} & \kw{and} \\
        \texttt{KwAssert} & \kw{assert} \\
        \texttt{KwAt} & \kw{at} \\
        \texttt{KwBreak} & \kw{break} \\
        \texttt{KwCancel} & \kw{cancel} \\
        \texttt{KwContinue} & \kw{continue} \\
        \texttt{KwDo} & \kw{do} \\
        \texttt{KwElse} & \kw{else} \\
        \texttt{KwExport} & \kw{export} \\
        \texttt{KwFalse} & \kw{false} \\
        \texttt{KwFn} & \kw{fn} \\
        \texttt{KwFor} & \kw{for} \\
        \texttt{KwFrom} & \kw{from} \\
        \texttt{KwFt} & \kw{ft} \\
        \texttt{KwFunc} & \kw{func} \\
        \texttt{KwGiven} & \kw{given} \\
        \texttt{KwIf} & \kw{if} \\
        \texttt{KwImport} & \kw{import} \\
        \texttt{KwInvert} & \kw{invert} \\
        \hline
    \end{tabular}
}
\hfill
\parbox[t]{0.45\linewidth}{
    \centering
    \begin{tabular}[t]{|ll|}
        \hline
        \textbf{Token Type} & \textbf{Keyword} \\
        \hline
        \texttt{KwIn} & \kw{in} \\
        \texttt{KwIs} & \kw{is} \\
        \texttt{KwLet} & \kw{let} \\
        \texttt{KwLoop} & \kw{loop} \\
        \texttt{KwMatch} & \kw{match} \\
        \texttt{KwModule} & \kw{module} \\
        \texttt{KwMut} & \kw{mut} \\
        \texttt{KwNot} & \kw{not} \\
        \texttt{KwOr} & \kw{or} \\
        \texttt{KwProc} & \kw{proc} \\
        \texttt{KwResume} & \kw{resume} \\
        \texttt{KwReturn} & \kw{return} \\
        \texttt{KwRule} & \kw{rule} \\
        \texttt{KwTest} & \kw{test} \\
        \texttt{KwTrue} & \kw{true} \\
        \texttt{KwUnit} & \kw{unit} \\
        \texttt{KwUse} & \kw{use} \\
        \texttt{KwWhen} & \kw{when} \\
        \texttt{KwWhile} & \kw{while} \\
        \texttt{KwYield} & \kw{yield} \\
        \hline
    \end{tabular}
}
\end{table}

\subsection{Reserved Words}

Some additional words are reserved for future usage, following the
same usage restrictions as keywords. This list is intentionally
very restrictive (who knows what ill advised feature I might add
to this language next) and will be relaxed over time as features
are ruled out or implemented as library code. Notably, this includes
most keywords I can think of from other languages.

\begin{table}[H]
\parbox[t]{0.45\linewidth}{
    \centering
    \begin{tabular}[t]{|ll|}
        \hline
        \textbf{Token Type} & \textbf{Keyword} \\
        \hline
        \texttt{KwAsync} & \kw{async} \\
        \texttt{KwAwait} & \kw{await} \\
        \texttt{KwCase} & \kw{case} \\
        \texttt{KwCatch} & \kw{catch} \\
        \texttt{KwClass} & \kw{class} \\
        \texttt{KwConst} & \kw{const} \\
        \texttt{KwData} & \kw{data} \\
        \texttt{KwDefer} & \kw{defer} \\
        \texttt{KwEnum} & \kw{enum} \\
        \texttt{KwExcept} & \kw{except} \\
        \texttt{KwExtends} & \kw{extends} \\
        \texttt{KwImplements} & \kw{implements} \\
        \texttt{KwInline} & \kw{inline} \\
        \texttt{KwInstanceof} & \kw{instanceof} \\
        \texttt{KwInterface} & \kw{interface} \\
        \texttt{KwLazy} & \kw{lazy} \\
        \texttt{KwLens} & \kw{lens} \\
        \texttt{KwMacro} & \kw{macro} \\
        \hline
    \end{tabular}
}
\hfill
\parbox[t]{0.45\linewidth}{
    \centering
    \begin{tabular}[t]{|ll|}
        \hline
        \textbf{Token Type} & \textbf{Keyword} \\
        \hline
        \texttt{KwNext} & \kw{next} \\
        \texttt{KwOper} & \kw{oper} \\
        \texttt{KwPrec} & \kw{prec} \\
        \texttt{KwProtocol} & \kw{protocol} \\
        \texttt{KwStatic} & \kw{static} \\
        \texttt{KwStruct} & \kw{struct} \\
        \texttt{KwSuper} & \kw{super} \\
        \texttt{KwSwitch} & \kw{switch} \\
        \texttt{KwTag} & \kw{tag} \\
        \texttt{KwTrait} & \kw{trait} \\
        \texttt{KwTry} & \kw{try} \\
        \texttt{KwType} & \kw{type} \\
        \texttt{KwTypeof} & \kw{typeof} \\
        \texttt{KwUnless} & \kw{unless} \\
        \texttt{KwUntil} & \kw{until} \\
        \texttt{KwVar} & \kw{var} \\
        \texttt{KwWhere} & \kw{where} \\
        \hline
    \end{tabular}
}
\end{table}

\subsection{Identifiers}

Identifiers in \Trilogy{} are currently restricted to US-ASCII characters,
with the intent being to support Unicode identifiers (including Emoji) in
a later edition.

Additionally, there are two other types of identifiers that are used in
special cases, those ending with an exclamation mark (\texttt{!}) used as
procedure names, and those ending with an equal sign (\texttt{=}) for
functions when used as an operator.

An additional restriction: the single underscore (\texttt{\_}) is never an
identifier, and is instead classified separately.

\begin{bnf*}
    \bnfprod{Discard}{\bnfts{\_}}\\
    \bnfprod{IdentifierEq}{\bnfpn{id}\bnfsp\bnfts{=}}\\
    \bnfprod{IdentifierBang}{\bnfpn{Identifier}\bnfsp\bnfts{!}}\\
    \bnfprod{Identifier}{\bnfpn{id}\quad\bnftd{excluding keywords}}\\
    \bnfprod{id}{\bnfts{[a-zA-Z\_][a-zA-Z0-9\_]*}} \\
\end{bnf*}

\subsection{Atom Literals}

Atoms are represented as identifiers with a leading single-quote (\texttt{'}).

\begin{bnf*}
    \bnfprod{Atom}{\bnfts{'}\bnfsp\bnfpn{Identifier}} \\
\end{bnf*}

\subsection{Numeric Literals}

Numbers in \Trilogy{} are real numbers, as in supporting arbitrary
precision and largeness. Complex numbers are supported as well
(not that I have ever required the use of complex numbers in any
piece of code) for completeness.

Numeric tokens are of the following structure:

\begin{bnf*}
    \bnfprod{Numeric}{\bnfpn{real}\bnfor\bnfpn{real}\bnfsp\bnfpn{complex}} \\
    \bnfprod{real}{\bnfpn{integer}\bnfor\bnfpn{rational}\bnfor\bnfpn{float}} \\
    \bnfprod{integer}{
        \bnfpn{decimal}
        \bnfor
        \bnfpn{hexadecimal}
        \bnfor
        \bnfpn{binary}
        \bnfor
        \bnfpn{octal}
    } \\
    \bnfprod{decimal}{\bnfts{[1-9][0-9\_]*}} \\
    \bnfprod{hexadecimal}{\bnfts{0x}\bnfsp\bnfts{[0-9A-Fa-f][0-9A-Fa-f\_]*}} \\
    \bnfprod{binary}{\bnfts{0b}\bnfsp\bnfts{[01][01\_]*}} \\
    \bnfprod{octal}{\bnfts{0o}\bnfsp\bnfts{[0-7][0-7\_]*}} \\
    \bnfprod{rational}{\bnfpn{integer}\bnfsp\bnfts{/}\bnfsp\bnfpn{integer}} \\
    \bnfprod{float}{\bnfpn{decimal}\bnfsp\bnfts{.}\bnfsp\bnfpn{decimal}} \\
    \bnfprod{complex}{\bnfts{i}\bnfsp\bnfpn{real}}
\end{bnf*}

Negative numbers do not exist in source code, and are actually
an appliction of the unary negation operator \op{-} to a non-negative number.

Numbers with exponential components also do not exist. Similar results
can be achieved using tagging.

Numbers may contain the underscore character (\texttt{\_}) as a separator.
There are no restrictions on the placement of underscores, beyond that they
may not start a number. \val{1000000}, \val{1\_000\_000}, and
\val{1\_0\_0\_\_\_000\_0} are all valid number literals representing one
million.

Worth noting as well, rational number literals (e.g. \texttt{2/3}) are
parsed as a single token, while runtime division (e.g. \texttt{2 / 3})
is parsed as three separate tokens. Be careful of this distinction later
when writing or parsing expressions, as these two representations will
lead to different results in ways not relevant to other languages without
rational literals.

As they are not often seen in other languages (and similarly unlikely to
be witnessed in \Trilogy{}), complex numbers are written such as \val{13i5}
to represent $13 + 5i$. On either side of the \texttt{i} may be a valid
real number (including rational or floating point numbers). Neither value
may be omitted (i.e. \val{0i5} not \texttt{i5}, and \val{13i0} not
\texttt{13i}). A negative complex number $15 - 5i$ is written \val{15 - 0i5}
(there is no literal, this is just a subtraction).

\subsection{String and Character Literals}

Strings in \Trilogy{} are always enclosed in double quotes (\texttt{"}).

At this time, one other special type of string, the ``template string'',
is prefixed with a \texttt{\$}. In future, other special strings may be %
implemented. String tokenization ends up producing many types of tokens
for template strings, where template strings containing no interpolations
are parsed as regular strings:

\begin{bnf*}
    \bnfprod{String}{
        \bnfts{"}
        \bnfsp
        \bnfpn{content}
        \bnfsp
        \bnfts{"}
        \bnfor
        \bnfts{\$"} %
        \bnfsp
        \bnfpn{content}
        \bnfsp
        \bnfts{"}
    }\\
    \bnfprod{TemplateStart}{
        \bnfts{\$"} %
        \bnfsp
        \bnfpn{content}
        \bnfsp
        \bnfts{\$\{} %
    } \\
    \bnfprod{TemplateMiddle}{
        \bnfts{\}}
        \bnfsp
        \bnfpn{content}
        \bnfsp
        \bnfts{\$\{} %
    } \\
    \bnfprod{TemplateEnd}{
        \bnfts{\}}
        \bnfsp
        \bnfpn{content}
        \bnfsp
        \bnfts{"}
    } \\
    \bnfprod{content}{\bnftd{Any UTF-8 characters or escape sequences}} \\
\end{bnf*}

Similarly, characters are always enclosed in single quotes (\texttt{'}).
Unlike many other dynamically typed languages, \Trilogy{} \emph{does} differentiate
between characters and strings. Character literals may contain either a
single (non-single-quote) character, or a single escape sequence.

\begin{bnf*}
    \bnfprod{Character}{\bnfts{'}\bnfsp\bnfpn{char}\bnfsp\bnfts{'}} \\
    \bnfprod{char}{\bnftd{One Unicode character}\bnfor\bnftd{One escape sequence}} \\
\end{bnf*}

\subsubsection{Escape Sequences}

Within character and string literals, escape sequences may be used to denote
special characters, typically ones that would otherwise be hard to physically
type or print and read.

Valid escape sequences and their meanings are listed here:

\begin{table}[H]
    \centering
\begin{tabular}[c]{|ll|}
    \hline
    \textbf{Code} & \textbf{Usage} \\
    \hline
    \texttt{\textbackslash n} & New line \\
    \texttt{\textbackslash r} & Carriage Return \\
    \texttt{\textbackslash t} & Tab \\
    \texttt{\textbackslash\textbackslash} & Backslash \\
    \texttt{\textbackslash 0} & Null \\
    \texttt{\textbackslash '} & Single Quote \\
    \texttt{\textbackslash "} & Double Quote \\
    \texttt{\textbackslash \$} & Dollar Sign \\ %
    \texttt{\textbackslash x41} & ASCII escape \\
    \texttt{\textbackslash u\{7FFF\}} & Unicode escape sequence \\
    \hline
\end{tabular}
\end{table}

Escape sequences in both strings and characters are resolved by the lexer
and replaced before the tokens are handed off to further passes.

\subsection{Whitespace}

\Trilogy{} aims to be (mostly) a ``free-form'' language, meaning that
whitespace holds no special meaning. The only exception to this rule
is when line breaks may lead to differing results, the specifics of which
are detailed later.

Otherwise, all whitespace may be treated equally, though the only supported
whitespaces at this time are the four ASCII whitespace characters listed:

\begin{itemize}
    \item \texttt{U+0009} (tab \texttt{'\textbackslash t'})
    \item \texttt{U+000A} (line feed \texttt{'\textbackslash n'})
    \item \texttt{U+000D} (carriage return \texttt{'\textbackslash r'})
    \item \texttt{U+0020} (space \texttt{' '})
\end{itemize}

For the purposes of ending lines, the only valid whitespace character is
\texttt{U+000A} (line feed \texttt{'\textbackslash n'}).
\texttt{U+000D} (carriage return \texttt{'\textbackslash r'}) is ignored,
as with any other whitespace, for any purpose other than separating tokens.

When \texttt{U+000A} (line feed \texttt{'\textbackslash n'}) is encountered
in the source code (other than within a string literal), the token
\texttt{EndOfLine} is emitted. No other whitespace constitutes a token.

\subsection{Comments}

Comments hold no semantic meaning, but they may constitute a line break,
if there is a line break within the comment. A line comment continues until and
includes the line feed character, and so are always considered as line breaks.
Comments which do not contain a line feed are treated as a non-line
break whitespace. Block comments may be nested.

To distinguish between line-breaking comments and non-line breaking comments,
block comments have two different token types.

\begin{bnf*}
    \bnfprod{CommentInline}{
        \bnfts{\#-}
        \bnfsp
        \bnftd{No line feed}
        \bnfsp
        \bnfpn{CommentInline}
        \bnfsp
        \bnftd{No line feed}
        \bnfsp
        \bnfts{-\#}
        \bnfor
    } \\
    \bnfmore{\bnfts{\#-} \bnfsp \bnftd{No line feed} \bnfsp \bnfts{-\#}} \\
    \bnfprod{CommentBlock}{
        \bnfts{\#-}
        \bnfsp
        \bnftd{Anything}
        \bnfsp
        \bnfpn{CommentBl}
        \bnfsp
        \bnftd{Anything}
        \bnfsp
        \bnfts{-\#}
        \bnfor
    } \\
    \bnfmore{\bnfts{\#-} \bnfsp \bnftd{Anything} \bnfsp \bnfts{-\#}} \\
    \bnfprod{DocOuter}{\bnfts{\#\#} \bnfsp \bnftd{All characters until line feed}} \\
    \bnfprod{DocInner}{\bnfts{\#!} \bnfsp \bnftd{All characters until line feed}} \\
    \bnfprod{CommentLine}{\bnfts{\#} \bnfsp \bnftd{All characters until line feed}} \\
\end{bnf*}

\subsection{Punctuation}

Punctuation includes symbols and operators that treated as whole tokens when encountered
on their own.

Keep in mind the slash (\texttt{/}) also appears as part of rational number literals, so
will not be exclusively identified as an individual token.

\begin{table}[H]
\parbox[t]{0.45\linewidth}{
    \centering
    \begin{tabular}[t]{|ll|}
        \hline
        \textbf{Token Type} & \textbf{Symbol} \\
        \hline
        \texttt{OpEq} & \op{=} \\
        \texttt{OpLt} & \op{<} \\
        \texttt{OpGt} & \op{>} \\
        \texttt{OpEqEq} & \op{==} \\
        \texttt{OpLtEq} & \op{<=} \\
        \texttt{OpGtEq} & \op{>=} \\
        \hline
        \texttt{OpAmp} & \op{\&} \\
        \texttt{OpPipe} & \op{|} \\
        \texttt{OpCaret} & \op{\textasciicircum} \\
        \texttt{OpTilde} & \op{\textasciitilde} \\
        \texttt{OpShr} & \op{\textasciitilde>} \\
        \texttt{OpShl} & \op{<\textasciitilde} \\
        \hline
        \texttt{OpAmpEq} & \op{\&=} \\
        \texttt{OpPipeEq} & \op{|=} \\
        \texttt{OpCaretEq} & \op{\textasciicircum=} \\
        \texttt{OpTildeEq} & \op{\textasciitilde=} \\
        \texttt{OpShrEq} & \op{\textasciitilde>=} \\
        \texttt{OpShlEq} & \op{<\textasciitilde=} \\
        \hline
        \texttt{OpAt} & \op{@} \\
        \texttt{OpDot} & \op{.} \\
        \texttt{OpDotDot} & \op{..} \\
        \texttt{OpComma} & \op{,} \\
        \texttt{OpColon} & \op{:} \\
        \texttt{OpSemi} & \op{;} \\
        \hline
    \end{tabular}
}
\hfill
\parbox[t]{0.45\linewidth}{
    \centering
    \begin{tabular}[t]{|ll|}
        \hline
        \textbf{Token Type} & \textbf{Symbol} \\
        \hline
        \texttt{OpPlus} & \op{+} \\
        \texttt{OpMinus} & \op{-} \\
        \texttt{OpStar} & \op{*} \\
        \texttt{OpSlash} & \op{/} \\
        \texttt{OpSlashSlash} & \op{//} \\
        \texttt{OpPercent} & \op{\%} \\
        \texttt{OpStarStar} & \op{**} \\
        \hline
        \texttt{OpPlusEq} & \op{+=} \\
        \texttt{OpMinusEq} & \op{-=} \\
        \texttt{OpStarEq} & \op{*=} \\
        \texttt{OpSlashEq} & \op{/=} \\
        \texttt{OpSlashSlashEq} & \op{//=} \\
        \texttt{OpPercentEq} & \op{\%=} \\
        \texttt{OpStarstarEq} & \op{**=} \\
        \hline
        \texttt{OpLtLt} & \op{<<} \\
        \texttt{OpGtGt} & \op{>>} \\
        \texttt{OpPipeGt} & \op{|>} \\
        \texttt{OpLtPipe} & \op{<|} \\
        \hline
    \end{tabular}
}
\end{table}

\subsection{Delimeters}

Delimiters typically surround other pieces of code, and must come in
matched opening-closing pairs.

\begin{table}[H]
    \centering
    \begin{tabular}[t]{|ll|ll|}
        \hline
        \textbf{Token Type} & \textbf{Symbol}& \textbf{Token Type} & \textbf{Symbol} \\
        \hline
        \texttt{OBrace} & \texttt{\{}& \texttt{CBrace} & \texttt{\}} \\
        \texttt{OParen} & \texttt{(} & \texttt{CParen} & \texttt{)} \\
        \texttt{OBrack} & \texttt{[} & \texttt{CBrack} & \texttt{]} \\
        \hline
    \end{tabular}
\end{table}

\subsection{Shebang}

A \Trilogy{} file may optionally begin with a
\href{https://en.wikipedia.org/wiki/Shebang_(Unix)}{shebang}. This has no bearing
on how the program is run or interpreted when provided to a \Trilogy{} implementation,
but may affect how the operating system treats the file.

Conveniently, this matches standard comment syntax. Inconveniently, it is not
likely that it is meant to be interpreted as a doc comment. Conveniently,
we don't care, and so it is up to the documentation generator to detect and
hide a shebang from generated documentation if so desired.
