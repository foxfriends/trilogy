\begin{titlepage}
    \maketitle
    \thispagestyle{empty}
    \pagenumbering{gobble}

    \begin{abstract}
        \noindent
        Likely an unnecessarily formal specification for this particular language,
        \Trilogy{} can be seen as a collection of at least three distinct languages
        in a single box set. I say ``at least'' because, depending on your
        interpretation of ``language'', you might find more than three. \Trilogy{}
        attempts to push the boundaries of language design to incorporate the best
        of features found in a wide range of programming languages including Rust,
        Haskell, Prolog, Javascript, ML, and more whose influences are subtle enough
        that I can't even identify them directly. Prioritizing practicality and
        experimentation equally, \Trilogy{} includes abstract concepts such as
        algebraic effects and inherent non-determinism as fundamental control flow
        mechanisms, while also including modern practical features---language-level
        support for unit testing, documentation, and code splitting, to name a
        few---that any new language these days would be foolish to leave out.
    \end{abstract}
\end{titlepage}

\pagenumbering{arabic}

\tableofcontents

\newpage
\import{introduction/}{index.tex}

\newpage
\import{overview/}{index.tex}

\newpage
\import{lexical-structure/}{index.tex}

\newpage
\import{type-system/}{index.tex}

\newpage
\import{syntax-and-semantics/}{index.tex}

\newpage
\import{programs/}{index.tex}

\newpage
\bibliographystyle{plain}
\bibliography{index}
